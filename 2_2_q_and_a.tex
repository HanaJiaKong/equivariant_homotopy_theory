\begin{ques}
So, why does anyone care about Spanier-Whitehead duality?
\end{ques}
One answer is that Spanier-Whitehead duality formally implies Poincaré duality. Poincaré duality is a remarkable
fact about the cohomology of manifolds, which is one good reason to care.

Another is that it's the real setting for the Euler characteristic --- really. The Euler characteristic arises as a
trace, and therefore some kind of categorical duality should appear.

Being more specific, let's consider a closed symmetric monoidal category $\fC$ with unit $\Sph$ (so secretly we're
thinking of the stable homotopy category). Spanier-Whitehead duality consists of two maps $\e\colon X\wedge
Y\to\Sph$ and $M\colon \Sph\to X\wedge Y$, such that the composition
\[\xymatrix{
	X\cong \Sph\wedge X\ar[r]^{M\wedge\id} & X\wedge Y\wedge X\ar[r]^{\id\wedge(\e\circ\tau)} &X\wedge\Sph\cong X
}\]
is the identity.

Using an adjunction, you can show $X\cong F(Y,\Sph)$ and $Y\cong F(X,\Sph)$, and so $X$ is the dual of its dual.
This has lots of formal consequences, includin Poincaré duality, the construction of things such as the Euler
characteristic, and more. The Spanier-Whitehead dual of $X$ is often denoted $DX$.

Suppose that we're in an algebraic category, say $H\Q$-module spectra (the rational stable homotopy category).
Then, $H\Q$ is the unit, so $DX = C^*(X;\Q)$ (maps from $X$ to $\Q$). That is, the Spanier-Whitehead dual embeds
the unstable homotopy theory of $X$, as long as you remember the $\E_\infty$-ring structure on $DX$ (which is
bizarre, e.g.\ it's not connective). There's a quite nontrivial theorem that $\E_\infty$ maps between $DX$ and $DY$
correspond to maps of \emph{spaces} between $X$ and $Y$.

This sort of data is useful for algebraicizing manifolds, and that would be nice for classifying manifolds, a goal
with fairly broad applications outside of homotopy theory. And Spanier-Whitehead duality has important consequences
on this: Sullivan began it by showing~\cite{SullivanQHT} that the $\E_\infty$-ring structure on $C^*(X;\Q)$
controls the rationalization $X_\Q$, and Mandell ended it by showing that the $\E_\infty$ structure on
$C^*(X;\F_p)$ controls $X_p^\wedge$.
\begin{ques}
What are some of the obstacles to extending equivariant homotopy theory to groups that aren't compact Lie groups?
\end{ques}
One cornerstone of homotopy theory is the Pontrjagin-Thom construction, which depends on the weak Whitney embedding
theorem: that a manifold can be embedded in some high-dimensional $\R^N$. This still works equivariantly \emph{but
only for compact Lie groups} --- a $G$-manifold $X$ can be embedded in some high-dimensional $G$-representation.
This is again an important piece in the equivariant Pontrjagin-Thom construction, and its failure for noncompact
groups is one major reason things don't work. There are other things that require compactness (e.g.\ descending
chain arguments). People have tried and not gotten anywhere.

In stable homotopy theory, there's a modification of the orbit category into something called the Burnside
category, and we'll see that it controls a lot of the stable homotopy theory of $G$-spaces. In fact, if you have
something that strongly resembles the Burnside category, you have something that looks like equivariant stable
homotopy theory. Clark Barwick and his collaborators have been working on studying presheaves on things that look
like Burnside categories.

A lot of it boils down to the fact that compact Lie groups have a tractable representation theory.
\begin{ques}
Recall $E\sF$, the classifying space for a family of subgroups. What is it used for?
\end{ques}
If you want to focus attention on a family of subgroups, you play with $E\sF$. One common example is $S^1$-spaces,
in which there are many constructions that are fixed by the finite subgroups of $S^1$, so having that family $\sF$
is helpful, and for this one can smash with $E\sF$.

There are various other applications. One is called \term{isotropy separation}, which splits up $\sF$ into pieces
that can be detected with different kinds of isotropy subgroups, and one can induct on this in nice cases.
\begin{ques}
What does it mean that $K$-theory and the Adams operations determine the tangent bundle?
\end{ques}
It's a somewhat implicit statement: any possible algebraic statement about the tangent bundle can be translated
into an equivalent statement in $K(X)$ that uses the Adams operations.
\begin{ques}
How many notions of $G$-spectra are there?
\end{ques}
One way you talk about equivariant stable homotopy theory is a \term{universe}, a countably infinite-dimensional
inner product space containing the irreducible representations you care about infinitely often. Then, you have
Spanier-Whitehead duality for $G/H$ iff $G/H$ embeds in the universe, and you get different flavors of homotopy
theory depending on which universes you use. There are lots of models here --- depending on what you mean, spectra
objects or diagrams on $\sO_G$ might not be the right thing for naïve spectra; instead, you need transfers, so you
have to consider sheaves on the Burnside category.

The choice of universe also affects which suspensions you invert: if $V$ is in your universe, you can invert
$\Sigma^V$.

Of course, there are probably many different point-set models for stable homotopy, but they'll give you the same
answer.
\begin{ques}
Can you go over how to design a coefficient system such that its cohomology is something obtained from $X$?
\end{ques}
For example, let's try to determine $M$ such that $H_G^*(X;M)\cong H^*(X)$. The reason it suffices to determine
what $M(G/H)$ is on all closed subgroups $H\subset G$ is the dimension axiom!

So if $G = \Z/p$, you can assign $M(G/e) \coloneqq H^0(G/e)$ and $M(G/G) \coloneqq H^0(G/G)$, and the map between
them is $H^0$ of the map $G/e\to G/G$.
\begin{ques}
So far, we've mostly seen $G$-equivariant homotopy theory where $G$ is a finite cyclic group. Is there anything
interesting for nonabelian groups, etc?
\end{ques}
Part of the problem is that computing examples is not easy, and it gets much harder when you have a complicated
lattice of subgroups.~\cite{LMS} is 500 pages, and there are scarcely any examples, because the computations for
nontrivial examples are so hard!

Peter May invented this stuff and set a bunch of grad students to work on it, but there wasn't a lot of buzz until
Carlsson proved the Segal conjecture, and then again more recently with Hill-Hopkins-Ravenel. So there haven't been
a lot of computations, period. If you do make an interesting computation with, say, the monster group, by all means
write it up!

So for the most part people have studied cyclic groups and $S^1$. There's been a little discussion of dihedral
actions, and some stuff with the symmetric groups.

There's also some stuff done for profinite groups. This is in some sense easy to set up formally, especially
because you mostly care about finite-index subgroups. People who study Galois actions (e.g.\ Carlsson's program to
lift the Quillen-Lichtenbaum conjecture into this context) care about this.
\begin{ques}
So we've spent some time looking at $EG$, a contractible space with a free $G$-action, and its quotient $BG$. What
things do people do with these objects?
\end{ques}
$BG$ is a classifying space for principal $G$-bundles (and therefore for $G = \O_n$ or $\U_n$, also vector bundles
of rank $n$). That is, homotopy classes of maps $X\to BG$ are identified with isomorphism classes of principal
$G$-bundles on $X$. There's a book by May, ``Classifying spaces and fibrations,'' which is excellent and goes into
great detail on this stuff. Because $B\U_n$ classifies complex vector bundles of rank $n$, it's used to construct
complex $K$-theory (and same for $B\O_n$ and real $K$-theory).

In our case, smashing with $EG$ is often a way to localize.

A third reason to care about this is group cohomology
and other purely algebraic stuff. The group cohomology of $G$ is the cohomology of $BG$, and there are plenty of
applications of group cohomology.

So though $BG$ may be infinite-dimensional, it's very simple homotopically.
\begin{ques}
What's going on with the construction to the right adjoint to the functor $\psi$ in the proof of Elmendorf's
theorem? What's a coend?
\end{ques}
Adrian said some stuff here; I wasn't able to get it down. He motivated the bar construction as the thing whose
homotopy colimits are ordinary colimits, I think?

Anyways, our setup is that we have a presheaf on the orbit category $X\in\Fun(\sO_G\op,\Top)$ and want to produce a
$G$-space. The right adjoint\footnote{Recall that the left adjoint was evaluation at $G/e$.} to $\psi$ is the
geometric realization of the bar construction $B_\bullet(X,\sO_G,M)$. The bar construction is a generalization of
the bar resolution to compute the derived tensor product: the functor $M\colon \sO_G\to\Top$ sending $G/H$ to the
space $G/H$ is thought of as a ``right $\sO_G$-module;'' instead of tensoring a bunch of elements togther, we get a
bunch of arrows, meaning we can replace with a coproduct. In some sense, it would be nice to take a tensor product
with $X$, but we have to do so in a derived sense, hence the bar construction.

A coend is a functor that behaves like geometric realization: there's two functors with opposite variance, and you
want to glue along their common edge, just like in geometric realization.
\begin{ques}
How did Elmendorf formalize his proof, given that it was done before model categories were available?
\end{ques}
He didn't: Elmendorf's paper is eminently readable, and simply provides an equivalence of homotopy categories. It
was not lifted into model categories unti much later. This stuff was all put into use fairly recently: for example,
not that long ago, it was known to experts but not written down that a left Quillen adjoint that's part of a
Quillen equivalence preserves homotopy limits.

Here, someone asked about the Freudenthal suspension theorem, and this led to a digression.
\begin{rem}
Modern cryptography depends on some hardness assumptions, that some functions, such as the discrete log in a finite
field, are hard to compute (but easy to check answers to). There's a paper by Impagliazzo which asks what
cryptography and security would look like if certain assumptions were false or true, with cute names for different
worlds. Imagine doing that for the Freudenthal suspension theorem --- what if the stable range were at about $n$
instead of about $2n$? What if it were $n/2$?
\end{rem}
\begin{ques}
In non-equivariant rational homotopy theory, there's a standard, completely algebraic description of the rational
homotopy category. Does this also work for the rational equivariant homotopy category?
\end{ques}
This is very hard --- someone was working on this, but the work actually depended on an incorrect calculation, and
some of it had to be redone. There are people working on an algebraic model for rational $G$-spectra, but the
algebraic models are \emph{very} complicated.
\begin{ques}
We've seen that representations of $G$ play a huge role in equivariant homotopy theory; these could be thought of
as relating to $K$-theory of $BG$ (e.g.\ the representation ring $\mathrm{RO}(G)$ is $KO^0(BG)$). More generally,
do $\mathit{KO}(BG)$ or $\mathit{MO}(BG)$ play a special role ``controlling'' the $G$-equivariant stable category?
% Strickland K(n)-local duality for finite groups and groupoids
\end{ques}
People certainly care about computing $K$-theory or bordism of $BG$. As to whether those control the $G$-equivariant stable category---that might be true, but I don't think anything like that has been said.
\begin{ques}
This question comes from GitHub: suppose you're defining a coefficient system $M$ over $G = C_p$. This is
determined by $M(C_p/e)$, $M(C_p/C_p)$, and a map between them. How do you get the map?
\end{ques}
The map is determined by functoriality: in the cases we used to prove \cref{smith}, which is where this question
arose, $M(C_p/e)$ and $M(C_p/C_p)$ are both defined to be $H^0$ of something, so the map between them is $H^0$ of
the map $C_p/e\to C_p/C_p$.
\begin{ques}
Do you want to talk about the unstable Steenrod functor $\Un$?
\end{ques}
You don't want to know.
\begin{ques}
Does $\Un$ appear in the theorem as a technical necessity or for deep reasons?
\end{ques}
It's very deep, and there's a whole theory of unstable modules over the Steenrod algebra. This involves some extra
restrictions.
\begin{ques}
So how many kids did Denis Sullivan have?
\end{ques}
About 7 or 8 by the end of~\cite{MITNotes}; one is now also a math professor. There are lots of other anecdotes in
the notes: once he was frustrated with some calculations he was doing, so he hurled the book into the Atlantic
ocean. While he was on a boat to Oxford.
\begin{ques}
Do we have any previews of the Clark school, besides what few things they've posted on the arXiv?
\end{ques}
They've been doing some really hard stuff: the idea is that $G$-spectra should be presheaves of spectra on Burnside
categories, or statements like $G$-spectra being the initial stable $G$-symmetric monoidal category\dots{} you know
what the theorems should be, but the proofs are technical. Riehl-Verity technology should make some of this easier,
hopefully.
\begin{ques}
What does it mean that ``the Spanier-Whitehead dual of a point involves taking shifts?''
\end{ques}
We've been a little careless about basepoints: this is really the pointed point, so $S^0$; thus, we really should
have said the Spanier-Whitehead duals of spheres.
\begin{ques}
The Fruedenthal suspension theorem requires inclusion of the basepoint to be a cofibration. If you add a disjoint
basepoint, is that the case?
\end{ques}
Yes. You can think of a cofibration as a slightly weaker version of a closed inclusion. This is no big technical
obstacle: the \term{whisker construction} replaces the basepoint with a line segment, so the homotopy theory is
unchanged and the other end of the line segment can be a basepoint that's cofibrantly included.
\begin{ques}
What's a natural example of a genuine $G$-spectrum of interest?
\end{ques}
One is the equivariant sphere spectrum. Thinking of cohomology theories, there's equivariant $K$-theory
$\mathit{KU}_G$, different kinds of equivariant bordism, and $\mathit{KR}$ (a genuine equivariant spectrum that's
built from $K$-theory).
\begin{ques}
Is there an equivariant approach to de Rham theory, e.g.\ starting with manifolds and equivariant differential
forms?
\end{ques}
Rationally, this can work, but integrally there are issues. Relatedly, defining a $G$-manifold is not too hard, and
there are nice things like the equivariant tubular neighborhood theorem, but there are some subtleties.
\begin{ques}
One of the interpretations of the homotopy category of a model category is localization at the weak equivalences.
This suggests that one only needs to look at weak equivalences. So if you only specify weak equivalences, do you
get a unique model category?
\end{ques}
No, you might not get one at all, and if you do, it won't be unique. What it means is (modulo some scary
set-theoretic issues) that the homotopy theory only depends on the weak equivalences. If you have a category with
weak equivalences, you can form the \term{Dwyer-Kan localization}, an associated simplicial category, and then take
the associated quasicategory or $\infty$-category. Under some hypotheses, you can then retreat to a
model-categorical structure. One commonly heard analogy is that a model category is like choosing a basis for a
vector space (where the basis-independent answer reflects the underlying $\infty$-category).

Another perspective is that model categories are like axiomatic obstruction theory, axiomatizing the cellular
inclusions (cofibrations) or extension questions common when considering CW theory.
\begin{ques}
One way to define the stable category is to invert the canonical map from a finite wedge of things to a product of
those things. How can we make this precise?
\end{ques}
The goal is to localize with respect to something. There's some set-theoretical issues that Bousfield was good at
addressing, but if you have a set of morphisms, you can localize with respect to them, and that's what we're doing
here. The Stacks project is a good resource for learning about localization.

The fact that finite wedges and finite products are the same is another way that spectra resemble abelian groups,
for which finite products and finite coproducts agree.
\begin{ques}
Where's the best place to learn non-equivariant stable homotopy theory?
\end{ques}
Adams' book~\cite{AdamsStableHomotopy}, with good calculations, but its construction of the stable category isn't
so pleasant. It might be better to consider~\cite{MMSS}. It assumes you already know why you care, then takes the
historical constructions and shows how they all relate together. It may seem scary, but looking at the
constructions shows they're all not so bad.

Adams' infinite loop spaces book~\cite{InfiniteLoopSpaces} is good bedtime reading.
\begin{ques}
We talked about the notion of an excisive functor, and there's a notion of an excisive pair in unstable homotopy
theory. Are the two related?
\end{ques}
Both deserve the name ``excisive'' because they satisfy some sort of Mayer-Vietoris principle: excisive functors
take pushouts to pullbacks, for example.
\begin{ques}
By $\RP^\infty$, do we just mean the colimit of $\RP^n$ over all $n$?
\end{ques}
Yes, and also for $\CP^\infty$. However, there are models for them based on the unit sphere in an
infinite-dimensional Hilbert space modulo a $\Z/2$ (resp.\ $S^1$) action. This is homotopic to, but not
homeomorphic to, the colimit realizations of $\RP^\infty$ and $\CP^\infty$.

