\begin{quote}\textit{
	``The sun, it burns!''
}\end{quote}

Before constructing the equivariant stable category, we'll provide some motivation, much of which is also good
motivation for the nonequivariant stable category. There's a choice here: you could just say ``let's take the
category of orthogonal $G$-spectra,'' but having some motivation for why we're doing what we're doing is important.

There are lots of ways to think about where stabilization comes from.
\begin{enumerate}
	\item The Freudenthal suspension theorem\index{Freudenthal suspension theorem} says that if $X$ is
	nondegenerately based (meaning the based inclusion map $*\inj X$ is a cofibration) and $n-1$-connected, then
	$\pi_q(X)\to\pi_{q+1}(\Sigma X)$ is an isomorphism for $q < 2n-1$ and a surjection when $q =
	2n-1$.\footnote{The map comes from the loop-suspension adjunction,\index{loop-suspension adjunction} which
	gives us a unit $X\to\Omega\Sigma X$, hence a map $\Omega^qX\to\Omega^{q+1}\Sigma X$, and the map on homotopy
	groups is $\pi_0$ of that map. This is the based version of the mapping space and Cartesian product adjunction:
	$\Sigma X\coloneqq S^1\wedge X$ and $\Omega X \coloneqq \Map(S^1, X)$ are adjoint functors.} It's easier to see
	that cohomology groups are stable under suspension, but this tells us that homotopy groups stabilize in a range
	that increases at about twice the rate that the connectivity of $X$ does. Since $\Sigma^n X$ is at least
	$n$-connected, this suggests you could replace $X$ by the sequence $X, \Sigma X,\dotsc,\Sigma^n X,\dotsc$, and
	keep track of that instead, regarding it as a repository for the \term{stable homotopy groups}
	$\pi_n^S(X)\coloneqq\colim_k \pi_{n+k}(\Sigma^k X)$.  One way to think of this is as formally making homotopy
	theory into a homology theory (which it isn't \latin{a prioiri}); you end up taking the same kind of colimit.

	You could do this equivariantly: we have representation spheres. But it's not entirely clear what to
	do.\index{stable homotopy category!as setting for stable homotopy groups}
	\item Another perspective is that the stable category is the result of inverting the canonical map\footnote{The
	existence of this map follows from the universal property of the product.}
	\begin{equation}
	\label{wedges_are_products}
	\bigvee_{i=1}^k X\longrightarrow\prod_{i=1}^k X.
	\end{equation}
	Again, this is something you can think about making precise; the stable category is the initial triangulated
	category constructed from $\Top$ in which~\eqref{wedges_are_products} is an isomorphism. In particular, this
	forces the homotopy category to be additive.

	Again, we could do this equivariantly.\index{stable homotopy category!as initial category where finite sums and
	products agree}
	\item Suppose we have a functor $F$ from finite CW complexes to $\Top$ such that $F(*) = *$, and suppose $F$
	commutes with filtered colimits and preserves weak equivalences (e.g.\ if it's topologically or simplicially
	enriched, for formal reasons). By taking colimits, we can obtain a functor $\widehat F$ from CW complexes to
	$\Top$. We say $F$ is \term[excisive functor]{excisive} if it takes pushouts to pullbacks; this is an old
	perspective, which was used to show the Dold-Thom theorem, that the infinite symmetric product
	$\mathrm{SP}^\infty$ is a cohomology theory. In any case, if $F$ is excisive, $\set{F(S^n)}$ represents a
	cohomology theory. Namely, the homotopy pushout\index{Dold-Thom theorem}
	\[\xymatrix{S^n\ar[r]\ar[d] & D^{n+1}\ar[d]\\
	D^{n+1}\ar[r] & S^{n+1}
	}\]
	creates suspension, and via $F$, becomes a pullback creating $\Omega$. Asking what the excisive functors are
	leads to the stable category, and is also something you could do equivariantly.\index{stable homotopy
	category!as setting of excisive functors}
	\item Another perspective: what is the Spanier-Whitehead dual of a point? Taking shifts, what's the
	Spanier-Whitehead dual of $S^n$? Equivariantly, one wants to know the Spanier-Whitehead duals of $G/H_+$. This
	is important for defining Poincaré duality, etc. Nonequivariantly, the best way to answer this is the
	Pontrjagin-Thom construction, which not only answers this, but provides a deep understanding for what the
	stable homotopy groups of the spheres are. In this chapter, we'll do this equivariantly, and it will tell us
	what the spheres are.\index{stable homotopy category!as setting of Spanier-Whitehead
	duality}\index{Spanier-Whitehead duality}
\end{enumerate}
Pursuant to motivating the stable category, this section is about duality. This is a reinvention of the original
construction of the stable category --- Spanier's original construction of spectra was motivated by answering
questions on duality, and we'll proceed similarly, if a bit ahistorically. Since we're in the equivariant setting,
the answers will be slightly different.

Alexander duality is a tale as old as time, from what could be called the prehistory of algebraic
topology.\footnote{If you read between the lines, you can find it in Poincaré's works, but it's from the 1940s or
50s stated explicitly.}
\begin{thm}[Alexander duality~\cite{Alexander}]
\index{Alexander duality}
Let $K\subset S^n$ be compact, locally contractible, and nonempty.\footnote{Classically, one works simplicially,
picking a triangulation of $S^n$ and letting $K$ be a subpolyhedron.} Then, $K$ and $S^n\setminus K$ are
\term*{Alexander dual} in that there is an isomorphism
\[\wH^{n-i-1}(K;\Z)\cong \wH_i(S^n\setminus K;\Z).\]
\end{thm}
The proof isn't too hard, e.g.\ Hatcher does it. This is closely related to considering embeddings in Euclidean
spaces, after you take the one-point compactification.

The good part of this proof is that it doesn't depend on the embedding. But there are a few drawbacks:
\begin{enumerate}
	\item $K$ does not determine the homotopy type of $S^n\setminus K$. Knot theory is full of examples, and they
	tell you that the issues arise for the fundamental group.\index{knot theory}
	\item $n$ can vary, and if you embed $S^n\inj S^{n+1}$ as the equator, you get different statements.
\end{enumerate}
Motivated by the second issue, Spanier defined the $S$-category in the 1950s.\footnote{The name ``$S$-category'' is
somewhat misleading: in those days, suspension was sometimes denoted $S$ instead of $\Sigma$ to make typeseting
easier, and the $S$ in $S$-category stood for suspension, not spheres.}
\begin{defn}
The \term[S-category@$S$-category]{$S$-category} $\cat S$ is the category whose objects are the objects in $\Top$
and whose morphisms are
\[\Map_{\cat S}(X, Y) \coloneqq \colim_n\Map_\Top(\Sigma^n X, \Sigma^n Y).\]
\end{defn}
By the Freudenthal suspension theorem, the hom-sets stabilize at some finite $n$. Spanier then defined
\term[S-duality@$S$-duality|see {Spanier-Whitehead duality}]{$S$-duality}, which we might call
\term{Spanier-Whitehead duality}, by specifying that $X$ and $Y$ are $S$-dual if $Y\cong S^n\setminus X$ in the
$S$-category.\index{Freudenthal suspension theorem}
\begin{rem}
The $S$-category has some issues: it's neither complete nor cocomplete. We like gluing stuff together, so this is
unfortunate.
\end{rem}
Spanier proved that to every $X\to S^n$, you can assign a dual $D_n X$, that $\Sigma D_n = D_{n+1}$, and
$D_{n+1}\Sigma = D_n$. That is, duality commutes with suspension, so in $\cat S$, every $X\to S^n$ has a unique
$S$-dual: the duals inside $S^n$ and $S^{n+1}$ are the same in the $S$-category for sufficiently large $n$.

Spanier and Whitehead then asked one of their graduate students, Elon Lima, to formalize this $S$-category, leading
to the first notion of the category of spectra.
\subsection*{An axiomatic setting for duality.}
The formal setting for duality is a \term[symmetric monoidal category!closed]{closed symmetric monoidal category}
$\fC$. We're not going to spell out the whole definition, but here are some important parts.
\begin{itemize}
	\item $\fC$ is symmetric monoidal, meaning there's a tensor product $\wedge\colon\fC\times\fC\to\fC$, which is
	(up to natural isomorphism) associative and commutative, and has a unit $S$. Commutativity is ensured by the
	\term{flip map} $\tau\colon X\wedge Y\isom Y\wedge X$.
	\item There is an internal \term{mapping object} $F(X,Y)\in\fC$ for any $X,Y\in\fC$.
	\item The functors $\bl\wedge X$ and $F(X,\bl)$ are adjoint (just like the tensor-hom
	adjunction).\index{tensor-hom adjunction}
\end{itemize}
The unit and counit of the tensor-hom adjunction are used to define duality.
\begin{defn}
The \term{evaluation map} is the unit $X\wedge F(X,Y)\to Y$, and the \term{coevaluation map} is the counit $X\to
F(Y,Y\wedge X)$. The \term*{dual} of $X$ is $DX = F(X,S)$.
\end{defn} % I missed the map epsilon
% Then I missed the point of the map F(X, Y( smash F(*V, W) smash V smash X -> Y smash W ...
You also get a natural map $\nu\colon F(X,Y)\wedge Z\to F(X,Y\wedge )$.
\begin{ex}
Check that $X\cong F(S,X)$, which follows directly from the axioms.
\end{ex}
The adjoint of $\e$ is a map $X\to DDX$.

There are a few good references for this: Dold-Puppe~\cite{DoldPuppe} is one, and~\cite{LMS} is another, though it
presents a somewhat old way of doing things.
\begin{defn}
$X\in\fC$ is \term{strongly dualizable} if there exists an $\eta\colon S\to X\wedge DX$ such that the following
diagram commutes.
\begin{equation}
\label{strongdual}
\gathxy{
	S\ar[r]^\eta\ar[d] & X\wedge DX\ar[d]^\tau\\
	F(X,X) & DX\wedge X.\ar[l]_\nu
}
\end{equation}
Here, the left-hand map comes from an adjunction to the identity $\id_X\colon X\cong X\wedge S\to X$. The lower map
is more explicitly $\nu\colon F(X,S)\wedge X\to F(X,X)$.
\end{defn}
\begin{exm}
Let $R$ be a commutative ring and $\fC = \Mod_R$, and let $X$ be a free $R$-module. If $\set{v_i}$ is a basis for
$X$ and $\set{f_i}$ is the dual basis, then the map $\eta\colon R\to X\otimes_R \Hom_R(X,R)$ is the map sending
\[1\mapsto \sum v_i\otimes f_i.\]
If you unravel what~\eqref{strongdual} is saying, it says that the map
\[x\mapsto \sum_i f_i(x)v_i\]
must be the identity. Thus, $X$ is strongly dualizable iff $X$ is finitely generated and projective. That is,
\emph{$X$ is strongly dualizable iff it's a retract of a free module}, which is a perspective that will be useful
later.
\end{exm}
Another way to think of this is that $X$ is strongly dualizable with dual $Y$ iff there exist maps $\e\colon
X\wedge Y\to S$ and $\eta\colon S\to X\wedge Y$ such that the compositions
\[\xymatrix{
	X\cong S\wedge X\ar[r]^-{\eta\wedge\id_X} & X\wedge Y\wedge X\ar[r]^-{\id_X\wedge\e} & X\wedge S\cong X
}\]
and
\[\xymatrix{
	Y\cong Y\wedge S\ar[r]^-{\id_Y\wedge\eta} & Y\wedge X\wedge Y\ar[r]^-{\e\wedge\id_Y} & S\wedge Y\cong Y
}\]
are the identity.\footnote{In some presentations, this is how duality in a symmetric monoidal category is defined;
the two approaches are equivalent.} From this and a diagram chase, you get some nice results.
\begin{prop} % are these isomorphisms natural?
If $X$ and $Y$ are dual, then there are isomorphisms $Y\isom F(X,S)$ and $X\isom DDX$.
\end{prop}
To paraphrase Lang, the best way to learn this is to prove all the statements without looking at the proofs, like
all diagram chases.

If you like string calculus, you can think of these in terms of $S$- or $Z$-shaped diagrams. In this form, these
results are sometimes known as the \term{Zorro lemmas}.\index{string calculus}

Another consequence of this formulation is that $\bl\wedge DX$ is right adjoint to $\bl\wedge X$, so by uniqueness
of adjoints there's a natural isomorphism $\bl\wedge DX\cong F(X,\bl)$.
\subsection*{Atiyah duality.}
The Whitney theorem tells us that for any manifold $M$ and sufficiently large $n$, there's an embedding
$M\inj\R^n$. This means we can compute the Spanier-Whitehead dual of a manifold, which is the setting of Atiyah
duality. We'll assume $M$ is compact.

By the tubular neighborhood theorem, there's an $\e > 0$ and a tubular neighborhood $M_\e$ such that $M_\e$ is the
disc bundle of the normal bundle $\nu\to M$.\footnote{Let $E$ be a vector bundle, and choose a metric on $E$; then,
the \footterm{disc bundle} $D(E)$ is the subset of vectors with $\norm x\le 1$, and the \footterm{sphere bundle}
$S(E)$ is the subset of vectors with $\norm x = 1$. These are fiber bundles with fiber $D^n$ and $S^{n-1}$,
respectively.}\index{tubular neighborhood theorem}

Thom's thesis~\cite{ThomThesis} supplied an amazing connection between cobordism groups and the stable homotopy
groups of the spheres by way of the \term{Pontrjagin-Thom map} $S^n\to \R^n/(\R^n\setminus M_\e)$ (heuristically,
crushing everything outside $M_\e$), and $\R^n/(\R^n\setminus M_\e)\cong D(V)/S(V)$, the \term{Thom space} $T\!\nu$
of the normal bundle to $M$.

Thus, we get a sequence of maps
\[\xymatrix@1{
	S^n\ar[r] & T\!\nu\ar[r] & T\!\nu\wedge M_+,
}\]
whose composition is called the \term{Thom diagonal}. This should look remarkably like the duality map $\eta$. To
construct $\e$, let $s\colon M\to\nu$ be the zero section; then, the composite
\[\xymatrix@1{
	M\ar[r]^-\Delta & M\times M\ar[r]^{s\times\id} &\nu\times M.
}\]
has trivial normal bundle, and the Pontrjagin-Thom construction yields a map
\[T\!\nu\wedge M_+\longrightarrow\Sigma^n M_+,\]
and projecting $M$ to a point, we get
\[\e\colon T\!\nu\wedge M_+\longrightarrow S^n.\]
The following theorem is nice, but quite nontrivial, at least from this approach.
\begin{thm}[Atiyah duality]
\label{atiyah_duality}
\index{Atiyah duality}
$\eta$ and $\e$ exhibit $T\!\nu$ and $M_+$ as $n$-dual in the $S$-category.
\end{thm}
We'd like $T\!\nu$ and $\Sigma^{-n}M_+$ to be dual, but we don't know how to show that yet. More accurately, let
$\Sigma^\infty\colon\Top\to\cat S$ be the functor that sends spaces and maps to themselves, so it makes a little
more sense to say that $\Sigma^\infty M_+$ and $\Sigma^{-n}\Sigma^\infty T\!\nu$ are dual in the $S$-category.

One surprising consequence is that the tangent bundle and normal bundle define stable homotopy invariants through
the Thom spectrum, which raises the question of what $T\!\nu$ looks like for particular examples.

Another is that we can immediately prove Poincaré duality, assuming the Thom isomorphism theorem. Namely, we can
establish an isomorphism\index{Thom isomorphism theorem}\index{Poincaré duality}
\begin{equation}
\label{HZatiyah}
[DX, H\Z] \cong [S, X_+\wedge H\Z].
\end{equation}
Here, we're using the corepresentability of cohomology: we know $H^n(\bl,\Z)$ is represented by $K(\Z, n)$, and
stitch these together (somehow) into $H\Z$. So the left-hand side is $H^*(DX;\Z)$, and the right-hand side is
$H_*(X;\Z)$.

Applying \cref{atiyah_duality} to~\eqref{HZatiyah},
\[[\Sigma^{-n} T\!\nu, H\Z]\cong H^{m-*}(X;\Z).\]
If $X$ is orientable, then the cohomology of the Thom spectrum is the same as the cohomology of $X$. The Thom
isomorphism theorem establishes a degree shift that gets rid of the dependency on $n$, the dimension of ambient
space.\index{Thom spectrum}
\subsection*{The equivariant setting.}
We'd like to establish Atiyah duality for $G/H$.
\begin{defn}
A \term[G-manifold@$G$-manifold]{$G$-manifold} is a manifold $M$ with a $G$-action by smooth maps.
\end{defn}
The theorems of differential topology that we need hold for $G$-manifolds. Namely, there is an equivariant tubular
neighborhood theorem, etc. In the smooth case, this goes back to the work of Andrew Gleason in the 1930s, and in
the PL case to Lashof in the 1950s. If you like manifold topology, these are really nice proofs to read, avoiding
triangulation arguments.\index{tubular neighborhood theorem!for $G$-manifolds}

We need one key fact.
\begin{thm}[Equivariant Whitney's theorem~\cite{Mostow, Palais}]
\index{Whitney embedding theorem!for $G$-manifolds}
Let $M$ be a compact $G$-manifold. Then, there is a $G$-equivariant embedding $M\inj V$, where $V$ is some
finite-dimensional real $G$-representation.
\end{thm}
Now we proceed as before: all of the arguments are exactly the same, including the equivariant Pontrjagin-Thom
construction.\index{Pontrjagin-Thom construction!for $G$-manifolds}

But this means that suspension has to be smashing with $S^V$, the representation sphere for $V$. So in order for
manifolds to have duals (meaning, in order to establish Poincaré duality), we need an $S$-category whose morphisms
are\index{Poincaré duality!for $G$-manifolds}
\[\Map_S(X,Y)\coloneqq \colim_V \Map(\Sigma^V X, \Sigma^V Y),\]
where $\Sigma^V X\coloneqq S^V\wedge X$. This is not sequential, but there is an equivariant Freudenthal suspension
theorem that stabilizes it.\index{Freudenthal suspension theorem!equivariant case}

Poincaré duality is one of the oldest results in algebraic topology, and if you don't have it in your theory, what
are you even doing? So to obtain Poincaré duality, \emph{we are inexorably forced to smash with representation
spheres that $G$-manifolds embed in}, namely finite-dimensional real representations. So many treatments of
equivariant homotopy theory start with defining $\Sigma^V$ and then they're off to the races, but this is why
they're doing this. If you don't need Poincaré duality, then you could do something different.

As before, we want a diagram category akin to the orbit category, but we want Atiyah duality to manifest in it.
\begin{ques}
In this setting, what replaces the orbit category?
\end{ques}
The naïve choice is the functor category from $\sO_G\op$ to the equivariant $S$-category, but this doesn't see
enough: it's like only choosing the trivial representation. This is nontrivial to see, though.

So we need extra structure in the orbit category, and this will come through extra structure in $\sO_G$. We'll add
extra maps between $G/H$ and $G/K$ called \term*{transfer maps}.\index{transfer map}

Let $M$ be a $G$-manifold embedded in some $G$-representation $V$, $\nu$ be its normal bundle, and $\tau$ be its
tangent bundle. Then, $\tau\oplus\nu$ is trivial, so the Thom construction applied to it is just smashing with
$S^V$. Thus, we obtain a sequence of maps
\begin{equation}
\label{PTtransfer}
\xymatrix@1{
	S^V\ar[r] & T\!\nu\ar[r] & T(\nu\oplus\tau)\ar[r] & S^V\wedge M_+.
}
\end{equation}
\begin{ex}
If you project $M\to *$, you get a map $S^V\to S^V$. Show that the degree of this map is the Euler characteristic
$\chi(M)$.\index{Euler characteristic}
\end{ex}
If $K\subset H$ are subgroups of $G$, then~\eqref{PTtransfer} defines a map $S^V\to S^V\wedge (H/K)_+$, and we can
induce this to get a map
\[(G/H)_+\wedge S^V\longrightarrow (G/K)_+\wedge S^V.\]
These are the additional maps we need. We'll go over them again in the next section, but these define the Burnside
category, which suffices to define the equivariant stable category! What's really nice is how this derives from
some of the oldest constructions and questions in homotopy theory: Thom's thesis can be considered the beginning of
modern homotopy theory.\index{Burnside category}
