\begin{quote}\textit{
	``Every tom Dieck and Harry works on this!''
}\end{quote}
In this section, we'll again assume that $G$ is a finite group. Some of this content extends to compact Lie groups,
but it's more complicated.\index{tom Dieck splitting}

Tom Dieck splitting is a decomposition of the fixed points of a suspension $G$-spectrum, either of the
statements\index{suspension spectrum!of a $G$-space}
\begin{subequations}
\begin{align}
\label{splevelTD}
(\Sigma^\infty X)^G &\cong \bigvee_{(H)\subseteq G} E\mathit{WH}_+\wedge_{\mathit{WH}} X^H\\
\pi_*(\Sigma^\infty X)^G &\cong\bigoplus_{(H)\subseteq G}
\pi_*^{\mathit{WH}}\paren{\Sigma^\infty E\mathit{WH}_+\wedge_{\mathit{WH}}X^H}.
\label{what_well_prove}
\end{align}
\end{subequations}
By $(H)\subseteq G$, we mean indexing over the conjugacy classes of subgroups of $G$. Here $\mathit{WH} =
\mathit{NH}/H$ as usual; if $H\trianglelefteq G$, then this is $G/H$. Tom Dieck splitting looks complicated; you
might have hoped that $\Sigma^\infty$ and $(\bl)^G$ commute, but they don't, and this is how they don't. But when
$X = S^0$, this tells you a decomposition of $\pi_*^G(\Sph)$.
\begin{rem}
Guillou-May~\cite{GM17} derive tom Dieck splitting as a consequence of the equivariant Barratt-Priddy-Quillen
theorem. We'll give a more elementary proof.\index{equivariant Barratt-Priddy-Quillen theorem}
\end{rem}
We have a cofiber sequence, called the \term{isotropy separation sequence},
\[\xymatrix@1{
	EG_+\ar[r] & S^0\ar[r] & \widetilde{EG},
}\]
which induces for any spectrum $X$ a cofiber sequence
\begin{equation}
\label{EGisot}
\xymatrix@1{
	(EG_+\wedge X)^G\ar[r] & X^G\ar[r] & (\widetilde{EG}\wedge X)^G.
}
\end{equation}
Take $X = \Sph = \Sigma^\infty S^0$, giving
\[\xymatrix@1{
	(\Sigma^\infty EG_+)^G\ar[r] & \Sph^G\ar[r] & (\Sigma^\infty\widetilde{EG})^G.
}\]
There are three key observations; one is easy, one is harder, and the one was a major structural theorem.
\begin{itemize}
	\item When $X$ is a suspension spectrum,~\eqref{EGisot} splits.
	\item The \term[Adams isomorphism]{Adams isomorphism} tells us the cofiber: $(\Sigma^\infty
	EG_+)^G\simeq\Sigma^\infty BG_+$.
	\item The last term is $(\widetilde{EG}\wedge X)^G\cong\Phi^G X$.
\end{itemize}
Tom Dieck proved tom Dieck splitting by inducting over isotropy sequences, adding one conjugacy class at a time.
This is one of the two major methods of induction in equivariant homotopy theory (the other being induction over
cells), and was a key step in the resolution of the Kervaire invariant $1$ problem~\cite{HHR}.\index{Kervaire
invariant $1$ conjecture}

There are several closely related versions of tom Dieck splitting, though going between them requires a deep result
called the Adams isomorphism. We'll return to that point later.\footnote{\TODO: I missed one version of the
splitting.} Anyways, we'll prove the version in~\eqref{what_well_prove}.

When $X = S^0$, this ``computes'' the equivariant stable homotopy groups of the spheres, and one consequence is
that
\begin{equation}
\label{piGs0}
\pi^G_0(\sus S^0) \cong \bigoplus_{(H)\subseteq G} \Z,
\end{equation}
i.e.\ it's free on the conjugacy classes of subgroups of $G$.

As a consequence of the proof of~\eqref{what_well_prove}, the generators are given by the Euler characteristics
$S\to \sus G/H_+\to S$.\footnote{\TODO: I didn't understand this.} This lets us identify homotopy classes
of stable maps $G/H_1\to G/H_2$; since we defined the Burnside category to be the full subcategory of $\Spc^G$ on
$\sus G/H_+$, this amounts to the computation of $\pi_0 B_G$.\index{Euler characteristic}\index{Burnside category}

Define an abelian group $A$ whose generators are diagrams of the form
\begin{equation}
\label{stableGHgen}
\xymatrix@1{
	G/H_1 & G/K\ar[l]_-{\theta_1}\ar[r]^-{\theta_2} & G/H_2,
}
\end{equation}
and for which two diagrams $G/H_1\gets G/K\to G/H_2$ and $G/H_1\gets G/K'\to G/H_2$ are equivalent if there is an
isomorphism $f\colon G/K\to G/K'$ such that the following diagram commutes.
\[\xymatrix@R=0.3cm{
	& G/K\ar[dd]_\wr^f\ar[dr]^{\theta_2}\ar[dl]_{\theta_1}\\
	G/H_1 && G/H_2.\\
	& G/K'\ar[ul]^{\theta_1'}\ar[ur]_{\theta_2'}
}\]
The point is that there's an isomorphism $\psi\colon A\congto\pi_0^{H_1}(\Sigma_+ G/H_2)$, the group of stable maps
$G/H_1\to G/H_2$. We'll calculate these maps in the algebraic Burnside category $\pi_0(B_G)$. The isomorphism
$\psi$ takes a diagram of the form~\eqref{stableGHgen} to the composite $\Sigma_+^\infty G/H_1\to \Sigma_+^\infty
G/K\to\Sigma_+^\infty G/H_2$, where the first map is the transfer for $\theta_1$ and the second is
$\sus\theta_2$.\index{Burnside category!algebraic Burnside category}
\begin{ex}
Show that $\psi$ is an isomorphism.
\end{ex}
This is really nice: we have a nice combinatorial description of the stable maps coming from the structure of $G$.
But what's the composition?

The above is more or less true for $G$ a compact Lie group. But what follows really requires $G$ to be finite.

It seems natural to play double coset games here, but this quickly turns into a notational mess. But if you know
about spans, there's only one thing it should be, which is the pullback:\index{double cosets}
\[\xymatrix@dr{
	G/K\times_{G/H_2} G/K'\ar[r]\ar[d] & G/K'\ar[r]\ar[d] & G/H_3.\\
	G/K\ar[r]\ar[d] & G/H_2\\
	G/H_1
}\]
This is true, and we'll give a real proof later. The upshot is really cool: we have a complete combinatorial
description of the algebraic Burnside category, including composition. This leads one to Mackey functors:
equivariant stable homotopy theory is controlled by a pair of functors out of the orbit category, one covariant and
one contravariant, and Mackey functors are nice examples of these.\index{Mackey functor}
\begin{rem}
We've been tacitly assuming a complete $G$-universe for this, but what if you'd prefer to leave out certain
suspensions $\Sigma^V$? In the case of an incomplete $G$-universe $U$, you only consider the orbits $G/H\inj U$,
and build the incomplete Burnside category and incomplete Mackey functors in this way.\index{universe}
\end{rem}
Now let's prove tom Dieck splitting.\index{tom Dieck splitting}
\begin{proof}[Proof of~\eqref{what_well_prove}]
We'll approach this inductively and look a single summand at a time, producing a map
\begin{equation}
\label{tdmap}
\pi_*^{\WH}\paren{\sus E\WH_+\wedge X^H}\longrightarrow \pi_*^G(\sus X).
\end{equation}
Recall that $\WH = \NH/H$, so we can define a sequence
\begin{equation}
\label{first_part_td}
\xymatrix{
	\pi_*^\WH\paren{\sus E\WH_+\wedge X^H}\ar[r]^-{\theta_1} & \pi_*^\NH\paren{\sus E\WH_+\wedge
	X^H}\ar[r]^-{\theta_2} &\pi_*^\NH\paren{\sus E\WH_+\wedge X},
}\end{equation}
where $\theta_1$ is induced by restriction $\pi_*^\WH\to\pi_*^\NH$, and $\theta_2$ is induced from the inclusion
$X^H\inj X$. The Wirthmüller isomorphism induces an isomorphism on homotopy groups\index{Wirthmüller isomorphism}
$\pi_*^H(X)\congto\pi_*^G(G\wedge_H X)$ (by an adjunction game), so we can extend~\eqref{first_part_td} with maps
\[\xymatrix{
	\pi_*^\NH\paren{\sus E\WH_+\wedge X}\ar[r]^-{\theta_3}_-\cong &\pi_*^G\paren{G\wedge_\NH \sus
	E\WH_+\wedge X}\ar[r]^-{\theta_4} &\pi_*^G(\sus X).
}\]
Here, $\theta_4$ is the adjoint of the $\NH$-map $\sus E\WH_+\wedge X\to\sus X$ that crushes
$E\WH\to *$. Thus, our goal is to prove that $\theta_2\circ\theta_1$ and $\theta_4$ are isomorphisms.

Recall that a $\Z$-graded homology theory on $G$-spaces is a collection of functors $E_q\colon G\Top\to\Ab$ for
$q\in\Z$ that are homotopy invariant and such that\index{homology theory!of $G$-spaces ($\Z$-graded)}
\[E_q\paren{\bigvee_k X_k} \cong\bigoplus_k E_q(X_k),\]
and a map $f\colon X\to Y$ induces a long exact sequence
\begin{equation}
\label{LESZgr}
\xymatrix{
	\dotsb\ar[r] & E_q(X)\ar[r] & E_q(Y)\ar[r] & E_q(Cf)\ar[r]^\delta & E_{q-1}(X)\ar[r] & \dotsb
}
\end{equation}
The key observation is that \emph{both sides of~\eqref{what_well_prove} specify $\Z$-graded homology theories}.
\begin{defn}
An $X\in G\Top$ is \term[concentration at a conjugacy class]{concentrated at a conjugacy class} $H$ if for all $K$
not in the conjugacy class of $H$ in $G$, $X^K$ is contractible.
\end{defn}
\begin{thm}
\label{lattice_induction}
Let $h\colon E_q\to\tilde E_q$ be a natural transformation of $\Z$-graded homology theories. If $h$ is an
isomorphism on all $X$ concentrated at a conjugacy class, then $h$ is an isomorphism.\footnote{The analogue of this
theorem for $\Z$-graded cohomology theories is also true.}
\end{thm}
This leads to a very innovative proof, invented by tom Dieck, whose pedagogical importance is just as significant
as the theorem statement: we'll set up a sequence of families of subgroups of $G$
\[\set e = \sF_0\subseteq\sF_1\subseteq\sF_2\many\subseteq \sF_n = \set{H\subseteq G}\]
such that $\sF_i$ is built from $\sF_{i-1}$ by adding a conjugacy class not already present. Then, we can induct on
$i$, and since $G$ is finite, this must terminate. This is one of the key techniques of induction in equivariant
homotopy theory, along with cellular induction.
\begin{proof}[Proof of \cref{lattice_induction}]
Let's induct: suppose that $\sF_i = \sF_{i-1}\cup (K)$ for some $K\subseteq G$, and consider any
$G$-space $X$. There is a natural map $f\colon X\wedge E\sF_{i-1}\to X\wedge E\sF_i$, and its cofiber $C(f)$ is
concentrated at $K$!

Suppose that $E_q(X\wedge E\sF_{i-1})\congto\tilde E_q(X\wedge E\sF_{i-1})$. By hypothesis,
$E_q(C(f))\congto\tilde E_q(C(f))$, since $C(f)$ is concentrated at $K$. Now, by the long exact
sequence~\eqref{LESZgr} and the five lemma, the map $E_q(X\wedge E\sF_i)\to E_q(X\wedge E\sF_i)$ is an isomorphism.
\end{proof}
Now, to prove tom Dieck splitting, it suffices to show that the map we defined (the direct sum of~\eqref{tdmap}
over all conjugacy classes) is an isomorphism when $X$ is concentrated at a single conjugacy class $(H)$.

First, we'll show that if $X$ is concentrated at $(H)$ and $K\not\in (H)$, then~\eqref{tdmap} is trivial for $K$.
Let's consider $\sus E\mathit{WK}_+\wedge X^K$. Look at the space $S^V\wedge E\mathit{WK}_+\wedge X^K$, and let's
consider its $L$-fixed points for $L\subseteq K$. If $L = \set e$, then $(X^K)^L$ is contractible,
and otherwise, $(E\mathit{WK}_+)^L$ is contractible, so the smash product is always contractible. We conclude that
$\sus E\mathit{WK}_+\wedge X^K$ is contractible, and therefore that
\[\pi_*^{\mathit{WK}}\paren{\sus E\mathit{WK}_+\wedge X^K} = 0.\]
Now we've reduced to showing that~\eqref{tdmap} is an isomorphism when $X$ is concentrated at $H$, i.e.\
that~\eqref{first_part_td} is an isomorphism (its components aren't) and that $\theta_4$ is an isomorphism.

Let's first tackle $\theta_4$.\footnote{\TODO: I got confused and should fix this.} This uses an important piece of
technology: suppose $X$ and $Y$ are $G$-spaces and $H\subseteq G$ is normal. Then, there is a
\term[restriction map!for fixed points of a $G$-space]{restriction map}
\[\rho_H\colon \Map^G(X,Y)\longrightarrow\Map^{G/H}(X^H, Y^H).\]
This map appears in an essential way in the definition of topological cyclic homology.\footnote{The new
construction of $\mathit{TC}$ by Nikolaus-Scholze~\cite{NikolausScholze} manages to avoid this in a way that is
considerably more natural.}\index{topological cyclic homology}
\begin{lem}
\label{restrlem}
Suppose $Y$ is concentrated at $(H)$; then, the restriction map $\rho_H\colon\Map^G(X,Y)\to\Map^{G/H}(X^H,Y^H)$ is
an acyclic fibration.\index{concentration at $(H)$|see {concentration at a conjugacy class}}
\end{lem}
We'll prove this below with cellular induction; in any case, we only need that $\rho_H$ is a weak
equivalence.

Anyways, we want to show that~\eqref{first_part_td} is an isomorphism, so we'll use the restriction map to define
an inverse $\pi_*^\NH(\sus E\WH_+\wedge X)\to\pi_*^\WH(\sus E\WH_+\wedge X^H)$. That is, consider
\[[S^{n+m\rho_{\NH}}, S^{m\rho_{NH}}\wedge E\WH_+\wedge X]\]
and take the $H$-fixed points on both sides; Lemma~\ref{restrlem} ensures this is an isomorphism.

We've reduced to showing that the composite
\[\xymatrix{
	\pi_*^\WH(\sus E\WH_+\wedge X^H)\ar[r]\ar[dr]^-{\theta_1} & \pi_*^\NH(\sus E\WH_+\wedge X^H)\ar[d]\\
	& \pi_*^\NH(\sus
	E\WH_+\wedge X)\ar[r]^-\cong & \pi_*^G(G\wedge_H\sus E\WH_+\wedge X)\ar[r]^-{\theta_2} &\pi_*^G(X)
}\]
is an isomorphism when $X$ is concentrated at the single conjugacy class $(H)$ for $G$. The composite of the first
two maps is denoted $\theta_1$. We did this by showing that the two sides of the tom Dieck isomorphism are
$\Z$-graded homology theories, and therefore are determined by their values on spaces concentrated at single
conjugacy classes. We'll show separately that $\theta_1$ and $\theta_2$ are isomorphisms.

First let's look at $\theta_2$, by looking at the spaces of the suspension spectrum for
\[\paren{G\wedge_\NH\paren{E\WH_+\wedge X\wedge S^V}}^K.\]
Fixed points do not commute with suspension on the spectrum level (after all, this is the content of tom Dieck
splitting), but they do commute with the smash product for spaces, so this is also
\[(G\wedge_H E\WH_+)^K\wedge X^K\wedge S^{V^K}.\]
If $K\not\in (H)$, then $X^K\simeq *$, since $X$ is concentrated at $(H)$. If $K\in (H)$, then $(E\WH_+)^K\simeq
*$, so either way what we obtain is an isomorphism for $\theta_2^K$.

For $\theta_1$, we needed to prove \cref{restrlem}, that if $H\trianglelefteq G$ and $Y$ is concentrated at $(H)$,
then the restriction map $\Map^G(X,Y)\to\Map^{G/H}(X^H,Y^H)$ is an isomorphism.
\begin{proof}[Proof of \cref{restrlem}]
The proof will proceed by cellular induction. Consider attaching a cell to a $G$-space $Z$ to form a space
$\tilde Z$, which is expressed by the pushout
\[\xymatrix{
	G/K_+\wedge S^{n-1}\ar[r]\ar[d] & Z\ar[d]\\
	G/K_+\wedge D^n\ar[r] & \pushout \tilde Z.
}\]
If we apply $\Map^G(\bl,Y)$ to this diagram, it becomes a pullback
\[\xymatrix{
	\Map^G(G/K_+\wedge S^{n-1}, Y) & \Map^G(Z,Y)\ar[l]\\
	\Map^G(G/K_+\wedge D^n, Y)\ar[u]_{\psi_1} & \Map^G(\tilde Z,Y)\ar[l]\ar[u]_{\psi_2}.
}\]
Since $\Map^G(G/K_+\wedge S^{n-1}, Y)\simeq \Map(S^{n-1}, Y^K)$, then $\psi_1$ is trivial, so $\psi_2$ is an
isomorphism.

Let
\[X_0\coloneqq\set{x\in X\mid G_x\not\subseteq H}\subseteq X.\]
Then, $X\setminus (X_0\cup X^H)$ is built by attaching cells $G/K_+\wedge D^n$ with $K\not\in (H)$.

The cell attachment computation we just made tells us that it suffices to understand $X_0\cup X^H$, for which we
use a Mayer-Vietoris argument associated to the pullback\index{Mayer-Vietoris principle}
\[\xymatrix{
	X_0\cup X^H & X^H\ar[l]_{\phi_1}\\
	X_0\ar[u] & X_0\cap X^H\ar[l]_{\phi_2}\ar[u].
}\]
We again apply $\Map^G(\bl,Y)$; this time, the same argument shows that $\phi_2^*$ is an isomorphism. Now the map
we want factors as
\[\xymatrix{
	\Map^G(X,Y)\ar[r] & \Map^G(X_0\cup X, Y)\ar[r] & \Map^G(X^H,Y)\ar[r] & \Map^{G/H}(X^H,Y^H).
}\]
The last map comes from the adjunction, and in particular is a known weak equivalence. And by induction, the first
two maps are weak equivalences.
\end{proof}
We'll use \cref{restrlem} to check that the map
\[\pi_*^\WH(\sus E\WH_+\wedge X^H)\stackrel{\theta_1}{\longrightarrow} \pi_*^\NH(\sus E\WH_+\wedge X)\]
is an isomorphism, by providing a map in the other direction. Specifically, this is a map of homotopy classes
\[[S^{n+m\rho}, \Sigma^{m\rho} E\WH_+\wedge X]\stackrel{\theta_1}{\longrightarrow} [S^{n+m\rho}, \Sigma^{m\rho}
E\WH_+\wedge X].\]
We've abused notation a bit here: on the left, $\rho$ is the regular representation for $\NH$, and on the right,
it's the regular representation for $\WH$. \Cref{restrlem} implies that the map going the other way is an
isomorphism when $X$ is concentrated at $(H)$:\index{regular representation}
\[[S^{n+m\rho}, S^{m\rho}\wedge E\WH_+\wedge X]\stackrel\cong\longrightarrow [S^{n + m\rho^H}, S^{m\rho^H}\wedge
E\WH_+\wedge X^H].\]
This is what we reduced the problem to, so we have completed the proof of tom Dieck splitting for $\pi_G^*(\sus X)$
on the level of homotopy groups.
\end{proof}
As this is a statement about stable homotopy groups, it's desirable to have a spectrum-level equivalence which
reduces to this isomorphism; the statement is~\eqref{splevelTD}, but the proof is different.
