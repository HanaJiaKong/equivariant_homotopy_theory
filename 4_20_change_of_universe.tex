\begin{quote}\textit{
	``Do I dare disturb the universe?''
}\end{quote}

In this section, we discuss the construction and some properties of the HHR norm $N_H^G\colon \Spc^H\to\Spc^G$ and
$N_H^G\colon \Spc^H[\P]\to\Spc^G[\P]$. This is the left adjoint to $i_H^*$ (for commutative algebras), and is the
basis for multiplicative induction in the spectral setting. Thus it specializes to the norm in algebra.

We know what we want to do, but the technicalities are tricky.
\begin{notdefn}
The \term{norm map} $N_H^G\colon \Spc^H\to\Spc^G$ sends
\[M\mapsto \theta^*\paren{\bigwedge_{g\in G/H} M},\]
which is the indexed smash product, in direct analogue with tensor induction.
\end{notdefn}
The reason we use the smash product instead of the wedge sum is that spectra are already abelian groups. The
additive analogue of this construction is the Wirthmüller isomorphism.

What makes this construction tricky is figuring out how to give $\bigwedge_{g\in G/H} M$ a $\Sigma_n\wr H$-action.
The issue is the universe.
\begin{exm}
Let $X\in\Spc$, and let's consider $N_e^{C_2}X$. If $U$ is the universe we started with, $X\wedge X$ is in $U\oplus
U$, and the smash product is a Kan extension along $U\oplus U\to U$.

If $X$ is a space, $C_2$ acts on $X\times X$ by switching the pairs. And we know the diagonal map $x\mapsto (x,x)$
defines an isomorphism $X\congto (N_e^{C_2} X)^{C_2}$. But if you have spectra instead, and the two universes are
different, i.e.\ $U_1\oplus U_2$ instead of $U\oplus U$, which makes it hard to resolve.
\end{exm}
The solution is a trick, due to unpublished work of Lydakis and elucidated by~\cite{HHR}. Consider $X$ in the
trivial $H$-universe, i.e.\ $X$ is an $H$-object in $\Spc$. Then there's no problem: $N_e^{C_2}X$ is a $C_2$-object
in $\Spc$, or $\Spc^{C_2}$ with the trivial universe. If we're not keeping track of the equivariant universe, it's
easy to make sense of the map $N_H^G\colon H\Spc\to G\Spc$.

The trick is the following point, which is extremely subtle: we can define $N_G^H\colon \Spc^H\to\Spc^G$ by
changing the universe from $\Spc^H$ to $H\Spc$ (i.e.\ the trivial universe), then using the tensor induction to get
to $G\Spc$ (i.e.\ the trivial universe), then changing universe again to get back to $\Spc^G$. As a point-set
construction, well, sure, this is a functor. But the crazy thing is that this works: you can make this a homotopical
functor such that its derived functor is the right thing.

Recall that $\Spc^G[\P]$ denotes the category of commutative ring objects in $G$-spectra (sometimes also written
$\cat{Comm}_H$). We'll show that $N_H^G\colon\Spc^H[\P]\to\Spc^G[\P]$ is left adjoint to $i^*_H$, and hence, if $R$
is a commutative ring $G$-spectrum, we have a counit $N_H^Gi_H^*R\to R$ as we did in group cohomology.

The principal player in~\cite{HHR} is obtained from this norm and a concrete $C_2$-spectrum: in \cref{MRdefn},
we'll define real bordism $\mathit{MR}$, the cobordism theory of stably almost complex manifolds with a
$C_2$-action by complex conjugation (there are some nuances to writing this down). Then, they consider the
$C_8$-theory $N_{C_2}^{C_8}\mathit{MR}$, and do a lot of hard work to show this detects the Kervaire invariant in
the stable homotopy groups of spheres. People are still working out the applications of this spectrum and this idea
more generally, and you could count yourself among their numbers.

Now we'll construct the norm map. This will involve some work with orthogonal spectra, so let's recall how
orthogonal $G$-spectra are defined, especially since we covered it rather quickly.\footnote{\TODO: there may be
notational inconsistencies with the first time orthogonal spectra were covered, which I should fix at some point.}

Recall that if $V$ and $V'$ are real $G$-representations on inner product spaces, we defined $I_G(V,V')$ to be the
$G$-space of linear isometries $V\to V'$. We defined $J_G(V,V')$ to be the Thom space for $E\subseteq
I_G(V,V')\times V'\to I_G(V,V')$, where $E$ is the \term{orthogonal complement}
\[E \coloneqq \set{(f,x)\mid x \in V - fV}.\]
When $V\subseteq V'$, you can write this explicitly as
\[J_G(V,V') = \O(V')_+\wedge_{\O(V' - V)} S^{V' - V}.\]
Now, let $J_G$ be the category whose objects are $G$-representations on finite-dimensional real inner product
spaces and whose morphism spaces are $J_G(V,V')$ as we defined them. In particular, there's a composition law
$J_G(V',V'')\times J_G(V,V')\to J_G(V, V'')$ and a symmetric monodial product
\begin{align*}
	(f,x) + (f',x') &= (f+f', x+x')\\
	(g,y)(f,x) &= (gf, gx+y).
\end{align*}
The unit is $(\id,0)$.

Now, we were able to define $\Spc^G = \Fun(J_G, G\Top_*)$, and we saw this has a symmetric monoidal structure.

One important point is that the Yoneda lemma implies
\begin{equation}
\label{tautpres}
\xymatrix{\bigvee_{V,W} S^{-W}\wedge J_G(V,W)\wedge X(V)\dblarrow[r]
& \bigvee_V S^{-V}\wedge X(V)\ar[r] &
X}
\end{equation}
is a coequalizer of $X$ in terms of representable objects. This is true in general for presheaves, and in this
context,~\cite{HHR} call it the \term{tautological presentation} of $X$.

Now we'll introduce the point-set change-of-universe functor $\Spc^G[U']\to\Spc^G[U]$, where $U$ and $U'$ are
$G$-universes. Let
\[I_{U'}^UX(V)\coloneqq J_G(\R^n, V)\wedge_{\O(n)} X(\R^n),\]
where $\dim V = n$. If $W$ is a $p$-dimensional $G$-representation, the action of $J_G(V,W)$ on this comes through
the diagram
\[\xymatrix{
	J_G(V,W)\wedge J_G(\R^n, V)\wedge_{\O(n)} X(\R^n)\ar[d] \\
	J_G(\R^n, W)\wedge_{\O(n)} X(\R^n)\ar[r] &J_G(\R^p,
	W)\wedge_{\O(p)} J_G'(\R^n,\R^p)\wedge_{\O(n)} X(\R^n)\ar[r] & J_G(\R^p, W)\wedge_{\O(p)} X(\R^p).
}\]
\begin{ex}
Show that $I_{U'}^U I_U^{U'} = \id$. This is not hard. Also, check that $I_{U'}^U$ is strong symmetric monoidal.
\end{ex}
Thus, for any $U$ and $U'$, $\Spc^G[U]$ and $\Spc^G[U']$ are equivalent! This is true for the same reason as
\cref{LReq}.
\begin{rem}
It's also quick to check that $I^{\R^\infty}_U$ is the forgetful functor.
\end{rem}
Let $U$ be a complete $H$-universe and $\widetilde U$ be a complete $G$-universe. Then, we'll define
$N_H^G\colon\Spc_U^H\to\Spc_{\widetilde U}^G$ by
\[X\mapsto I_{\R^\infty}^{\widetilde U} \theta^*\paren{\bigwedge_{G/H} X} I_U^{\R^\infty} X.\]
Here $\theta$ is the wreath action we introduced last time. This functor is clearly strong symmetric monoidal, and
turns out to be left-adjoint to $i_H^*\colon\Spc^G_{\widetilde U}[\P]\to\Spc_U^H[\P]$, i.e.\ on the categories of
commutative algebras.

The miracle is that this can be derived in the usual model structure by cofibrant replacement, so it's
homotopically the right thing. We won't prove this; it's one of the technically hardest parts of~\cite{HHR}: it's
not a left adjoint on modules, so it doesn't commute with colimits, and one has to carefully analyze what it does
to pushouts.

There are a few key facts:
\begin{itemize}
	\item $N_H^G$ is symmetric monoidal.
	\item $N_H^G S^{-V} = S^{-\Ind_H^G V}$.
	\item There's a diagonal map $\Phi^HX\to\Phi^G N_H^G X$, and it's an equivalence. In fact, when $X$ is
	cofibrant, this is a point-set equivalence!
\end{itemize}
We'll begin showing this by describing $\Phi^G$, which is a strong symmetric monoidal functor such that if $Z$ is a
space, $\Phi^G(S^{-V}\wedge Z)\cong S^{-V^G}\wedge Z^G$. In the $\infty$-categorical sense, it commutes with
colimits, but this is not true in the point-set case; in any case, mirroring the tautological
presentation~\eqref{tautpres}, $\Phi^GX$ is the coequalizer
\[\xymatrix{
	\bigvee_{V,W} S^{-W^G}\wedge J_G(V,W)^G\wedge X(V)^G\dblarrow[r] & \bigvee_V S^{-V^G}\wedge X(V)^G\ar[r] &
	\Phi^G X.
}\]
It's not too hard to check this is the same as the model described in~\citeme{Mandell-May}. We'll use this
description to construct the diagonal map. Using the fact that $Z^H\cong (N_H^G Z)^G$,
\[\Phi^G N_H^G S^{-V}\cong \Phi^G\paren{S^{-\Ind_H^G V}} \cong S^{-V^H}\cong\Phi^H S^{-V}.\]
Therefore we have an isomorphism
\[\Phi^H(S^{-V}\wedge Z)\stackrel\cong\longrightarrow \Phi^G N_H^G(S^{-V}\wedge Z).\]
Applying this termwise to the tautological construction, we obtain a map
\[\xymatrix{
	\bigvee_{V,W} S^{-W^H}\wedge I_H(V,W)^G\wedge X(V)^H\dblarrow[r] & \bigvee_V S^{-V^G}H\wedge X(V)^H\ar[r] &
	\Phi^G N_H^G X,
}\]
hence by the universal property of the coequalizer, a map $\Phi^HG\to\Phi^G N_H^G X$.
