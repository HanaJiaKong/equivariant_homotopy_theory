%!TEX root = m392c_EHT_notes.tex

% TODO: would be good to place this on or immediately after the titlepage
\versioninfo

These notes were taken in UT Austin’s M392C (Topics in Algebraic Topology) class in Spring 2017, taught by Andrew
Blumberg. I (Arun) live-\TeX ed them using \texttt{vim}, so there may be typos; please send questions, comments,
complaints, and corrections to
\href{mailto:a.debray@math.utexas.edu?subject=M392C\%20Lecture\%20Notes}{\texttt{a.debray@math.utexas.edu}}.
Alternatively, these notes are hosted on Github at \url{https://github.com/adebray/equivariant_homotopy_theory},
and you can submit a pull request. Thanks to Rustam Antia-Riedel, Omar Antolín Camarena, Thomas Brazelton, Gill
Grindstaff, Keeley Hoek, Hana Jia Kong, Charles Rezk, Sean Tilson, and Yixian Wu for catching a few errors; to
Andrew Blumberg, Ernie Fontes, Tom Gannon, Tyler Lawson, Richard Wong, Valentin Zakharevich, an anonymous reader,
and the users of the homotopy theory Mathoverflow chatroom for some clarifications and suggestions; to Adrian
Clough, Richard Wong, and Yuri Sulyma for adding some examples, remarks, and references; and to Richard Wong for
the cover page.

% TODO: will probably need to fix acknowledgements at the end

\orbreak

These notes are an overview of equivariant stable homotopy theory. We're in the uncomfortable position where this
is a big subject, a hard subject, and one that is poorly served by its textbooks. Algebraic topology is like this
in general, but it's particularly acute here. Nonetheless, here are some references:
\begin{itemize}
	\item Adams, ``Prerequisites (on equivariant stable homotopy) for Carlsson's lecture.''~\cite{Adams84}. This is
	old, and some parts of it don't reflect how we do things now.
	\item The Alaska notes~\cite{AlaskaNotes}, edited by May, are newer, and are written by many authors. Some
	parts are a grab bag, and some parts (e.g.\ the rational equivariant bits) aren't entirely right. The notes are
	also not a textbook.
	\item Appendix A of Hill-Hopkins-Ravenel~\cite{HHR}. This is a paper which resolved an old conjecture on
	manifolds using equivariant stable homotopy theory, but let this be a lesson on referee reports: the authors
	were asked to provide more background, and so wrote a 150-page appendix on this material. Their suffering is
	your gain: the appendix is a well-written introduction to equivariant stable homotopy theory, albeit again not
	a textbook.
\end{itemize}

There are arguably two very serious modern applications of equivariant stable homotopy theory: 

\begin{itemize}
	\item The first is trace methods in algebraic $K$-theory:\index{trace methods} Hochschild
	homology\index{Hochschild homology} and its topological cousins are equipped with natural $S^1$-actions (the
	same $S^1$-action coming from field theory). This is how people other than Quillen compute
	algebraic $K$-theory.\index{algebraic $K$-theory}
	\item The other major application is Hill-Hopkins-Ravenel's settling of the Kervaire invariant $1$ conjecture
	in~\cite{HHR}.\index{Kervaire invariant $1$ conjecture}
\end{itemize}
The nice thing is, however you feel about the applications, both applications require developing new theory in
equivariant stable homotopy theory. Hill-Hopkins-Ravenel in particular required a clarification of the foundations
of this subject which has been enlightening.

In these notes, we hope to cover the foundations of equivariant stable homotopy theory. On the one hand, this will
be a modern take, insofar as we emphasize the norm and presheaves on orbit categories (these will be explained in
due time), the modern emerging consensus on how to think of these things, different than what's written in
textbooks. The former is old, but has gained more attention recently; the latter is new. Moreover, there's an
increasing sense that a lot of the foundations here are best done in
$\infty$-categories.\index{infinity-category@$\infty$-category} We will not take this approach in order to avoid
getting bogged down in $\infty$-categories; moreover, the notes are supposed to be rigorous. It will sometimes be
clear to some people that $\infty$-categories lie in the background, but we won't talk very much about them.

We'll cover some old topics such as Smith theory and the Segal conjecture, and newer ones such as trace methods
and Hill-Hopkins-Ravenel, depending on student interest. We will not have time to discuss many topics, including
equivariant cobordism or equivariant surgery theory.
\subsection*{Prerequisites.} If you don't know these prerequisites, that's okay; it means you're willing to read
about them on your own.
\begin{itemize}
	\item Foundations of unstable homotopy theory at the level of May's \textit{A Concise Course in Algebraic
	Topology}~\cite{ConciseCourse}. For example, we'll discuss equivariant CW complexes, so it will help to know
	what a CW complex is.
	\item A little bit of category theory, e.g.\ found in Mac Lane~\cite{MacLane} or Riehl~\cite{RiehlCTC}.
	\item These notes will not require much in the way of simplicial methods (simply because it's hard to reconcile
	simplicial methods with non-discrete Lie groups), but you will want to know the bar construction. An excellent source for this is \cite[Chapter 4]{RiehlCHT}.
	\item A bit of abstract homotopy theory, e.g.\ what a model structure is. Good sources for model categories are \cite[Part III]{RiehlCHT} and \cite{Hovey}.
\end{itemize}
If you don't know these, feel free to ask the professor for references. His advisor suggested learning
nonequivariant stable homotopy theory by reading Lewis-May-Steinberger's book~\cite{LMS} on equivariant stable
homotopy theory and letting $G = *$, but this may not appeal to everyone. In any case, perhaps this isn't
necessarily a good reference for nontrivial groups.

Unstable equivariant questions are very natural, and somewhat reasonable. But stable questions are harder; they
ultimately arise from reasonable questions, but their formulations and answers are hard: even discussing the
equivariant analogue of $\pi_0S^0$ (see~\eqref{piGs0}) requires some representation theory --- and yet of course it
should. Thus there's a lot of foundations behind hard calculations.
\subsection*{Categories of topological spaces.}
The category of topological spaces we consider is $\Top$, the category of compactly generated, weak Hausdorff
spaces (and continuous maps); we'll also consider $\Top_*$, the category of based, compactly generated, weak
Hausdorff spaces and continuous, based maps. This is an important and old trick which eliminates some pathological
behavior in quotients. It's reasonable to imagine that point-set topology shouldn't be at the heart of foundational
issues, but there are various ways to motivate this, e.g.\ to make $\Top$ more resemble a topos or the category of
simplicial sets.
\begin{defn}
Let $X$ be a topological space.
\begin{itemize}
	\item A subset $A\subseteq X$ is \term{compactly closed} if for every compact Hausdorff space $Y$ and $f\colon
	Y\to X$, $f^{-1}(A)$ is closed.
	\item $X$ is \term[compactly generated!topological space]{compactly generated} if every compactly closed subset
	of $X$ is closed.
	\item $X$ is \term{weak Hausdorff} if for every compact Hausdorff space $Y$ and continuous map $f\colon Y\to
	X$, $f(Y)$ is a closed subset of $X$.\footnote{When $X$ is compactly generated, this is equivalent to being
	\term[k-Hausdorff@$k$-Hausdorff]{$k$-Hausdorff}, i.e.\ the diagonal map $\Delta\colon X\to X\times X$ is closed
	when $X\times X$ has the compactly generated topology. See~\cite{StricklandCGWH, RezkCGWH} for more details.}
\end{itemize}
\end{defn}
The intuition behind compact generation is that the topology is determined by compact Hausdorff spaces. The weak
Hausdorff condition is strictly stronger than $T_1$ (points are closed), but strictly weaker than being Hausdorff.
Any space you can think of without trying to be pathological will meet these criteria.

There is a functor $k$ from all spaces to compactly generated spaces which adds the necessary closed sets. This has
the unfortunate name of \term[k-ification@$k$-ification]{$k$-ification} or
\synonym{$k$-ification}{kaonification};\footnote{Kaonification is of course distinct from koanification, the
process which makes statements more confusing.\index{pun}} by putting the compactly generated topology on $X\times
X$, we mean taking $k(X\times X)$. There's also a ``weak Hausdorffification'' functor $w$\index{weak Hausdorff!weak
Hausdorffification} which makes a space weakly Hausdorff, which is some kind of quotient.\footnote{The $k$ functor
is right adjoint to the forgetful map, which tells you what it does to limits.}

When computing limits and colimits, it's often desirable to compute them in the category of spaces and then apply
$k$ and $w$ to return to $\Top$. This works correctly for limits, but for colimits, $w$ is particularly badly
behaved: you cannot compute the colimit in $\Top$ by computing it in $\Set$ and figuring out the topology. In
general, you'll need to take a quotient.

Nonetheless, there are nice theorems which make things work out anyways.
\begin{prop}
Let $Z = \colim(X_0\to X_1\to X_2\to\dots)$ be a sequential colimit (sometimes called a \term{telescope}); if each
$X_i$ is weak Hausdorff, then so is $Z$.
\end{prop}
\begin{prop}
Consider a diagram
\[\xymatrix{
	A\ar[r]^f\ar[d] &B\\
	C
}\]
where $f$ is a closed inclusion. If $A$, $B$, and $C$ are weakly Hausdorff, then $B\amalg_A C$ is weakly Hausdorff.
\end{prop}
These are the two kinds of colimits people tend to compute, so this is reassuring.

One reason we require regularity on our topological spaces is the following, which is not true for topological
spaces in general.
\begin{lem}
Let $X$, $Y$, and $Z$ be in $\Top$; then, the natural map
\[\Map(X\times Y, Z)\longmapsto\Map(X, \Map(Y, Z))\]
is a homeomorphism.
\end{lem}
\subsection*{Enrichments.}
The categories $\Top$ and $\Top_*$ are enriched over themselves (as are categories of $G$-spaces, which we'll see
soon). This merits a brief digression into enriched categories.
\begin{defn}
Let $(\fV,\otimes, \boldsymbol 1)$ be a symmetric monoidal\index{symmetric monoidal category}
category.\footnote{Briefly, this means $V$ has a tensor product $\otimes$ and a unit $\boldsymbol 1$; there are
certain axioms these must satisfy.} Then, an \term[enriched category]{enrichment} of a category $\fC$ over $\fV$
means
\begin{itemize}
	\item for every $x,y\in\fC$, there is a hom-object $\underline\fC(x,y)$, which is an object in $\fV$,
	\item for every $x\in\fC$, there is a unit $\fV$-morphism $\boldsymbol 1\to\underline\fC(x,x)$,
	\item composition $\underline\fC(x,y)\otimes\underline\fC(y,z)\to\underline\fC(x,z)$ is associative and unital,
	and
	\item the underlying category is recovered as $\fC(x,y) = \Map(\boldsymbol 1, \underline\fC(x,y))$.
\end{itemize}
\end{defn}
A great deal of category theory can be generalized to enriched categories, including $\fV$-enriched functors,
$\fV$-enriched natural transformations, $\fV$-enriched limits and colimits, and more. The canonical reference is
Kelley~\cite{Kelley}, available free and legally online. It covers just about everything we need except for the Day
convolution,\index{Day convolution} which can be read from Day's thesis~\cite{DayThesis}. Another good source, with
a view towards homotopy theory, is \cite[Chapter 3]{RiehlCHT}.
\begin{defn}
Let $\fC$ and $\fD$ be enriched over $\fV$. Then, an \term{enriched functor} $F\colon\fC\to\fD$ is an assignment
of objects in $\fC$ to objects in $\fD$ and maps $\underline\fC(x,y)\to\underline\fD(Fx, Fy)$ that are $\cat
V$-morphisms, and commute with composition.
\end{defn}
A category enriched over $\Top$ is called a \term{topological category}.
\begin{ex}
Work out the definition of enriched natural transformations.\index{enriched natural transformation}
\end{ex}
This brings us to the beginning.
