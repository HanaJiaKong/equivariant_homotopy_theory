\begin{quote}\textit{
	``I thought it was obvious, but it's obviously not obvious.''
}\end{quote}
In the back of your mind, you should be wondering about the connection between $G$-spectra, equivariant cohomology
theories, and Mackey functors. In the nonequivariant setting, Brown representability establishes an (almost)
equivalence between spectra and cohomology theories. Equivariantly, things are very similar: $G$-spectra play the
role of spectra, equivariant cohomology theories play the role of cohomology theories, and, less tautologically,
Mackey functors play the role of abelian groups. We'll talk about rings later.\index{Brown
representability}\index{Mackey functor}

First, we should discuss $\RO(G)$, the ring we'll index cohomology groups on.
\begin{defn}
Let $G$ be a group and $R$ be a ring. Then, the \term{representation ring} of $G$ over $R$ is the Grothendieck
construction applied to the category of finite-rank $G$-representations over $R$, i.e.\ the ring generated by
isomorphism classes of finite-rank $G$-representations over $R$, with the relations $[V\oplus W] = [V]\oplus [W]$
and $[V\otimes W] = [V]\cdot [W]$.\footnote{Not every element of $R(G)$ is the class of a representation: some
elements are formal differences of two representations. These elements are sometimes called \term*{virtual
representations}.\index{virtual representation}} By \emph{the} representation ring, we'll mean
$\RO(G)\coloneqq\R(G)$, the ring of real representations.\index{RO(G)@$\RO(G)$|see {representation ring}}
\end{defn}
We can think of $\R(G)$ and $\C(G)$ as rings of characters: given a representation $V$, its \term{character} is the
map $\chi_V\colon G\to\R$ (or to $\C$) where $\chi_V(g) = \tr(g\cdot)$. Since $\chi_{V\oplus W} = \chi_V + \chi_W$,
$\chi_{V\otimes W} = \chi_V\cdot\chi_W$, and for irreducible representations, $\chi_V = \chi_W$ iff $V\cong W$,
then the characters detect isomorphism classes, so $\R(G)$ and $\C(G)$ can be thought of as the rings of functions
$G\to\R$ (resp.\ $\C$) that are characters.
\begin{comp}{ex}{enumerate}
	\item Compute $\C(G)$ for $G = C_p$, and then for $C_m$ where $m$ isn't prime.
	\item Harder: compute $\R(G)$ for the same $G$.
\end{comp}
Roughly speaking, an $\RO(G)$-graded cohomology theory should be some collection of functors $E^\alpha$ indexed by
$\alpha\in\RO(G)$ such that for any $G$-representation $V$, $E^\alpha(X) \cong E^{\alpha+V}(\Sigma^V X)$, and any
fixed $\alpha$ should satisfy the wedge and cofiber axioms, taking wedge products to coproducts and cofiber
sequences to long exact sequences.\index{RO(G)-graded cohomology theory@$\RO(G)$-graded cohomology theory}

The (well, an) issue that makes this tricky is that you can't really work with isomorphism classes of
representations, even though you need that to define $\RO(G)$. As a workaround, let's fix a
$G$-universe\index{universe} $U$ and define $\cat{RO}(G;U)$ to be the category whose objects are finite-dimensional
representations $V$ that embed equivariantly in $U$ and whose morphisms $V\to W$ are $G$-equivariant isometric
isomorphisms.\footnote{The data of the embedding $V\inj U$ is not used in $\cat{RO}(G;U)$. When we discuss
multiplicative structures, this will change.} We say that two maps $f,g\colon V\to W$ are \term*{homotopic} if their
one-point compactifications $\overline f, \overline g\colon S^V\to S^W$ are stably homotopic. We don't have enough
structure to set up a model category or anything, but we can define the homotopy category $\Ho(\cat{RO}(G;U))$ by
passing to homotopy classes of maps.

A finite-dimensional $G$-representation defines a suspension functor
\[\Sigma^W\colon \Ho(\cat{RO}(G;U))\times\Ho(G\Top)\longrightarrow \Ho(\cat{RO}(G;U))\times\Ho(G\Top)\]
which sends
\[(V,X)\mapsto (V\oplus W, \Sigma^W X).\]
This feels a lot like ordinary $S^W$-suspension, but we're keeping track of the automorphisms of $W$.
\begin{defn}
An \term[RO(G)-graded cohomology theory@$\RO(G)$-graded cohomology theory]{$\RO(G)$-graded cohomology theory} is a
functor $E\colon\Ho(\cat{RO}(G;U))\times\Ho(G\Top)\op\to\Ab$, where $E(V,X)$ is generally written $E^V(X)$,
together with isomorphisms
\[\sigma_W\colon E^V(X)\longrightarrow E^{V\oplus W}(\Sigma^W X),\]
such that for each $V$, $E^V(\bl)$ satisfies the wedge and cofiber axioms and for each isometric isomorphism
$\alpha\colon W\to W'$, the map
\[\xymatrix{
	E^V(X)\ar[r]^-{\sigma_W}\ar[d]^{\sigma_{W'}} & E^{V\oplus W}(\Sigma^W X)\ar[d]^{(\id\oplus\alpha, \id)}\\
	E^{V\oplus W'}(\Sigma^{W'}X)\ar[r]^{(\id\oplus\id, \alpha)} & E^{V\oplus W'}(\Sigma^W X).
}\]
We further require that $\sigma_0 = \id$ and the $\set{\sigma_W}$ are transitive: $\sigma_W\circ\sigma_V =
\sigma_{V\oplus W}$.
\end{defn}
\begin{defn}
The \term*{formal differences} of objects in $\cat{RO}(G;U)$ are equivalence classes of pairs $(V,W)$ of objects in
$\cat{RO}(G;U)$, where $(V,W)\sim(V',W')$ if $V\oplus W'\cong V'\oplus W$. The pair $(V,W)$ is also denoted
$V\ominus W$.
\end{defn}
We can extend an $\cat{RO}(G;U)$-graded cohomology theory to formal differences by $E_{V\ominus W}(X)\coloneqq
E_V(\Sigma^W X)$.

Let $\tau\colon V\oplus W\to W\oplus V$ be the transposition natural isomorphism. Then, the coherence condition on
the $\sigma_{(\bl)}$ means that the diagram
\[\xymatrix{
	E_V(\Sigma^W X)\ar[r]\ar[d] & E_{V\oplus W'}(\Sigma^{W\oplus W'}X)\ar[d]^{\tau_*}\\
	E_{V'}(\Sigma^{W'}X)\ar[r] & E_{V'\oplus W}(\Sigma^{W'\oplus W}X)
}\]
commutes.
\begin{rem}
When you add ring structures to this story, the coherence conditions become gnarlier: there's a cocycle arising
from the pentagon diagram\index{pentagon diagram} for associativity. Unfortunately, this is not treated well in the
literature; Lewis' thesis~\cite{LewisThesis} discusses it, but was not fully published. \cite[Appendix
A]{LewisMandell} talks about it, and \citeme{Dugger} worked it out in the abstract in the motivic setting.
\end{rem}
\begin{defn}
When $G$ is finite, it's possible to extend an $\RO(G)$-graded cohomology theory $E$ to a Mackey-functor-valued
theory $\underline E$.

Let $X$ be a $G$-space and $V\in\RO(G)$. Then, the \term{Mackey functor-valued $E$-cohomology} of $X$, denoted
$\underline E^V(X)$, is the Mackey functor defined as follows:
\begin{itemize}
	\item On an orbit $G/H$,
	\[\underline E^V(X)(G/H)\coloneqq E^V(G/H\times X).\]
	\item Given subgroups $H\subseteq K$, we get a covering map $\pi\colon G/H\times X\to G/K\times X$. The
	restriction map for $\underline E^V(X)$ is the pullback $\pi^*\colon E^V(G/K\times X)\to E^V(G/H\times X)$ and
	the transfer map is the Gysin map $\pi_!\colon E^V(G/H\times X)\to E^V(G/K\times X)$.\footnote{\TODO:
	double-check to make sure this is right. Why does the Gysin map exist? Are all the arrows in the right
	direction?}
\end{itemize}
\end{defn}
There's a similar construction for $E$-homology. We'll see this structure explicitly in examples in
\S\S\ref{const_comp} and \ref{burncomp}.

We'll also need to talk about the equivalence between $\Spc^G$ and $\RO(G)$-graded cohomology theories; one
direction is Brown representability, and the other is more or less straightforward.\index{Brown representability}

You may be asking why we use $\RO(G)$-graded theories at all. One great reason is that there are known examples of
$\Z$-graded cohomology theories that are extremely chaotic, but the $\RO(G)$-graded theories have nice patterns.
% earliest example I know is big hard book by Gaunce Lewis
Another, more recent, realization is that in many cases the $\RO(G)$-graded theories carry more useful information.
However, of course, they're extremely hard to compute.

%Anyways, recall that we defined $\RO(G)$-graded cohomology theories $\set{E_G^V(X)}$ to be Mackey functors graded
%on the category $\cat{RO}(G;U)$, where $U$ is the ambient universe, together with isomorphisms $\sigma^W\colon
%E_G^V(X)\cong E_G^{V\oplus W}(\Sigma^W X)$, together with some additional data.
%
%Before we go on, though, we should talk a little about $\RO(G)$.
One of the important connections in ordinary stable homotopy theory is that cohomology theories are closely related
to spectra: every spectrum determines a cohomology theory, and up to a very small ambiguity, a cohomology theory is
represented by a spectrum.
\begin{defn}
Let $E$ be a $G$-spectrum. Then, we define
\term[RO(G)-graded cohomology theory@$\RO(G)$-graded cohomology theory!represented by a $G$-spectrum]{$E$-cohomology}
to be the functor $E_G^V(X)\coloneqq [S^V\wedge X, E]_G$.
\end{defn}
This is an $\RO(G)$-graded cohomology theory, but this is something to check. If you want to restrict your
universe, there are some nuances that have to be overcome.\index{universe}

If you put in finite $G$-sets (basically points), what you get is a Mackey functor: considering maps out of $X$
ensures the transfer maps have the correct variance.\index{Mackey functor}

That was the easy direction: the other direction requires Brown representability. This is originally due to
\cite{Brown}, and we'll give Neeman's interpretation~\cite{Neeman}. It works in any triangulated category, and is
very close to the small object argument, which already makes it a good thing.\index{triangulated
category}\index{small object argument}

Fix a triangulated category $\fC$ that has small coproducts. The stable homotopy categories $\Ho(\Spc)$ and
$\Ho(\Spc^G)$ are the examples to keep in mind.
\begin{defn}
An $x\in\fC$ is \term[compact object!in a triangulated category]{compact} if for all countable coproducts over
$y_i\in\fC$,
\[\Map_\fC\paren{x, \coprod_i y_i} \cong \coprod_i \Map_\fC(x, y_i).\]
\end{defn}
This is not the usual definition of compactness in category theory, which uses filtered colimits, but in the stable
setting these are the same as coproducts, motivating our definition. For example, this definition does not
characterize compact topological spaces.
\begin{comp}{defn}{itemize}
	\item A \term[generating set!of a triangulated category]{generating set} of a category $\fC$ is a set
	$T\subseteq\operatorname{ob}(\fC)$ that \term*{detects zero}, i.e.\ for all $x\in\fC$, $x = 0$ iff
	$\Map_\fC(z,x) = 0$ for all $z\in T$. If $\fC$ is triangulated, we additionally require $T$ to be closed under
	shift.
	\item $\fC$ is \term[compactly generated!triangulated category]{compactly generated} if it has a generating set
	consisting of compact objects.
\end{comp}
We introduce these to skate around set-theoretic issues: at some point, we'd like to take a coproduct over all
objects in the category, but that's too large. Instead, taking the coproduct over all generators will have the same
power, and is actually well-defined.

For example, dualizable spectra (resp.\ dualizable $G$-spectra) are a compact generating set for $\Ho(\Spc)$
(resp.\ $\Ho(\Spc^G)$).
\begin{thm}[Brown representability~\cite{Brown, Neeman}]
\label{brown_rep}
\index{Brown representability}
Let $\fC$ be a compactly generated triangulated category and $H\colon\fC\op\to\Ab$ be a functor such that
\begin{enumerate}
	\item
	\[H\paren{\coprod_i X_i}\cong\prod_i H(X_i)\]
	and
	\item $H$ sends exact triangles $X\to Y\to Z\to X[1]$ to long exact sequences
	\[\xymatrix{
		H(X)\ar[r] & H(Y)\ar[r] & H(Z)\ar[r] & H(X[1])\ar[r] & H(Y[1])\ar[r] & H(Z[1])\ar[r] & H(X[2])\ar[r] &
		\dotsb
	}\]
\end{enumerate}
Then, $H$ is \term[representable functor!in a triangulated category]{representable}, i.e.\ there's an $X\in\fC$ and
a natural isomorphism $\Hom_\fC(\bl,X)\to H$.
\end{thm}
\begin{proof}
We're going to build $X$ inductively. Fix a generating set $T$ for $\fC$ of compact objects.

In the base case, let
\[U_0\coloneqq \coprod_{x\in T} H(x)\qquad\text{and}\qquad X_0 \coloneqq \coprod_{\substack{(\alpha,t)\in
U_0\\\alpha\in H(t)}} t.\]
Thus,
\[H(X_0) = H\paren{\coprod_{(\alpha, t)\in U_0} t}\cong\prod_{(\alpha, t)\in U_0} H(t),\]
and in particular there is a distinguished element $\alpha_0\in H(X_0)$ which is $\alpha$ at the $(\alpha,t)$
factor. By the Yoneda lemma, $\alpha_0$ specifies a natural transformation $\theta_0\colon \Map_\fC(\bl,X_0)\to H$,
and by construction, $\theta_0$ is surjective for each $t\in T$.\index{Yoneda lemma}

Now we induct: assume we have $X_i$ and $\alpha_i$ specifying a natural transformation $\theta_i\colon
\Map_\fC(\bl,X_i)\to H$. Then, define
\[U_{i+1}\coloneqq \coprod_{t\in T} \ker(\theta_i(t)\colon\Map_\fC(t,X_i)\to H(t)).\qquad\text{and}\qquad
K_{i+1}\coloneqq \coprod_{(f,t)\in U_{i+1}} t.\]
There's a natural map $K_{i+1}\to X_i$ which applies $f$; let $X_{i+1}$ be its cofiber. Applying $H$, this
produces a map
\[\xymatrix{
	H(X_{i+1})\ar[r] & H(X_i)\ar[r] & H(K_{i+1}),
}\]
and by construction, $\alpha_i\mapsto 0$, so by exactness, we can lift to $\alpha_{i+1}\in H(X_{i+1})$. This
$\alpha_{i+1}$ specifies a natural transformation $\theta_{i+1}\colon \Map_\fC(\bl,X_{i+1})\to H$, and the
following diagram commutes:
\[\xymatrix{
	\Map_\fC(\bl,X_i)\ar[r]^-{\theta_i}\ar[d] & H\\
	\Map_\fC(\bl,X_{i+1}).\ar[ur]_{\theta_{i+1}}
}\]
Thus we have a tower $\N\to\fC$ sending $i\mapsto X_i$, and we can define
\[X\coloneqq \hocolim_i X_i,\]
which we'll show represents $H$. The \term{homotopy colimit} in a triangulated category is \TODO.
\begin{rem}
Though triangulated categories are a good language to know, as they're historically interesting and often useful,
they have drawbacks: the octahedral axiom is awkward, for example, and sometimes the triangulated structure works
against you.

For this reason, it can be useful to remember that triangulated categories arise as the homotopy categories of
stable $\infty$-categories, where there's additional versatility simplifying some arguments. For this reason, we
use words such as ``cofiber'' and ``homotopy colimit,'' because they're secretly the same thing. % TODO cofiber is
% assoc to any map...
\index{infinity-category@$\infty$-category}
\end{rem}
\begin{ex}
The homotopy colimit is also called the \term[telescope!in a triangulated category]{telescope}. Relate this to the
usual definition of a telescope.
\end{ex}
Anyways, we'll construct a natural transformation $\theta\colon\Map_\fC(\bl,X)\to H$ as follows. Since $H$ sends
cofiber sequences to long exact sequences, applying it to
\[\xymatrix{
	\coprod_i X_i\ar[r] & \coprod_i X_i\ar[r] & X
}\]
produces a long exact sequence
\[\xymatrix{
	\dotsb\ar[r] & H(X)\ar[r] & \prod_i H(X_i)\ar[r] & \prod_i H(X_i)\ar[r] & \dotsb,
}\]
so in particular we can lift the $\alpha_i\in X_i$ to an $\alpha\in X$, which defines $\theta$ as before, and there
is a commutative diagram
\[\xymatrix{
	\Map_\fC(\bl,X_0)\ar[r]^-{\theta_0}\ar[d] & H.\\
	\Map_\fC(\bl,X)\ar[ur]_{\theta}
}\]
In particular, $\theta$ is surjective on $T$ (namely, $\Map_\fC(t,X)\to H(t)$ is surjective for $t\in T$).

To see that $\theta$ is injective on $T$, we use compactness. Let $f\in\Map_\fC(t,X)$ be such that $\theta(f) = 0$;
we wish to show that $f = 0$. Since $t$ is compact, $f$ lies in some $\Map_\fC(t,X_i)$, i.e.\
$f\in\ker(\Map_\fC(t,X_i)\to H(t))$. Therefore, $f\in U_{i+1}$, so it's killed in $X_{i+1}$, and thus is $0$ in the
colimit. Therefore $\theta$ is injective on $T$, hence and isomorphism.

There exists a largest full triangulated subcategory $\fC'$ of $\fC$ that's closed under small coproducts
and such that $\theta|_{\fC'}$ is a natural isomorphism; we'll show that $\fC' =
\fC$.\footnote{Dissecting what ``largest full triangulated subcategory'' means requires a little care, but such a
$\fC'$ exists.}

Running the whole argument again,\footnote{It's important that we use the same generating set $T$ for this.} we
obtain a $Z\in\fC'$ and a natural transformation $\widetilde\theta\colon\Map_{\fC'}(\bl,Z)\to H$.  We know
$\Map_{\fC'}$ and $\Map_\fC$ are isomorphic, and proved that $\theta$ and $\widetilde\theta$ are isomorphisms on
the generating set, we can consider the cofiber of
\[\Map_\fC(\bl,Z)\longrightarrow\Map_\fC(\bl,X).\]
By the Yoneda lemma,\index{Yoneda lemma} such natural transformations are naturally identified with
$\Map_\fC(Z,X)$, and you can compute the cofiber of $Z\to X$ in terms of these maps. This is $0$, so $Z\cong X$.

So we've proven that $H$ is representable when restricted to the objects in the generating set $T$ in $\fC$.
Furthermore, both $H$ and $\Hom_\fC(\bl,X)$ take coproducts to products and take cofiber sequences to long exact
sequences. As a consequence, we can conclude that the full subcategory $\fC'$ of $\fC$ on which the natural
transformation $\theta\colon \Hom_\fC(\bl,X)\to H$ is an isomorphism is a full triangulated subcategory of $\fC$
which is closed under small coproducts (in $\fC$) and contains $T$.\footnote{Terminology: a triangulated
subcategory that is closed under small coproducts (taken in the ambient category) is called
\index{localizing subcategory!of a triangulated category}\term*{localizing}.}

Therefore, it suffices to show that for any compactly generated triangulated category $\fC$, a full triangulated
subcategory $\fC'$ that contains $T$ and has small coproducts must in fact be all of $\fC$. We show this as
follows. Without loss of generality assume that $\fC'$ is the smallest full triangulated subcategory containing $T$
and small coproducts; that is, take $\fC'$ to be the smallest localizing subcategory containing $T$.

Now fix an object $c\in\fC$ and consider the functor $\Hom_\fC(\bl,c)$. Applying the construction from above, we
obtain an object $z$ and a natural transformation $\Hom_\fC(\bl,z) \to \Hom_\fC(\bl,c)$. By construction, the
object $z\in\fC'$ --- each $X_i$ is in $\fC'$, and so the homotopy colimit is too. Moreover, the map
$\Hom_\fC(t,z)\to \Hom_\fC(t,c)$ is an isomorphism for all $t\in T$.

The Yoneda lemma now implies that we have a map $z\to c$ which corresponds to the natural transformation
$\Hom_\fC(\bl,z) \to \Hom_\fC(\bl,c)$. Consider the cofiber $c/z$. By the previous paragraph and the fact that
$\Hom_\fC(t,\bl)$ takes triangles to long exact sequences, $\Hom_\fC(t, c/z) = 0$ for all $t\in T$, and so $c/z =
0$. Therefore $c$ and $z$ are isomorphic. Since $z$ is in $\fC'$, $c\in\fC'$. As $c$ was chosen arbitrarily, we
conclude that $\fC$ and $\fC'$ coincide. 
\end{proof}
In the next section, we'll discuss what Brown representability means for $G$-spectra, and use it to define
Eilenberg-Mac Lane spectra as the spectra representing certain cohomology theories. One fun fact about $G$-spectra
is that every $\Z$-graded cohomology theory uniquely extends to an $\RO(G)$-graded cohomology theory by taking
shifts, but this is not true for $G$-spaces!
