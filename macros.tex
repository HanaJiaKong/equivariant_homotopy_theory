% File for macros, shortcuts, etc.
% Feel free to add your own, but check whether it's in the .cls file if you think
% I might have added it already.

\institution{UT Austin}
\coursenum{M392C}
\coursename{Topics in Algebraic Topology}
\semester{Spring 2017}
\teacher{Andrew Blumberg}
\author{Arun Debray}
\date{\today}
\email{a.debray@math.utexas.edu}
\thanks{Thanks to Rustam Antia-Riedel, Gill Grindstaff, Yixian Wu, and an anonymous reader for catching a few
errors; to Ernie Fontes, Tom Gannon, Richard Wong, and Valentin Zakharevich for some clarifications; and to Yuri
Sulyma for adding some remarks and references.}

\usepackage[charter]{mathdesign}
\usepackage[T1]{fontenc}
\usepackage{tikz}
\usetikzlibrary{shapes.geometric}

\renewcommand{\term}[1]{{\bfseries #1}}
\newcommand{\sSet}{\cat{sSet}}
\DeclareMathOperator{\Map}{Map}
\DeclareMathOperator{\rank}{rank\,}
\DeclareMathOperator{\Un}{Un}
\newcommand{\KU}{\mathit{KU}}
\newcommand{\isom}{\stackrel\cong\to}
\DeclareMathOperator{\Ind}{Ind}
\DeclareMathOperator{\CoInd}{CoInd}
\newcommand{\Spc}{\cat{Sp}}
\newcommand{\overwedge}{\mathbin{\overline\wedge}}
\DeclareMathOperator{\Ev}{Ev}
\newcommand{\sW}{\mathscr W}
\DeclareMathOperator{\Ho}{Ho}
\newcommand{\simeqto}{\stackrel\simeq\to}
\newcommand{\congto}{\stackrel\cong\to}

% derived functors
\newcommand{\LD}{\mathbf L}
\newcommand{\RD}{\mathbf R}

% Kan extensions
\DeclareMathOperator{\Lan}{Lan}
\DeclareMathOperator{\Ran}{Ran}

% 2x2 matrices
\newcommand{\matr}[4]{
  \begin{bmatrix} #1 & #2 \\ #3 & #4 \end{bmatrix}
}

% nice arrows
\newcommand{\morph}{\mathop{\longrightarrow}\limits}
\newcommand{\twomorph}{\mathop{\Longrightarrow}\limits}

% fancy diagram shortcuts
% double adjunction
\newcommand{\dadjnctn}[6][2em]{
  \xymatrix@1@C=#1{
    #2 \ar@<2ex>[r]^-{#4} \ar@<-2ex>[r]_-{#6}
    \ar@{}@<1.2ex>[r]|-{\scriptscriptstyle\bot} \ar@{}@<-1.2ex>[r]|-{\scriptscriptstyle\bot}
    &
    #3 \ar[l]|-{#5}
  }
}

\theoremstyle{definition}
\newtheorem{ques}[equation]{Question}

\newtheoremstyle{warning}
  {5pt}   % ABOVESPACE
  {5pt}   % BELOWSPACE
  {\normalfont}  % BODYFONT
  {0pt}       % INDENT (empty value is the same as 0pt)
  {\bfseries} % HEADFONT
  {!}         % HEADPUNCT
  {5pt plus 1pt minus 1pt} % HEADSPACE
  {}          % CUSTOM-HEAD-SPEC
\theoremstyle{warning}
\newtheorem*{warn}{Warning}
