% File for macros, shortcuts, etc.
% Feel free to add your own, but check whether it's in the .cls file if you think
% I might have added it already.

\institution{UT Austin}
\coursenum{M392C}
\coursename{Topics in Algebraic Topology}
\semester{Spring 2017}
\teacher{Andrew Blumberg}
\author{Arun Debray}
\date{\today}
\email{a.debray@math.utexas.edu}
\thanks{Thanks to Rustam Antia-Riedel, Gill Grindstaff, Sean Tilson, and Yixian Wu for catching a few errors; to
Andrew Blumberg, Ernie Fontes, Tom Gannon, Tyler Lawson, Richard Wong, Valentin Zakharevich, an anonymous reader,
and the users of the homotopy theory Mathoverflow chatroom for some clarifications and suggestions; and to Yuri
Sulyma for adding some remarks and references.}


\usepackage[charter]{mathdesign}
\usepackage[T1]{fontenc}
\usepackage{tikz}
\usetikzlibrary{shapes.geometric}

\renewcommand{\term}[1]{{\bfseries #1}}
\newcommand{\sSet}{\cat{sSet}}
\DeclareMathOperator{\Map}{Map}
\DeclareMathOperator{\rank}{rank\,}
\DeclareMathOperator{\Un}{Un}
\newcommand{\KU}{\mathit{KU}}
\newcommand{\isom}{\stackrel\cong\to}
\DeclareMathOperator{\Ind}{Ind}
\DeclareMathOperator{\CoInd}{CoInd}
\newcommand{\Spc}{\cat{Sp}}
\newcommand{\overwedge}{\mathbin{\overline\wedge}}
\DeclareMathOperator{\Ev}{Ev}
\newcommand{\sW}{\mathscr W}
\DeclareMathOperator{\Ho}{Ho}
\newcommand{\simeqto}{\stackrel\simeq\to}
\newcommand{\congto}{\stackrel\cong\to}
\newcommand{\wI}{\widetilde I}
\newcommand{\WH}{\mathit{WH}}
\newcommand{\NH}{\mathit{NH}}
\newcommand{\sus}{\Sigma^\infty}
\DeclareMathOperator{\Span}{Span}
\DeclareMathOperator{\tr}{tr}
\newcommand{\RO}{\mathrm{RO}}
\DeclareMathOperator{\Res}{Res}
\DeclareMathOperator{\Tr}{Tr}
\newcommand{\fV}{\cat V}
\newcommand{\rop}{^{\mathrm{op}}} % for "op" of things that aren't categories
\newcommand{\cotensor}{\mathbin\square}
\newcommand{\Mac}{\cat{Mac}}
\newcommand{\Bispan}{\cat{Bispan}}
\newcommand{\Orb}{\cat{Orb}}
\newcommand{\THH}{\mathit{THH}}
\newcommand{\MU}{\mathit{MU}}
\renewcommand{\MR}{\mathit{MR}}
\newcommand{\tH}{\widehat H} % TODO too big
\newcommand{\tEG}{\widetilde{EG}}
\newcommand{\MO}{\mathit{MO}}
\DeclareMathOperator{\Tot}{Tot}
\newcommand{\sk}{\mathrm{sk}}
\newcommand{\KR}{\mathit{KR}}

% derived functors
\newcommand{\LD}{\mathbf L}
\newcommand{\RD}{\mathbf R}

% Kan extensions
\DeclareMathOperator{\Lan}{Lan}
\DeclareMathOperator{\Ran}{Ran}

% 2x2 matrices
\newcommand{\matr}[4]{
  \begin{bmatrix} #1 & #2 \\ #3 & #4 \end{bmatrix}
}

% nice arrows
\newcommand{\morph}{\mathop{\longrightarrow}\limits}
\newcommand{\twomorph}{\mathop{\Longrightarrow}\limits}

% fancy diagram shortcuts

% pushout/pullback decorations
\newcommand{\pullback}{\ar@{}[dr]|<<{\mbox{\Large$\lrcorner$}}}
\newcommand{\pushout}{\ar@<2pt>@{}[ul]|<{\mbox{\Large$\ulcorner$}}}

% \lon = co\colon --- used in the inline version of \adjnctn, just being careful about kerning
\newcommand*\lon{%
        \nobreak
        \mskip6mu plus1mu
        \mathpunct{}%
        \nonscript
        \mkern-\thinmuskip
        {:}%
        \mskip2mu
        \relax
}

% smart adjunction macro --- displays differently in inline vs display mode
\newcommand{\adjnctn}[4]{
  \mathchoice{
    \xymatrix@1{
      #1 \ar@<1ex>[r]^-{#3} \ar@{}[r]|-{\bot} & #2 \ar@<1ex>[l]^-{#4}
    }
  }{
    \xymatrix@1@C=2em{
      \rule[-1ex]{0pt}{3.5ex}
      #3 \colon #1 \ar@<1ex>[r] \ar@{}[r]|-{\bot} & #2 \lon #4 \ar@<1ex>[l]
    }
  }{}{}
}

% double adjunction
\newcommand{\dadjnctn}[6][2em]{
  \xymatrix@1@C=#1{
    #2 \ar@<2ex>[r]^-{#4} \ar@<-2ex>[r]_-{#6}
    \ar@{}@<1.2ex>[r]|-{\scriptscriptstyle\bot} \ar@{}@<-1.2ex>[r]|-{\scriptscriptstyle\bot}
    &
    #3 \ar[l]|-{#5}
  }
}

\theoremstyle{definition}
\newtheorem{ques}[equation]{Question}
\newtheorem{cons}[equation]{Construction}
\newtheorem{beast}[equation]{Bestiary}
\newtheorem{notdefn}[equation]{``Definition''}

\newtheoremstyle{warning}
  {5pt}   % ABOVESPACE
  {5pt}   % BELOWSPACE
  {\normalfont}  % BODYFONT
  {0pt}       % INDENT (empty value is the same as 0pt)
  {\bfseries} % HEADFONT
  {!}         % HEADPUNCT
  {5pt plus 1pt minus 1pt} % HEADSPACE
  {}          % CUSTOM-HEAD-SPEC
\theoremstyle{warning}
\newtheorem*{warn}{Warning}

% macros for TODOs. There should probably also be a comment macro
\newcommand{\TODO}{\textcolor{red}{TODO}}
\newcommand{\citeme}[1]{[\textcolor{red}{CITE ME}: #1]}

% version info from Github, using gitinfo2 package (see README.md for details)
\newcommand{\versioninfo}{
	\ifnum\pdfstrcmp{\gitHash}{MISSING}=0
	% git data missing
		\relax
	\else % git data present
		{\small\begin{center}Git hash:
		\href{https://github.com/adebray/equivariant_homotopy_theory/commit/\gitAbbrevHash}
		{\texttt{\gitAbbrevHash}} (Updated \gitAuthorIsoDate)\gitDirty\end{center}}
	\fi
}

% possibly: \EM{R} -> Eilenberg-Mac Lane HR. \mathit?
