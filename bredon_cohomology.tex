%!TEX root = m392c_EHT_notes.tex

\begin{quote}\textit{
	``Smith's theorem was proven by Smith, hence the name.''
}\end{quote}
Today I was running about two minutes late. Sorry, everybody! (What I missed: Andrew discussed holding a ``Q and
A'' session next week on a weeknight, perhaps with pizza. If you're interested in this, email him to let him know
which nights work for you.)

Today we're going to talk more about Bredon cohomology, including an analogue of the Eilenberg-Steenrod axioms for
it. Then, we'll turn to the circle of ideas around Smith theory, including the Sullivan conjecture (which we won't
prove, because it's hard). Smith theory discusses when one can recover $H_*(X^G)$ from a $G$-action on $X$ and on
$H_*(X)$.

Recall that if $X$ is a $G$-CW complex, we defined Bredon cohomology as follows: we set up a chain complex of
coefficient systems (i.e.\ functors $\sO_G\op\to\Ab$) $\underline C_*(X)$, where
\[\underline C_n(G/H)\coloneqq H^n((X^n)^H, (X^{n-1})^H; \Z).\]
The differential $\partial:\underline C_n(X)\to\underline C_{n-1}(X)$ is induced by the connecting morphism in the
long exact sequence for the triple $(X^n, X^{n-1}, X^{n-2})$; that $\partial^2 = 0$ is something you have to check,
though it's not very difficult.

With this you can define two things:
\begin{itemize}
	\item The Bredon cohomology with coefficients in a coefficient system $M$ is
	\[H_G^n(X;M) \coloneqq H^n(\Hom_{\Fun(\sO_G\op, \Ab)}(\underline C_*(X), M)).\]
	We defined this last time.
	\item For homology to be covariant, we need the coefficient system $M$ to be a functor $\sO_G\to\Ab$ rather
	than $\sO_G\op\to\Ab$ (e.g.\ $\underline H^*(X)$ for a $G$-space $X$, which sends $G/H\mapsto\set{H^n(X^H)}$).
	With $M$ such a coefficient system, the \term{Bredon homology} with coefficients in $M$ is
	\[H_n^G(X; M) \coloneqq H_n(\underline C_n(X)\otimes_{\sO_G} M).\]
	By this tensor product, we mean a coend:
	\[\underline C_*(X)\otimes_{\sO_G} M \coloneqq \coprod_{G/H} \underline C_n(X)(G/H)\times M(G/H)/\sim,\]
	where if $f\in\Map_{\sO_G}(G/H, G/K)$, $(f^*y, z)\sim (y, f_*z)$.\footnote{Tensor products are particular
	instances of coends; instead of inducing an equivalence $mr\otimes n\sim m\otimes rn$, you flip a map across
	the two objects. One might write $y\cdot f$ for $f^* y$ and $f\cdot z$ for $f_* z$ to emphasize this point of view.}
\end{itemize}
The whole philosophy of Bredon (co)homology is that you understand the cohomology or homology through the
fixed-point sets and the lattice of subgroups of $G$.
\begin{exm}
Let's return to Example~\ref{S2pi}. We had a $C_2$-action on $S^2$. There were two $0$-cells, one $1$-cell
$C_2\times I$, and one $2$-cell $C_2\times D^2$, and the chain complexes $\underline C_n(C_2/H)$ as we wrote down
last lecture. We aim to compute the Bredon cohomology for the constant coefficient system $M = \underline\Z$ (so
$M(C_2/H) = \Z$ for all $H$).

The orbit category for $C_2$ is particularly simple:
\[\xymatrix{
	C_2/e\ar[d]\ar@(ul, ur)\\
	C_2/C_2.
}\]
So a coefficient system is determined by a map $M_1\to M_2$ and an involution on $M_2$. In our case, $\underline
C_k(X)(C_2/e) = \Z\oplus\Z$ for $k = 0,1,2$, and the involution flips the two factors.  Therefore we obtain a chain complex of coefficient systems which is
\[
\shortexact[f_1][f_2]{\Z\oplus\Z}{\Z\oplus\Z}{\Z\oplus\Z}\quad
\]
at $C_2/e$ and
\[
\shortexact{0}{0}{\Z\oplus\Z}\quad
\]
at $C_2/C_2$.
Next we need to determine the differentials.  By definition, these are determined by the cellular boundary maps, i.e.\ the attaching maps. Let $(x,y)$ denote the standard basis of $\Z\oplus\Z$.
\begin{itemize}
	\item $f_1$ sends $(x,y)\mapsto (x+y, x+y)$.
	\item $f_2$ sends $(x,y)\mapsto (x-y, y-x)$, 
\end{itemize}

\iffalse
In terms of matrices, the differentials are given by
\[\xymatrix{
	0 \ar[r] & \Z\oplus\Z \ar[r]^-{\text{\tiny$\matr 1{-1}1{-1}$}} & \Z\oplus\Z \ar[r]^-{\text{\tiny$\matr 1{-1}{-1}1$}} & \Z\oplus\Z \ar[r] & 0
}\]
\fi

Next, we compute the $\Hom$ from this coefficient system to the constant coefficient system $\Z$.  These groups are evidently determined by what happens at the $C_2/e$ spot.  For the first two terms, we are looking at maps $\Z \oplus \Z \to \Z$ that are equivariant (where $g \in C_2$ acts by transposition on the left and trivially on the right).  Such a map is determined by where it sends $(1,0)$, and so these two terms are isomorphic to $\Z$.  The last term is simply all homomorphisms $\Z \oplus \Z \to \Z \oplus \Z$, which is isomorphic to $\Z \oplus \Z$.
Therefore, we have the cochain complex
\[
\shortexact{\Z \oplus \Z}{\Z}{\Z}.
\]
Computing the differentials, we find that the first map is $(a,b) \mapsto a+b$ and the second map is $0$.

Taking the cohomology of this complex then produces
\[H^n(S^V;\underline\Z) = \begin{cases}
	\Z, & n = 0, 2\\
	0, &\text{otherwise.}
\end{cases}\]
\end{exm}
\begin{rem}
We're used to reading off properties of a space from its cohomology, and this is still true here, if harder. For
example, $H^0$ tells us the number of connected components of $S^2/C_2$.
\end{rem}
We want cohomology to have nice formal properties analogous to the Eilenberg-Steenrod axioms. We'll think of $H_G$
as a general $G$-equivariant cohomology theory of pairs $H_G^n(X, A; M)$, but in our case this will just be
$H_G^n(X/A; M)$.
\begin{enumerate}
	\item $H_G^*$ should be invariant under weak equivalences: a weak equivalence $X\to Y$ should induce an
	isomorphism $H_G^n(Y;M) \morph^{\cong} H_G^n(X;M)$.
	\item Given a pair $A\subseteq X$, we get a long exact sequence
	\[\xymatrix{
		\dotsb\ar[r] & H_G^n(X, A; M)\ar[r] & H_G^n(X;M)\ar[r] &H_G^n(A;M)\ar[r]^-\delta &H^{n+1}(X,A;M)\ar[r] &
		\dotsb
	}\]
	\item The excision axiom: if $X = A\cup B$, then
	\[H_G^n(X/A; M)\cong H_G^n(B/(A\cap B); M).\]
	\item The Milnor axiom:
	\[H_G^n\paren{\bigvee_i X_i; M} = \prod_i H_G^n(X_i; M).\]
	\item Finally, for now we impose the dimension axiom: our points are orbits $G/J$, so we ask that $H^n(G/H; M)$
	is concentrated in degree $0$.
\end{enumerate} %TODO: some underlines in the proof may be missing
Some of these are easier than others: Bredon cohomology is manifestly homotopy-invariant in the same ways as
ordinary cohomology, so invariance under weak equivalence and the Milnor axiom are immediate, and excision follows
because if all spaces involved are CW complexes, $X/A\cong B/(A\cap B)$.

What takes more work is the dimension axiom and the long exact sequence. We'll show that $\underline C_n(X)$ is a
projective object, and hitting projective objects with $\Hom$ produces a long exact sequence by homological
algebra. (Recall that an object $P$ in a category where you can do homological algebra is \term{projective} if
$\Hom(P,\bl)$ is exact, which is equivalent to maps to $P$ lifting across surjections $M\surj P$.)
\begin{proof}[Proof of the dimension and long exact sequence axioms]
Observe that $\underline C_n(X)$ splits as a direct sum of pieces $H_n(G/H_+\wedge S^n)\cong \widetilde H_0(G/H)$
indexed by the cells $G/H$ of $X$. At $G/K$, $H_0(G/H) = \Z[\pi_0(G/H)^K]$. This is a free abelian group, and
we'll directly use the lifting criterion to prove this is projective. That is, we'll write down an isomorphism
\[\vp\colon\Hom_{\Fun(\sO_G\op, \Ab)}(\underline H_0(G/H), M) \morph^{\cong} M(G/H).\]
This immediately proves $\underline H_0(G/H)$ is projective: evaluating a coefficient system on an exact sequence
produces an exact sequence, and we've shown $\Hom(\underline H_0(G/H), \bl)$ is evaluation of a coefficient system.

The map $\vp$ takes a homomorphism $\theta$ and applies it to $\id_{G/H}$, which produces something in $M(G/H)$.
Why is this an isomorphism? The Yoneda lemma is a fancy answer, but you can prove it in a more elementary manner.
\begin{ex}
Calculate that any $\theta\in\Hom_{\Fun(\sO_G\op, \Ab)}(H_0(G/H), M)$ is determined by where the identity
$\id\in\Map_{\sO_G}(G/H, G/H)$ is sent, implying $\vp$ is an isomorphism.
\end{ex}
Thus, we've effectively calculated the value at $G/H$, proving the dimension axiom as well.
\end{proof}
The Eilenberg-Steenrod axioms hold for Bredon homology, and the proof is the same.
\begin{rem}
Let's foreshadow a little bit. In ordinary homotopy theory, one can show that the Eilenberg-Steenroad axioms
plus the value on points determine a cohomology theory, and this is still true in the equivariant case. But then
one wonders about Brown representability and what happens when you remove the dimension axiom --- and indeed
there are lots of interesting examples of generalized equivariant cohomology theories.
\end{rem}
\begin{ex}
We constructed a functor $\underline H:\Fun(\sO_G\op,\Top)\to \Fun(\sO_G\op, \Ab)$. Show that $\underline HM$
represents $H_G^n(\bl; M)$.
\end{ex}
This slick definition of $\underline H$ is one of the advantages of working with presheaves on the orbit category.
\begin{rem}
You might also want to have a universal coefficient sequence, but it's more complicated. The short exact sequence
in ordinary homotopy theory depends on the existence of short projective resolutions. Here, we have enough
projectives and injectives, but resolutions are longer. Thus, taking an injective resolution of $M$ and filtering
the resulting double complex, one obtains a spectral sequence
\[\Ext^{p,q}(\underline C_*(X), M)\Longrightarrow H_G^{p+q}(X;M).\]
{\color{red}Warning}: indexing might be slighly off.

There's a corresponding $\Tor$ spectral sequence.
\end{rem}
Since the category is more complicated, one expects to have to do more work. But sometimes there are nice results
nonetheless; Smith theory is an example. This theorem is very old, from the 1940s, so none of the cohomology in the
statement is equivariant.
\begin{thm}[Smith]
\label{smith}
Let $G$ be a finite $p$-group and $X$ be a finite $G$-CW complex such that (the underlying topological space of)
$X$ is an $\F_p$-cohomology sphere.\footnote{A space $X$ is an \term{$\F_p$-cohomology sphere} if there is an
isomorphism of graded abelian groups $H^*(X;\F_p)\cong H^*(S^n;\F_p)$ for some $n$.} Then, $X^G$ is either empty or
an $\F_p$-cohomology sphere of smaller dimension.
\end{thm}
There are sharper statements, but we can prove this one. It's the start of a long program to understand $H_*(X^G)$
using algebraic data calculated from $X$ and the action of $G$ on $X$. One useful tool in this is the \term{Borel
construction} (well, \emph a Borel construction) $EG\times_G X$ (where this is the usual balanced product, not a
pullback). This is a ``fattened up'' version of $X/G$.
\begin{defn}
The \term{Borel cohomology} of $X$ is $H_G^*(X) \coloneqq H^*(EG\times_G X)$.
\end{defn}
\begin{warn}
This notation is potentially confusing. Outside of the field of equivariant homotopy theory, ``equivariant
cohomology'' generally means Borel cohomology, not Bredon cohomology. In equivariant homotopy theory, $H_G^*$ can
be used to denote both. We will always make the coefficient system for Bredon cohomology explicit, so that
$H_G^*(X)$ unambiguously refers to Borel cohomology, as is standard in the literature.\footnote{The nLab uses a
different convention: Borel cohomology is $H_G^*(X;A)$, and Bredon cohomology is $\mathbf H^*(X_G)$.}
\end{warn}
The finiteness in Theorem~\ref{smith} is key: Elmendorf's theorem lets us build a $C_p$-complex with
non-equivariant homotopy type $S^n$ and any set of fixed points. Thus, in the infinite-dimensional case, we should
be looking at a different thing than the fixed points, namely the homotopy fixed points.
\begin{defn}
The \term{homotopy fixed points} of a $G$-space $X$ is $X^{hG}\coloneqq \Map(EG, X)^G$.
\end{defn}
This can also be defined as a homotopy limit. This is ``easy'' to compute (relative to the rest of equivariant
homotopy theory, that is), as there are spectral sequences, nice models for $EG$, and other tools.

The terminal map $EG\to *$ induces a map $X^G\to X^{hG}$. The Sullivan conjecture is all about when this happens:
when you $p$-complete, things become really nice. (There will be a precise statement next lecture.)

Returning to Theorem~\ref{smith}, we can use Bredon cohomology to give an easy, modern proof. The idea is to
adroitly choose coefficient systems such that we recover $H^n(X)$, $H^n(X^G)$, and $H^n((X/G)/G)$ from Bredon
cohomology. They'll fit into exact sequences, and using tools like Mayer-Vietoris, we'll get inequalities on the
ranks of groups. This will also use the fact that a short exact sequence of coefficient systems induces a long
exact sequence on $H_G^*$. The proof will be beautiful and short, unlike Smith's original proof!

Theorem~\ref{smith} is sufficiently classical that there are several different proofs. Ours illuminates Bredon
cohomology at the expense of obscuring the overarching goal of Smith theory. There's another proof by Dwyer and
Wilkerson in ``Smith theory revisited'' \cite{SmithRevisit}, a short, beautiful paper which is highly recommended. It uses the unstable
Steenrod algebra to prove Theorem~\ref{smith} and more.

Over the next day or so, there will be a problem set and a course webpage. Doing the problem sets is recommened!
