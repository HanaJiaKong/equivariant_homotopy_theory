In this section, we'll take a historical detour and discuss the difference between $\Gamma$-$G$-spaces and
$\Gamma_G$-spaces. This arises from a reasonable question: what's the analogue of a $\Gamma$-space in the
equivariant context?

Recall that $\Gamma$ is the category of based finite sets and based maps, and the category of $\Gamma$-spaces is
the functor category $\Fun(\Gamma, \Top_*)$, as we discussed in \cref{gamma_spaces}. A special $\Gamma$-space $X$
is the analogue of an abelian group: we ask that the induced map
\[X(n)\longrightarrow\prod_{i=1}^n X(1)\]
is an equivalence.

In the equivariant setting, there are two things we can do:
\begin{itemize}
	\item \term{$\Gamma$-$G$-spaces} are $\Gamma$-objects in $G\Top_*$. These were introduced by
	Segal~\cite{SegalEquivariant}.
	\item Let $\Gamma_G$ be the category of finite based $G$-sets and all based maps, so it's $G$-enriched, where
	$G$ acts by conjugation on the mapping space. Then, \term{$\Gamma_G$-spaces} are the enriched functors
	$\Gamma_G\to G\Top_*$. These were introduced by Shimakawa~\cite{Shi89}.
\end{itemize}
Which of these is right? Turns out both are.
\begin{thm}[Shimakawa~\cite{Shimakawa91}]
\label{shimgamma}
These two categories are equivalent.
\end{thm}
We can also line up the special conditions in a nice way. The proof uses that one weird trick in diagram spectra.
The manifestation in orthogonal spectra means that the category of orthogonal spectra on the trivial universe is
equivalent (on the point-set level) to orthogonal spectra on the complete universe! And in fact this is true for
any universe.

The reason this happens is that diagram spectra mix the point-set and homotopical data in a way that can be
confusing. Of course, the homotopy groups depend on the universe, so the homotopy theory and homotopy category are
genuinely different.
\begin{rem}
You might be thinking that this is an artifact of one's desire for concreteness and model categories, but the
$\infty$-categorical view isn't necessarily easier: yes, you don't have to worry as much about point-set phenomena,
and all functors are derived, but it's a lot harder to infer the homotopy theory from first principles and you can
write stuff down, but it's not at all clear how to prove it's the right thing.

To be sure, some of the difficulties in writing down the correct norm map were point-set in nature; but not all of
them were.
\end{rem}
Since finite based sets are finite based $G$-sets (with $G$ acting trivially), there is an inclusion
$\Gamma\to\Gamma_G$, which induces a functor
\[R\colon\Fun(\Gamma_G, G\Top_*)\longrightarrow\Fun(\Gamma, G\Top_*).\]
It has a left adjoint
\[L\colon\Fun(\Gamma, G\Top_*)\longrightarrow\Fun(\Gamma_G, G\Top_*),\]
which is a Kan extension, explicitly given by the coend
\[LX = \int^\Gamma \Hom(n, \bl)\times X(\bl) = \coprod_{n\ge 0} \Map(n, S)\times X(n)/\sim,\]
where $(sf, x)\sim (s, X(f)x)$ for any $\Gamma$-morphism $f\colon n\to m$. The $G$-action is $g[s,x] = [gs,gx]$.

Now, $L$ is an equivariant functor: if $f\colon S\to T$ is a $\Gamma_G$-morphism,
\begin{align*}
	LX(gfg^{-1})[s,x] &= [gfg^{-1}s, x]\\
	&= g[fg^{-1}s, g^{-1}x]\\
	&= gLX(f)[g^{-1}s, g^{-1}x].
\end{align*}
\begin{ex}
Check that $L$ and $R$ are adjoint. This follows from essentially formal reasons.
\end{ex}
\begin{prop}
\label{LReq}
$(L,R)$ are adjoint equivalences.
\end{prop}
\begin{proof}
We need to check that the natural transformations $\id\to RL$ and $LR\to\id$ are natural isomorphisms. The first is
formal: we're Kan extending along a full subcategory, and in this case $\id\to RL$ is always an equivalence.

Now, we want to recover $X(S)$ from $X(n)$, where $\abs S = n$.\footnote{We're using $n$ to denote both the set
$\set{1,\dotsc,n}$ and its cardinality; hopefully they can be distinguished from context.} Choose a bijection
$f\colon S\to n$. The key is that this defines a map $\rho\colon G\to\Hom(n, n)\cong\Sigma_n$, through the
$G$-action on $S$. But now this looks familiar from the story of $G$-operads. Why were we able to get away with
indexing them on finite sets rather than on finite $G$-sets? Well, from the $\infty$-categorical perspective, maybe
it would be better, but we were able to get away with using just finite sets by the same trick!

So now we have a commutative diagram
\[\xymatrix{
	S\ar[r]^f\ar[d]^g & n\ar[d]^{\rho(g)}\\
	S\ar[r]_f &n. 
}\]
Endow $X(n)$ with a new $G$-action\footnote{$X(n)$ was already a $G$-space, but we're putting a different
$G$-action on it.} via the map $G\times X(n)\to X(n)$ sending
\[g,x\mapsto X(\rho(g))(gx).\]
We'll call this $G$-space $X(n)_\rho$. This is another manifestation of a common principle from this class: just
like finite sums and finite products are the same in the stable setting, in the equivariant stable setting, finite
$G$-indexed sums and finite $G$-indexed products are the same. Since $X(n)$ is a product of $n$ copies of $X(1)$,
this is a $G$-indexed product.

Now, we claim that as $G$-spaces, $X(S)\cong X(n)_\rho$.
\begin{ex}
Prove this. There's an obvious map, and you just have to check that it's $G$-equivariant, like we did for $L$
above.
\end{ex}
Assuming this exercise, we get that $LR\to\id$ is a natural isomorphism.
\end{proof}
Now, what does it mean to be special in the equivariant setting?
\begin{comp}{defn}{itemize}
	\item A \term{special $\Gamma_G$-space} is one for which the natural map
	\[X(S)\stackrel\cong\longrightarrow \prod_S X(1)\]
	is an equivalence.
	\item A \term{special $\Gamma$-$G$-space} is one for which the natural map
	\[X(S)\stackrel\cong\longrightarrow \prod_{i=1}^n X(1)\] 
	is an equivalence\footnote{\TODO: I might have gotten this criterion wrong.} --- not just of $G$-spaces, but
	for the family of subsets $H\subset G\times\Sigma_n$ such that $H\cap\set{(e,\sigma)} =
	\emptyset$.\footnote{This is what Segal missed.}
\end{comp}
The definition for $\Gamma$-$G$-spaces again is reminiscent to the operadic story: we're asking for homomorphisms.
\begin{ex}
Show that under the equivalence of \cref{LReq}, these two notions of special objects are identified.
\end{ex}
\begin{comp}{rem}{enumerate}
	\item You can run the rest of Segal's story: there's a notion of a very special $\Gamma$-$G$-space or
	$\Gamma_G$-space, which is a condition on fixed points, and using similar machinery as before, you can turn
	very special objects into connective genuine $G$-spectra.
	\item We didn't discuss the homotopy theory of these two categories: the natural notions of fibrancy for
	$\Gamma_G$-spaces and $G$-$\Gamma$-spaces are different. This is analogous to the difference between the naïve
	homotopy groups of $G$-spectra on the trivial universe, and the more sophisticated homotopy groups that you
	could define on them, and which agree with the homotopy groups of $G$-spectra on the complete universe. This
	is really saying something about Kan extensions and diagram spaces. \qedhere
\end{comp}
