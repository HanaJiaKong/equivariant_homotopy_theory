Now, we'll shift gears and talk about ring structures on $G$-spectra. This will lead eventually to multiplicative
structures on $\RO(G)$-graded cohomology theories, through Green functors (simpler) and Tambara functors (more
sophisticated), but the topology comes from the algebra, so let's start with the algebra.

By construction, $\Spc^G$ is symmetric monoidal: it's a diagram category, and as such has a symmetric monoidal Day
convolution. By a \term{ring spectrum} we mean a monoid in $\Spc^G$, and by a \term{commutative ring spectrum} we
mean a commutative monoid.

One issue with this approach: why is there only one kind of ring? We found lots of different kinds of abelian
groups in $\Spc^G$: different subcategories of $\sO_G$ or different universes $U$ lead to different definitions. In
other words, a choice of abelian group structure amounts to a choice of compatible transfers. We want a parallel
story for rings, meaning we'll want some sort of ``multiplicative transfer'' maps.

We'll say this in two ways: first in an impressionistic way, then more formally.

\begin{ex}
Show that in any symmetric monoidal category $\fC$ (e.g.\ $\Spc$, $\Spc^G$, $\Ab$), the coproduct in the category
of commutative monoid objects in $\fC$ is the same as the monoidal product in $\fC$.
\end{ex}
This implies, for example, that the tensor product is the coproduct for rings.
Commutative rings are tensored over finite sets, and multiplication is encoded by maps of sets: given a finite set
$\underline n$, we let $R\otimes\underline n\cong R^{\lor n}$, and the multiplication map is the universal map
$R^{\vee n}\to R$, i.e.\ the coproduct. The point is that this structure determines a ring structure on an abelian
group, so we can use it in less familiar settings.

In $\Spc^G$, this says that equivariant multiplication should be controlled by tensoring with $G$-sets (at least
when $G$ is a finite group): that is, if $T$ and $S$ are finite $G$-sets, we want a ring structure to be defined by
maps
\[\bigvee_T R\longrightarrow R\qquad\qquad\text{and}\qquad\qquad\bigvee_T
R\longrightarrow\bigvee_S R.\]
This is where the other notions of rings come from: it's reasonable to restrict to certain subcollections of finite
$G$-sets. In $G$-spaces, this amounts to choosing transfers, and for $G$-spectra, this amounts to choosing
multiplicative transfers, which are called norms; suspension will send transfers to norms.

Operadic multiplication will make this precise. $\Spc^G$ and $G\Top$ are enriched over $G\Top$, so we'll work with
operads in $G\Top$.\footnote{There are more general settings you can work in, e.g.\ using operads in $\Spc^G$ or
$\Spc$ in the stable setting. This is reasonable, but we won't need to do this.}
\begin{defn}
A \term{$G$-operad} is a collection of $G\times\Sigma_n$-spaces $\set{O(n)}$ along with structure maps
\[s_{k,\vec n}\colon O(k)\times O(n_1)\times O(n_2)\many\times O(n_k)\longrightarrow O(n_1\many + n_k)\]
which are suitably equivariant, i.e.\ $G$-equivariant and equivariant under the permutations in $\Sigma_{n_1\many+
n_k}$ that arise as block permutations in $(n_1,\dotsc,n_k)$, and are associative and unital.
\end{defn}
You should think of the structure map $s_{k,\vec n}$ as grafting leaves from $O(n_1),\dotsc,O(n_k)$ onto a tree
$O(k)$.
\begin{defn}
Let $O$ be a $G$-operad. An \text{$O$-algebra}, in either $G\Top$ or $\Spc^G$, is an object $X$ with structure maps
\[O(n)\times_{\Sigma_n} X^{\times n}\longrightarrow X\]
compatible with the operadic multiplication. Here, $\times$ denotes the product in $G\Top$ or $\Spc^G$.
\end{defn}
\begin{exm}\hfill
\label{Gopexm}
\begin{enumerate}
	\item Any operad in $\Top$, such as the associative or commutative operads, defines a $G$-operad with trivial
	$G$-action.
	\item Let $U$ be a $G$-universe, and $O(n)$ be the space of linear isometries from $U^{\oplus n}$ to $U$; the
	structure maps come from composition.\footnote{Interestingly, there's something important about spaces here: if
	you take chains, the operadic axioms fail.} This operad is called $\mathcal L_U$, the \term{linear isometries
	operad}.
	\qedhere
\end{enumerate}
\end{exm}
\begin{defn}[Blumberg-Hill~\cite{NInfty}]
\label{ninfty}
An \term{$N_\infty$-operad}\footnote{The $N$ in $N_{\infty}$ stands for ``norm.''} is a $G$-operad $O = \set{O(n)}$
such that
\begin{enumerate}
	\item $\Sigma_n$ acts freely on $O(n)$ (one says $O(n)$ is \term{$\Sigma_n$-free}), and
	\item $O(n)$ is a universal space for some family $\sF_n$ of subgroups of $G\times\Sigma_n$ containing
	$H\times\set 1$ for all subgroups $H\subseteq G$. That is, if $\Gamma\subset G\times\Sigma_n$,
	\[O(n)^\Gamma = \begin{cases}
		*, &\Gamma\in\sF_n\\
		\varnothing, &\text{otherwise.}
	\end{cases}\]
	\item $O(n)$ is nonequivariantly contractible.\footnote{This means that if you forget from $G$-spaces to
	spaces, $N_\infty$-operads become $E_\infty$-operads.}
\end{enumerate}
\end{defn}
That $O(n)$ is $\Sigma_n$-free places strong constraints on $\set{\sF_n}$. For example, if $\Gamma\in\sF_n$, then
$\Gamma\cap\paren{\set 1\times\Sigma_n} = \set 1$.
\begin{lem}
\label{admlem}
Let $\Gamma\subseteq G\times\Sigma_n$ be a subgroup. Then, the following are equivalent.
\begin{enumerate}
	\item $\Gamma\cap(\set 1\times\Sigma_n) = \set 1$.
	\item $\Gamma$ is the graph of a homomorphism, i.e.\ there's an $H\subseteq G$ and a group homomorphism
	$p\colon H\to\Sigma_n$ such that $\Gamma = \set{(h,p(h))\mid h\in H}$.
\end{enumerate}
\end{lem}
The proof is two lines, neither of which is too hard. In the context of a family $\sF_n$, subgroups therefore
define maps $p\colon H\to\Sigma_n$ and therefore $H$-set structures on $\underline n$; such $H$-sets will be called
\term{admissible}.
\begin{exm}
The operad $\mathcal L_U$ from \cref{Gopexm} is $N_\infty$, which is a nontrivial fact: unlike for
finite-dimensional spaces, the space of linear isometries from $U^n$ to $U$ is contractible. This is a nice thing
to think through. The rest of the axioms are easier.
\end{exm}
You might wonder whether the little discs operad appears in this context; it turns out that fitting everything
together into a colimit behaves poorly, and so one uses something else called the Steiner operad, which maybe will
appear in these notes.

In the nonequivariant case, the linear isometries operad is a model for the $E_\infty$-operad. But in the
equivariant case, there are $N_\infty$-operads that aren't the linear isometries operad, and this is a disturbing
representation-theoretic fact.
