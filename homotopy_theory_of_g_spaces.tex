%!TEX root = m392c_EHT_notes.tex
\begin{quote}\textit{
	``It's nice to write down, but oh so false.''
}\end{quote}
Last time, we saw the definition of a $G$-CW complex, but no examples were provided. Today, we'll start with some
examples.

Recall that a $G$-CW complex is a sequential colimit $X = \colim_n X_n$, where $X_n$ is formed by attaching cells
$G/H\times D^n$ along maps $G/H\times S^{n-1}\to X_{n-1}$: just like the CW complexes we know and love, but with
new cells $G/H$ indexed by the closed subgroups $H\subset G$. The idea is that you're building up a space by
attaching different spaces with different isotropy groups ($G/H$ has isotropy group $H$, just by construction).
\begin{exm}[Zero-dimensional complexes]
The zero-dimensional complexes are $G/H$ or disjoint unions $\amalg_i G/H_i$. This is an instance of the slogan
that ``orbits are points.'' Keep in mind that if $G$ is a compact Lie group, this might not be zero-dimensional in
other, more familiar kinds of dimension.
\end{exm}
\begin{exm}
Let $S^1$ act on $\R^2$ by rotation along the origin. This also induces a $C_n$-action, as $C_n\subseteq S^1$ as
the $n^{\text{th}}$ roots of unity. Let $V$ denote this $C_n$-space.

Let $D(V)$ denote the unit disc in $V$, and $S^V$ denote its one-point compactification, a representation sphere.
Then, $D(V)$ looks like wedges of pie, as the origin is fixed. On $S^V$, the point at infinity is also fixed, so we
obtain a beachball. % TODO picture

Now let's consider $V$ as an $S^1$-space, and write down the CW structure on $S^V$. There are two fixed points, and
each one is a $0$-cell $S^1/S^1\times *$, but there is one $1$-cell $S^1\times I$ attached to the endpoints
(thought of as a meridian rotated around the sphere).

Now let's consider the beachball for $C_2$ on $S^V$, where there are two hemispheres and $C_2$ rotates by a
half-turn. What's the $G$-CW structure on this?
\begin{itemize}
	\item There are two $0$-cells $C_2/C_2\times *$, corresponding to the two fixed points, the north and south
	poles.
	\item There is a single free $1$-cell $C_2\times I$, corresponding to the boundary of the hemispheres.
	\item There is a single $2$-cell $C_2\times D^2$.\qedhere
\end{itemize}
\end{exm}
Last time, we discussed additional cells $G\times_H S(V)$ and $G\times_H D(V)$; these can be decomposed in terms of
our actual $G$-cells, but it's also worth mentioning that the action of $G$ on our preferred $G$-cells is cellular,
unlike for the other cells.
\begin{ex}
$C_2$ also acts on $S^2$ by the antipodal map, which has no fixed points. Write a $C_2$-CW cell structure for this
$C_2$-space.
\end{ex}

\begin{ex}
The torus $S^1 \times S^1$ has one group action $z(z_1, z_2) = (zz_1, z_2)$. With this action, the torus can be viewed as a $G$ CW complex with one zero cell $S^1/e \times *$ and one one cell $S^1 \times [0,1]$ with attaching map sending $0$ and $1$ to $*$.
\end{ex}
There will be additional examples of $G$-CW complexes on the homework, some with richer structure.
\begin{rem}
At this point in class, the professor mentioned that these notes are hosted on Github at
\url{https://github.com/adebray/equivariant_homotopy_theory}. Since there aren't very many sources for learning
this material, and existing ones tend to have few examples, the hope is that these notes can be turned into a good
source of lecture notes for learning this material. So as you're learning this material, feel free to add
examples, insert comments (e.g.\ ``this section is confusing/unmotivated''), and let me know if you want access to
the repository.
\end{rem}
\begin{comp}{rem}{enumerate}
	\item There is a technical issue of a $G$-CW structure on a product of $G$-CW complexes; namely, there are
	technical difficulties in cleanly putting a $G$-CW structure on $G/H_1\times G/H_2$ involving triangulation.
	We won't digress into this: it's straightforward for finite groups, but a theorem for compact Lie groups, and
	required revisiting the foundations. Similarly, if $H\subset G$, we'd like the forgetful functor $G\Top\to
	H\Top$ to send $G$-CW complexes to $H$-CW complexes. This is again possible, yet involves technicalities.
	\item A nicer fact is that computing the fixed points of a $G$-CW complex is straightforward. Recall that
	$(\bl)^H$ is a right adjoint, which can be seen by realizing it as the limit of the diagram
	\[\xymatrix{
		\bullet\ar@(ur, ul)_H\ar[r] & \Top.
	}\]
	Thus, we don't expect it to commute with colimits in general. However, it does commute with many important
	ones, as in the following proposition.\qedhere
\end{comp}
\begin{prop}
The fixed point functor $(\bl)^H$ commutes with
\begin{enumerate}
	\item pushouts where one leg is a closed inclusion, and
	\item sequential colimits along closed inclusions.
\end{enumerate}
\end{prop}
This is great, because it means we can commute $(\bl)^H$ through the construction of a $G$-CW complex! In
particular, on each cell,
\[(G/K\times D^n)^H\cong (G/K)^H\times D^n,\]
so we need to understand $(G/K)^H\cong\Map^G(G/H,G/K)$. We will return to this important point.
\subsection*{Two approaches to the Whitehead theorem.}
We'll now discuss some homotopy theory of $G$-spaces and the Whitehead theorem. The first will be a hands-on proof
using the HELP lemma. This is an elegant approach to unstable homotopy theory due to Peter May in which one lemma
gives quick proofs of several theorems. In the equivariant case, it allows a quick reduction to the non-equivariant
case; it will be useful to see a proof of this nature. Ultimately, we will take a different approach involving
model categories, and this will be the second perspective.
\begin{defn}
Let $X,Y\in\Top$ and $f:X\to Y$ be continuous. Then, $f$ is \term{$n$-connected} if $\pi_q(f):\pi_q(X)\to\pi_q(Y)$
is an isomorphism when $q < n$ and surjective when $q = n$.
\end{defn}
We wish to generalize this to the equivariant case.
\begin{defn}
Let $\theta\colon\set{\text{conjugacy classes of subgroups of $G$}}\to\set{x\in\Z\mid x\ge -1}$.
\begin{itemize}
	\item A map $f:X\to Y$ of $G$-spaces is \term{$\theta$-connected} if for all $H\subset G$, $f^H$ is
	$\theta(H)$-connected.
	\item A $G$-CW complex is \term{$\theta$-dimensional} if all cells of orbit type $G/H$ have (nonequivariant)
	dimension at most $\theta(H)$.
\end{itemize}
\end{defn}
\begin{thm}[Equivariant HELP lemma]
\label{HELP}
Let $A$, $X$, $Y$, and $Z$ be $G$-CW complexes such that $A\subseteq X$ is $\theta$-dimensional and let $e:Y\to Z$
be a $\theta$-connected $G$-map. Given $g\colon A\to Y$, $h\colon A\times I\to Z$, and $f\colon X\to Z$ such that $eg = hi_0$ and
$fi = hi_1$, there exist maps $\tilde g\colon X\to Y$ and $\tilde h\colon X\times I\to Z$ that make the following diagram commute:
\[\xymatrix{
	A \ar[dd] \ar[rr]^-{i_0} && A\times I \ar[dd] \ar[dl]_-h && \ar[ll]_-{i_1} \ar[dl]_-g A \ar[dd]\\
	& Z && \ar[ll]_(0.35)e Y\\
	X \ar[ur]^-f \ar[rr]_-{i_0} && \ar@{-->}[ul]^-{\tilde h} X\times I && \ar[ll]^-{i_1} \ar@{-->}[ul]^-{\tilde g} X
}\]
\end{thm}
This is a massive elaboration of the idea of a Hurewicz cofibration. The best way to understand this is to prove
it (though it's not an easy proof).

In the non-equivariant case, one reduces to working one cell at a time, inductively extending over the cells of $X$
not in $A$.\footnote{This requires reducing to the case where $X$ is a finite CW complex, but taking a sequential
colimit recovers the theorem for all CW complexes $X$.} In this case, look at $S^{n-1}\subseteq D^n$. Now you just
do it: at this point, there's no way to avoid writing down explicit homotopies.
\begin{ex}
Think about this argument, and then read the proof in~\cite{ConciseCourse}.
\end{ex}
The equivariant case is very similar: in the same way, one can reduce to inductively attaching a single cell in the
case where $X$ is a finite CW complex. This comes via a map $G/H\times S^{n-1}\to G/H\times D^n$, but the only
interesting content is in the nonequivariant part, so we can reduce again to $S^{n-1}\to D^n$ with trivial
$G$-action! This allows us to finish the proof in the same way. It also says that the homotopy theory of $G$-spaces
is lifted from ordinary homotopy theory, in a sense that model categories will allow us to make precise.

The first consequence of Theorem~\ref{HELP} is:
\begin{thm}
\label{estar}
Let $e:Y\to Z$ be a $\theta$-connected map and $e_*: [X,Y]\to [X,Z]$ be the map induced by composition.
\begin{itemize}
	\item If $X$ has dimension less than $\theta$, $e_*$ is a bijection.\footnote{{\color{red}TODO}: What does it
	mean to have dimension less than $\theta$?}
	\item If $X$ has dimension $\theta$, $e_*$ is a surjection.
\end{itemize}
\end{thm}
The proof is an exercise; filling in the details is a great way to get your hands on what the HELP lemma is
actually doing. Hint: consider the pairs $\emptyset\to X$ and $X\times S^0\to X\times I$, and apply the HELP lemma.
\begin{cor}[Equivariant Whitehead theorem]
\label{eqWhite}
Let $e:Y\to Z$ be a weak equivalence of $G$-CW complexes. Then, $e$ is a $G$-homotopy equivalence.
\end{cor}
\begin{proof}
This is also a standard argument: using Theorem~\ref{estar}, $e_*$ is a bijection, so we can pull back
$\id_Z\in[Z,Z]$ to an inverse $(e_*)^{-1}(\id_Z)\in [Z,Y]$, which is a homotopy inverse to $e$.
\end{proof}
One can continue and prove the cellular approximation theorem in this way, and so forth. We won't do this, because
we'll approach it from a model-categorical perspective.

One thing that's useful, not so much for this class as for enriching your life, is to learn how to approach this
from the perspective of abstract homotopy theory, learning about disc complexes and so forth. You can prove
theorems such as the HELP lemma and its consequences in a general setting, and then specialize them to the cases
you need. This is a great way to ``just do it'' without needing model categories.

Anyways, we'll now define a model structure on $G\Top$ and $G\Top_*$. If you don't know what a model category is,
now is a good time to review.
\begin{prop}
There is a model structure on $G\Top$ (and on $G\Top_*$) defined by the following data.
\begin{description}
	\item[Cofibrations] The maps $f:X\to Y$ such that for all $H\subset G$, $f^H:X^H\to Y^H$ is a cofibration.
	\item[Weak equivalences] The maps $f:X\to Y$ such that for all $H\subset G$, $f^H:X^H\to Y^H$ is a weak
	equivalence.
\end{description}
\end{prop}
So we once again parametrize everything over subgroups of $G$ and use fixed points. This is a cofibrantly generated
model category; the cofibrations are specified by generators of acyclic cofibrations in a similar manner to $\Top$.
That is, in $\Top$, one can choose generators $I = \set{S^{n-1}\to D^n}$ and $J = \set{D^n\to D^n\times I}$; in
$G\Top$, we instead take $I_G = \set{G/H\times I}$ and $J_G = \set{G/H\times J}$.

These are cells that we used to define $G$-CW complexes, and this is no coincidence: it's a general fact about
cofibrantly generated model categories that follows from the small object argument\footnote{The small object
argument is a beautiful piece of basic mathematics that everybody should know. If you don't know it, your homework
is to read enough about model categories to get to that point. In general, there may be large objects and
transfinite induction, but for the case we care about large cardinals won't arise.} that cofibrant objects are
retracts of ``cell complexes'' built from the things in $I$, and cofibrations are retracts of cellular inclusions
of cell complexes. In this sense, CW complexes are inevitable.

The Whitehead theorem (Corollary~\ref{eqWhite}) now falls out of the general theory of model categories.
\begin{thm}[Whitehead theorem for model categories]
Let $f:X\to Y$ be a weak equivalence of cofibrant-fibrant objects in a model category. Then, $f$ is a homotopy
equivalence.
\end{thm}
In $\Top$ and $G\Top$, all objects are fibrant, so this is particularly applicable.
\subsection*{The orbit category.}
We'll begin talking about the orbit category in the rest of today's lecture, and discuss the bar construction next
class.
\begin{defn}
The \term{orbit category} $\sO_G$ is the full subcategory of $G\Top$ on the objects $G/H$.
\end{defn}
That is, its objects are the spaces $G/H$, where $H\subset G$ is closed, and its morphisms are $\Map^G(G/H,
G/K)\cong (G/K)^H$. These maps are the same thing as subconjugacy relations, i.e.\ those of the form
\begin{equation}
\label{subconj}
gHg^{-1}\subseteq K;
\end{equation}
a $G$-map $f:G/H\to G/K$ is completely specified by what it does to the identity coset $f(eH) = gK$, and this $g$
implies the subconjugacy relation~\eqref{subconj}.

There's another description of the orbit category.
\begin{prop}
Let $G$ be a finite group. Then, the orbit category $\sO_G$ is equivalent to the category of finite transitive
$G$-sets and $G$-maps.
\end{prop}
The observation that ignites the proof is that if $x\in X$ has isotropy group $H$, then its orbit space is
isomorphic to $G/H$.
\begin{defn}
Given a $G$-space $X$, we obtain a presheaf on the orbit category, namely a functor $X^{(\bl)}:\sO_G\op\to\Top$, by
sending $G/H\to X^H$. This assignment itself is a functor $\psi:G\Top\to\Fun(\sO_G\op, \Top)$.
\end{defn}
\begin{prop}
$\Fun(\sO_G\op, \Top)$ has a projective model structure where the weak equivalences and fibrations are taken
pointwise.
\end{prop}
The point is the following result, a revisionist interpretation of Elmendorf's theorem.\footnote{Elmendorf proved
that these two categories have the same homotopy theory, but his proof was more explicit.}
\begin{thm}
$\psi$ is the right adjoint in a Quillen equivalence; the left adjoint is evaluation at $G/e$.
\end{thm}
The point is, these two model categories have the same homotopy theory.
\begin{ex}
Check that evaluation at $G/e$ is a left adjoint to $\psi$.
\end{ex}
