\begin{quote}\textit{
	``I came here to chew bubblegum and compute cohomology groups, and I'm all out of bubblegum.''
}\end{quote}
\label{burncomp}
Now we'll repeat this all in the slightly harder case of the Burnside Mackey functor $A_{C_2}$. The published
source is~\cite{Lew88}, building on unpublished work of Stong. The answer is somewhat crazy, and we'll have to
define a whole bunch of Mackey functors along the way. Recently, Basu and Ghosh~\cite{BasuGhosh} generalized to $G
= C_{pq}$. There are also some computations for nontrivial spaces, e.g.\ Megan Shulman's
thesis~\cite{MeganShulmanThesis} computes the $\RO(C_p)$-graded cohomology of $\mathit{BO}$ and
$\mathit{BSO}$.
\index{Burnside Mackey functor}
\begin{beast}
We use Lewis' notation~\cite{Lew88} for the various Mackey functors that will arise in the
computation.\footnote{\TODO: However, for consistency with the notation for the orbit category in previous
sections, we've switched which of $C_2/e$ and $C_2/C_2$ is on top. Should this be changed? Whichever we choose, it
should be consistent.}
\label{best_bestiary}
\begin{enumerate}
	\item The Burnside Mackey functor $A_{C_2}$ takes the form\footnote{For $G = C_p$, the Burnside Mackey functor
	looks very similar: $A_{C_p}(C_p/C_p) = \Z[x]/(x^2-px)$, and the restriction map sends $1\mapsto 1$ and
	$x\mapsto p$.}\index{Burnside Mackey functor!for $C_p$}
	\begin{equation}
	\label{burnmack}
	\gathxy{
		\Z\ar@(ur,ul)_\id\ar@/_0.4cm/[d]_{1\mapsto x}\\
		\Z[x]/(x^2-2x).\ar@/_0.4cm/[u]_{\substack{1\mapsto 1,\\x\mapsto 2}}
	}
	\end{equation}
	\item For any abelian group $G$, there's a Mackey functor $\ang G$ for which all of the maps are
	$0$:
	\[\xymatrix{
		0\ar@/_0.4cm/[d]_0\ar@(ur,ul)_0\\
		G.\ar@/_0.4cm/[u]_0
	}\]
	\item Given a $C_2$-Mackey functor, we can forget to a $\Z[C_2]$-module by evaluating at $C_2/e$. This has
	both left and right adjoints: if $M$ is a $\Z[C_2]$-module, the left adjoint, denoted $L(M)$, is
	\[\xymatrix{
		M\ar@(ur,ul)_\tau\ar@/_0.4cm/[d]_\pi\\
		M/C_2.\ar@/_0.4cm/[u]_{\overline\tr}
	}\]
	where $\pi$ is projection, $\tau$ is the action of the nontrivial element of $C_2$, and $\overline\tr([x]) =
	\sum_{g\in G} g\cdot x$ for any $x$ in the coset $[x]$. The right adjoint, denoted $R(M)$, is
	\[\xymatrix{
		M\ar@(ur,ul)_\tau\ar@/_0.4cm/[d]_\tr\\
		M^{C_2}.\ar@/_0.4cm/[u]_\iota
	}\]
	where $\iota$ is inclusion and $\tr(x) = \sum_{g\in G} g\cdot x$.\footnote{These generalize \latin{mutatis
	mutandis} to $G = C_p$.}
	\item Regarding $\Z$ as a trivial $\Z[C_2]$-module, we obtain $L\coloneqq L(\Z)$ and $R\coloneqq R(\Z)$, which
	are explicitly
	\[\gathxy{
		\Z\ar@/_0.4cm/[d]_\id\ar@(ur,ul)_\id\\
		\Z\ar@/_0.4cm/[u]_{\cdot 2}
	}\qquad\text{and}\qquad
	\gathxy{
		\Z\ar@/_0.4cm/[d]_{\cdot 2}\ar@(ur,ul)_\id\\
		\Z,\ar@/_0.4cm/[u]_\id
	}
	\]
	respectively.
	\item If $\Z\sigma$ denotes $\Z$ as a $\Z[C_2]$-module with the sign action, we obtain $L_-\coloneqq
	L(\Z\sigma)$ and $R_-\coloneqq R(\Z\sigma)$, which are explicitly
	\[\gathxy{
		\Z\ar@/_0.4cm/[d]_\pi\ar@(ur,ul)_{-1}\\
		\Z/2\ar@/_0.4cm/[u]_0
	}\qquad\text{and}\qquad
	\gathxy{
		\Z\ar@/_0.4cm/[d]_0\ar@(ur,ul)_{-1}\\
		0,\ar@/_0.4cm/[u]_0
	}
	\]
	respectively.
\end{enumerate}
\end{beast}
\begin{ex}
Show that $L(\bl)$ and $R(\bl)$ are really left and right adjoints to the forgetful map
$\cat{Mac}\to\Mod_{\Z[C_2]}$.
\end{ex}
It's possible to write down lots of Mackey functors, and it's entertaining to do so. But the key examples arise as
$\pi_0 X$, where $X$ is a $G$-spectrum. You might ask if all Mackey functors arise this way --- and for once we
know the answer: we constructed Eilenberg-Mac Lane spectra for every Mackey functor, so the answer is
yes!\index{equivariant Eilenberg-Mac Lane spectrum}

The next question: do functors on the category of Mackey functors arise as $\pi_0$ of functors on $\Spc^G$? You can
also ask lots of questions from commutative algebra, e.g.\ how to think of rings and modules in this context, and
the answers are generally harder than in the purely algebraic case.

Anyways, using the bestiary, we can state the answer. It's complicated, but the dimension axiom is complicated in
equivariant cohomology too.\index{dimension axiom}
\begin{thm}
\label{Aval}
\[H_{C_2}^{p,q}(*; A_{C_2}) = \begin{cases}
	A, &p = q = 0\\
	R, &p+q = 0,\ p < 0,\ p\text{\rm{} even}\\
	R_-, &p+q = 0,\ p \le 1,\ p\text{\rm{} odd}\\
	L, &p+q = 0,\ p > 0,\ p\text{\rm{} even}\\
	L_-, &p+q = 0,\ p < 1,\ p\text{\rm{} odd}\\
	\ang\Z, &p+q\ne 0, p = 0\\
	\ang{\Z/2} & p+q > 0,\ p < 0,\ p\text{\rm{} even}\\
	\ang{\Z/2} & p+q < 0,\ p > 1,\ p\text{\rm{} even}\\
	0, &\text{\rm otherwise.}
\end{cases}\]
\end{thm}
\begin{rem}
The description of $H_{C_2}^*(*;A_{C_2})$ in~\cite[Thm.~2.1]{Lew88} uses a different grading, indexing by
$(p+q,p)$.
\end{rem}
So the good news is, the answer is determined by the fixed and total dimensions.\index{fixed dimension!of a
$C_2$-representation}\index{total dimension!of a $C_2$-representation} This also generalizes nicely to $C_p$ when
$p$ is odd. Though this looks kind of frightening, Lewis~\cite{Lew88} proved that if $X$ has even-dimensional
cells, then its cohomology is free over that of a point, which is good. Moreover, the $\RO(C_2)$-grading makes
these statements cleaner; things like this just aren't true for the $\Z$-graded theories.

The computation follows by analyzing the following cofiber sequence(s),\footnote{In the nonequivariant case, you
probably wouldn't dwell on this so explicitly, just like we mentioned Spanier-Whitehead
duality\index{Spanier-Whitehead duality} for a point to make it clearer what's going on.} and proceeds in a similar
way as for $\underline\Z$, just with a harder answer.\footnote{\TODO: some confusion as to which cofiber sequences
work out correctly.}
\begin{gather}
%\xymatrix{
%	(C_2)_+\ar[r] & S(\eta)_+\ar[r] & \Sigma(C_2)_+
%}\\
% \xymatrix{
% 	S(\eta)_+\ar[r] & D(\eta)_+\ar[r] & S^\eta
% }\\
\label{sigmacofib}
\xymatrix{
	(C_2)_+\cong S(\sigma)_+\ar[r] & D(\sigma)_+\ar[r] & S^\sigma
}
\end{gather}
Here $\sigma$ is the sign representation, and we'd like $\eta$ to be the nontrivial irreducible complex
$C_2$-representation, regarded as a two-dimensional real representation.\index{sign representation}

Recall that we defined a zoo of Mackey functors for $C_2$ in \cref{best_bestiary}: we will continue to use that
notation here. We stated the answer to the calculation of $\underline\wH_{C_2}^{p,q}(S^0;A_{C_2})$ (the underline
means calculating the Mackey functors) in \cref{Aval}, and it may help to see it as a plot indexed in $x = p$
(fixed dimension) and $y = p+q$ (total dimension), as in \cref{ROgdiag}.
\begin{figure}[h!]
\[\xymatrix@R=0.2cm@C=0.2cm{ % TODO wrong
	& & \vdots     && \vdots     && \vdots\\
	& & \ang{\Z/2} && \ang{\Z/2} && \ang\Z\\
	& & \ang{\Z/2} && \ang{\Z/2} && \ang\Z\\
	& & \ang{\Z/2} && \ang{\Z/2} && \ang\Z\\
	& & \ang{\Z/2} && \ang{\Z/2} && \ang\Z\\
	\dotsb & R^- & R & R^- & R & R^- & A & R^- & L & L^- & L & L^- & \dotsb\\
	& & & & & & \ang\Z && \ang{\Z/2} && \ang{\Z/2}\\
	& & & & & & \ang\Z && \ang{\Z/2} && \ang{\Z/2}\\
	& & & & & & \ang\Z && \ang{\Z/2} && \ang{\Z/2}\\
	& & & & & & \ang\Z && \ang{\Z/2} && \ang{\Z/2}\\
	& & & & & & \vdots && \vdots && \vdots
}\]
\caption{$\underline H_{C_2}^{p,q}(S^0;A_{C_2})$, indexed by $(p,p+q)$.}
\label{ROgdiag}
\end{figure}

The approach we'll use to prove \cref{Aval}, which is a general strategy for understanding $H_G^*(S^0)$, is to
calculate $H_0^*(S^V)$ and use the Künneth spectral sequence\index{Künneth spectral sequence} to combine
$S^{V\oplus W}$. However, this has only been done for cyclic groups and $D_3$; if you do it for pretty much any
other group, it would be novel (and publishable). If you like spectral sequences and homological algebra, you could
work on it --- but it might not be interesting. On the other hand, the engine that lets \cite{HHR} prove the
Kervaire invariant $1$ problem is a computation of $\RO((C_2)^n)$-graded cohomology of a point with coefficients in
constant Mackey functors, which leads to the gap theorem, using a method Doug Ravenel suggested in the 1970s, so
who knows what you could discover?\index{Kervaire invariant $1$ conjecture}

In our calculation, we'll exploit the cofiber sequence~\eqref{sigmacofib} heavily. The first example is the
calculation of $H_{C_2}^{p,q}((C_2)_+)$.
\begin{prop}
\[H_{C_2}^{p,q}((C_2)_+) = \begin{cases}
	A_{C_2}, &p +q = 0\\
	0, &\text{\rm otherwise.}
\end{cases}\]
\end{prop}
\begin{proof}[Proof sketch]
Let $M$ be an abelian group and $M^2$ denote the $\Z[C_2]$-module $M\oplus M$, where the nontrivial element of
$C_2$ acts by switching the factors.\footnote{\TODO: I didn't follow this proof at all.}
\begin{ex}
Show that $L(M^2) = R(M^2)$.
\end{ex}
We'll let $M_{C_2}$ denote $L(M^2)$. If $M$ is a Mackey functor, the notation $M_{C_2}$ means $L(M(e)^2)$.

Observe that $\underline H_{C_2}^{p,q}((C_2)_+)\cong (\underline H_{C_2}^{p,q}(S^0)(e))_{C_2}$.
\end{proof}
Using the long exact sequence associated to~\eqref{sigmacofib}, the map
\[H_{C_2}^{p,q-1}(S^0)\cong H_{C_2}^{p,q}(S^\sigma)\longrightarrow H_{C_2}^{p,q}(S^0)\]
is injective if $p+q\ne 1$ and surjective when $p+q\ne 0$, hence an isomorphism when $p+q\ne 0,1$.

Therefore, we conclude that there are $0$s in the upper right and lower left quadrants (indexing by $(p,p+q)$ as in
\cref{ROgdiag}): when $H_{C_2}^{p,q-1}(S^0)\cong H_{C_2}^{p,q}(S^0)$, we can pull the $\sigma$ factors out of the
representation and use the dimension axiom.
\begin{prop}
When $p+q = 1$, we have
\[\underline H_{C_2}^{p,q}(S^0)\cong \coker\paren{\underline H_{C_2}^{p,q-1}((C_2)_+)\longrightarrow \underline
H_{C_2}^{p,q-1}(S^0)}\]
and when $p+q = -1$,
\[\underline H_{C_2}^{p,q}(S^0)\cong \ker\paren{\underline H_{C_2}^{p,q+1}(S^0)\longrightarrow \underline
H_{C_2}^{p,q+1}((C_2)_+)}.\]
We also have
\[(\underline H_{C_2}^{p,q}(S^0))(e) = 0.\]
\end{prop}
These follow from the long exact sequences for~\eqref{sigmacofib} in cohomology and homology, using the dualities
\[\underline H_{C_2}^{p,q}(S^0)\cong \underline H_{-p,-q}^{C_2}(S^0)\qquad\qquad\text{and}\qquad\qquad\underline
H_{C_2}^{p,q}((C_2)_+)\cong \underline H_{-p,-q}^{C_2}((C_2)_+).\]
\begin{lem}
$\underline H_{C_2}^{1,-1}(S^0)\cong R^-$.
\end{lem}
\begin{proof}
Once again we use the long exact sequence
\[\xymatrix{
	H_{C_2}^{0,0}(S^0)\ar[r] & H_{C_2}^{0,0}((C_2)_+)\ar[r] & H_{C_2}^{1,0}(S^\sigma)\ar[r] &
	H_{C_2}^{1,0}(S^0).
}\]
Since $H_{C_2}^{0,0}(S^0) = A$, $H_{C_2}^{0,0}((C_2)_+)= A_{C_2}$, $H_{C_2}^{1,0}(S^\sigma) \cong
H_{C_2}^{1,-1}(S^0)$, and $H_{C_2}^{1,0}(S^0) = 0$, then this long exact sequence is actually
\[\xymatrix{
	A\ar[r] & A_{C_2}\ar[r] & H_{C_2}^{1,-1}(S^0)\ar[r] & 0,
}\]
from which the result follows.
\end{proof}
Thus, the only piece left is the case when $p+q = 0$.
\begin{ex}
Compute $\underline H_{C_2}^{p,q}(S^0)$ when $p+q = 0$. This will again involve a manipulation of the long exact
sequence.
\end{ex}
\begin{ex}
What happens for other primes? Though it's more difficult, it isn't that different from the case where $p = 2$, and
$H^\alpha$ still only depends on $\abs\alpha$ and $\abs{\alpha^{C_p}}$. The key will be to use two cofiber
sequences associated to an irreducible $C_p$-representation $V$:
\begin{subequations}
\begin{gather}
	\xymatrix{
		(C_p)_+\ar[r] & S^0\ar[r] & S^V
	}\\
	\xymatrix{
		(C_p)_+\ar[r] & S(V)\ar[r] & \Sigma(C_p)_+.
	}
\end{gather}
\end{subequations}
\end{ex}
If you like computations, working through this will be enlightening.
\begin{rem}
This $\RO(G)$-graded structure is a lot of extra data and pain, but there are good reasons for it: the nice
symmetry present in \cref{ROgdiag} collapses into confusing data in the $\Z$-graded theory, and freeness results
that ought to be true require the $\RO(G)$-grading.
\end{rem}
There are a few other cases for which the $\RO(G)$-graded cohomology of a point has been calculated: Ferland and
Lewis~\cite{FerlandLewis} worked out a formula for an arbitrary Mackey functor over $C_p$, and explicate it for the
cases $H_{C_p}^{*,*}(*; \ang G)$, $H_{C_p}^{*,*}(*; L)$, and $H_{C_p}^{*,*}(*; R)$.
