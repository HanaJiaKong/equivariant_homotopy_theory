%!TEX root = m392c_EHT_notes.tex
Today, we're going to finish the proof of tom Dieck splitting, at least on the level of homotopy
groups~\eqref{what_well_prove}. We'll defer the spectrum-level theorem~\eqref{splevelTD} to later.

We've reduced to showing that the composite
\[\xymatrix{
	\pi_*^\WH(\sus E\WH_+\wedge X^H)\ar[r]\ar[dr]^-{\theta_1} & \pi_*^\NH(\sus E\WH_+\wedge X^H)\ar[d]\\
	& \pi_*^\NH(\sus
	E\WH_+\wedge X)\ar[r]^-\cong & \pi_*^G(G\wedge_H\sus E\WH_+\wedge X)\ar[r]^-{\theta_2} &\pi_*^G(X)
}\]
is an isomorphism when $X$ is concentrated at the single conjugacy class $(H)$ for $G$. The composite of the first
two maps is denoted $\theta_1$. We did this by showing that the two sides of the tom Dieck isomorphism are
$\Z$-graded homology theories, and therefore are determined by their values on spaces concentrated at single
conjugacy classes. We'll show separately that $\theta_1$ and $\theta_2$ are isomorphisms.

First let's look at $\theta_2$, by looking at the spaces of the susmension spectrum for
\[\paren{G\wedge_\NH\paren{E\WH_+\wedge X\wedge S^V}}^K.\]
Fixed points do not commute with suspension on the spectrum level (after all, this is the content of tom Dieck
splitting), but they do commute with the smash product for spaces, so this is also
\[(G\wedge_H E\WH_+)^K\wedge X^K\wedge S^{V^K}.\]
If $K\not\in (H)$, then $X^K\simeq *$, since $X$ is concentrated at $(H)$. If $K\in (H)$, then $(E\WH_+)^K\simeq
*$, so either way what we obtain is an isomorphism for $\theta_2^K$.

For $\theta_1$, we needed to prove \cref{restrlem}, that if $H\trianglelefteq G$ and $Y$ is concentrated at $(H)$,
then the restriction map $\Map_G(X,Y)\to\Map_{G/H}(X^H,Y^H)$ is an isomorphism.
\begin{proof}[Proof of \cref{restrlem}]
The proof will proceed by cellular induction. Consider attaching a cell to a $G$-space $Z$ to form a space
$\tilde Z$, which is expressed by the pushout
\[\xymatrix{
	G/K_+\wedge S^{n-1}\ar[r]\ar[d] & Z\ar[d]\\
	G/K_+\wedge D^n\ar[r] & \pushout \tilde Z.
}\]
If we apply $\Map_G(\bl,Y)$ to this diagram, it becomes a pullback
\[\xymatrix{
	\Map_G(G/K_+\wedge S^{n-1}, Y) & \Map_G(Z,Y)\ar[l]\\
	\Map_G(G/K_+\wedge D^n, Y)\ar[u]_{\psi_1} & \Map_G(\tilde Z,Y)\ar[l]\ar[u]_{\psi_2}.
}\]
Since $\Map_G(G/K_+\wedge S^{n-1}, Y)\simeq \Map(S^{n-1}, Y^K)$, then $\psi_1$ is trivial, so $\psi_2$ is an
isomorphism.

Let
\[X_0\coloneqq\set{x\in X\mid G_x\not\subseteq H}\subseteq X.\]
Then, $X\setminus (X_0\cup X^H)$ is built by attaching cells $G/K_+\wedge D^n$ with $K\not\in (H)$.

The cell attachment computation we just made tells us that it suffices to understand $X_0\cup X^H$, for which we
use a Mayer-Vietoris argument associated to the pullback
\[\xymatrix{
	X_0\cup X^H & X^H\ar[l]_{\phi_1}\\
	X_0\ar[u] & X_0\cap X^H\ar[l]_{\phi_2}\ar[u].
}\]
We again apply $\Map_G(\bl,Y)$; this time, the same argument shows that $\phi_2^*$ is an isomorphism. Now the map
we want factors as
\[\xymatrix{
	\Map_G(X,Y)\ar[r] & \Map_G(X_0\cup X, Y)\ar[r] & \Map_G(X^H,Y)\ar[r] & \Map_{G/H}(X^H,Y^H).
}\]
The last map comes from the adjunction, and in particular is a known weak equivalence. And by induction, the first
two maps are weak equivalences.
\end{proof}
We'll use \cref{restrlem} to check that the map
\[\pi_*^\WH(\sus E\WH_+\wedge X^H)\stackrel{\theta_1}{\longrightarrow} \pi_*^\NH(\sus E\WH_+\wedge X)\]
is an isomorphism, by providing a map in the other direction. Specifically, this is a map of homotopy classes
\[[S^{n+m\rho}, \Sigma^{m\rho} E\WH_+\wedge X]\stackrel{\theta_1}{\longrightarrow} [S^{n+m\rho}, \Sigma^{m\rho}
E\WH_+\wedge X].\]
We've abused notation a bit here: on the left, $\rho$ is the regular representation for $\NH$, and on the right,
it's the regular representation for $\WH$. \Cref{restrlem} implies that the map going the other way is an
isomorphism when $X$ is concentrated at $(H)$:
\[[S^{n+m\rho}, S^{m\rho}\wedge E\WH_+\wedge X]\stackrel\cong\longrightarrow [S^{n + m\rho^H}, S^{m\rho^H}\wedge
E\WH_+\wedge X^H].\]
This is what we reduced the problem to, so we have completed the proof of tom Dieck splitting for $\pi_G^*(\sus X)$
on the level of homotopy groups.
\subsection*{The Burnside category.}
Let $\fC$ be a small category with finite limits and finite coproducts. In $\fC$, consider the diagram
\[\xymatrix{
	X\ar[r]^{\vp_1}\ar[d] & Y\ar[d] & Z\ar[d]\ar[l]_{\vp_2}\\
	X'\ar[r] & X'\amalg Z' & Z'.\ar[l]
}\]
Both of these squares are pullbacks iff $Y\cong X\amalg Z$ with $\vp_1$ and $\vp_2$ the universal maps. This tells
us something about how coproducts interact with pullbacks.
\begin{defn}
Let $\Span(\fC)$ be the category whose objects are those of $\fC$ and whose morphisms $\Hom_{\Span(\fC)}(X,Y)$ are
equivalences classes of diagrams
\begin{equation}
\label{span1mor}
\begin{gathered}
\xymatrix@dr{
	A\ar[r]\ar[d] & Y,\\
	Z
}
\end{gathered}
\end{equation}
where $X\gets A\to Y$ and $X\gets A'\to Y$ are equivalent if there's an isomorphism $h\colon A\congto A'$ such that
the following diagram commutes.
\begin{equation}
\label{span2mor}
\begin{gathered}
\xymatrix@R=0.4cm{
	& A\ar[dr]\ar[dl]\ar[dd]^h\\
	X && Y\\
	& A'.\ar[ul]\ar[ur]
}
\end{gathered}
\end{equation}
Composition is defined by pullback of spans
\[\xymatrix@dr{
	A\times_Y A'\ar[r]\ar[d] & A'\ar[r]\ar[d] & Z.\\
	A\ar[r]\ar[d] & Y\\
	X
}\]
\end{defn}
\begin{rem} % TODO: citation for category number?
If your category number is at least $2$, you can construct $\Span(\fC)$ as a $2$-category, where the $1$-morphisms
between $X$ and $Y$ are spans~\eqref{span1mor}, and the $2$-morphisms between $X\gets A\to Y$ and $X\gets A'\to Y$
are the diagrams~\eqref{span2mor}.
\end{rem}
There's a sum on $\Hom_{\Span(\fC)}(X,Y)$ given by the coproduct of $X\gets A\to Y$ and $X\gets A'\to Y$, which
define a map $X\gets A\amalg A'\to Y$. You might worry whether this is compatible with composition, which is why
we took the morphisms to be equivalence classes of spans.

This sum does not make $\Span(\fC)$ into an additive category, so let $\Span^+(\fC)$ denote the \term{preadditive
completion} of $\Span(\fC)$, i.e.\ $\Hom_{\Span^+(\fC)}(X,Y)$ is the Grothendieck group of
$\Hom_{\Span(\fC)}(X,Y)$.
\begin{ex}
Show that the category $G\Set^{\mathrm{fin}}$ of finite $G$-sets and $G$-equivariant maps satisfy the axioms we
asked of $\fC$.
\end{ex}
\begin{defn}
The \term{Burnside category} $B_G$ is $\Span^+(G\Set^{\mathrm{fin}})$.
\end{defn}
In \cref{burn1}, we defined the Burnside category differently, as the full subcategory of $\Spc^G$ on $\sus G/H_+$;
these two definitions are equivalent, and the content is that every finite $G$-set is a coproduct of orbits
$G/K_i$.
\begin{prop}
Let $\tilde B_G$ denote the full subcategory of $\Spc^G$ spanned by finite $G$-sets. Then, $B_G$ and $\pi_0\tilde
B_G$ are equivalent.
\end{prop}
\begin{proof}
The functor in question sends $T\mapsto\sus T_+$ and a span
\[\xymatrix@dr{
	Z\ar[r]^\theta\ar[d]_\tau & Y\\
	X
}\]
to $\theta\circ\tr(\tau)$, where $\tr$ is the transfer map. Because the transfer behaves well under composition,
this is a functor $\Span(G\Set^{\mathrm{fin}})\to\pi_0\tilde B_G$, and therefore uniquely determines the functor
$B_G\to\pi_0\tilde B_G$. This is essentially surjective (i.e.\ for objects), and we've already computed that it's
an isomorphism on hom sets.\footnote{\TODO: I think this was tom Dieck splitting, but want to make sure.}
\end{proof}
\begin{defn}
A \term{Mackey functor} is a functor $B_G\op\to\Ab$.
\end{defn}
Restricting to spans of the form $X\stackrel\id\gets X\to Y$ defines a covariant functor
$\sO_G\to B_G$, and restricting to spans of the form $Y\gets X\stackrel\id\to X$ defines a contravariant functor
$\sO_G\op\to B_G$. Composing with a Mackey functor $F$ defines a pair of functors $\sO_G\op\to\Ab$ and
$\sO_G\to\Ab$. We'll denote the image of an $f\colon X\to Y$ under these functors as $f^*$ and $f_*$, and given a
diagram
\[\xymatrix{
	X'\ar[r]^{f'}\ar[d]^{g'} & X\ar[d]^g\\
	Y'\ar[r]^f & Y,
}\]
Mackey functors satisfy the \term{Chevalley condition} $g^*f_* = f_*'(g')^*$.
\begin{prop}
$\pi+K^H(\bl)$ defines a Mackey functor.
\end{prop}
The two functors $\sO_G,\sO_G\op\to\Ab$ are evident, coming from the usual maps in the orbit category and the
transfer maps, but there are axioms to check.

Let $M$ be a Mackey functor. Then, there exists an object $\mathit{HM}\in\Spc^G$, called the \term{equivariant
Eilenberg-Mac Lane spectrum} for $M$, whose homotopy groups $\pi_G^*$ are determined by $M$.
\begin{defn}
The \term{cohomology} of an $X\in\Spc^G$ with coefficients in a Mackey functor $M$ is
\[H^n(X;M)\coloneqq [X, \Sigma^n \mathit{HM}],\]
which defines a $\Z$-graded cohomology theory. Moreover, one can define an $\RO(G)$-graded cohomology theory
\[H^V(X;M)\coloneqq [X, \Sigma^V \mathit{HM}],\]
where $V$ is a representation in $U$.
\end{defn}
The definition of an $\RO(G)$-graded cohomology theory requires some care, and we will investigate this; the naïve
definition (which is worth writing down) you might write down is likely to be wrong for subtle reasons.

The central algebraic analogue of the equivariant stable category is given by the category of Mackey functors.
Mackey functors form a symmetric monoidal category under something called the box product; we'll define this in
detail later, but it's familiar, arising as a Kan extension just like Day convolution.

The point is, it makes sense to talk about commutative monoids in the category of Mackey functors, which are called
\term{Green functors}. These aren't quite the right objects, however: we need to account for stable transfer data,
which is what makes the equivariant stable category act the way it does.

The more sophisticated notion of a commutative ring in Mackey functors involves additional multiplicative transfer
data; these are called \term{Tambara functors} or \term{TNR functors}.

Up to now, everything we've done was known to tom Dieck and can be found (perhaps not in this language)
in~\cite{LMS}. But this stuff is more modern: this perspective on Tambara functors was popularized by~\cite{HHR}.
Multiplicative transfers were originally considered by \citeme{Evens} in representation theory, and
\citeme{Greenlees-May} also wrote about them, but~\cite{HHR} innovated the norm and how it interacts with these
multiplicative transfer maps.

This leads to an interesting question about calculations, which appears as soon as you try to start doing things.
\begin{ques}
On the homotopy groups $\pi_*"\bl$, we have restrictions and transfer maps. What can we say about the composite of
a restriction map and a transfer map, or a transfer map and a restriction map?
\end{ques}
The answer involves double coset formulae, and can be thought of in a few ways: we'll start with the classical
perspective and become more modern.

Next time, we'll talk more about Mackey functors and $\RO(G)$-graded cohomology theories, including how to
construct Eilenberg-Mac Lane spectra, which is considerably more difficult than in the nonequivariant case.
