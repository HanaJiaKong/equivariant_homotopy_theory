Today we're going to talk about some applications of the norm map, including why $\pi_0$ of an $E_\infty$-ring
spectrum is a Tambara functor and how the norm construction interacts with the resolution of the Kervaire invariant
$1$ problem. In the next two classes we'll talk about the Tate construction and the Tate spectral sequence; the
last class is more open, and we can talk about current directions in research or anything else.

Over the summer, there will be more work on the notes: if you did any exercises while working this material out,
feel free to include them, even if they're very rough. If anything is unclear or not useful or you have any other
comments, feel free to leave them in the notes.

Today, $R$ will be a commutative ring in orthogonal $G$-spectra, i.e.\ an object of $\Spc^G[\P]$. The forgetful
functor from $G$-ring spectra to $H$-ring spectra is lax symmetric monoidal and has a left adjoint, which is the
HHR norm; it was known that this left adjoint existed for a long time, and that it wasn't the same adjoint as for
$G$-spectra without a ring structure, but it wasn't understood how to think of it until recently.

One consequence is the existence of the counit $\eta\colon N_H^Gi_H^*R\to R$. If $X\in\Spc^H$, then $R_H^0(X) = [X,
i_H^*R]_H$, so the norm provides us with a map $R_H^0(X)\to R_G^0(N_G^HX)$: given a map $f\colon X\to i_H^*R$,
norm it and apply the counit: $\e\circ N_H^G(f)\colon N_H^GX\to N_H^G i_H^*R\to R$. This and a few similar
constructions are reminiscent of the Evens norm: if $V$ is an $H$-representation and $X = S^V$, this is a map
$R_H^0(S^V)\to R_G^0(N_H^GS^V) = R_G^0(S^{\Ind_H^GV})$, and these are homotopy groups: $R_H^0(S^V) = [S^V, i_H^*R]
= \pi_V(i_H^*R)$, so we obtain a map $\pi_V(i_H^*R)\to\pi_{\Ind_H^GV}(R)$.

For example, if $A$ is a $G$-space, given a map $f\colon\sus A\to i_H^*R$, we can construct a map $R_H^0(i_H^*A)\to
R_G^0(A)$ sending $\sus A\to\sus N_H^G A$, which is homeomorphic to $N_H^G\sus A$ (here this is the norm map on
spaces); then, using $f$, map to $N_H^Gi_H^*R$ then to $R$ by the counit.
\begin{ex}
This is nearly saying that $\pi_0R = R_{(\bl)}^0$ is a Tambara functor; check that this is really the case.
\end{ex}
There's lots of things to check: figuring out why the double coset formula holds is work, though not too
conceptually hard if you've followed everything so far. The original papers are complicated, because this check is
inevitably notation-heavy.

On the other hand, there's a converse:
\begin{thm}[Ullman~\cite{Ullman}]
All Tambara functors arise as $\pi_0R$ for some ring spectrum $R$, e.g.\ an Eilenberg-Mac Lane spectrum. However,
there is no Eilenberg-Mac Lane functor that is symmetric monoidal.
\end{thm}
\begin{rem}
There's a cool connection between the structures of Tambara functors and Witt vectors. For example,
\citeme{Hesselholt-Madsen} identify $\pi_0\THH$ and $\pi_0\THH^{C_{p^n}}$ with certain rings of Witt vectors. This
fits into a bigger picture.
\end{rem}
$N_\infty$-operads come back. We used them to specify which transfer maps you have, and here they will control
which norm maps you get. There are some complicated questions here: distributivity of norm and transfer maps is
intricate, and the interactions between two different $N_\infty$-operads, neither of which is complete, is not
written down.

The rough idea is that any commutative $G$-ring spectrum should be charactrized by having certain norm maps
$N_H^Gi_H^*X\to X$.
\begin{lem}
Let $T$ be an admissible $H$-set for an $N_\infty$-operad $O$. Then,
\[(G\times\Sigma_{\abs T}/\Gamma_T)_+\wedge_{\Sigma_{\abs T}} X^{\abs T} \cong G_+\wedge_H \bigwedge_i
N_{K_i}^Hi_{K_i}^*X.\]
\end{lem}
The idea is that we can write $T$ as a coproduct of orbits $H/K_i$.

Using this lemma, since $T$ is admissible, we get a map $G\times\Sigma_{\abs T}/\Gamma_{\abs T}\to O(\abs T)$, and
therefore if $X$ is an $O$-algebra (i.e.\ is in $\Spc^G[O]$), this becomes a map
\[\xymatrix{
	G_+\wedge_H\bigwedge N_{K_i}^H i_{K_i}^*X\ar[r] & O(\abs T)\wedge_{\Sigma_{\abs T}} X^{\abs T}\ar[r] & X,
}\]
so for $X$ an $N_\infty$-algebra over $O$, $N_H^G: \Spc^H[O]\rightleftarrows \Spc^G[O]: i_H^*$ is homotopical
whenever $G/H$ is admissible. In this case, $\pi_0X$ is an incomplete Tambara functor: we asked for it to contain
the norm maps for $O$, and we have to accept certain other ones.

If instead $A$ is an $N_\infty$-algebra in $G$-spaces, then $\Sigma_+^\infty A$ is an $N_\infty$-algebra in
$\Spc^G$, and is in fact the spectral group ring (also denoted $\Sph[A]$). The multiplication comes from the
isomorphism
\[\O(n)\times_{\Sigma_n} (\sus_+A)^n\cong \sus_+(\O(n)\times_{\Sigma_n} A^n).\]
\begin{rem}
The nonequivariant analogue of this is called the \term{spherical group ring} of a space $M$, $\Sph[\Omega
M]\coloneqq \Sigma_+^\infty \Omega M$. It's worth thinking about the similarities between these two cases.
\end{rem}
What ends up happening is that $\sus_+$ takes transfers to norms, which is a good slogan to remember.

We can also look at what happens to the units of a commutative ring spectrum. Nonequivariantly, let $R$ be a
commutative ring spectrum. Its \term{group of units} was initially defined as the space $\GL_1R$ that fits into the
pullback diagram
\begin{equation}
\label{GL1}
\gathxy{
	\GL_1R\pullback\ar[r]\ar[d] & \Omega^\infty R\ar[d]\\
	(\pi_0\Omega^\infty R)^\times\ar[r] & \pi_0\Omega^\infty R.
}\end{equation}
This definition makes sense for any $A_\infty$-ring spectrum, but behaves better when $R$ is commutative.
\begin{comp}{fct}{enumerate}
	\item If $R$ is an $E_\infty$-ring spectrum, then $\GL_1R$ is an $E_\infty$-space, and therefore extends to a
	spectrum called $\mathrm{gl}_1R$.
	\item $\GL_1R$ is (homotopically) the right adjoint to $\sus_+$.\qedhere
\end{comp}
This relates to the theory of orientations, and in fact this was originally considered for this purpose by Dennis
Sullivan, and expanded on by~\cite{MQR}. The modern approach was written down in~\cite{ABGHR2, ABGHR1}.
\citeme{May} has three articles from the 2009 Banff conference about these things that are particularly clear.

Another characterization of $\GL_1R$ is as the homotopy automorphisms of $R$ in the category of $R$-modules:
$\GL_1R = \LD\Aut_{\Mod_R}(R)$.

Anyways, there's an analogous construction in the equivariant setting~\citeme{???}, where you have to work
out~\eqref{GL1} for all fixed points simultaneously. But this works out, and you can show that $\GL_1$ sends norms
to transfers, which makes sense.
\subsection*{An outline of the Kervaire invariant 1 proof.}
Of course, this is the proof in~\cite{HHR}. \citeme{Kervaire} connected his invariant to the stable homotopy groups
of spheres, and therefore was able to show that every $10$-dimensional smooth manifold has Kervaire invariant zero,
then produced a PL manifold with Kervaire manifold $1$ (hence a nonsmoothable manifold).

Browder managed to connect this to elements $\theta_j\in\pi_{2^{j+1}-2}^s(\Sph)$, so that a manifold of Kervaire
invariant one exists in a certain dimension iff there's a differential in the Adams spectral sequence showing that
$\theta_j\ne 0$.

The proof idea was concocted by Ravenel in the 1970s, but the tools to implement it weren't available until very
recently. The idea is to construct a designer cohomology theory $\Omega$ meeting the following criteria.
\begin{enumerate}
	\item $\Omega$ is multiplicative, therefore admitting a map $\Sph\to\Omega$ such that the image of $\theta_j$
	is nonzero.
	\item The homotopy groups $\pi_i\Omega$ are zero in degrees $0 < i < 4$.
	\item $\pi_*\Omega\cong\pi_{*+256}\Omega$.
\end{enumerate}
These suffice: the idea is that for $j\ge 7$, $\theta_j$ is detected by $\Omega$, but lands in degree between $0$
and $4\bmod 256$, so must be zero.
\begin{rem}
For a while, people suspected there were infinitely many dimensions in which manifolds with Kervaire invariant $1$
existed. This simplifies a lot of Adams spectral sequence computations. See~\citeme{Mahowald} for a ``doomsday
conjecture'' about this.
\end{rem}
Then, Hill, Hopkins, and Ravenel constructed $\Omega$ as $N_{C_2}^{C_8}\MR[\kappa^{-1}]$: take real bordism, norm
it up from $C_2$ to $C_8$, and then invert something (an analogue of the Bott element), and they showed it meets
the desired criteria. \citeme{Schwede's notes}
\begin{defn}
Consider the space
\[\MU_n\coloneqq E\U(n)_+\wedge_{\U(n)} S^{\C^n},\]
where $\U(n)$ is the group of unitary matrices. Then, $\MU_n$ is a $C_2$-space, where the action is by complex
conjugation.\footnote{Here, $E\U(n) = B(\U(n), \U(n), *)$ is a contractible space with a free $\U(n)$-action, and
the quotient is $B\U(n) = B(*, \U(n), *)$.}

We'll use these spaces to define a $C_2$-ring spectrum. First, the decomposition $\C^{n+m}\cong\C^n\oplus\C^m$
induces a multiplication map $\MU_n\wedge\MU_m\longrightarrow \MU_{n+m}$, and the map $\U(n)\to B(\U(n), \U(n), *)$
defines a map $S^{\C^n}\to E\U(n)\times_{\U(n)} S^{\C^n}$; together, these define $C_2$-equivariant structure maps
$S^{\C^n}\wedge\MU_m\to \MU_{n+m}$.

This is very close to the definition of an orthogonal ring spectrum, but indexed on complex representations rather
than real ones. One of the lessons of this class is that the diagram you index spectra by doesn't matter all that
much, and there's a way to turn these \term{unitary spectra} into orthogonal ones.

Namely, using the decomposition $\C = \R\oplus i\R$, we can define
\[\MR_n\coloneqq \Map(i\R^n, \MU_n),\]
where $\O_n$ and $C_2$ both act by conjugation. The structure maps for $\MU_n$ factor as
\[S^n\wedge S^{i\R^n}\wedge\MU_k\longrightarrow \MU_{n+k},\]
where $C_2$ acts on $\R^n$ by the sign representation and on $\MU_N$ by complex conjugation.

Now, we get a $C_2$-object in $\Spc$:
\[\MR_V \coloneqq \Map(S^{iV}, E\U(V_\C)_+\wedge_{\U(V(\C))}S^{V_\C}),\]
where $V_\C\coloneqq V\otimes_\R\C$.
\end{defn}
Real bordism relates to usual cobordism spectra in nice ways: the geometric fixed points
$\Phi^{C_2}\MR\cong\mathit{MO}$, and $i_{C_2}^*\MR = \MU$.

From here, one uses spectral sequences to compute that $\Omega$ satisfies the various criteria, including the
homotopy fixed points spectral sequence and the slice spectral sequence.
