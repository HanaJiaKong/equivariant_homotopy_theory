\begin{quote}\textit{
	``The large print giveth and the small print taketh away.''
}\end{quote}
The big goal of today's lecture is to construct Eilenberg-Mac Lane spectra: if $\underline M$ is a Mackey functor,
we'll produce a $G$-spectrum $H\underline M$ which represents ordinary cohomology with coefficients in $\underline
M$. This will take some work, but will have some nice payoffs. We'll begin some computations: for $G = C_2$, we'll
compute the Bredon cohomology $H_{C_2}(\pt;\underline\Z)$ and $H_{C_2}(\pt;A)$, where $\underline\Z$ is the
constant $\Z$-valued Mackey functor and $A$ is the Burnside Mackey functor (\cref{Mackeyexm}).

Last time, we proved Brown representability (\cref{brown_rep}) for triangulated categories, which are stable. We
also want an unstable variant.
\begin{thm}[Unstable Brown representability]
Let $\fC$ denote the category of based, $G$-connected $G$-CW complexes and $H\colon \Ho(\fC)\op\to\Set$ be a
functor such that
\begin{enumerate}
	\item $H$ satisfies the wedge axioms, and
	\item $H$ takes homotopy pushpouts to (weak) pullbacks.
\end{enumerate}
Then, $H$ is represented, and the converse is true.
\end{thm}
Though we stated this theorem for $H$ going to $\Set$, it turns out all such functors factor through the forgetful
functor $\Ab\to\Set$, and therefore can be taken to be $\Ab$-valued.
\begin{warn}
The basepoint and connectivity hypotheses are important: Brown representability does not hold in the unstable case
if you fail to impose those requirements.
\end{warn}
We can use Brown representability to prove the following result.
\begin{prop}
Any $\RO(G)$-graded cohomology theory is represented by a $G$-spectrum.
\end{prop}
\begin{proof}
Let $E$ be an $\RO(G)$-graded cohomology theory. We can apply Brown representability to $E_G^V(\bl)$ for each $V$,
and the identification $E_G^V(\Sigma^W X)\cong E_G^{V\oplus W}(X)$ gives rise to a homotopy class of structure maps
on representing objects, but there's a coherence issue: we constructed $G$-spectra with actual structure maps, not
homotopy classes of maps.

The usual way to solve this is with a trick: restrict attention to a cofinal indexing sequence $V_n\inj V_{n+1}\inj
V_{n+2}\inj\dots$, and then observe that the homotopy theory of $G$-prespectra on this sequence is equivalent to
$\Ho(\Spc^G)$, which is guaranteed by cofinality: limits are preserved, and the homotopy groups are the same. This
fixes the coherence problem: we can pick a representative for each map and force transitivity. Then, we can left
Kan extend back to an orthogonal $G$-spectrum.
\end{proof}
This would be an exercise, except that maybe it's a bit too hard to be one.
\begin{ques}
Can we build an object of $\Spc^G$ directly? It would be nice not to worry about prespectra on a cofinal sequence.

There are different ways to think about doing this. One approach goes through the model of the $G$-stable category
as spectral presheaves.
\end{ques}
This provides the connection between cohomology theories on $G$-spaces and $G$-spectra, just like in the
nonequivariant world. They're almost the same, except for the existence of phantom maps,\footnote{A \term{phantom
map} is a nonzero morphism between spectra that's not detected by any compact object. In particular, this means
that as a natural transformation of cohomology theories, it's $0$, but arises from a nonzero morphism. These encode
the non-uniqueness of Brown representability.} just like in the nonequivariant case. These are poorly understood,
but of course we want to understand them better in the nonequivariant case.
\begin{cor}
\label{ZtoROG}
A $\Z$-graded cohomology theory on $\Spc^G$ is represented by an object of $\Spc^G$.
\end{cor}
The proof is that $\Sigma^V$ and $\Omega^V$ are inverse equivalences in $\Ho(\Spc^G)$, so we can use them to extend
to $\RO(G)$-graded theories. \Cref{ZtoROG} feels like a weird coincidence, but it's a key technical result that
we'll use more than once. The unstable case is different, though.
\begin{cor}
\label{unstZtoR}
A $\Z$-graded cohomology theory on $G\Top$ with coefficients in a coefficient system $M$ extends to an
$\RO(G)$-graded theory iff $M$ extends to (the contravariant part of) a Mackey functor.
\end{cor}
Now we can build Eilenberg-Mac Lane spectra for a Mackey functor $\underline M$. There will be two constructions:
one may be more satisfying. You might hope to use obstruction theory to build Eilenberg-Mac Lane spaces and chain
them together, but this is a mess. We'll adopt one unfamiliar-looking approach, which always works, and also see a
more familiar one.
\begin{cons}
Given $\underline M$, we'll construct a $\Z$-graded cohomology theory on $\Spc^G$; by \cref{unstZtoR}, this is
represented by an object of $\Spc^G$, which will be $H\underline M$.

Let $X$ be a $G$-CW complex, and define $\underline C_n(X)\coloneqq \underline\pi_n(X_n/X_{n-1})$ (where $X_n$ is
the $n$-skeleton of $X$). Now we'll replicate the construction of Bredon cohomology; in particular, the long exact
sequence for the triple $(X_n, X_{n-1}, X_{n-2})$ induces a differential $\underline C_n(X)\to\underline
C_{n-1}(X)$. If $\cat{Mac}$ denotes the category of Mackey functors, consider the cochain complex
\[\Hom_{\cat{Mac}}(\underline C_n(X), \underline M),\]
and take the cohomology of this cochain complex.

This is represented by an \term{Eilenberg-Mac Lane spectrum} $H\underline M\in\Spc^G$, and one can calculate that
\[\pi_k^H(H\underline M) = \begin{cases}
	0, &k\ne 0\\
	\underline M(G/H), &k = 0.
\end{cases}\]
From this perspective, it's easy to show that $H$ is functorial, which is nice. But you might wonder if it's
symmetric monoidal: in the nonequivariant case, $H(R\otimes S)\simeq HR\wedge HS$. This is false, and is one of
the unfortunate things about the equivariant setting.
\end{cons}
Though the cohomology theory $H\underline M$ represents is explicit, the spectrum itself is not. Brown's original
proof of Brown representability \cite{Brown} looks a lot like Postnikov induction, which closely resembles one
classical construction of nonequivariant Eilenberg-Mac Lane spaces, which suggests a more explicit approach is
possible.
\begin{cons}
One can build some Eilenberg-Mac Lane spectra via the equivariant Dold-Thom construction; this approach only works
in certain settings (we'll say which ones in a bit), but is interesting enough to mention.

First let's discuss the nonequivariant case. Let $\Sp^\infty\colon\Top_*\to\cat{TopCMod}$ be the free functor
from based spaces to topological abelian monoids (i.e.\ adjoint to the forgetful functor). The proof of Dold-Thom
amounts to check that $\Sp^\infty$ is a fibrant $\sW$-space, which means checking that it takes homotopy pushouts
to homotopy pullbacks. We can identify this with $H\Z$.

We'll now introduce the \term{McCord construction}: McCord's original paper is a good reference. Nick Kuhn has a
talent for clarifying constructions and finding cool uses for them, and his papers on the McCord construction
(which should have ``McCord'' in the title) are also great.

Let $A$ be an abelian group and $X$ be a based space. Let $A\otimes X$ denote the free abelian topological monoid
on pairs $(a,x)$, written $x^a$, subject to the relations
\begin{enumerate}
	\item $*^a = *$,
	\item $x^0 = *$, and
	\item $x^ax^b = x^{a+b}$.
\end{enumerate}
That is, $Y$ inherits the quotient topology from the free abelian topological monoid.
\begin{rem}
The McCord construction can be carried out very generally, e.g.\ Kuhn uses a closely related model to analyze the
categorical tensor product in ring spectra.
\end{rem}
One can prove a version of the Dold-Thom theorem that implies that $\set{A\otimes\bl}$ is a model for $HA$. This is
a $\sW$-space, and we can think of this as $\set{A\otimes S^n}$. There's also a $\Gamma$-space $\underline n\mapsto
\underline n\otimes A$, and one can check this is fibrant, i.e.\ very special.

In our case, suppose $M$ is a $\Z[G]$-module, and that $G$ is finite. We can extract a Mackey functor $\underline
M(G/H)\coloneqq M^H$, where the transfer $\Tr_H^K$ sends
\[x\mapsto \sum_{hH\in K/H} hx.\]
You can check this is a Mackey functor. The equivariant McCord construction produces Eilenberg-Mac Lane spectra for
the Mackey functors that arise in this way \citeme{Shimakawa, Dos Santos, Nies}.

If $A$ is a based $G$-space and $M$ is a $\Z[G]$-module, then $A\otimes M$ is a $G$-space using the diagonal
$G$-action. What you obtain is an equivariant $\Gamma$-space. We haven't talked much about these, and we won't, but
they're interesting. Segal's preprint on equivariant $\Gamma$-spaces \citeme{Segal} is highly influential and
widely circulated, but the statement has a mistake (though the proof is correct).
\begin{ex}
Find the mistake, and fix it. At this point, you have the background to do so.
\end{ex}
The correction was written up by~\citeme{Shimakawa}, and a related question about equivariant $\sW$-spaces is
written up in Andrew Blumberg's thesis.
\begin{thm}
$\set{S^V\otimes M}$ is a model of $H\underline M$.
\end{thm}
So this construction is nice and explicit, but only works when $G$ is finite, and for Mackey functors that arise
from $\Z[G]$-modules.
\end{cons}
Understanding the Dold-Thom construction is equivalent to understanding infinite loop space theory, and for general
compact Lie groups we understand neither.

There aren't many examples of calculations in Bredon cohomology, except for what people needed for other things:
Stong has some lecture notes which might be published, and the motivic homotopy theorists (specifically Duggar and
Voevodsky) used the $C_2$-equivariant category as a model for the motivic setting, and wrote up some calculations.

The takeaway is that, even for groups such as $C_4$, calculating the Bredon cohomology of a point would almost be a
paper. People probably know what it is, but it would be interesting to understand. The equivariant Steenrod and
Dyer-Lashof algebras are also not understood well; Hill-Hopkins-Ravenel figured some of it out, but might not have
written all of it down. This would be hard to attack via a Cartan-style seminar, unfortunately.

Thus, it's not at all trivial that we're computing $H_{C_2}^*(\pt;\underline\Z)$. Recall that $\underline\Z$ comes
from the coefficient system
\[\xymatrix{
	M(C_2/C_2)\\
	M(C_2/e),\ar[u]\ar@(dl,dr)
}\]
but as a Mackey functor there's also a transfer map $f_*$:\footnote{\TODO: fix this \Xy{} collision with the text.}
\[\xymatrix{
	M(C_2/C_2)\ar@/^0.4cm/[d]^{f_*}\\
	M(C_2/e).\ar@/^0.4cm/[u]^{f^*}\ar@(dl,dr)_\gamma
}\]
\begin{ex}
Show that the double coset formulae force $f_*f^* = 1+\gamma$ and $f^*f_* = 2$.\footnote{\TODO: I'd like to
double-check that I got this correct.}
\end{ex}
For $\underline\Z$,
\begin{itemize}
	\item The coefficients are $M(C_2/C_2) = \Z$ and $M(C_2/e) = \Z$.
	\item $f^* = \id$, $f_*$ is multiplication by $2$, and $\gamma = \id$.
\end{itemize}
We'll introduce a bigrading on $H^*_{C_2}$, since $C_2$ has two irreducible $1$-dimensional representations: the
triival representation on $\R$ and the sign representation $\sigma$. Thus, let $H_{C_2}^{p,q}$ denote $H_{C_2}^V$,
where $V$ contains $p$ copies of the trivial representation and $q$ copies of $\sigma$.
\begin{rem}
There's an alternative \term{motivic indexing} $H_{C_2}^{p+q,q}$, where $q$ counts the number of copies of $\sigma$
and $p+q$ counts the total underlying dimension. We're not going to use this grading.
\end{rem}
We'll first compute the additive structure, then sew things together.

By adjunction, $H_{C_2}^{p,q}(C_2)\cong H^{p+q}(*)$ (ordinary cohomology), because $[C_{2+}\wedge X, Y]_{C_2}\cong
[X,Y]$. Thus,
\[H_{C_2}^{p,q}(C_2) = H^{p+q}(*) = \begin{cases}
	\Z, &p+q=0\\
	0, &\text{otherwise.}
\end{cases}\]
Next we'll compute $H_{C_2}^{p,q}(*)$. Let $S^{\sigma q}$ denote the representation sphere that's the one-point
compactification of $\R^{\sigma q}$ (i.e.\ $q$ copies of the sign representation); then, if $q > 0$,
\[H_{C_2}^{p,-q}(*)\cong \wH_{C_2}^{p,-q}(S^0)\cong \wH_{C_2}^{p,0}(S^{\sigma q}).\]
If $S(\sigma q)$ denotes the unit sphere inside $\R^{\sigma q}$ and $D(\sigma q)$ denotes the unit disc, then there
is a cofiber sequence
\[\xymatrix{
	S(\sigma q)\ar[r] & D(\sigma q)\ar[r] & S^{\sigma q},
}\]
so $\wH_{C_2}^{p,0}(S^{\sigma q})\cong\wH_{C_2}^{p,0}(\Sigma S(\sigma q))$. This is the $\Z$-graded Bredon
cohomology of $\Sigma S(\sigma q)$, which if the $C_2$-action is free is the same thing as the ordinary cohomology
of the quotient. The $C_2$-action on $S(\sigma q)$ is the antipodal map, so the quotient is $\RP^{q-1}$, so
\[\wH_{C_2}^{p,0}(S^{\sigma q})\cong \wH_{C_2}^{p,0}(\Sigma S(\sigma q))\cong \wH^p(\Sigma\RP^{q-1}),\]
and we know what this is.

When $q > 0$, we use equivariant Spanier-Whitehead duality:
\[H_{C_2}^{p,q}(*)\cong \wH_{C_2}^{p,q}(S^0)\cong \wH_{-p,-q}^{C_2}(S^0)\cong \wH_{-p,0}(S^{\sigma q}).\]
Once again we'll apply $\wH_{*,0}$ to the cofiber sequence, and conclude when $k\ne 0,1$, $\wH_{k,0}(S^{\sigma
q})\cong \wH_{k-1,0}(S(\sigma q))$, which will again relate to projective spaces; for $k = 0,1$, we'll have to open
up the long exact sequence.

Inducting over the cofiber sequence is a common trick, and is a good one to carry with you.

We'll probably spend most of the next class on calculations.
