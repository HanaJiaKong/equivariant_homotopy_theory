In this section, we'll conver the Wirthmüller isomorphism, the equivariant analogue of the nonequivariant statement
that in the stable homotopy category $\Ho(\Spc)$, finite sums and products are equivalent:
\begin{equation}
\label{additivespec}
\bigvee_{i=1}^n X_i\stackrel\cong\longrightarrow \prod_{i=1}^n X_i.
\end{equation}
This is a backbone of the stable category: among other things, it ensures $\Ho(\Spc)$ is additive.

Tom Dieck splitting is another important structural result about $\pi_G^*(\Sigma^\infty X)$. When $X = S^0$, this
is used to compute the equivariant stable homotopy groups of the spheres (which is, of course, not completely
known). This can be used to recover the Burnside category.

Transfers play a key role in both of these results: it's somewhat implicit in the Wirthmüller isomorphism, but is
pretty explicit for tom Dieck splitting.
\subsection*{The Wirthmüller isomorphism.}
In this section, we assume $G$ is a finite group. There is a more general statement for compact Lie groups, but
it's more complicated.
\begin{thm}[Wirthmüller isomorphism]
\label{Wirthiso}
Let $H$ be a subgroup of $G$ and $X$ be an object of $\Sp^H$.\footnote{Here, we're implicitly using the complete
universe $U$ to define $\Spc^G$ and $\Spc^H$.} Then, there is a $\pi_*$-isomorphism $G\wedge_H X\simeqto F_H(G,X)$.
\end{thm}
\begin{ex}
Adapt the proof of~\eqref{additivespec} for nonequivariant spectra to show that the natural map from finite sums to
finite products in $\Spc^G$ is also an isomorphism.
\end{ex}
However, if you only have~\eqref{additivespec}, you've described the category of $G$-spectra structured by a
universe with only trivial representations!

Applying \cref{Wirthiso} to $X = S^0$ computes the Spanier-Whitehead duals of orbits.
\begin{cor}
\label{wirthcor}
In the situation of \cref{Wirthiso}, $\Sigma^\infty(G/H)_+\cong F_H(G,S^0)\cong F((G/H)_+, S^0)$.
\end{cor}
That is, $\Sigma^\infty (G/H)_+$ is its own Spanier-Whitehead dual. The slogan is that orbits are self-dual in the
equivariant stable category when $G$ is finite.\footnote{When $G$ is a compact Lie group, there's a degree shift
arising from the tangent representation of $G$ on the Lie algebra of $H$. \TODO: double-check this.}

Recall that the forgetful map $i_H^*\colon \Spc^G\to\Spc^H$ has a left and a right adjoint, respectively
$G\wedge_H\bl$ and $F_H(G,\bl)$. These spectrum-level constructions are induced by applying the space-level
functors levelwise.
\begin{ex}
Show that these definitions of $G\wedge_H\bl$ and $F_H(G,\bl)$ are consistent with the structure maps.
\end{ex}
There are three approaches to proving \cref{Wirthiso}.
\begin{enumerate}
	\item One is a connectivity argument: show that on the space level, the maps get more and more highly
	connected.
	\item\label{transferproof} Another approach is to construct an explicit inverse, using a transfer map.
	\item One can also set up a Grothendieck six-functor formalism to prove it, which sets up a general theory for
	when a lax monoidal functor's left and right adjoints coincide. See~\cite{FHM, WirthRevisited, BDS} for a proof
	using this approach; the first and the third papers set up general theory, and the second shows that it applies
	to the Wirthmüller map. This makes interesting contact with base-change theory in algebraic geometry.
\end{enumerate}
\begin{warn}
The proof of \cref{Wirthiso} given in~\cite{LMS} is incorrect, and the fix is nontrivial. It takes
approach~\eqref{transferproof}.
\end{warn}
We'll use a connectivity argument, which will introduce some tools we'll find useful later.
\begin{proof}[Proof of \cref{Wirthiso} ($X$ connected)]
Though we assume $X$ is connected, the same proof can be adapted when $X$ is bounded-below. The theorem is true for
general $X$, but this proof may not work in that case.

Let $X\in H\Top_*$. Then, there's a map $\theta\colon G\wedge_H X\to F_H(G,X)$ defined by
\[\theta(g_1,x)(g_2) = \begin{cases}
	g_2g_1x, &g_2g_1\in H\\
	*, &\text{otherwise.}
\end{cases}\]
\begin{ex}
Show that $\theta$ is a $G$-map.
\end{ex}
$\theta$ induces a map $\overline\theta\colon G\wedge_H X\to F_H(G,X)$ in $\Spc^G$. We'll show that it's an
equivalence by computing the connectivity of $\theta$.

Let $K\subseteq G$, $\rho$ be the regular representation of $G$, and $m\in\N$. Then, we'll compute the connectivity
of 
\[\theta_{m,K}\colon \paren{G\wedge_H \Sigma^{m\rho}X}^K\longrightarrow \paren{G_H(G, \Sigma^{m\rho} X)}^K.\]
When $G$ is finite, the sequence $(\rho, 2\rho, 3\rho,\dotsc)$ is cofinal (i.e.\ the colimits are the same) in the
filtered diagram of finite-dimensional representations of the complete universe $U$. Thus, to understand
$\colim_{V\subset U} \Omega^V\Sigma^V X$, it suffices to understand what happens when $V = m\rho$.
\begin{ex}
The reason this is true is that when $G$ is finite, the regular representation $\rho$ contains a copy of every
irreducible as a summand. Prove this by doing a character computation.
\end{ex}
We'll show the connectivity of $\theta_{m,K}$ is increasing in $m$, which means that
$\pi^K_*\overline\theta\colon\pi_*^K(G\wedge_H X)\to\pi_*^K(F_H(G,X))$ is an isomorphism.

The calculation itself will use the $K, H$ double coset decomposition of $G$ to identify the $K$-fixed points as a
sum and as a product, and we understand the connectivity of both of these from nonequivariant homotopy theory.

Let $S\subset G$ be a set of
representatives for the double coset partition of $G$, i.e.\ under the $K\times H$-action $(k,h)\cdot g =
k^{-1}gh$. Then,
\begin{equation}
\label{GZKfixed}
\paren{G\wedge_H Z}^K = \paren{\coprod_{g\in S} (KgH)_+\wedge_H Z}^K.
\end{equation}
Either $g^{-1}Kg\subseteq H$ or it isn't.
\begin{itemize}
	\item If $g^{-1}Kg\subseteq H$, then the action is only in $Z$, so $((KgH)_+\wedge_H Z)^K = Z^{g^{-1}Kg}$.
	\item If $g^{-1}Kg\not\subseteq H$, then the action is free, so $((KgH)_+\wedge_H Z)^K = *$.
\end{itemize}
In particular~\eqref{GZKfixed} simplifies to
\[\paren{G\wedge_H Z}^K\cong \bigvee_{\substack{g\in S\\g^{-1}Kg\subseteq H}} Z^{g^{-1}Kg}.\]
Next we look at $(F_H(G,Z))^K$. If $S$ is a set of representatives of the double cosets, so is
$S^{-1}\coloneqq \set{g^{-1}\mid g\in S}$, so we can decompose
\begin{align*}
\paren{F_H(G,Z)}^K &\cong \paren{F_H\paren{\coprod_{g\in S} Kg^{-1}H, Z}}^K\\
&\cong \prod_{g\in S} F_H(Kg^{-1}H, Z)^K\\
&\cong \prod_{g\in S} Z^{(gKg^{-1})\cap H} \cong \prod_{g\in S} Z^{(g^{-1}Kg)\cap H}.
\end{align*}
This identification sends $f\mapsto\set{f(g)}$; the last equivalence is by using $S^{-1}$ as the set of
representatives instead of $S$.

In particular, the space-level Wirthmüller isomorphism may be written
\begin{equation}
\label{fixedwirth}
\bigvee_{\substack{g\in S\\g^{-1}Kg\subseteq H}} Z^{g^{-1}Kg}\stackrel\cong\longrightarrow \prod_{g\in S}
Z^{(g^{-1}Kg)\cap H}.
\end{equation}
This is nice-looking, but the indexing sets are slightly different. To overcome this, we'll
factor~\eqref{fixedwirth} as
\[\xymatrix{
	\bigvee_{\substack{g\in S\\g^{-1}Kg\subseteq H}} Z^{g^{-1}Kg}	\ar[r]^-{\vp_1}
	& \prod_{\substack{g\in S\\g^{-1}Kg\subseteq H}} Z^{(g^{-1}Kg)\cap H}\ar[r]^-{\vp_2}
	& \prod_{g\in S} Z^{(g^{-1}Kg)\cap H},
}\]
where $\vp_1$ is the natural map from the sum to the product and $\vp_2$ is inclusion. Then, we will estimate the
connectivity of $\vp_1$ and $\vp_2$ separately in the case when $Z = \Sigma^{m\rho}X$. Namely, we'd like to show
that they're both $(m[G:K] + O(1))$-connected.

First, what are the fixed points in $(S^{m\rho})^{g^{-1}Kg}$? This is a sphere whose dimension has been shrunk by
$K$, so smashing with it produces something $m[G:K]$-connected. Then, the usual argument about turning sums into
products says that $\vp_1$ is about $2m[G:K]$-connected.

For $\vp_2$, what's the connectivity of the missing factors in the domain? In this case, the connectivity is about
$m[G:K]$.

Thus, the connectivity of $\theta_{m,K}$ is $m([G,K] + O(1))$,\footnote{\TODO: I want to run through the
connectivity carefully and ensure I made no typos.} so the spectrum-level map $\overline\theta$ is a
$\pi_*^K$-isomorphism. More precisely, the cofiber of $\theta\colon G\wedge_H X\to F_H(G,X)$ has trivial $\pi_*^K$,
and there's a little bit to do here to check this.
\end{proof}
\begin{rem}
Let's see how this connects to the transfer. Let $V$ be a $G$-representation such that there's an embedding
$G/H\inj V$. Then, we obtain an $H$-map $G\wedge_H S^V\to S^V$ that takes points in $G\setminus H$ to the
basepoint. The adjoint of this map is the $G$-map $G\wedge_H S^V\to F_H(G,S^V)$.

On the other hand, we have a Pontrjagin-Thom map $S^V\to G\wedge_H S^V$ induced as follows: $G/H\inj V$ induces a
map $G\wedge_H D(V)\to V$, and therefore a map $S^V\to G\wedge_H D(V)/(G\wedge_H S(V))\cong G\wedge_H S^V$. The
observation is that the composition $S^V\to G\wedge_H S^V\to S^V$ is the identity (you can and should think about
this: it's possible to write down an explicit homotopy inverse). This motivates the slogan that ``the Wirthmüller
isomorphism is the inverse of the transfer map,'' which we'll say more about later.
\end{rem}
You can use \cref{wirthcor} to explicitly write down the transfer map: given subgroups $K\subseteq H\subseteq G$,
we have the ``right-way'' map $G/H\to G/K$, and therefore a map $\sus(G/K_+)\cong D(G/K_+)\to
D(G/H_+)\cong\sus(G/H_+)$. By the Wirthmüller isomorphism, this is the same as a map $F(G/K_+,S)\to F(G/H_+,S)$,
which is exactly the transfer map!

This is a little circular; if you look carefully into the proof of \cref{Wirthiso}, the transfer map is already
there. But the point is more philosophical: the existence of transfers is the same thing as the Wirthmüller
isomorphism.
\begin{rem}
Transfer maps can be studied in more generality, e.g.\ in the context of equivariant vector bundles on homogeneous
spaces. Rothenberg wrote some stuff, but~\cite{LMS} is probably the best source.
\end{rem}
