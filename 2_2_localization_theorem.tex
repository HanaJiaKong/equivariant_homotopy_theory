\begin{quote}\textit{
	``The number of children he had was a monotonically increasing function.''
}\end{quote}
There are three goals in today's lecture.
\begin{enumerate}
	\item We'll discuss the localization theorem (\cref{localization}) again, and prove it, providing a bit more
	context for where it came from.
	\item We'll talk about the Sullivan conjecture. This is really Sullivan's attack on Adams' conjecture, and is a
	very important story. We won't prove the conjecture, because it's hard, but the context around it was a major
	motivation for a lot of the work in algebraic and geometric topology in the past 40 years. Sullivan wrote up
	some notes for a class of his at MIT, which have been published as~\cite{MITNotes}, and you should read them:
	they are enlightening and contain all of the jokes he told in class!
	\item We'll introduce the stable category, discussing Spanier-Whitehead duality and the Pontrjagin-Thom
	construction. If you want points (orbits) to have Spanier-Whitehead duals, you're inexorably forced to
	construct this stable category.
\end{enumerate}
\subsection*{The localization theorem.}
Recall that the Borel cohomology $H^*(EG\times_G X)$ is an $H^*(BG)$-module, through the map $EG\times_G X\to
EG\times_G * = BG$. Thus, we should compute $H^*(BG)$ as a ring. In the case we care about, $G = (\Z/p)^n$; let's
start with $n = 1$.

There's a nice geometric model for $B\Z/p$ as an ``infinite-dimensional lens space:'' let $S^\infty$ denote the
unit sphere in an infinite-dimensional complex Hilbert space,\footnote{Alternatively, you could choose $S^\infty$
to be the colimit of $S^n$ for all $n$, through the inclusion $S^n\inj S^{n+1}$ at the equator. These two choices
are not homeomorphic, but produce homotopy equivalent models of $B\Z/p$.} and give it a $C_p$-action by $z\mapsto
e^{2\pi i/p}z$. Then, the infinite-dimensional lens space is $S^\infty/C_p$.

The quotient $S^\infty\surj S^\infty/C_p$ is a covering map, and $S^\infty$ is contractible. The proof goes through
some version of the Eilenberg swindle:
\begin{itemize}
	\item There is a homotopy from $\id_{S^\infty}$ to $s: (x_1,x_2,\dotsc)\mapsto (0, x_1, x_2, \dotsc)$. In fact,
	it's a straight-line homotopy.
	\item There is a straight-line homotopy from $s$ to $(0,x_1,\dotsc)\mapsto (1, 0, 0,\dotsc)$.
\end{itemize}
Thus $S^\infty/C_p$ is the quotient of a contractible space by a free $G$-action, so it deserves to be called
$B\Z/p$.

You can set up a cell structure on $B\Z/p$ as with finite-dimensional lens spaces, and therefore compute that
\[H^k(B\Z/p) \cong\begin{cases}
	\Z/p, &k\text{ even}\\
	0, &k\text{ odd.}
\end{cases}\]
Using the universal coefficient theorem, you can then deduce that
\[H^k(B\Z/p;\Z/p) \cong \Z/p\]
for all $k$. Now we want to deduce the ring structure.

Recall that if
\[\shortexact{M}{L}{N}{}\]
is a short exact sequence, it induces a long exact sequence in cohomology:
\[\xymatrix{
	\dotsb\ar[r] & H^k(\bl;M)\ar[r] &H^k(\bl;L)\ar[r] &H^k(\bl;N)\ar[r]^\beta &H^{k+1}(\bl;M)\ar[r] & \dotsb
}\]
Let's apply this to the short exact sequences
\[\xymatrix{
	0\ar[r] & \Z\ar[r]^{\cdot p}\ar[d]&\Z\ar[r]\ar[d] & \Z/p\ar[r]\ar[d] &0\\
	0\ar[r] & \Z/p\ar[r]^{\cdot p} &\Z/p^2\ar[r] & \Z/p\ar[r] &0.
}\]
This is a map of short exact sequences, inducing a map of their long exact sequences.
\begin{equation}
\begin{gathered}
\xymatrix{
	\dotsb\ar[r] & H^q(X;\Z)\ar[r]\ar[d] & H^q(X;\Z)\ar[r]\ar[d] & H^q(X;\Z/p)\ar[r]\ar[d] & H^{q+1}\ar[r]\ar[d] &
	\dotsb\\
	\dotsb\ar[r] & H^q(X;\Z/p)\ar[r] & H^q(X;\Z/p^2)\ar[r] & H^q*(X;\Z/p)\ar[r]^\beta &H^{q+1}(X;\Z/p)\ar[r] &
	\dotsb
}
\end{gathered}
\end{equation}
The map $\beta:H^q(X;\Z/p)\to H^{q+1}(X;\Z/p)$ will be called the \term{Bockstein homomorphism}, and is a simple
example of a cohomology operation.

We now assume $p$ is odd.
\begin{lem}
If $n$ is odd, the Bockstein for $B\Z/p$ is an isomorphism; if $n$ is even, it's $0$.
\end{lem}
\begin{proof}
First, observe that the diagram
\[\xymatrix{
	H^n(B\Z/p;\Z)\ar[r]^f & H^n(B\Z/p;\Z/p)\ar[r]^g\ar[dr]_\beta & H^{n+1}(B\Z/p;\Z)\ar[d]^\pi\ar[r] &
	H^{n+1}(B\Z/p;\Z)\\
	&& H^{n+1}(B\Z/p;\Z/p)
}\]
commutes, and the top row is exact.
\begin{itemize}
	\item If $n$ is even, $f$ is an isomorphism, so $g = 0$, so $\beta = 0$.
	\item if $n$ is odd, $g$ and $\pi$ are surjections, so $\beta$ is a surjection between two $\F_p$-vector spaces
	of the same rank, so $\beta$ is an isomorphism.\qedhere
\end{itemize}
\end{proof}
Next, a nice way to compute the cup product. The map $B\Z/p\to\CP^\infty$ is cellular, and is a homeomorphism when
restricted to even-dimensional cells. As the cell structure determines the cup product structure, the cup products
on $H^{\text{even}}(B\Z/p;\Z/p)$ and $H^{\text{even}}(\CP^\infty;B\Z/p)\cong \Z/p[y_1]$ agree. On the
odd-dimensional cells, the cup product is graded commutative rather than strictly commutative, so we get an
exterior algebra. Therefore we conclude that
\[H^*(B\Z/p;\Z/p)\cong \Lambda(x_1)\otimes \Z/p[y_1],\]
where $\abs{x_1} = 1$, $\abs{y_1} = 2$, and $\beta x_1 = y_1$.

By essentially the same argument, one can compute the cohomology ring for $B(\Z/p)^n$.
\begin{prop}
As rings, there is an isomorphism
\[H^*(B(\Z/p)^n; \Z/p) \cong \Lambda(x_1,\dotsc,x_n)\otimes\Z/p[y_1,\dotsc,y_n],\]
where $\beta x_i = y_i$.
\end{prop}
\begin{ex}
Figure out the slight changes needed for $p = 2$.
\end{ex}
Recall that the localization theorem asserted that if $G$ is a finite $p$-group, $S_H^{-1}H^*(EG\times_G
X)\cong S_H^{-1}H^*(EG\times_G X^G)$, where $S$ is the multiplicative system generated by images of the Bockstein
homomorphism in $H^*(BG;\Z/p)$. We'll prove this in the case when $G$ is abelian.

We mentioned last time that it's possible to inductively reduce to considering $H^*(BH)\otimes H^*\paren{\bigvee
S^{q_i}}$. It's now clear\footnote{{\color{red}TODO}: I didn't follow this proof in class.} that something in $S$
restricts to $0$ in $H^*(BH)$, completing the proof (in the abelian case).

The localization fails terribly for infinite nonabelian compact Lie groups $G$. For example, for any topological
space $K$, there exists a $G$-CW complex $X$ such that $X$ is nonequivariantly contractible, $X$ is
finite-dimensional, and $X^G\simeq K$.
\subsection*{The Sullivan conjecture.}
\begin{thm}[Sullivan conjecture]
Let $G$ be a finite, abelian $p$-group. Then, $X^{hG}\to H^G$ is an equivalence on $p$-completions.
\end{thm}
Recall that $X^{hG} \coloneqq\Map(EG, X)^G$, so this asserts a weak equivalence (after $p$-completing) $\Map(EG,
X)^G\to\Map(*, X)^G$.

By \term{$p$-completion}, we mean Bousfield localization at $\F_p$ cohomology. This produces the category of
spaces where equivalences are detected by $H^*(\bl;\F_p)$. The most familiar example of completion is
\term{rationalization}, a localization where equivalences are detected by rational homotopy groups, and one of
Sullivan's biggest achievements was providing a completely algebraic description of the rational homotopy
category in~\cite{SullivanQHT}.
More broadly, he had the insight that to study a problem in homotopy theory, one could localize at $\Q$ and at each
$\F_p$ and study each piece, which has been a very fruitful approach.

$p$-completion falls into the collection of basic life skills for homotopy theorists, so if you haven't seen it
before, you should read about it. The standard reference is~\cite{BousfieldKan}, but this is 500 pages and hard to
read.

Sullivan's conjecture is an algebro-geometric attack on the Adams conjecture. This was within Sullivan's program to
find algebraic models of manifolds. This is still being done today, and is what led Sullivan to think about string
topology and related things.

Let $X$ be a manifold. We first have the homotopical data of $X$, $C^*(X;\Q)$ and $C^*(X;\F_p)$, which are
``commutative rings.''\footnote{They're not literally commutative; instead, they're $\E_\infty$ dg algebras. They
also have more structure as modules over the Steenrod algebra.} That $X$ is a manifold means we can see Poincaré
duality, which doesn't appear in all $\E_\infty$ dg algebras. But we still need some way to encode additional
geometry obstructions, e.g.\ a way to encode the tangent and normal bundles.

It's been an interesting, but as yet unsuccessful attack, to use a Frobenius algebra structure to try to obtain
this geometric data. It's neat to think about what the $\E_\infty$ analogue of a Frobenius algebra is, and this is
intimately related to Lurie's approach to the cobordism hypothesis~\cite{Lur09}, thinking about fully dualizable
objects in a symmetric monoidal $\infty$-category.

This circle of ideas is also related to surgery theory; there's been lots of cool work by smart people in it, and
$L$-theory was invented basically as an algebraic home for these geometric objects.

Sullivan was interested in the Adams conjecture because it says that one can identify the tangent bundle inside
$K$-theory $K(X)$ with its Adams operations $\psi^k$, as fiberwise homotopy types.

Sullivan's idea, motivated by Quillen, was to use the theory of étale homotopy types. This translates some
questions about scheme theory into homotopy theory. For example, if $X$ is a variety (more generally a scheme), one
can assign some profinite topological object, built out of something like a system of hypercovers.\footnote{If you
don't know what this is, it's an example of an extremely interesting construction which you should look up
sometime.} So if you take the profinite completion of the complex points $X(\C)^\land$, it has an action of the
absolute Galois group $\Gal(\overline\Q/\Q)$ --- and another crazy interpretation of the Adams conjecture is that
the profinite completion of $K(X)$ can be interpreted in this way, and has an action of $(\widehat\Z)^*$
($\widehat\Z$ is the Galois group of the maximal abelian extension over $\Q$). The conjecture is that this action
is by the Adams operations.

There's a deep and inadequately understood story (which could be an opportunity for you) connecting $p$-adically
completed complex $K$-theory $\KU_p^\land$ to number theory, specifically the Iwasawa algebra. Adams noticed this,
but it's too interesting to be a coincidence.

Anyways, stable fiberwise homotopy types are invariant under this $(\widehat\Z)^*$-action, which led Sullivan to
ask questions about $(X(\C)^\wedge)^{hC_2}$ versus $X(\R)^\wedge$. The references~\cite{MITNotes, Genetics} are
both excellent for this.

For reasons of scope, we can't go into too much more detail, but you should definitely look this stuff up. The
takeaway is that equivariant homotopy theory has been motivated by seemingly unrelated questions about manifolds.
There's been a lot of interesting interplay between algebraic and geometric topology in the last half century, and
this is one of the cites of contact.
\subsection*{The stable category and stable phenomena.}
There are lots of ways to think about where stabilization comes from.
\begin{enumerate}
	\item The Freudenthal suspension theorem says that if $X$ is nondegenerately based (meaning the based inclusion
	map $*\inj X$ is a cofibration) and $n-1$-connected, then $\pi_q(X)\to\pi_{q+1}(\Sigma
	X)$ is an isomorphism for $q < 2n-1$ and a surjection when $q = 2n-1$.\footnote{The map comes from the
	loop-suspension adjunction, which gives us a unit $X\to\Omega\Sigma X$, hence a map
	$\Omega^qX\to\Omega^{q+1}\Sigma X$, and the map on homotopy groups is $\pi_0$ of that map. This is the based
	version of the mapping space and Cartesian product adjunction: $\Sigma X\coloneqq S^1\wedge X$ and $\Omega X
	\coloneqq \Map(S^1, X)$ are adjoint functors.} It's easier to see that cohomology groups are stable under
	suspension, but this tells us that homotopy groups stabilize in a range that increases at about twice the rate
	that the connectivity of $X$ does. Since $\Sigma^n X$ is at least $n$-connected, this suggests you could
	replace $X$ by the sequence $X, \Sigma X,\dotsc,\Sigma^n X,\dotsc$, and keep track of that instead, regarding
	it as a repository for the \term{stable homotopy groups} $\pi_n^S(X)\coloneqq\colim_k \pi_{n+k}(\Sigma^k X)$.
	One way to think of this is as formally making homotopy theory into a homology theory (which it isn't \textit{a
	prioiri}); you end up taking the same kind of colimit.

	You could do this equivariantly: we have representation spheres. But it's not entirely clear what to do.
	\item Another perspective is that the stable category is the result of inverting the canonical map\footnote{The
	existence of this map follows from the universal property of the product.}
	\begin{equation}
	\label{wedges_are_products}
	\bigvee_{i=1}^k X\to\prod_{i=1}^k X.
	\end{equation}
	Again, this is something you can think about making precise; the stable category is the initial triangulated
	category constructed from $\Top$ in which~\eqref{wedges_are_products} is an isomorphism. In particular, this
	forces the homotopy category to be additive.

	Again, we could do this equivariantly.
	\item Suppose we have a functor $F$ from finite CW complexes to $\Top$ such that $F(*) = *$, and suppose $F$
	commutes with filtered colimits and preserves weak equivalences (e.g.\ if it's topologically or simplicially
	enriched, for formal reasons). By taking colimits, we can obtain a functor $\widehat F$ from CW complexes to
	$\Top$. We say $F$ is \term{excisive} if it takes pushouts to pullbacks; this is an old perspective, which was
	used to show the Dold-Thom theorem, that the infinite symmetric product $\mathrm{SP}^\infty$ is a cohomology
	theory. In any case, if $F$ is excisive, $\set{F(S^n)}$ represents a cohomology theory. Namely, the homotopy
	pushout
	\[\xymatrix{S^n\ar[r]\ar[d] & D^{n+1}\ar[d]\\
	D^{n+1}\ar[r] & S^{n+1}
	}\]
	creates suspension, and via $F$, becomes a pullback creating $\Omega$. Asking what the excisive functors are
	leads to the stable category, and is also something you could do equivariantly.
	\item Another perspective: what is the Spanier-Whitehead dual of a point? Taking shifts, what's the
	Spanier-Whitehead dual of $S^n$? Equivariantly, one wants to know the Spanier-Whitehead duals of $G/H_+$. This
	is important for defining Poincaré duality, etc. Nonequivariantly, the best way to answer this is the
	Pontrjagin-Thom construction, which not only answers this, but provides a deep understanding for what the
	stable homotopy groups of the spheres are. We'll do this equivariantly next time, and it will tell us what the
	spheres are.
\end{enumerate}
There's a choice here: you could just say ``let's take the category of orthogonal $G$-spectra,'' but having some
motivation for why we're doing and what we're doing is important.
