Today, we're going to discuss the homotopy theory of spectra in the nonequivariant setting. Last time, we discussed
$\fD$-spaces, where $\fD$ is a small topological category, and saw that the category of $\fD$-spaces is symmetric
monoidal. We discussed a few examples:
\begin{itemize}
	\item Consider $\fD = \N$ with all maps isomorphisms, as in Example~\ref{prespectra}. There's a
	\emph{non}commutative monoid $S_\N\colon n\mapsto S^n$, and the category of $S_\N$-modules is called prespectra
	$\Spc^\N$;\footnote{There are many different choices of notation for diagram spectra, as seen in~\cite{MMSS}
	and \cite{MandellMay}, with various fancy decorations.} it's a monoidal category, but is \emph{not} symmetric.
	\item In Example~\ref{orthogonal_spectra}, we considered $\fD = \sI$, the category whose objects are real
	finite-dimensional inner product spaces and whose morphisms are the linear isometric isomorphisms. Here, the
	monoid $S_\sI\colon V\mapsto S^V$ is commutative, and the category of $S_\sI$-modules, called orthogonal
	spectra $\Spc^\sI$, is symmetric monoidal.
\end{itemize}
Those are the main examples for us, but there are others:
\begin{itemize}
	\item When $\fD$ is the category of finite CW complexes, as in Example~\ref{wspaces}, we obtain $\sW$-spaces.
	\item When $\fD$ is the category of finite, based sets and based maps, as in Example~\ref{gamma_spaces}, we
	recover Segal's $\Gamma$-spaces.
\end{itemize}
Because these diagram categories are presheaves on nice categories, they inherit some good properties from $\Top_*$;
in particular, they are complete and cocomplete, and limits and colimits may be taken pointwise. This is also true
for categories of modules over monoids in $\fD$-spaces, though it requires more work to prove: computing colimits
is a bit harder, just like how the free product of groups is more complicated than the direct prodict. Categories
of rings are \emph{not} bicomplete, though.
\begin{defn}
Let $f\colon X\to Y$ be a map of $\fD$-spaces (or prespectra or orthogonal spectra).
\begin{itemize}
	\item $f$ is a \term{level equivalence} if for all $d\in\fD$, $f(d)\colon X(d)\simeqto Y(d)$ is a weak
	equivalence.
	\item $f$ is a \term{level fibration} if for all $d\in\fD$, $f(d)$ is a fibration.
\end{itemize}
\end{defn}
That is, $f$ is a natural transformation, and it acts through weak equivalences (resp.\ fibrations).
\begin{thm}
The category of $\fD$-spaces has a model structure, called the \term{level model structure}, in which the weak
equivalences are the level equivalences and the fibrations are level fibrations. Moreover, this model category is
cofibrantly generated.
\end{thm}
\begin{ex}
Starting with the usual model structure on $\Top_*$, construct the level model structure.
\end{ex}
The ``cofibrantly generated'' part means that cofibrant objects behave like CW complexes, and in particular there
is a theory of cellular objects. If $F_d\colon\Top_*\to\Fun(\fD,\Top_*)$ is the left adjoint to $\Ev_d$ we
constructed last time, then the \term{generating cofibrations} are the maps $F_d(S_+^{n-1}\to D_+^n)$ for each
$n\ge 1$ and $d\in\fD$, and the \term{acyclic cofibrations} are $F_d(D_+^n\to (D_n\times I)_+)$ for each $n\ge 1$
and $d\in\fD$.

Since all spaces are fibrant, all $\fD$-spaces are fibrant in the level model structure. The cofibrant objects are
the retracts of \term{cellular objects}, which are built by iterated pullbacks
\[\xymatrix{
	\bigvee F_d S_+^{n-1}\ar[r]\ar[d] & X_n\ar[d]\\
	\bigvee F_dD_+^n\ar[r] & X_{n+1}.
}\]
While this is all nice, it's not what we're looking for, as it contains no information about stable phenomena. It's
like the category of spaces, just with more of them. For example, it's not even true that $X\to\Omega\Sigma X$ is a
weak equivalence, which is important if you want $\Omega$ and $\Sigma$ to be homotopy inverses.

Recall that we defined a $\pi_*$-isomorphim of prespectra to be a map $f\colon X\to Y$ such that $\pi_qf\colon
\pi_q X\to\pi_q Y$ is an isomorphism for all $q$. We'll extend this to orthogonal spectra: let
$U\colon\Spc^\sI\to\Spc^\N$ be pullback by the map $[n]\mapsto\R^n$, i.e.\ $UX([n]) = X(\R^n)$. $U$ is right
adjoint to a left Kan extension $P\colon\Spc^\N\to\Spc^\sI$.
\begin{defn}
A map of orthogonal spectra $f\colon X\to Y$ is a \term{$\pi_*$-isomorphism} if $Uf\colon UX\to UY$ is a
$\pi_*$-isomorphism of prespectra.
\end{defn}
We also defined an $\Omega$-spectrum in prespectra, or an $\Omega$-prespectrum, to be a prespectrum where the
adjoints to the structure maps $X_n\congto\Omega^mX_{n+m}$ are homeomorphisms. This is a pretty rigid condition,
and so $\Omega$-prespectra have nice properties.
\begin{defn}
Similarly, we define an \term{$\Omega$-spectrum} in orthogonal spectra to be an orthogonal spectrum $X$ such that
the adjoints to the structure maps $X(U)\congto\Omega^V X(U\oplus V)$ are homeomorphisms.
\end{defn}
Classically, there were prespectra and then there were spectra (or $\Omega$-spectra), and you would use some
``spectrification'' functor that took a prespectrum and produced a spectrum of the same homotopy type. Turning the
adjoint maps into homeomorphisms looks difficult and is, as it involves some categorical and point-set wizardry. If
you like this stuff, check out the appendix of~\cite{LMS}. The first point-set symmetric monoidal model for the
stable category~\cite{EKMM} relies on this and even more magic, both clever and surprising.

Since $\fD$-spaces are enriched over spaces, it's possible to tensor with the interval and therefore define
homotopies of $\fD$-spaces in the same way as for spaces. As usual, $[X,Y]$ will denote the set of homotopy classes
of maps $X\to Y$.
\begin{defn}
\label{stable_equivalence}
A \term{stable equivalence} of prespectra is a map $f\colon X\to Y$ such that for all $\Omega$-prespectra $Z$, the
induced map $[Y,Z]\to[X,Z]$ is an isomorphism.
\end{defn}
\begin{thm}
There are \term{stable model structures} on the categories of $\fD$-spaces, prespectra, and orthogonal spectra in
which the weak equivalences are stable equivalences. For $\Spc^\N$ and $\Spc^\sI$, the stable equivalences are the
same as the $\pi_*$-isomorphisms.
\end{thm}
\begin{rem}
You can make the same construction for symmetric spectra, but the stable equivalences are not the same as
$\pi_*$-isomorphisms, and homotopy groups are consequently finickier. This ultimately comes from the fact that
quotients $\O(n+k)/\O(n)$ get more highly connected as $n$ and $k$ grow in a way that quotients of symmetric groups
don't. In any case, since symmetric spectra don't behave so well in the equivariant case, we won't use them.
\end{rem}
The stable model structure does in fact manifest stable phenomena: the map $X\to\Omega\Sigma X$ is a weak
equivalence.\footnote{If you like $\infty$-categories, you can say that the level model category is an
$\infty$-category of presheaves, and this is different from the $\infty$-category of spectra, which is presented by
the stable model category.}

If $X$ and $Y$ are $\Omega$-prespectra and $f\colon X\to Y$ is a level equivalence, then $f$ is a
$\pi_*$-isomorphism. That is, the weak equivalences in the stable model structure contain the weak equivalences in
the level model structure, and it's possible to use \term{Bousfield localization} to obtain the stable model
structure from the level model structure.

Let $\fC$ be a model category and $S$ be a set of maps in $\fC$.\footnote{We want $S$ to contain the weak
equivalences in $\fC$, but there are important set-theoretic issues. Often, one specifies that $S$ contains a
generating set (under filtered colimits) of the weak equivalences of $\fC$.} Bousfield localization produces a new
model structure $L_S\fC$ on $\fC$ in which the morphisms in $S$ are weak equivalences. The homotopy category and
homotopy (co)limits change, but all point-set phenomena remain the same. Localization is given by fibrant
replacement.
\begin{defn}
Let $\fC$ be a topologically enriched model category and $S$ be as above.
\begin{itemize}
	\item An \term{$S$-local object} in $\fC$ is an object $X$ such that for all $f\colon Y\to Z$ in $S$, the
	induced map
	\[\Map_\fC(Z,X)\stackrel\simeq\longrightarrow \Map_\fC(Y, X)\]
	is a weak equivalence.
	\item A map $f\colon X\to Y$ is an \term{$S$-local equivalence} if for all $S$-local objects $Z$, the induced map
	\[\Map_\fC(Y,Z)\stackrel\simeq\longrightarrow \Map_\fC(X,Z)\]
	is a weak equivalence.
\end{itemize}
\end{defn}
The following theorem is due to many people, but Hirschhorn's formulation is particularly nice.
\begin{thm}[\cite{Hirschhorn}]
Let $\fC$ be a cofibrantly generated, $\Top$-enriched model category. Then, the Bousfield localization $L_S\fC$
always exists. Moreover, the weak equivalences are the $S$-local equivalences, the fibrant objects are the
$S$-local objects, and the cofibrations are exactly those in $\fC$.
\end{thm}
\begin{exm}
One way this is used is to localize the category of spectra such that the $S$-local equivalences are detected by
$\bl\wedge H\Q$ or $\bl\wedge H\F_p$. This is a slick way to construct the rationalization or $p$-completion,
respectively, and in particular makes the localization map functorial. Thus one obtains the rational (or
$p$-completed) stable homotopy category.
\end{exm}
The stable model structure is obtained by Bousfield localization at the stable equivalences
(Definition~\ref{stable_equivalence}). It follows immediately that $\Omega$-prespectra are local objects and stable
equivaiences are weak equivalences.
\begin{prop}
The stable model structure is stable, i.e.\ $X\to\Omega\Sigma X$ is a $\pi_*$-isomorphism.
\end{prop}
\begin{proof}
On homotopy groups, this is asking for $\colim_n\pi_{q+n} X_n\to\colim_n\pi_{q+n}\Omega\Sigma X_n$ to be an
isomorphism. But by the Freudenthal suspension theorem, these colimits stabilize to the same stable homotopy group.
\end{proof}
This implies that when $X$ is an $\Omega$-prespectrum, $\pi_q X = \pi_q X_0$, and for $q < 0$, we can define $\pi_q
X =\pi_0(X_{-q})$.
\begin{thm}
The adjunction $P: \Spc^\N\rightleftarrows \Spc^\sI: U$ is a Quillen equivalence, and therefore
$\Ho(\Spc^\N)\cong\Ho(\Spc^\sI)$ as triangulated categories.
\end{thm}
This homotopy category is called the stable category. In a sense, it's the original triangulated category.
Suspension $\Sigma$ is the shift functor, turning a sequence $X\stackrel f\to Y\to Cf$ (where $Cf$ is the homotopy
cofiber) into a fiber sequence $Ff\to X\to Y$.\footnote{{\color{red}TODO}: I think I got something wrong.}

Another sense in which the stable category is stable is that both fiber and cofiber sequences induce long exact
sequences of homotopy groups, instead of just fiber sequences.

We'd like to construct a Quillen adjunction $\Sigma^\infty:\Top_*\rightleftarrows\Spc^\sI: \Omega^\infty$.
$\Omega^\infty$ is just $\Ev_0$, evaluating at the zero space. If $F_d$ is the adjoint to $\Ev_d$ (so that
$(F_dA)(e) = \Map_\fD(d, e)_+\wedge A$), then we can define $\Sigma^\infty A\coloneqq F_0A\wedge S$, where $S =
S_\N$ for prespectra and $S = S_\sI$ for orthogonal spectra.

If $R$ is a monoid in $\fD$-spaces, the category of $R$-modules is equivalent to a category of diagram spaces over
a more complicated diagram $\fD_R$. This is useful because diagrams are nice, and some things become less
complicated. The recipe is that $\fD_R$ is the category whose objects are the same as $\fD$ and whose morphisms are
\begin{equation}
\label{DSmaps}
\Map_{\fD_R}(d,e) = \Map_{\Mod_R}(F_d S^0\wedge R, F_eS^0\wedge R).
\end{equation}
That is, we take the suspension spectrum of the sphere shifted by $d$ and that of the sphere shifted by $e$.
\begin{ex}
Show that for $\fD = \N$, the structure maps for prespectra come out of~\eqref{DSmaps} for $R = S_\N$.  (This is an
adjunction game.) The same is true for orthogonal spectra and $S_\sI$.
\end{ex}
This feels like a Spanier-Whitehead trick, but constructs the right category. In particular, in $\fD_R$-spaces,
$\Sigma^\infty$ is the left adjoint to evaluating at $0$.

It's possible to bootstrap this to define model categories of ring spectra, i.e.\ algebras over $S_\sI$.
\begin{defn}
Let $\T\colon \Spc^\sI\to\Spc^\sI$ denote the free associative algebra monad, i.e.\
\[\T X\coloneqq\bigvee_{n\ge 0} X^{\wedge n},\]
so that the category of $\T$-algebras $\Spc^\sI[\T]$ is the category of associative monoids in $\Spc^\sI$.
Similarly, let $\P\colon\Spc^\sI\to\Spc^\sI$ denote the free commutative algebra monad, so
\[\P X\coloneqq \bigvee_{n\ge 0} X^{\wedge n}/\Sigma_n,\]
where $\Sigma_n$ denotes the action of the symmetric group by permutations; thus, the category of $\P$-algebras
$\Spc^\sI[\P]$ is the category of commutative monoids in $\Spc^\sI$ (i.e.\ commutative ring spectra).
\end{defn}
The stable model structure on $\Spc^\sI$ induces one on $\Spc^\sI[\T]$, in which the weak equivalences and
fibrations are detected by those in $\Spc^\sI$. For $\Spc^\sI[\P]$, the quotient makes things harder: it has a
model structure where the weak equivalences are \term{positive equivalences} (so those which are detected by
\term{positive $\Omega$-spectra}, i.e.\ those where $X_0\to\Omega X_1$ need not be an equivalence). But this still
doesn't behave very well. Namely, if you try to set the theory up for $\P$ to be the same as that for $\T$, then
you get that $\Omega^\infty\Sigma^\infty S^0$ is a commutative topological monoid. It's a fact that any commutative
topological monoid is a product of Eilenberg-Mac Lane spaces, and that $\Omega^\infty\Sigma^\infty S^0$ has
nontrivial $k$-invariants. There's a short paper by Lewis which addresses this~\cite{Lewis91}, showing that there
are five very reasonable axioms for the stable category that can't all be true! So people decided to forget about
letting $\Sigma^\infty$ be left adjoint to evaluation at $0$, and it's okay, if not perfect.
\begin{exm}[$\Gamma$-spaces]
\label{gamma_spaces}
Let $\fD$ be the category of finite based sets and based maps, e.g.\ $n_+ = \set{0,1,\dotsc,n}$ with $0$ as the
basepoint. $\fD$-spaces are called \term{$\Gamma$-spaces}, and agree with Segal's notion of $\Gamma$-spaces, which
are defined differently. The multiplication comes from the map $\psi\colon 2_+\to 1_+$ sending $1,2\mapsto 1$.

Let $d_i:n_+\to 1_+$ send $j\mapsto\delta_{ij}$ (i.e.\ $1$ if $i = j$, and $0$ otherwise). A $\Gamma$-space is
\term{special} if the induced map
\[X(n_+)\stackrel{\vp_n}{\longrightarrow} \prod_n X(1_+)\]
is a weak equivalence; it's \term{very special} if in addition the composition
\[\xymatrix{
	X(1_+)\times X(1_+) & X(2_+)\ar[l]^-\simeq_-{\vp_2}\ar[r]^{\psi_*} &X(1_+)
}\]
induces a commutative monoid structure on $\pi_0 X(1_+)$.

Kan extension defines a functor from $\fD$ to the category of finite CW complexes, and working with
$\pi_*$-equivalences of these, one obtains a model structure on the category of $\Gamma$-spaces. This is Quillen
equivalent to the category of \term{connective spectra}, i.e.\ those whose negative homotopy groups vanish.
\end{exm}
In the equivariant case, there's even more structure, and notions of ``extra special'' $\Gamma$-spaces. We'll spend
most of our time on orthogonal $G$-spectra, which have nuances of their own, as we'll see starting next time.
