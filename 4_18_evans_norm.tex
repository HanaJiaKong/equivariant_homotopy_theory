\begin{quote}\textit{
	``Speaking as someone who spent three years thinking about this as a graduate student, this is hard.''
}\end{quote}
\label{evenssec}
Today, we're going to talk about the Evens norm in group cohomology, tensor induction, and the norm in $\Spc^G$.
All of this is about motivating the norm structure on Tambara functors, an extra covariant map $N_H^G$. We're going
to explain where this is coming from in the context of topology and/or group cohomology. We've seen one approach
already, using the dependent sum and dependent product, but that was kind of abstract, and it's always useful to
have more than one approach.

Even if you don't care about equivariant ring spectra, the Evens norm is a cool construction, less well known than
it should be, and is a useful tool for proving things. It was introduced by Evens in~\citeme{Evens} and also written
about in~\citeme{Greenlees-May}.

Today, $H$ will be a finite-index subgroup of $G$. Of course, if $G$ is finite, any subgroup will do.

Let's recall first the definition of the transfer map $H^*(H;M)\to H^*(G;M)$. Here, the $H$-action on the
$\Z[G]$-module $M$ is through restriction $i_H^*$. We can also write $H^*(H;M)\cong\Ext_{\Z[H]}(\Z, M)$, the
derived functors of $M^H$.

The right adjoint of restriction is $\Hom_{\Z[G]}(\Z[H], \bl)$, but the Wirthmüller isomorphism, in this
group-theoretic context, says this is the same as the left adjoint $\Z[G]\otimes_{\Z[H]}\bl$. Thus, we get a map
\[H^*(H; i^*M)\cong H^*(G; \Hom_{\Z[G]}(\Z[H], M))\cong H^*(G, \Z[G]\otimes_{\Z[H]} i_H^*M)\longrightarrow
H^*(G;M),\]
where the map is induced by the counit $\Z[G]\otimes_{\Z[H]} i_H^*M\to M$.

Another way to describe the transfer is through the map $M^H\to M^G$ sending
\begin{equation}
\label{addtransfer}
x\mapsto \sum_{g\in G/H} gx.
\end{equation}
Then, the transfer is the induced map between derived functors. It's not hard to show these are the same: since
both of them are effaceable $\delta$-functors, it suffices to check on degree $0$, where it's straightforward. The
idea is that information about derived functors can be obtained from their degree-$0$ terms.

In~\eqref{addtransfer}, we added. But if $M$ is a $\Z[G]$-algebra, then we should also be able to multiply. This
will give us a variant description of
\[\Z[G]\otimes_{\Z[H]} M\cong \bigoplus_{g\in G/H} g\otimes M,\]
where $g\otimes M$ denotes $M$ with the action twisted by $g$.

We want to describe this more explicitly, and will do so with a wreath product.
\begin{defn}
The \term{wreath product} of $\Sigma_n$ and $H$ is the group
\[\Sigma_n\wr H\coloneqq \set{(\sigma, h_1,\dotsc,h_n)\mid \sigma\in\Sigma_n, h_i\in H}\]
with the multiplication
\[(\sigma, h_1,\dotsc, h_n)(\tau, \overline h_1,\dotsc,\overline h_n) = (\sigma\tau, h_{\tau(1)}\overline h_1,
h_{\tau(2)}\overline h_2,\dotsc, h_{\tau(n)}\overline h_n).\]
\end{defn}
This is a kind of semidirect product. If $n = [G:H]$, there's a natural homomorphism $\theta\colon G\to\Sigma_n\wr
H$, because $G$ acts by permuting the elements of $H$. Namely, if $g_1,\dotsc,g_n$ is a set of coset
representatives for $G/H$, let $\pi_g\in\Sigma_n$ and $h_i(g)\in H$ be such that $gg_i = g_{\pi(i)} h_i(g)$. Then,
$\theta$ sends $g\mapsto (\pi, h_1(g),\dotsc,h_n(g))$.
\begin{comp}{prop}{enumerate}
	\item\label{ishom} $\theta$ is a group homomorphism.
	\item\label{isocnj} A different choice of coset representatives determines a map conjugate to $\theta$.
	\item\label{action} $\theta$ induces a natural $G$-action on $\bigoplus_{i=1}^n M$, which is naturally a
	$\Sigma_n\wr H$-object.
	\item The $\Z[G]$-module $\bigoplus_{i=1}^n M$ induced by $\theta$ is precisely $\Z[G]\otimes_{\Z[H]} M$.
\end{comp}
\begin{proof}[Partial proof]
For~\eqref{ishom},
\begin{align*}
	(\gamma_1\gamma_2)g &= \gamma_1(\gamma_2 g_i)\\
	&= \gamma_1(g_{\pi_1(i)} h_i(\gamma_2))\\
	&= g_{\pi_2(\pi_1(i))} h_{\pi_1(i)}(\gamma_1)h_i(\gamma_2).
\end{align*}
This plays well with the group structure on $\Sigma_n\wr H$, and implies that $\theta$ is a homomorphism.
\eqref{isocnj} follows from a similar calculation.

The action of $\Sigma_n\wr H$ on $\bigoplus_{i=1}^n M$ is
\[(\sigma,h_1,\dotsc,h_n)\cdot (m_1,\dotsc,m_n) = (h_{\sigma^{-1}(1)}m_{\sigma^{-1}(1)},
h_{\sigma^{-1}(2)}m_{\sigma^{-1}(2)},\dotsc, h_{\sigma^{-1}(n)} m_{\sigma^{-1}(n)}). \qedhere\]
\end{proof}
The key observation is that you can do this in any symmetric monoidal category: we haven't used anything specific
about the symmetric monoidal structure, just the fact that there's an $\Sigma_n$-action on an $n$-fold monoidal
product.
\begin{defn}
Let $M$ be an object in a symmetric monoidal category $\fC$. Its \term{tensor induction} is the $G$-object
\[N_H^GM\coloneqq \theta^*\paren{\bigotimes_{i=1}^n M},\]
where we give the iterated tensor product the $\Sigma_n\wr H$-action described above.
\end{defn}
There are a lot of interesting properties of tensor induction, which you can work out.
\begin{comp}{prop}{enumerate}
	\item Let $H_2\subseteq H_1\subseteq G$. Then, $N_{H_1}^G N_{H_2}^{H_1}\cong N_{H_2}^G$.
	\item $N_H^G(M_1\oplus M_2)$ is isomorphic to $N_H^GM_1\oplus N_H^G M_2$ plus some norms to $K$, where $K$ is a
	proper subgroup containing $\bigcap {^gH}$.
\end{comp}
The Evens norm is a kind of multiplicative transfer. Segal~\citeme{Segal} made this literal, writing down an
equivariant cohomology theory whose transfer map is the Evens norm.
\begin{thm}[Evens]
If $R$ is a Noetherian ring, then $H^*G;R)$ is finitely generated over $R$.
\end{thm}
Evens wrote a book~\cite{EvensBook} on group cohomology which adopts the Evens norm as an organizing principle, and
this theorem is probably proved in there.
\begin{defn}
The \term{Evens norm} is a map $H^p(H;R)\to H^{p^n}(G;R)$ (where $R$ is a $\Z[G]$-algebra) which sends $\alpha\in
H^p(H;R)$ to $\alpha^{\otimes n}$ in $N_H^GR$ (with its $\Sigma_n\wr H$-action), then pulls back by
$\theta^*(\alpha^{\otimes n})$ to define the Evens norm of $\alpha$.
\end{defn}
We will assume $\alpha$ is in even degree: otherwise, there are factors $(-1)$ to the sign of a permutation. The
trick is that instead of using the counit $\Z[G]\otimes_{\Z[H]} i_H^*M\to M$, we use the twisted counit
\[\bigotimes_{i=1}^n R\longrightarrow R\]
coming from multiplication. The interesting piece is that this plays well with the $G$-action: it's really a map
$N_H^G(i_H^*R)\to R$. In other words, it's easy to get a map $H^p(H;M)\to H^{p^n}(G;R^n)$, and we use the ring
structure to get back to $R$. There's a nice analogy of the norm as the multiplicative transfer, where we've
replaced one counit with another. From this perspective, the definition of an $N_\infty$-operad makes more sense.

More explicitly, let $P_\bullet\to R$ be a projective resolution of $R$ as a $\Z[H]$-module, so we obtain a
resolution $P_\bullet^{\otimes n}\to R^{\otimes n}\to R$. Then, let $\e\colon Q_\bullet\to R$ be a resolution of
$R$ as a $\Z[\Sigma_n]$-object (induced from the $\Sigma_n\wr H$-action).\footnote{Don't we need to act on
$R^{\oplus n}$, rather than $R$?} We can join these into a resolution
\[\xymatrix{
	Q_\bullet\otimes_R P_\bullet^{\otimes n}\ar[r] & R\otimes_R R^{\otimes n}\ar[r] & R,
}\]
and given some cocycle $f\colon P_p\to R$, $\e\otimes f^n$ is a cocycle for $H^{pn}$ whose cohomology class is the
Evens norm of $f$. This is far from obvious, but comes from the fact that
\[\xymatrix{
	(P_\bullet^{\otimes n})_{np}\ar[r] & P_p^{\otimes n}\ar[r] & R
}\]
produces cocycles for $H^{np}(H;R)$.\footnote{\TODO: I didn't follow this.}
\begin{comp}{rem}{enumerate}
	\item The Evens norm satisfies a double coset formula, similarly to the transfer map.
	\item The Evens norm is part of a Tambara functor structure where
	\[G/H\mapsto \bigoplus_{i=0} H^{2i}(H; \bl).\]
	\item Multiplicative transfers, e.g.\ power operations, are ubiquitous, e.g.\ the Steenrod operations. In fact,
	one can realize the Steenrod operations as a homotopical version of this norm construction, and this is a
	fruitful approach to research for some homotopy theorists. The analogue of the twisted action is a map
	$E\Sigma_n\times_{\Sigma_n} X^n\to X$, which is used to construct the Steenrod operations. \qedhere
\end{comp}
We'll soon bring this from group cohomology into equivariant stable homotopy theory, where it becomes the
Hill-Hopkins-Ravenel norm.
