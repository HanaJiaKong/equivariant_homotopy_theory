\begin{ques}
Are orthogonal spectra the same as diagram $G$-spectra indexed on all finite-dimensional representations of $G$? If
so, what are the advantages of working with only orthogonal representations? Is orthogonality needed to make sense
of $\Omega$-spectra?
\end{ques}
Much of it is that the representations in question often come with a natural inner product. Also, it's useful to
take orthogonal complements sometimes. Nonetheless, it's probably not strictly necessary
% cpx for k theory
\begin{ques}
What conditions need to be put on a diagram category $\fD$ such that $\fD$-spectra are Quillen equivalent to
orthogonal spectra?
\end{ques}
It's hard to be precise about this, but one thing that's common to all of them is an embedding of $\N$ into $\fD$
producing an adjunction between the forgetful functor and the prolongation functor. We used the forgetful functor
to define the homotopy groups. So if you can embed $\N$ into it, it's probably OK, but it's doubtful that a
complete, explicit classification exists. $\Gamma$-spaces are kind of strange, though.
\begin{ques}
There are geometric fixed-points, homotopy fixed-points, and categorical fixed-points. What's the intuition for
what these three constructions are doing?
\end{ques}
Recall that the categorical fixed points are $(\bl)^H = F(\Sigma_+ G/H,\bl)$, which tom Dieck splitting tells us is
complicated and hard to compute in general. The homotopy fixed points $F(EG_+,\bl)^G$ are more computable: there's
a spectral sequence arising from the filtration of $BG$. Geometric fixed points $\Phi^HX$, which satisfy the
property $\Phi^H\Sigma^\infty X \cong \Sigma^\infty X^H$ (and are uniquely characterized by this, being strong
symmetric monoidal, and preserving homotopy colimits), have nice properties.

You can also analyze them in terms of what they do to $S^{-V}$, i.e.\ the Spanier-Whitehead dual of $S^V$:
geometric fixed points will send it to $S^{-V^H}$, whereas categorical fixed points will look more like a tom Dieck
splitting.

It's possible to write every spectrum as a colimit of suspension spectra, and this ``canonical presentation'' is a
good way to dissect the differences between these fixed points. We might talk more about this later.

Both the categorical and geometric fixed points detect weak equivalences, so you could use, e.g.\ the geometric
fixed points to define homotopy groups.

Tate spectra fit into this story: the Tate spectrum $X^{tC}$ of $X$ fit into a homotopy pullback
diagram~\eqref{tatedefn} where the other three corners are the three kinds of fixed points. This looks like an
arithmetic square, and this analogy can be made precise.
\begin{ques}
Is there a similar story for coinvariants? Are there categorical, homotopical, and geometric coinvariants?
\end{ques}
Not really, so far: Lewis and May remark that the construction you would write down is hard to control, and hard to
derive. It's possible that new work on the foundations of equivariant homotopy theory could enable a better answer,
but nothing's been done right now.
\begin{ques}
Are there notions of naïve and genuine $G$-\emph{spaces}? (This question was asked in the homotopy theory chatroom
on MathOverflow.)
\end{ques}
The terminology is less common than that of naïve and genuine $G$-spectra; in some sense, it's a little weird to
be using terms like ``genuine'' and ``naïve'' in the first place to describe mathematical objects. In any case, by
a genuine $G$-space, we mean a functor out of the orbit category, but a naïve $G$-space would be a functor out of
the category containing just $G/e$, so capturing only the theory of spaces over $BG$.

On the subject of mathematical literature, the HULK smash product~\cite{HulkSmash} is possibly the best
mathematical joke of all time.
\begin{ques}
You've said that Hill-Hopkins-Ravenel~\cite{HHR} has significantly affected the direction of equivariant stable
homotopy theory. Can you elaborate on this?
\end{ques}
There's been some interest in $S^1$-equivariant methods, but a big piece has been a focus on computations, now that
some computations look possible. The use of trace methods is also exciting. There's some work on relating
equivariant homotopy theory to motivic homotopy theory, which looks exciting.

Another generalization is to global homotopy theory; Stefan Schwede has a program for setting this up. One of the
eventual payoffs will be an equivariant $\mathit{TMF}$, which Lurie has sketched a construction of.
\begin{ques}
Has the Tannakian formalism been applied to this context? Maybe instead of a universe of representations, it would
be easier to take a category.
\end{ques}
This hasn't really been considered; it might be interesting, and it might be hard to work with.

\begin{ques}
If you like to think of spectra as infinite loop spaces, what changes in the equivariant case?
\end{ques}
For $G$ a finite group, everything is the same: you can use $\Gamma$-$G$-spaces or $\Gamma_G$-spaces (which are
different notions for the same thing by \cref{shimgamma}); you can use functors from $\Gamma$ to $G$-spaces, and a
few other ones.  Then, loop spaces come from a $E_\infty$-operad action; for technical reasons you may need to
assume there are trivial representations in the universe.

In the case of compact Lie groups, nobody really knows how to set up the theory of infinite loop spaces. In
particular, being an $E_\infty$-$G$-space is not the same as being the zero space of a $G$-spectrum.

One way to think about this is that on the space level, the $E_\infty$ operad controls the transfers. The issue is
that we only know how to handle finite-index transfers and norms; maybe you should put them in freely, but you get
weird obstructions. There's lots of interesting questions to be asked here, but they are hard and likely not worth
your time.
\begin{ques}
What is a cyclotomic spectrum?
\end{ques}
Fix a prime $p$; a \term{$p$-cyclotomic spectrum} is an $S^1$-spectrum $X$ together with data of an equivalence
with its $C_p$-geometric fixed points as $S^1$-spectra.

Recall that the \term{loop space} of $X$ is $LX\coloneqq\Map(S^1,X)$. Its suspension spectrum is an example of a
cyclotomic spectrum: there is an identification $\Phi^{C_p}(\Sigma_+^\infty LX)\cong\Sigma_+^\infty (LX)^{C_p}\cong
\Sigma_+^\infty LX$.

If $X$ is an $S^1$-spectrum, we can take $X^{C_{p^n}}\cong (X^{C_p})^{C_{p^{n-1}}}\to (\Phi^{C_p}
X)^{C_{p^{n-1}}}\to X^{C_{p^{n-1}}}$, and the homotopy limit of these maps is topological cyclic homology.
\begin{ques}
What are cyclonic spectra?
\end{ques}
It's not a standard term, and the professor didn't remember off the top of his head.
\begin{ques}
Is there an analogue of the regular representation in stable homotopy theory?
\end{ques}
The answer is, not really. This relates to the complicated structure of tom Dieck splitting --- even in the case
for compact Lie groups, where the proofs we gave don't completely generalize, there are versions of the
Wirthmüller isomorphism and tom Dieck splitting that have shifts by the tangent representation at the identity
coset of $G/H$. This shift pops up again and again.
\begin{ques}
What is currently happening in EDAG? What roadblocks are currently in the way?
\end{ques}
When you look at TMF, especially with level structure, it really looks like there should be some underlying
equivariant spectrum, and it would be nice to have a way to make this explicit.
% mike (?) and Lennart Meier give nice proofs from this perspective

There are issues with localization behaving poorly, and $\infty$-operads were introduced to make things behave
better. It's also conceptually interesting and cool how the action of an $\infty$-operad controls what transfers
it has, and suspension converts the transfers into norms, and you get a ring spectrum: it's a nice conceptual
picture that explains what's going on.
% \begin{ques}
% When we defined $\Z$-graded (co)homology theories, was there any analogue of the Mayer-Vietoris axiom?
% \end{ques}
% Yes: we required them to send cofiber sequences to long exact sequences. We also used this in the
% proof.
\begin{ques}
What applications have there been besides~\cite{HHR} and trace methods?
\end{ques}
Both of those are good applications. A lot of the interesting work in chromatic homotopy theory regarding, e.g.\
homotopy fixed points of subgroups of the Morava stabilizer group, uses $G$-actions. Stable representation theory
of finite groups also uses this material. Goodwillie calculus also uses equivariant methods: Nick Kuhn proved
in~\cite{Kuhn04} that the vanishing of extensions in the Taylor tower is equivalent to the vanishing of the Tate
spectrum.

There's an analogue of Goodwillie calculus for the equivariant setting, an ``equivariant calculus;'' since calculus
is about passage from the unstable to the stable category, the equivariant stable category plays the analogous role
in equivariant calculus.

Completion with respect to the $(\Sigma^\infty, \Omega^\infty)$-adjunction is important in calculus, in terms of a
simplicial resolution, and it's a little weird that you get the same answer if you complete with the correct or the
trivial choice.\footnote{\TODO: better word than ``choice.''}
\begin{ques}
In a vague sense, the classification of groups splits into two parts: find the simple pieces (finite simple
groups) and figure out how to put then together (which is also hard). Algebraic $K$-theory throws out the second
part: is there any way to understand the category of finitely-presented groups in this way?
\end{ques}
Jeff Smith used a similar method in a different context, but it hasn't been extremely fruitful. People haven't
looked into the $K$-theory of finitely presented groups, and it might be interesting.
\begin{ques}
% the generators of pi_0^G(S) are Euler characteristics...?
What did you mean by the generators of $\pi_0^G(S^0)$ being Euler characteristics as a consequence of tom Dieck
splitting?
\end{ques}
$\pi_0^G(S^0)$ is equivariant homotopy classes of maps from $S^0$ to itself. This is free on conjugacy classes ---
well, what maps generate? The answer is the composites of the maps $S\to\Sigma^\infty G/H_+$ and the crush map back
to $S$: these are Euler characteristics, in that these are equivariant analogues of the maps whose degrees give
Euler characteristics in the nonequivariant case?
\begin{ques}
How much more bananas is the equivariant sphere than the sphere?
\end{ques}
It's way more complicated: it has $\Omega^V S^V$ in it for all $V$, and the decomposition is not nice. Maybe for
specific groups you can say something nice.
\begin{ques}
Knowledge of facts about the sphere spectrum encodes information interesting in other fields. Does knowledge about
the equivariant sphere spectrum tell us anything more than the equivariant generalizations of those results?
\end{ques}
Presently, no such applications are known.
\begin{ques}
Under certain conditions, you can detect the homotopy groups of a spectrum by looking at the $0$ level. Can you do
this for $G$-spectra by looking at the regular representation?
\end{ques}
Better, in fact: in the case of an $\Omega$-$G$-spectrum, you can compute the homotopy groups at the
\emph{trivial} representation.
\begin{ques}
What is the status of Grothendieck's approach to homotopy theory?
\end{ques}
That's a good question; it's not so clear. People like derivators, because they're $2$-categorical and therefore
easier to deal with, but they might go away once the higher-categorical foundations get standardized.
\begin{ques}
In equivariant homotopy theory, there are multiple kinds of suspension functor. How does one then conclude that
cofibers are the same as fibers up to some shift, if there are multiple choices of shift to make?
\end{ques}
This is a great exercise for you to work out.


% TODO: make a fun cover!
