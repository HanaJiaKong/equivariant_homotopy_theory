%!TEX root = m392c_EHT_notes.tex
\label{monads}
\begin{quote}\textit{
	``May your first talk be more peaceful.''
}\end{quote}
Today, we'll discuss a model for the equivariant stable category, but first we'll say something about algebras and
monads that went by quickly last time.

Last time, we introduced two monads $\T$ and $\P$ on $\Spc^\sI$, defined as
\[\T X\coloneqq \bigvee_{k\ge 0} X^{\wedge k}\qquad\qquad \P X\coloneqq \bigvee_{k\ge 0} X^{\wedge k}/\Sigma_k.\]
Let's delve into this a little more.
\begin{defn}
A \term{monad} $M$ on a category $\fC$ is an endofunctor of $\fC$ which is a monoid in the functor category
$\Fun(\fC,\fC)$.
\end{defn}
That is, there's a natural transformation $\mu\colon M^2\to M$ and a unit, and $\mu$ is associative and unital in
that the relevant diagrams commute.

There are lots of examples: we've already seen that if $G$ is a group, the assignment $X\to G\times X$ is a monad.
More generally, algebraic structures can usually be obtained monadically.
\begin{defn}
Let $M$ be a monad on $\fC$. Then, the category $\fC[M]$ of \term{algebras over $M$} is the category whose objects
are pairs $X\in\fC$ and structure maps $m\colon MX\to X$ satisfying associativity and unitality for $M$, and whose
morphisms are the $\fC$-morphisms that are compatible with the structure maps.
\end{defn}
The associativity diagram, for example, is
\[\xymatrix{
	M^2X\ar[r]^m\ar[d]^\mu & MX\ar[d]^m\\
	MX\ar[r]^m & X.
}\]
We require this to commute.

Lots of structures are monadic, e.g.\ groups are algebras over the free group monad in $\Set$, and similarly for
abelian groups, rings, etc. Monads very generally come from free structures in algebra; they also arise from
adjunctions: an adjunction $\adjnctn\fC\fD FG$ defines a monad $GF$, with the structure map defined by
the unit map $G(FG)F\to GF$. Many monads arise in this way.

There is always a free-forgetful adjunction $\adjnctn \fC{\fC[M]}FU$, which has to do with the
Barr-Beck theorem. Suppose $\fC$ is a model category. We'd like to lift this to a model structure on $\fC[M]$ ---
when does $\fC[M]$ have a model structure where the weak equivalences are determined by the forgetful functor
$U$?\footnote{This is not how we defined the model structure on $G$-spaces, but we'll use it to define model
structures on rings and module spectra.} There are two issues.
\begin{enumerate}
	\item\label{caveat1} Since $\fC$ is complete, so is $\fC[M]$: the arrows point the right way. But it's not
	always cocomplete.
	\item How do we define the model structure?
\end{enumerate}
In order for $\fC[M]$ to be cocomplete, we'll use a criterion about preserving certain colimits. Since monads tend
to arise from adjunctions, the criterion definitely won't be true in general! There are seventeen versions of this
criterion in Mac Lane, but we'll only need one, following Hopkins and McClure.
\begin{defn}
Let $f,g\colon X\rightrightarrows Y$ be two maps. A \term{reflexive coequalizer} is a coequalizer for $f$ and $g$
together with a simultaneous section $s\colon Y\to X$ for both $f$ and $g$.
\end{defn}
\begin{ex}
Prove that if $M$ preserves reflexive coequalizers, then $\fC[M]$ is cocomplete. (This is hard, but worthwhile.)
\end{ex}
\begin{prop}[Hopkins-McClure]
Under very mild hypotheses, $\T$ and $\P$ preserve reflexive coequalizers.
\end{prop}
See~\cite{EKMM} for a proof. $\Spc^\sI$ satisfies these hypotheses, so we've addressed caveat~\eqref{caveat1}.
\begin{thm}[Schwede-Shipley~\cite{SchwedeShipley}]
Under mild hypotheses, $\fC[M]$ inherits a model structure from $\fC$, where the fibrations are detected by $U$,
and the generating cofibrations are $M$ applied to the generating cofibrations of $\fC$.
\end{thm}
This is a very general theorem; the hard step is constructing a nice enough filtration on pushouts.
\begin{warn}
These hypotheses are met for the associative monad $\T$, but are \emph{not} met by the commutative monad $\P$!
This is another formulation of Lewis' paradox: for commutative ring spectra, you have to change the underlying
homotopy theory.
\end{warn}
\begin{rem}
You might be used to thinking of these as the associative and commutative operads. Every operad determines a monad,
and the operadic algebras become monadic algebras, but for $\P$ and $\T$, the explicit form of the monad makes it
easier to analyze from the monadic viewpoint.
\end{rem}
Returning to diagram spectra, we've been putting a huge emphasis on strict symmetric monoidal structures, rather
than just commutativity in the homotopy category. This is useful because it lets you do algebra: if $R$ is a ring
in the homotopy category, it's very hard to control the category of modules, e.g.\ the cofiber of a map of modules
may not even be a module, the cyclic bar construction isn't a simplicial object, etc. Some people have tried to use
operads to fix this, and this is extremely hard: operadic ring spectra are fine, but their modules are not. In a
sense, Lurie's $\infty$-categorical machinery is designed to do this in a more modern way.

A lot of modern homotopy theory has been importing algebraic constructions about rings into homotopy theory,
replacing tensor products with smash products. This has been very useful, yet can be hard, and everything is much
more tractable with a point-set symmetric monoidal structure.
\subsection*{$G$-spectra.}
In this context, we must talk about universes, making choices of which orbits $G/H$ are dualizable. This is closely
related to choosing representations $V$ such that $G/H\inj V$ equivariantly.
\begin{defn}
A \term{universe} is a countably infinite-dimensional real inner product space with a $G$-action through
isometries. There is some collection $R$ of irreducible representations such that $U$ contains countably many
copies of each irreducible in $R$, and $R$ always has all of the trivial representations.\footnote{Ideally, $U$
should not contain other representations. Sometimes people are vague about this, where there may be other
representations and you ignore them, but it's better to just not have them.}
\end{defn}
Sometimes we'll ask for more structure, such as $R$ being closed under tensor products. The inclusion of the
trivial representation is what makes this a strict generalization of ordinary stable homotopy theory: for example,
if $U = \R^\infty$ with the trivial $G$-action, the homotopy groups are the nonequivariant homotopy
groups.\footnote{More explicitly, this is a stabilization of the category of presheaves on $BG$. Depending on your
definitions, this is more or less a tautology --- stabilization in $\infty$-categories is defined by taking
spectrum objects.} Another common choice of $U$ is a \term{complete universe}, containing all irreducible
representations infinitely often, so all orbits are dualizable.

$G$-prespectra were defined classically in a manner similar to prespectra. They mix point-set and homotopical data,
which is a little odd.
\begin{comp}{defn}{itemize}
	\item A \term{$G$-prespectrum} $X$ is the data of a $G$-space $X(V)$ for every finite-dimensional subspace
	$V\subset U$ and $G$-maps $S^W\wedge X(V)\to X(V\oplus W)$ for each pair $V$ and $W$, such that the structure
	maps are associative.
	\item A $G$-prespectrum is an \term{$\Omega$-prespectrum} if for all $V$ and $W$, the adjoint to the structure
	map is a homeomorphism: $X(V)\congto\Omega^W X(V\oplus W)$.
	\item The \term{homotopy groups} of a $G$-prespectrum are
	\begin{align*}
		\pi_q^H(X) &\coloneqq \colim_V \pi_q^H \Omega^V X(V)\\
		\pi_{-q}^H(X) &\coloneqq \colim_{\substack{V\supset\R^m\\m > 0}} \pi_0^H\Omega^{V - \R^m}X(V).
	\end{align*}
	Here $q\ge 0$, $\R^m$ denotes the trivial representation of dimension $m$, which we specified was in $U$, and
	$V - \R^n$ denotes the orthogonal complement.
\end{comp}
\begin{prop}
There is a model structure on $G$-prespectra where the weak equivalences are the stable equivalences (equivalently,
$\pi_*$-isomorphisms).
\end{prop}
The proof is identical to the nonequivariant case. One cool aspect of this is that homotopy groups are determined
by trivial representations, and we'll see that for orthogonal $G$-spectra, this is also true at the point-set
level. It's nice to not have to carry around the entire universe, just the trivial representations. We'll use this
philosophy in the proof of \cref{Wirthiso}.
\begin{defn}
The \term{equivariant stable category} (\term{structured by $U$}) is the homotopy category of $G$-prespectra.
\end{defn}
This is the category in which we have Poincaré duality for the orbits $G/H$ that embed in $U$. The functors
$\Sigma^V\coloneqq S^V\wedge\bl$ and $\Omega^V\coloneqq\Map(S^V,\bl)$ are inverse equivalences.
\begin{rem}
It would be nice to write down some sort of equivariant analogue of a triangulated structure on the equivariant
stable category, but trying to get an $\RO(G)$-grading (i.e.\ shifts by arbitrary representations of $G$) does not
work except in ad hoc ways for $G = C_2$. However, you can keep track of the action of the Picard group
$\operatorname{Pic}(\Spc^G)$ on the equivariant stable category.
\end{rem}
\subsection*{Orthogonal $G$-spectra.}
Let's build a point-set model of the equivariant stable category. Let $V$ and $V'$ be finite-dimensional
irreducible $G$-representations (by which we mean finite-dimensional subspaces of $U$), and let $I_G(V,V')$ be the
linear, $G$-equivariant isometries.\footnote{Unlike the nonequivariant case, we're not taking isometric
isomorphisms, so maps in $I_G(V,V')$ must be injective, but may have nontrivial cokernel.}
\begin{defn}
The \term{complement bundle} $E(V,V')$ is the subbundle of the product bundle $I_G(V,V')\times V'$ consisting of
pairs $(f,x)$ such that $x\in V' - f(V)$. Let $\wI_G(V,V')$ denote the Thom space of this bundle.
\end{defn}
We're trying to create spheres: $\wI_G(V,V')$ should topologize $S^{V' - f(V)}$. For example, if $V\subset V'$,
this is isomorphic to $\O(V')\wedge_{\O(V'-V)} S^{V' - V}$.

Let $\wI_G$ denote the category whose objects are finite-dimensional subspaces $V$ of $U$ and whose morphisms are
$\wI_G(V,V')$.
\begin{defn}
An \term{orthogonal $G$-spectrum} is a $\Top$-enriched functor $\wI_G\to G\Top$. The category of orthogonal
$G$-spectra is denoted $\Spc^G$, or $\Spc^G_U$ if the universe needs to be explicit.
\end{defn}
For the nonequivariant case, we defined $\fD$-spaces and spectra to be modules over a certain monoid in them, and
then showed that you could think of them as diagram spaces for a more complicated diagram. Here, we've done this
all at once --- it's simpler to define $\wI_G$ than find a module in a simpler category of diagram spaces.
% see the definition for Mackey functors later
% \begin{exm}[Equivariant Eilenberg-Mac Lane spectra]
% Let $G$ be finite and $M$ be a countable $\Z[G]$-module (equivalently an abelian group with an action of $G$).
% Following \cref{orthogonal_HA}, we can define an \term{equivariant Eilenberg-Mac Lane spectrum} $HM$ to be the
% orthogonal $G$-spectrum that assigns to a representation $V$ the $M$-linearization $HM(V)\coloneqq M[S^V]$.  Here,
% the orthogonal group acts through its action on $S^V$, and $G$ acts on both $M$ and $S^V$ from the right. The
% structure map $S^V\wedge M[S^W]\to M[S^{V\oplus W}]$ is as in~\eqref{orth_HA_structure_map}.
% 
% If $M$ is also a $\Z[G]$-algebra, then $HM$ is an orthogonal $G$-ring spectrum through the multiplication map
% $\mu\colon M[S^V]\wedge M[S^W]\to M[S^{V\oplus W}]$ defined by
% \[\paren{\sum_i m_i\cdot v_i}\wedge\paren*[\Bigg]{\sum_j m_j'\cdot w_j} \mapsto \sum_{i,j} (m_im_j')\cdot
% (v_i\wedge w_j).\]
% This example is discussed further in~\cite[Example 2.13]{SchwedeEquivariant}.
% \end{exm}

There is a forgetful functor $U$ from $\Spc^G$ to the category of $G$-prespectra; we define a map $f$ in $\Spc^G$
to be a \term{$\pi_*$-isomorphism} if $Uf$ is a $\pi_*$-isomorphism. $\Spc^G$ is symmetric monoidal under Day
convolution, just as before.
\begin{thm}
$\Spc^G$ has a stable model structure in which the weak equivalences are $\pi_*$-isomorphisms, the fibrations are
levelwise fibrations, and the generating cofibrations are
\[F_V\paren{(G/H\times S^{n-1})_+\longrightarrow (G/H\times D^n)_+}.\]
\end{thm}
Here, $F_V$ is adjoint to evaluation at $V$, as before.

Moreover, the smash product is compatible with the model structure. If $X$ is cofibrant, then $X\wedge Y$ computes
Tor $X\wedge^\LD Y$. Another way to say this is that if $X$ is cofibrant, $X\wedge\bl$ preserves weak equivalences.
The proof is where cofibrantly generated model categories: $X\wedge\bl$ is a colimit, so it commutes with colimits,
and cofibrant objects are retracts of cell complexes, so you can reduce to the case where $X$ is a single cell,
which admits a direct proof. Once again the stable homotopy category behaves like homological algebra.
\subsection*{Functors on $\Spc^G$.}
Let $H\subset G$ be a subgroup, so restriction defines a functor $i_H\colon \Spc^G\to\Spc^H$. Just as for
$G$-spaces, $i_H$ has a left adjoint $G_+\wedge_H\bl$ and a right adjoint $F_H(G_+,\bl)$, and these are constructed
space-wise.
\begin{prop}
$G_+\wedge_H\bl$ and $F_H(G_+,\bl)$ are left (resp.\ right) Quillen functors.
\end{prop}
\begin{rem}
You might worry what universe you end up in after applying $i_H$ --- naïvely you get the $H$-representations that
arise from restriction of $G$-representations, but you might want all $H$-representations. The way to address this
is to use a change-of-universe functor, which we'll mention shortly. This is a good thing to not be sloppy about,
but is annoying, and is why people try to get the universe out of the point-set category. In any case, as in group
cohomology, induction and coinduction are isomorphic through the \term{Wirthmüller isomorphism}
(\cref{Wirthiso}).
\end{rem}
One thing that's nice to do with $G$-spectra is take fixed points. There are \emph{three} different ways to do
this, namely homotopy fixed points, categorical fixed points, and geometric fixed points, and keeping them separate
in your head is important.

$\Spc^G$ is a \emph{closed} symmetric monoidal category, meaning it has internal function objects. This is true for
$\fD$-spaces in general, with internal function objects
\[F_\fD(X,Y)(n) \coloneqq F(X, Y(n)).\]
This is exactly like the mapping complex of two chain complexes: you're looking at chain maps from $X$ to shifts of
$Y$.

$\Spc^G$ is tensored and cotensored over $G$-spaces, meaning we can smash $G$-spectra with $G$-spaces, and take
function objects from $G$-spaces to $G$-spectra: if $A$ is a $G$-space, tensoring with $A$ is $A_+\wedge\bl$
applied levelwise, and cotensoring with $A$ is $F(A_+,\bl)$ applied levelwise.
\begin{comp}{defn}{itemize}
	\item The \term{homotopy fixed points} is the functor $X\mapsto F(EG_+, X)^G$. (Here, $EG_+$ is a $G$-space, so
	we're using the cotensor of $G$-spaces and $G$-spectra.)
	\item The \term{categorical fixed points} is the functor $(\bl)^H\colon X\mapsto F(G/H_+, X)$. As the notation
	suggests, this is analogous to $(\bl)^H$ on $G$-spaces.
\end{comp}
\begin{warn}
You might hope that $(\Sigma^\infty A_+)^H = \Sigma^\infty A_+^H$, so that fixed points of spaces lead to fixed
points of spectra, but this is \emph{not} true. It has a more complicated description called \term{tom Dieck
splitting} as a wedge of other pieces.
\end{warn}
\begin{defn}
The \term{geometric fixed points}, denoted $\Phi^H$ or also $X^{gH}$, is the unique functor
such that
\begin{enumerate}
	\item it's (derived) symmetric monoidal: if $X$ and $Y$ are cofibrant,
	$\Phi^H(X)\wedge\Phi^H(Y)\congto\Phi^H(X\wedge Y)$.
	\item $\Phi^H$ preserves homotopy colimits.
	\item $\Phi^H\Sigma^\infty X \cong \Sigma^\infty X^H$.
\end{enumerate}
\end{defn}
Of course, uniqueness (up to a contractible space of choices) is a theorem, but it's true. We'll also give a
preferred construction.

We'll prove that categorical and geometric fixed points detect weak equivalences.
\begin{comp}{thm}{enumerate}
	\item A map $X\to Y$ in $\Spc^G$ is a weak equivalence iff $X^H\to Y^H$ is a weak equivalence for all subgroups
	$H\subset G$.
	\item A map $X\to Y$ in $\Spc^G$ is a weak equivalence iff $\Phi^H X\to \Phi^H Y$ is a weak equivalence for all
	subgroups $H\subset G$.
\end{comp}

This leads one to consider diagrams on categorical or geometric fixed points: because of this perspective, one can
define Tate spectra using this philosophy, and this is particularly useful when applied to $S^1$-spectra. It was
originally developed by John Greenlees, and has recently been repopularized and is being mined for applications,
such as Nikolaus-Scholze's description of topological cyclic homology.

We're going to try to understand the equivariant stable category from a few other perspectives: the Wirthmüller
isomorphism is an analogue of the isomorphism between finite products and coproducts in spectra, and we'll also
learn more about tom Dieck splitting. You can also ask how the transfer maps we talked about
characterize the equivariant stable category.
