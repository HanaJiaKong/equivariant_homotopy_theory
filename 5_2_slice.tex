In this section, we discuss the slice spectral sequence, the equivariant version of the Postnikov spectral
sequence. Very little is known about the slice spectral sequence, so you could pursue it in your research. For
example, its use in~\cite{HHR} depended on an explicit identification of the slices with something else, which does
not generalize.

\subsection*{Postnikov towers.} For this section, we work in the category $\Top$ of spaces. We want a collection
of functors $P_n\colon\Top\to\Top$, called \term{Postnikov sections}, together with natural transformations $P_n\to
P_{n-1}$ such that
\begin{itemize}
	\item
	\[\pi_i(P_nX) = \begin{cases}
		\pi_i(X), &0\le i\le n\\
		0, &\text{otherwise}
	\end{cases}\]
	\item $X\cong\holim P_nX$, and
	\item the homotopy fiber of $P_nX\to P_{n-1}X$ is weakly equivalent to $K(\pi_n(X), n)$.
\end{itemize}
So we obtain a \term{Postnikov tower} $\dotsb\to P_2X\to P_1X\to P_0X$, which is in a sense dual to the cellular
filtration.
\begin{rem}
The axiomatization of the idea of a Postnikov tower in a triangulated category is called a \term{$t$-structure}.
The dual notion of a cellular filtration is axiomatized as a \term{weight structure}.
\end{rem}
These $P_nX$ have only finitely many nonzero homotopy groups, hence are going to be very large spaces. There will
also be different point-set models for them. Here's one model.

Choose an $\alpha\in\pi_{n+1}(X)$, which defines a homotopy class of maps $f_\alpha\colon S^{n+1}\to X$. Let
$\widetilde X$ be the pushout
\[\xymatrix{
	S^{n+1}\ar[r]^{f_\alpha}\ar[d] & X\ar[d]\\
	D^{n+2}\ar[r] & \widetilde X.\pushout
}\]
This kills $\alpha$, and since $S^{n+1}$ is $n$-connected, the induced map $\pi_k(X)\to\pi_k(\widetilde X)$ is an
isomorphism for $k < n+1$. Then we can iterate.

This has a fatal flaw: it's not functorial. As usual, we fix this with the small object argument. Consider
\emph{all} of the maps $S^{n+1}\to X$ and take the pushout
\[\xymatrix{
	\bigvee S^{n+1}\ar[r]\ar[d] & X\ar[d]\\
	\bigvee D^{n+2}\ar[r] & X_{n+1}.
}\]
Continue this way for maps $S^N\to X$ for $N\ge n+1$ and let $P_nX\coloneqq \colim_N X_N$. This is functorial.

% Mandell-Shipley telescope in title
$P_nX$ can be described as localization, which allows for some slick high-tech setups, e.g.\ if you work with
presentable $\infty$-categories, you can describe $P_n$ as adjoint to the inclusion of spaces with homotopy groups
within $[0,n]$

The Postnikov tower leads to the \term{Atiyah-Hirzebruch spectral sequence} (AHSS), which for a generalized
homology theory $E$, has signature
\[E_{p,q}^2 = H_p(X; E_q(*))\Longrightarrow E_{p+q}(X).\]
\begin{ex}[Maunder~\cite{Maunder}]
If you play the same game with the CW filtration, you obtain an isomorphic spectral sequence. It's a good exercise
to work this out yourself.
\end{ex}
This works in more than just spaces: you can set it up for spectra, and (with a little technical work) for
(commutative) ring spectra.
\subsection*{The equivariant case.}
We'd like to do this in $\Spc^G$. The slice spectral sequence will be an analogue to the Atiyah-Hirzebruch spectral
sequence. One difficulty will be understanding the associated graded, which is considerably more complicated than
the nonequivariant Postnikov or CW associated graded complexes, and this is ultimately because there are more
spheres around.

The first piece of the slice spectral sequence was worked out by Dugger~\cite{DuggerKR}. His motivation was the
analogy between $C_2$-equivariant homotopy theory and motivic homotopy theory.  \citeme{Bloch-Lichtenbaum,
Grayson} The goal was to approach the Quillen-Lichtenbaum conjecture, the existence of a spectral sequence
\[H^p(X; \Z(-\e/2))\Longrightarrow K^{p+q}(X).\]
Here, $H^*(\bl,\Z(-\e/2))$ is ``motivic cohomology,'' which was not well-understood when the conjecture was
formulated. The $-\e/2$ is a \term{Tate twist}, which is akin to Bott periodicity for algebraic $K$-theory of
finite fields. See~\citeme{Bloch-Lichtenbaum} and Grayson~\cite{Grayson} for details or Mitchell~\cite{Mitchell}
for an exposition. Dugger replaced this with \TODO, using Atiyah's $\mathit{KR}$-theory~\cite{AtiyahKR} as the
target.

The general formulation of the slice spectral sequence was worked out in~\cite{HHR}; see also
Hu-Kriz-Ormsby~\cite{HKO11}. The exposition in~\cite{HHR} is pretty good, and you should also check out Mike Hill's
introduction~\cite{HillSlice}.  There are also some worked-out computations with the splice spectral sequence due
to Hill~\cite{HillRealBordism}, Yarnall~\cite{Yarnall}, Hill-Hopkins-Ravenel~\cite{HHR_HZ, HHR_C4},
Hill-Meier~\cite{HillMeier}, Hill-Yarnall~\cite{HillYarnall}, and Greenlees~\cite{GreenCalc}.

To get at the spectral sequence, we first approach the filtration. The motivation is to have a Postnikov section
$P_H$ for the regular representation $\rho_H$ of $H$. The \term{slice cells} will be the cells
\[\set{G_+\wedge_H S^{m\rho_H}, \Sigma^{-1}G_+\wedge_H S^{m\rho_H}}.\]
The \term{dimensions} of these slice cells are $m\abs H$, resp.\ $m\abs H - 1$.

Slice cells are well-behaved under the ``change functors'' $i_K^*$, $G_+\wedge_K\bl$, and $N_K^G$: all of these
preserve slice cells.
\begin{defn}
Let $X\in\Spc^G$.
\begin{itemize}
	\item $X$ is \term{slice $n$-null} if $\Map(\hat S, X)$ is contractible (as a $G$-space) for all slice
	cells $\hat S$ of dimension greater than $n$. One also says $X$ is \term{slice $\le n$}.
	\item $X$ is \term{slice $n$-positive} (also \term{slice $\ge n$}) if $\Map(\hat S, X)$ is contractible (as
	a $G$-space) for all slice cells $\hat S$ of dimension at most $n$.
	\item $X$ is \term{slice $n$-positive}\footnote{\TODO: these two notions are different, and so we need
	different names for them.} if there's a filtration $X_0\subseteq X_1\subseteq\dotsb\subseteq X$ such that
	$X_i/X_{i-1}$ is a wedge of slice cells of dimensions greater than $n$.
\end{itemize}
\end{defn}
We localize at the slices of dimension greater than $n$ to obtain $P^nX$, the \term{$n$-slice} of $X$. This is a
bit tricky, because there are maps $S^V\to S^W$ when $V\subseteq W$ that aren't null-homotopic, frustrating our
approach to defining the Postnikov tower, but the key is that if $V$ contains a trivial representation, all such
maps are null-homotopic.

Localization means there's a map $X\to P^nX$ by fiat, and one can show that the homotopy fiber of this map is slice
$n$-positive. We obtain a tower $\dotsb\to P_nX\to P_{n-1}X\to \dotsb$. Let $P_n^nX$ denote the homotopy fiber of
$P_nX\to P_{n-1}X$.\footnote{\TODO: superscripts or subscripts?}

\cite{HHR} heavily use the fact that the regular
representation contains a copy of all irreducibles to understand what's going on, which is great for finite groups,
but doesn't work for compact Lie groups. There's a different perspective adopted by Dugger, that $G/H_+\wedge S^n$
detects homotopies, so we can restrict to slice cells $G/H_+\wedge S^V$ where $V$ contains a copy of the trivial
representation (and $\Sigma^{-1}$ of these cells). The slice filtration has some other nice properties: $\holim
P^nX\cong X$, and $\colim P^nX\simeq *$.

These localizations are controlled by the subcategory of $\Spc^G$ determined by positive shifts and cofibers of
slice cells. Negative shifts are not allowed, so when passing to the homotopy category, this isn't a triangulated
subcategory.

This slice filtration produces the \term{slice spectral sequence}, which is a strongly convergent spectral sequence
\begin{equation}
\label{SlSS}
E_2^{s,t} = \pi_s^G P_t^tX\Longrightarrow \pi_{s-t}^G X.
\end{equation}
Here we use the Adams grading, so this may look funny compared to the usual grading on, e.g.\ the Serre spectral
sequence. You can refine the slice spectral sequence into a spectral sequence of Mackey functors, and there is an
$\RO(G)$-graded version. If you think about it, you'll see this is a first- and third-quadrant spectral sequence.

The key to understanding the slice spectral sequence is understanding the slices $P_t^tX$, and this is hard.
In~\cite{HHR}, they use the fact that real bordism has nice slices, and~\citeme{Dugger} works out some computations
for $C_2$, but there are few calculations in the literature, and more would be welcome.
%
%This concludes the content of the course; on Thursday there will be donuts and discussion of future research. Some
%of these are important open questions that may be intractable, and others could be research projects; hopefully
%it will be clear which is which.

These notes will hopefully be a living document; feel free to update with better explanations, or to fix mistakes,
or to add examples, or anything like that. There are relatively few references for equivariant homotopy theory, and
it would be great for the notes to be one more, hopefully in a presentable way.
