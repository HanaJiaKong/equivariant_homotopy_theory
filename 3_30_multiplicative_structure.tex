\begin{quote}\textit{
	``If you liked it then you should have put a ring on it.''
}\end{quote}
\TODO: this section may need to be reorganized.

Now, we'll shift gears and talk about ring structures on $G$-spectra. This will lead eventually to multiplicative
structures on $\RO(G)$-graded cohomology theories, through Green functors (simpler) and Tambara functors (more
sophisticated), but the topology comes from the algebra, so let's start with the algebra.\index{Green
functor}\index{Tambara functor}

By construction, $\Spc^G$ is symmetric monoidal: it's a diagram category, and as such has a symmetric monoidal Day
convolution. By a \term{ring spectrum} we mean a monoid in $\Spc^G$, and by a \term{commutative ring spectrum} we
mean a commutative monoid.\index{Day convolution}

One issue with this approach: why is there only one kind of ring? We found lots of different kinds of abelian
groups in $\Spc^G$: different subcategories of $\sO_G$ or different universes $U$ lead to different definitions. In
other words, a choice of abelian group structure amounts to a choice of compatible transfers. We want a parallel
story for rings, meaning we'll want some sort of ``multiplicative transfer'' maps.\index{multiplicative
transfer|see {norm}}

We'll say this in two ways: first in an impressionistic way, then more formally.

\begin{ex}
Show that in any symmetric monoidal category $\fC$ (e.g.\ $\Spc$, $\Spc^G$, $\Ab$), the coproduct in the category
of commutative monoid objects in $\fC$ is the same as the monoidal product in $\fC$.
\end{ex}
This implies, for example, that the tensor product is the coproduct for rings.
Commutative rings are tensored over finite sets, and multiplication is encoded by maps of sets: given a finite set
$\underline n$, we let $R\otimes\underline n\cong R^{\lor n}$, and the multiplication map is the universal map
$R^{\vee n}\to R$, i.e.\ the coproduct. The point is that this structure determines a ring structure on an abelian
group, so we can use it in less familiar settings.

In $\Spc^G$, this says that equivariant multiplication should be controlled by tensoring with $G$-sets (at least
when $G$ is a finite group): that is, if $T$ and $S$ are finite $G$-sets, we want a ring structure to be defined by
maps
\[\bigvee_T R\longrightarrow R\qquad\qquad\text{and}\qquad\qquad\bigvee_T
R\longrightarrow\bigvee_S R.\]
This is where the other notions of rings come from: it's reasonable to restrict to certain subcollections of finite
$G$-sets. In $G$-spaces, this amounts to choosing transfers, and for $G$-spectra, this amounts to choosing
multiplicative transfers, which are called norms; suspension will send transfers to norms.\index{norm}

Operadic multiplication will make this precise. $\Spc^G$ and $G\Top$ are enriched over $G\Top$, so we'll work with
operads in $G\Top$.\footnote{There are more general settings you can work in, e.g.\ using operads in $\Spc^G$ or
$\Spc$ in the stable setting. This is reasonable, but we won't need to do this.}\index{operad}
\begin{defn}
A \term[G-operad@$G$-operad]{$G$-operad} is a collection of $G\times\Sigma_n$-spaces $\set{O(n)}$ along with
structure maps
\[s_{k,\vec n}\colon O(k)\times O(n_1)\times O(n_2)\many\times O(n_k)\longrightarrow O(n_1\many + n_k)\]
which are suitably equivariant, i.e.\ $G$-equivariant and equivariant under the permutations in $\Sigma_{n_1\many+
n_k}$ that arise as block permutations in $(n_1,\dotsc,n_k)$, and are associative and unital.
\end{defn}
You should think of the structure map $s_{k,\vec n}$ as grafting leaves from $O(n_1),\dotsc,O(n_k)$ onto a tree
$O(k)$.
\begin{defn}
Let $O$ be a $G$-operad. An \text[algebra!over a $G$-operad]{$O$-algebra}, in either $G\Top$ or $\Spc^G$, is an
object $X$ with structure maps
\[O(n)\times_{\Sigma_n} X^{\times n}\longrightarrow X\]
compatible with the operadic multiplication. Here, $\times$ denotes the product in $G\Top$ or $\Spc^G$.
\end{defn}
\begin{exm}\hfill
\label{Gopexm}
\begin{enumerate}
	\item Any operad in $\Top$, such as the associative or commutative operads, defines a $G$-operad with trivial
	$G$-action.
	\item Let $U$ be a $G$-universe, and $O(n)$ be the space of linear isometries from $U^{\oplus n}$ to $U$; the
	structure maps come from composition.\footnote{Interestingly, there's something important about spaces here: if
	you take chains, the operadic axioms fail.} This operad is called $\mathcal L_U$, the \term{linear isometries
	operad}.\index{universe}
	\qedhere
\end{enumerate}
\end{exm}
\begin{defn}[Blumberg-Hill~\cite{NInfty}]
\label{ninfty}
An \term[Ninfinity-operad@$N_\infty$-operad]{$N_\infty$-operad}\footnote{The $N$ in $N_{\infty}$ stands for
``norm.''} is a $G$-operad $O = \set{O(n)}$ such that
\begin{enumerate}
	\item $\Sigma_n$ acts freely on $O(n)$ (one says $O(n)$ is \term[Sigman-free G-operad@$\Sigma_n$-free
	$G$-operad]{$\Sigma_n$-free}), and
	\item $O(n)$ is a universal space for some family $\sF_n$ of subgroups of $G\times\Sigma_n$ containing
	$H\times\set 1$ for all subgroups $H\subseteq G$. That is, if $\Gamma\subset G\times\Sigma_n$,
	\[O(n)^\Gamma = \begin{cases}
		*, &\Gamma\in\sF_n\\
		\varnothing, &\text{otherwise.}
	\end{cases}\]
	\item $O(n)$ is nonequivariantly contractible.\footnote{This means that if you forget from $G$-spaces to
	spaces, $N_\infty$-operads become $E_\infty$-operads.\index{Einfinity-operad@$E_\infty$-operad}}
\end{enumerate}
\end{defn}
That $O(n)$ is $\Sigma_n$-free places strong constraints on $\set{\sF_n}$. For example, if $\Gamma\in\sF_n$, then
$\Gamma\cap\paren{\set 1\times\Sigma_n} = \set 1$.
\begin{lem}
\label{admlem}
Let $\Gamma\subseteq G\times\Sigma_n$ be a subgroup. Then, the following are equivalent.
\begin{enumerate}
	\item $\Gamma\cap(\set 1\times\Sigma_n) = \set 1$.
	\item $\Gamma$ is the graph of a homomorphism, i.e.\ there's an $H\subseteq G$ and a group homomorphism
	$p\colon H\to\Sigma_n$ such that $\Gamma = \set{(h,p(h))\mid h\in H}$.
\end{enumerate}
\end{lem}
The proof is two lines, neither of which is too hard. In the context of a family $\sF_n$, subgroups therefore
define maps $p\colon H\to\Sigma_n$ and therefore $H$-set structures on $\underline n$; such $H$-sets will be called
\term[admissible $H$-set]{admissible}.\index{family of subgroups}
\begin{exm}
The operad $\mathcal L_U$ from \cref{Gopexm} is $N_\infty$, which is a nontrivial fact: unlike for
finite-dimensional spaces, the space of linear isometries from $U^n$ to $U$ is contractible. This is a nice thing
to think through. The rest of the axioms are easier.
\end{exm}
You might wonder whether the little discs operad appears in this context; it turns out that fitting everything
together into a colimit behaves poorly, and so one uses something else called the Steiner operad, which maybe will
appear in these notes.\index{little $n$-cubes operad}\index{Steiner operad}

In the nonequivariant case, the linear isometries operad is a model for the $E_\infty$-operad. But in the
equivariant case, there are $N_\infty$-operads that aren't the linear isometries operad, and this is a disturbing
representation-theoretic fact. The goal is to figure out what the right notion of a multiplicative $\RO(G)$-graded
cohomology theory is --- in the nonequivariant case, ring spectra are the objects that represent multiplicative
cohomology theories, and we want to discover the equivariant analogue.\index{RO(G)-graded cohomology
theory@$\RO(G)$-graded cohomology theory!multiplicative structure}

Operadic multiplication is one way to approach this, an algebraic (or infinite loop space) approach to this by
thinking of $\Spc^G$ as algebras in spaces.\index{infinite loop space}

In the nonequivariant case, $\Omega^\infty$ defines an equivalence between connective spectra\index{connective
spectrum} and grouplike $E_\infty$-spaces ($X$ is \term[grouplike $E_\infty$ space]{grouplike} if $\pi_0(X)$ is a
group). These are spaces that have homotopy coherent abelian group structures, and is a fancy way of saying that
spectra behave like abelian groups. We want to do this equivariantly, and the big question therein is: \emph{how do
we operadically encode transfers?}

Let's first recall what's going on in the nonequivariant case. We'll think of connective spectra as \term[infinite
loop space]{infinite loop spaces}: if $X$ is an $\Omega$-spectrum, $X_0 \cong \Omega X_1 \cong \Omega^2 X_2 \cong
\dotsb$, and therefore is an $n$-fold loop space for all $n$.

We want to recognize $n$-fold loop spaces: how do you know when $X\simeq\Omega^n Y$ for some $Y$? One way to
approach this is to define a delooping construction $B$, and be able to use it $n$ times. Beck (of Barr-Beck fame)
provided an answer in~\cite{Beck}: $(\Sigma^n, \Omega^n)$ define an adjunction $\Top\rightleftarrows\Top$, whose
associated comonad is $\Sigma^n\Omega^n$ and whose associated monad is $\Omega^n\Sigma^n$.

What Beck observed is that if $X = \Omega^n Y$, $X$ admits an action of the monad $\Omega^n\Sigma^n$:
\[\xymatrix{
	\Omega^n\Sigma^n (\Omega^n Y) = \Omega^n (\Sigma^n\Omega^n) Y\ar[r]^-\eta &\Omega^n Y,
}\]
where $\eta\colon\Sigma^n\Omega^n\to\id$ is the counit of the adjunction.
\begin{thm}[Beck \cite{Beck}]
If $B(\Sigma^m, \Omega^n\Sigma^n, \Omega^n Y)$ is the simplicial bar construction, i.e.\ the simplicial
space\index{simplicial bar construction}
\[[q]\mapsto (\Sigma^m)(\Omega^n\Sigma^n)\Omega^n Y,\]
then its geometric realization is equivalent to $\Omega^{n-m}Y$:
\[\abs{B(\Sigma^m, \Omega^n\Sigma^n, \Omega^n Y)}\simeq \Omega^{n-m}Y.\]
\end{thm}
This recognition principle is nice: it means that if $Z$ is any algebra over $\Omega^n\Sigma^n$, then $Z$ is an
$n$-fold loop space. Great! Except that we don't know any $\Omega^n\Sigma^n$-algebras other than $\Omega^n Y$: they
don't arise in nature.

Boardman-Vogt and May realized it's convenient to work with operads $C_n$ such that there's a map of monads from
$\overline C_n\to\Omega^n\Sigma^n$.
\begin{defn}
Let $O$ be an operad. Then, $\overline O$ denotes the associated monad\index{monad!associated to an operad}
\[X\mapsto\bigvee_n O(n)\times_{\Sigma_n} X^n.\]
\end{defn}
The point is that algebras for the monad $\overline O$ are the same thing as algebras for the operad $O$, but the
monadic category is easier to deal with. For example, you can use this construction to show that $\Alg_O$ is
complete and cocomplete under mild hypotheses on $O$.\index{algebra!over an operad}\index{algebra!over a monad}
\begin{exm}
The \term{little $n$-cubes operad} $C$ is the operad such that $C_n(m)$ is the space of configurations of $m$
disjoint closed cubes in an $n$-dimensional cube. There is a map $\overline C_n\to\Omega^n\Sigma^n$ which arises
from an action of $\overline C_n$ on
\[C_n(m)\times (\Omega^n X)^m\to \Omega^n X\]
The idea\footnote{\TODO: I don't understand this.} is, given $m$ $n$-cubes inside a big $n$-cube, crush the
boundary of each cube to a point, and send everything outside the little cubes to the basepoint. It's not too hard
to write this out explicitly.
\end{exm}
This is a fattened-up version of a configuration space, and so the topology of $C_n(m)$ is already very interesting
(and why the little $n$-cubes operad can be used to define $E_n$-algebras).\index{En-algebra@$E_n$-algebra}

Now, given a connected $C_n$-space $X$, it's possible to deloop through the bar construction:\index{bar
construction}
\[\xymatrix{
	X & B(C_n,C_n,X)\ar[l]_-\simeq\ar[r]^-{\theta_1} & B(\Omega^n\Sigma^n, C_n, X)\ar[r]^-{\theta_2} &\Omega^n
	B(\Sigma^n, C_n, X).
}\]
The map $B(C_n,C_n, X)\to X$ is the easy part: it's an equivalence by standard methods, and should be thought of as
the free $C_n$-resolution of $X$. The map $\theta_1$ comes from the map $C_n\to\Omega_n\Sigma_n$, and is an
equivalence. But $\theta_2$ is hard: it's a theorem that geometric realization commutes with homotopy limits in
this context, and uses the fact that $X$ is connected. But the point is, with all this setup, $C_n$-algebras can be
delooped. This approach is studied in~\cite{MayGILS}.

Now, taking the sequential colimit along $C_n\to C_{n+1}\to\dotsb$, we get an $E_\infty$-operad whose algebras are
called \term[Einfinity-space@$E_\infty$-space]{$E_\infty$-spaces}.
\begin{defn}
An operad $O$ is an \term[Einfinity-operad@$E_\infty$-operad]{$E_\infty$-operad} if $O(n)$ is contractible for all
$n$ and $\Sigma_n$ acts freely on each $O(n)$.
\end{defn}
\begin{thm}
If $X$ is an $E_\infty$-space and $\pi_0(X)$ is a group, then $X\cong\Omega^\infty Z$ for a spectrum $Z$.
\end{thm}
That is, we can deloop $X$ arbitrarily many times. This is the classical way to understand infinite loop spaces.

Suppose $f\colon O\to O'$ is a map of operads. Then, we get functors between algebras in both directions: there's a
pullback $f^*\colon\Top[O']\to\Top[O]$ and a \term[change of scalars!between operads]{change of scalars} operad
$B(O',O,\bl)\colon\Top[O]\to\Top[O']$; the bar construction model for this is a nice thing to have
around.\index{algebra!over an operad}

When do these induce an equivalence on the categories of algebras? The answer for $E_\infty$-operads,
is when $f_n\colon O(n)\to O'(n)$ is a (nonequivariant) equivalence.\footnote{The \term{commutative operad}, for
which $O(n) = *$ for all $n$, parametrizes commutative algebras, and it's not $E_\infty$! This is a source of much
frustration; for example, it's the reason that you generally have to use a different model structure on categories
of commutative ring spectra.} But there's always a map, given by the push-pull construction on
\[\xymatrix@dr{
	O\times O'\ar[r]\ar[d] & O'.\\
	O
}\]
So it really doesn't matter which $E_\infty$-operad you use.
\subsection*{The equivariant case.} What does this look like equivariantly? We defined
$N_\infty$-operads to be those for which $O(0)\simeq *$, the $\Sigma_n$-action on $O(n)$ is free, and $O(n)\simeq
E\sF_n$ for some family $\sF_n$ of $G\times\Sigma_n$ that contains $H\times\set 1$ (see
\cref{ninfty}).\index{Ninfinity-operad@$N_\infty$-operad}

We're going to show that algebras for an $N_\infty$-operad $O$ in $\Top$ will correspond to some kinds of $G$-spectrum
with transfer structures by $\set{\sF_n}$. If $O = \mathcal L_U$ (the linear isometries operad in \cref{Gopexm}),
then we get $G$-spectra dictated by the universe $U$, but these are not the only examples.\index{universe}

This leaves a few questions.
\begin{enumerate}
	\item Why do we get transfer maps? How can we see this from the operadic structure?\index{transfer map}
	\item What can we say about the collection $\set{\sF_n}$? You can't just pick any of them: they have
	complicated interrelationships.
\end{enumerate}
Looking at the second question, suppose $\Gamma\subset G\times\Sigma_n$ is such that $\Gamma\cap\paren{\set
1\times\Sigma_n} = \set 1$. We proved in \cref{admlem} that this is equivalent to $\Gamma$ being the graph of a
homomorphism $\rho\colon H\to\Sigma_n$, which is equivalent to the data of an $H$-set of cardinality $n$. We called
such $H$-sets admissible.
\begin{defn}
Let $\cat{Cat}$ denote the category of small categories and functors, and $\operatorname{\otimes\cat{Cat}}$ denote
the category of small symmetric monoidal categories and strong monoidal functors.
\begin{itemize}
	\item A \term{categorical coefficient system} is a functor $\sO_G\op\to\cat{Cat}$.
	\item A \term{symmetric monoidal categorical coefficient system} (SMCCS) is a functor
	$\sO_G\op\to\operatorname{\otimes\cat{Cat}}$.\index{SMCCS|see {symmetric monoidal categorical coefficient
	system}}
\end{itemize}
\end{defn}
\begin{exm}
The most important example is $\underline\Set$, the functor sending $G/H\mapsto H\Set$, the category whose objects
are $H$-sets, morphisms are $H$-maps, and monoidal product is disjoint union. $\underline\Set$ is a symmetric
monoidal categorical coefficient system.

Similarly, $\underline\Top$ denotes the functor sending $G/H\mapsto H\Top$, with monoidal product disjoint union.
This is also a symmetric monoidal categorical coefficient system.
\end{exm}
\begin{lem}
Let $\set{\sF_n}$ be a collection of families of subgroups of $G$. Then, $\set{\sF_n}$ arises from an
$N_\infty$-operad iff it's a sub-SMCCS of $\underline\Set$.\footnote{\TODO: did I get this right?}\index{family of
subgroups}
\end{lem}
That is, these are exactly the functors sending $G/H$ to admissible $H$-sets.\index{admissible $H$-set}
\begin{defn}
An \term{indexing system} $\underline I$ is a sub-SMCCS of $\underline\Top$ which
is
\begin{enumerate}
	\item closed under self-induction,
	\item closed under Cartesian product,
	\item closed under passage to self-objects, and
	\item contains all trivial sets.
\end{enumerate}
Here, \term{self-induction} means that if $H/K\in\underline I(H)$ and $T\in\underline I(K)$, then $H\times_K
T\in\underline I(H)$.
\end{defn}
These axioms appear somewhat arbitrary, but there's a nice categorical description in terms of bispans, which we'll
discuss later.\index{bispan}
\begin{defn}
Let $O$ be an $N_\infty$-operad, so that it defines a collection of families $\set{\sF_n}$ of subgroups of $G$.
Let $\underline O$ denote the functor sending $G/H$ to the collection of admissible $H$-sets for
$\set{\sF_n}$.\index{Ninfinity-operad@$N_\infty$-operad}
\end{defn}
\begin{prop}
\label{Ninfind}
Let $O$ be an $N_\infty$-operad; then, $\underline O$ is an indexing system.
\end{prop}
We say that a map $f\colon O\to O'$ of $N_\infty$-operads is a \term{homotopy equivalence} if it defines a
$G\times\Sigma_n$-homotopy equivalence $O(n)\to O'(n)$ for all $n$.
\begin{thm}
The assignment $O\mapsto\underline O$ is a fully faithful embedding from the homotopy category of
$N_\infty$-operads to the poset of indexing systems.
\end{thm}
In fact, this can be upgraded to an equivalence of categories.
\begin{thm}[Rubin~\cite{Rubin}]
The assignment $O\mapsto\underline O$ is an equivalence of categories from the homotopy category of
$N_\infty$-operads to the poset of indexing systems.
\end{thm}
\begin{proof}[Partial proof of \cref{Ninfind}]
It's clear that trivial sets are admissible for an $N_\infty$-operad $O$.\footnote{As a corollary, this means that
if $O$ is an $N_\infty$-operad, then $i_e^*O$ is an $E_\infty$-operad, where $i_e\colon\Top\to G\Top$ prescribes
the trivial action.\index{Einfinity-operad@$E_\infty$-operad}}

We want to show that $\underline O$ is closed under coproduct: if $S$ and $T$ are admissible $H$-sets, we'd like
for $S\amalg T$ to be admissible. Well, $S$ is the graph of a $p_S\colon H\to\Sigma_{\abs S}$ and $T$ is the graph
of $P_T\colon H\to\Sigma_{\abs T}$. Let $\Gamma_S$ be the associated subgroup of $G\times\Sigma_{\abs S}$, and
define $\Gamma_T$ in the same way.

Let
\[\Gamma\coloneqq \set{(h, p_S(h)\amalg p_T(h))\mid h\in H}.\]
This acts on $S\amalg T$, and gives it the structure of an admissible $H$-set. Given the map
\[O(2)\times O(\abs S)\times O(\abs T)\longrightarrow O(\abs S + \abs T),\]
we'd like to show that the $\Gamma$-fixed point set of $O(\abs S + \abs T)$ is nonempty. But by hypothesis,
$(O(2)\times O(\abs S)\times O(\abs T)^\Gamma$ is nonempty, so we're done.

The pattern of proof is the same for the rest of the conditions.
\end{proof}
We've been talking about transfers, so let's see why algebras over an $N_\infty$-operad have
transfers.\index{algebra!over an $N_\infty$-operad}

Let $T$ be an admissible $H$-set. Then, the space of maps
\[G\times\Sigma_{\abs T}/\Gamma_T\longrightarrow O(\abs T)\]
is contractible.
\begin{lem}
\[(G\times\Sigma_{\abs T}/\Gamma_T)\times_{\Sigma_{\abs T}} X^{\abs T} \cong G\times_H F(T,X).\]
\end{lem}
% ???
Thus, if $X$ is an algebra over an $N_\infty$-operad $O$, we have maps
\[\xymatrix{
	G\times_H F(T,X)\ar[r] & (G\times\Sigma_{\abs T}/\Gamma_T)\times\Sigma_{\abs T} X^{\abs T}\ar[r] & O(\abs
	T)\times_{\Sigma_{\abs T}} X^{\abs T}\ar[r] & X,
}\]
and this is where the transfer maps on $\pi_0(X)$ come from.\index{transfer map!for an $N_\infty$-algebra}

If $X$ is a $G$-space and $T$ is an $H$-set, let
\[N^TX\coloneqq G\times_H F(T, i_H^*X).\]
\begin{lem}
A map $S\to T$ of $H$-sets induces a map $N^S X\to N^T X$.
\end{lem}
The proof is the same as for the construction of the transfer maps $H/K\to H/H$: build this transfer, and then
induce up. In general, an $H$-map $S\to T$ induces a map
\[F(T, i_H^*X)\longrightarrow F(S, i_H^*X).\]
Passing to fixed points and taking $\pi_0$, this is precisely a system of transfers on $\pi_0$.
\begin{rem}
One consequence of this is a conceptual description of the double coset formula: the restriction to $K'$ of a
transfer on $T$ is just the transfer on $i_{K'}T$. We'll unpack this in the next section.\index{double cosets}
\end{rem}
Another conceptual takeaway is that $N_\infty$-operads are the right way to discuss transfers in the equivariant
setting. So we may as well port them to $G$-spectra, since they're also enriched over $G$-spaces. But when you do
this, you stumble upon one of the basic conundrums of equivariant homotopy theory.

Let $X$ be a space. Then, $X^n$ is a $C_n$-space, where $C_n$ acts by permuting the factors. The diagonal map
$\Delta\colon X\to X^n$ induces an equivalence $X\cong (X^n)^{C_n}$. The problem is to do this with spectra: we
want a map $\Sp\to\Sp^{C_n}$ such that we have a diagonal.

The naïve approach is to send $X\mapsto X^{\wedge n}$, but what universe does $X^{\wedge n}$ live in? Naïvely,
we're in $U^n$, which maps back down to $U$. But in $U^n$, we have good fixed-point behavior for representations of
the form $(V,V,\dotsc,V)$ --- and only for representations of this form. So it's not clear how to make this
work.\index{universe}

Bökstedt's construction of $\mathit{THH}$~\cite{Bokstedt} needed something like this, and so Bökstedt did the
minimal amount necessary to get something to work. The technical solution to this uses ideas from group cohomology,
specifically the Evens norm, which we'll discuss in \S\ref{evenssec}. Figuring out how to make this work in
homotopy theory was one of the important new ideas in~\cite{HHR}.\index{topological Hochschild
homology}\index{Evens norm}

We'll eventually get to ring spectra and spectra with norm maps, which encode a kind of twisted multiplication.

