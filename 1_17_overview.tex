%!TEX root = m392c_EHT_notes.tex

This class will be an overview of equivariant stable homotopy theory. We're in the uncomfortable position where
this is a big subject, a hard subject, and one that is poorly served by its textbooks. Algebraic topology is like
this in general, but it's particularly acute here. Nonetheless, here are some references:
\begin{itemize}
	\item Adams, ``Prerequisites (on equivariant stable homotopy) for Carlsson's lecture.''~\cite{Adams84}. This is
	old, and some parts of it don't reflect how we do things now.
	\item The Alaska notes~\cite{AlaskaNotes}, edited by May, are newer, and are written by many authors. Some
	parts are a grab bag, and some parts (e.g.\ the rational equivariant bits) aren't entirely right. The notes are
	also not a textbook.
	\item Appendix A of Hill-Hopkins-Ravenel~\cite{HHR}. This is a paper which resolved an old conjecture on
	manifolds using equivariant stable homotopy theory, but let this be a lesson on referee reports: the authors
	were asked to provide more background, and so wrote a 150-page appendix on this material. Their suffering is
	your gain: the appendix is a well-written introduction to equivariant stable homotopy theory, albeit again not
	a textbook.
\end{itemize}

There are arguably two very serious modern applications of equivariant stable homotopy theory: 

\begin{itemize}
	\item The first is trace methods in algebraic $K$-theory: Hochschild homology and its topological cousins are
	equipped with natural $S^1$-actions (the same $S^1$-action coming from field theory). This is how people
	other than Quillen compute algebraic $K$-theory.
	\item The other major application is Hill-Hopkins-Ravenel's settling of the Kervaire invariant $1$ conjecture
	in~\cite{HHR}.
\end{itemize}
The nice thing is, however you feel about the applications, both applications require developing new theory in
equivariant stable homotopy theory. Hill-Hopkins-Ravenel in particular required a clarification of the foundations
of this subject which has been enlightening.

In this class, we hope to cover the foundations of equivariant stable homotopy theory. On the one hand, this will
be a modern take, insofar as we emphasize the norm and presheaves on orbit categories (these will be explained in
due time), the modern emerging consensus on how to think of these things, different than what's written in
textbooks. The former is old, but has gained more attention recently; the latter is new. Moreover, there's an
increasing sense that a lot of the foundations here are best done in $\infty$-categories. We will not take this
approach in order to avoid getting bogged down in $\infty$-categories; moreover, this class is supposed to be
rigorous. It will sometimes be clear to some people that $\infty$-categories lie in the background, but we won't
talk very much about them.

We'll cover some old topics such as Smith theory and the Segal conjecture, and newer ones such as trace methods
and Hill-Hopkins-Ravenel, depending on student interest. We will not have time to discuss many topics, including
equivariant cobordism or equivariant surgery theory.
\subsection*{Prerequisites.} If you don't know these prerequisites, that's okay; it means you're willing to read
about them on your own.
\begin{itemize}
	\item Foundations of unstable homotopy theory at the level of May's \textit{A Concise Course in Algebraic
	Topology}~\cite{ConciseCourse}. For example, we'll discuss equivariant CW complexes, so it will help to know
	what a CW complex is.
	\item A little bit of category theory, e.g.\ found in Mac Lane~\cite{MacLane} or Riehl~\cite{RiehlCTC}.
	\item This class will not require much in the way of simplicial methods (simply because it's hard to reconcile
	simplicial methods with non-discrete Lie groups), but you will want to know the bar construction. An excellent source for this is \cite[Chapter 4]{RiehlCHT}.
	\item A bit of abstract homotopy theory, e.g.\ what a model structure is. Good sources for model categories are \cite[Part III]{RiehlCHT} and \cite{Hovey}.
\end{itemize}
If you don't know these, feel free to ask the professor for references. His advisor suggested learning
nonequivariant stable homotopy theory by reading Lewis-May-Steinberger's book~\cite{LMS} on equivariant stable
homotopy theory and letting $G = *$, but this may not appeal to everyone. In any case, perhaps this isn't
necessarily a good reference for nontrivial groups.

Unstable equivariant questions are very natural, and somewhat reasonable. But stable questions are harder; they
ultimately arise from reasonable questions, but their formulations and answers are hard: even discussing the
equivariant analogue of $\pi_0S^0$ (see~\eqref{piGs0}) requires some representation theory --- and yet of course it
should. Thus there's a lot of foundations behind hard calculations. There will be problem sets; if you want to
learn the material (or are an undergrad), you should do the problem sets.
\subsection*{Categories of topological spaces.}
The category of topological spaces we consider is $\Top$, the category of compactly generated, weak Hausdorff
spaces (and continuous maps); we'll also consider $\Top_*$, the category of based, compactly generated, weak
Hausdorff spaces and continuous, based maps. This is an important and old trick which eliminates some pathological
behavior in quotients. It's reasonable to imagine that point-set topology shouldn't be at the heart of foundational
issues, but there are various ways to motivate this, e.g.\ to make $\Top$ more resemble a topos or the category of
simplicial sets.
\begin{defn}
Let $X$ be a topological space.
\begin{itemize}
	\item A subset $A\subseteq X$ is \term{compactly closed} if for every compact Hausdorff space $Y$ and $f\colon
	Y\to X$, $f^{-1}(A)$ is closed.
	\item $X$ is \term{compactly generated} if every compactly closed subset of $X$ is closed.
	\item $X$ is \term{weak Hausdorff} if the diagonal map $\Delta\colon X\to X\times X$ is closed when $X\times X$
	has the compactly generated topology.
\end{itemize}
\end{defn}
The intuition behind compact generation is that the topology is determined by compact Hausdorff spaces. The weak
Hausdorff condition is strictly stronger than $T_1$ (points are closed), but strictly weaker than being Hausdorff.
Any space you can think of without trying to be pathological will meet these criteria.

There is a functor $k$ from all spaces to compactly generated spaces which adds the necessary closed sets. This has
the unfortunate name of \term{$k$-ification} or \term{kaonification};\footnote{Kaonification is of course distinct
from koanification, the process which makes statements more confusing.} by putting the compactly generated topology
on $X\times X$, we mean taking $k(X\times X)$. There's also a ``weak Hausdorffification'' functor $w$ which makes a
space weakly Hausdorff, which is some kind of quotient.\footnote{The $k$ functor is right adjoint to the forgetful
map, which tells you what it does to limits.}

When computing limits and colimits, it's often desirable to compute them in the category of spaces and then apply
$k$ and $w$ to return to $\Top$. This works correctly for limits, but for colimits, $w$ is particularly badly
behaved: you cannot compute the colimit in $\Top$ by computing it in $\Set$ and figuring out the topology. In
general, you'll need to take a quotient.

Nonetheless, there are nice theorems which make things work out anyways.
\begin{prop}
Let $Z = \colim(X_0\to X_1\to X_2\to\dots)$ be a sequential colimit (sometimes called a \term{telescope}); if each
$X_i$ is weak Hausdorff, then so is $Z$.
\end{prop}
\begin{prop}
Consider a diagram
\[\xymatrix{
	A\ar[r]^f\ar[d] &B\\
	C
}\]
where $f$ is a closed inclusion. If $A$, $B$, and $C$ are weakly Hausdorff, then $B\amalg_A C$ is weakly Hausdorff.
\end{prop}
These are the two kinds of colimits people tend to compute, so this is reassuring.

One reason we require regularity on our topological spaces is the following, which is not true for topological
spaces in general.
\begin{lem}
Let $X$, $Y$, and $Z$ be in $\Top$; then, the natural map
\[\Map(X\times Y, Z)\longmapsto\Map(X, \Map(Y, Z))\]
is a homeomorphism.
\end{lem}
\subsection*{Enrichments.}
The categories $\Top$ and $\Top_*$ are enriched over themselves (as are categories of $G$-spaces, which we'll see
soon). This merits a brief digression into enriched categories.
\begin{defn}
Let $(\fV,\otimes, \boldsymbol 1)$ be a symmetric monoidal category.\footnote{Briefly, this means $V$ has a tensor
product $\otimes$ and a unit $\boldsymbol 1$; there are certain axioms these must satisfy.} Then, an
\term{enrichment} of a category $\fC$ over $\fV$ means
\begin{itemize}
	\item for every $x,y\in\fC$, there is a hom-object $\underline\fC(x,y)$, which is an object in $\fV$,
	\item for every $x\in\fC$, there is a unit $\fV$-morphism $\boldsymbol 1\to\underline\fC(x,x)$,
	\item composition $\underline\fC(x,y)\otimes\underline\fC(y,z)\to\underline\fC(x,z)$ is associative and unital,
	and
	\item the underlying category is recovered as $\fC(x,y) = \Map(\boldsymbol 1, \underline\fC(x,y))$.
\end{itemize}
\end{defn}
A great deal of category theory can be generalized to enriched categories, including $\fV$-enriched functors,
$\fV$-enriched natural transformations, $\fV$-enriched limits and colimits, and more. The canonical reference is
Kelley~\cite{Kelley}, available free and legally online. It covers just about everything we need except for the Day
convolution, which can be read from Day's thesis~\cite{DayThesis}. Another good source, with a view towards
homotopy theory, is \cite[Chapter 3]{RiehlCHT}.
\begin{defn}
Let $\fC$ and $\fD$ be enriched over $\fV$. Then, an \term{enriched functor} $F\colon\fC\to\fD$ is an assignment
of objects in $\fC$ to objects in $\fD$ and maps $\underline\fC(x,y)\to\underline\fD(Fx, Fy)$ that are $\cat
V$-morphisms, and commute with composition.
\end{defn}
A category enriched over $\Top$ is called a \term{topological category}.
\begin{ex}
Work out the definition of enriched natural transformations.
\end{ex}
This brings us to the beginning.

\orbreak

Let $G$ be a group. We'll generally restrict to finite groups or compact Lie groups; this is not because these are
the only interesting groups, but rather because they are the only ones we really understand. If you can come up
with a good equivariant homotopy theory for discrete infinite groups, you will be famous. Throughout, keep in mind
the examples $C_p$ (the cyclic group of order $p$, sometimes also denoted $\Z/p$), $C_{p^n}$, the symmetric group
$S_n$, and the circle group $S^1$.

There's a monad\footnote{We're going to say more about monads in \S\ref{monads}.} $M_G$ on $\Top$ which sends
$X\mapsto G\times X$, and analogously a monad $M_G^*$ on $\Top_*$ sending $X\to G_+\land X$. One can define the
category of \term{$G$-spaces} $G\Top$ (resp.\ \term{based $G$-spaces} $G\Top_*$) to be the category of algebras
over $M_G$ (resp.\ $M_G^*$). This is probably not the most explicit way to define $G$-spaces, but it makes it
evident that $G\Top$ and $G\Top_*$ are complete and cocomplete.

More explicitly, $G\Top$ is the category of spaces $X\in\Top$ equipped with a continuous action $\mu\colon G\times
X\to X$. That is, $\mu$ must be associative and unital. Associativity is encoded in the commutativity of the
diagram
\[\xymatrix{
	G\times G\times X\ar[r]^{1\times\mu}\ar[d]^m & G\times X\ar[d]^\mu\\
	G\times X\ar[r]^\mu & X.
}\]
The morphisms in $G\Top$ are the \term{$G$-equivariant} maps $f\colon X\to Y$, i.e.\ those commuting with $\mu$:
\[\xymatrix{
	G\times X\ar[r]\ar[d]^{\mu_X} & G\times Y\ar[d]^{\mu_Y}\\
	X\ar[r]^f & Y.
}\]
It's possible (but not the right idea) to let $\underline G$ denote\footnote{There isn't really a standard notation for this category, but the closest is $BG$. This notation emphasizes the fact that groupoids are Quillen equivalent to 1-truncated spaces.} the category with an object $*$ such that
$\underline G(*, *) = G$. Then, $G\Top$ is also the category of functors $\underline G\to\Top$, with morphisms as
natural transformations. This realizes $G\Top$ as a \term{presheaf category}; it will eventually be useful to do
something like this, but in a different way described by Elmendorf's theorem (\cref{elmendorf}).

When we write $\Map(X, Y)$ in $G\Top$ or $G\Top_*$, we could mean three things:
\begin{enumerate}
	\item\label{geqset} The set of $G$-equivariant maps $X\to Y$.
	\item\label{geq} The space of $G$-equivariant maps $X\to Y$ in the subspace topology of all maps from $X\to Y$.
	As this suggests, $G\Top$ admits an enrichment over $\Top$ (resp.\ $G\Top_*$ admits an enrichment over
	$\Top_*$).
	\item\label{justmap} The $G$-space of all maps $X\to Y$, where $G$ acts by conjugation: $f\mapsto
	g^{-1}f(g\cdot)$. This realizes $G\Top$ as enriched over itself, and similarly for $G\Top_*$.
\end{enumerate}
Each of these is useful in its own way: for constructions it may be important to be self-enriched, or to only look
at $G$-equivariant maps. We will let $\Map^G(X,Y)$ or $\Map(X,Y)$ denote~\eqref{geq} or its underlying
set~\eqref{geqset}, and $G\Map(X,Y)$ denote~\eqref{justmap}.

It turns out you can recover $\Map^G$ from $\Map$: the equivariant maps are the fixed points under conjugation of
all maps. This is written $\Map(X,Y)^G = \Map^G(X,Y)$.

Throughout this class, ``subgroup'' will mean ``closed subgroup'' unless specified otherwise.
\begin{defn}
Let $X$ be a $G$-set and $H\subseteq G$ be a subgroup. Then, the \term{$H$-fixed points} of $X$ is the space
$X^H\coloneqq \set{x\in X\mid hx = x\text{ for all } h\in H}$. This is naturally a $\WH$-space, where $\WH = \NH/H$
(here $\NH$ is the normalizer of $H$ in $G$).\footnote{If $H\trianglelefteq G$, then $X^H$ is also a $G/H$-space.}
\end{defn}
\begin{defn}
The \term{isotropy group} of an $x\in X$ is $G_x\coloneqq \set{h\in G\mid hx = x}$.
\end{defn}
Isotropy groups are useful in the following two ways.
\begin{enumerate}
	\item Often, it will be helpful to reduce questions from $G\Top$ to $\Top$ using $(-)^H$.
	\item It's also useful to induct over isotropy types.
\end{enumerate}
Now, we'll see some examples of $G$-spaces.
\begin{exm}
Let $H$ be a subgroup of $G$; then, the \term{orbit space} $G/H$ is a useful example, because it corepresents the
fixed points by $H$. That is, $X^H \cong G\Map(G/H, X)$. These spaces will play the role of points when we build
things such as equivariant CW complexes.
\end{exm}
\begin{exm}
Let $H\subset G$ as usual and $U\colon G\Top\to H\Top$ be the forgetful functor. Then, $U$ has both left and right
adjoints:
\begin{itemize}
	\item The left adjoint sends $X$ to the \term{balanced product} $G\times_H X\coloneqq G\times X/\sim$, where
	$(gh, x)\sim (g, hx)$ for all $g\in G$, $h\in H$, and $x\in X$. Despite the notation, this is \emph{not} a
	pullback! (In the based case, the balanced product is $G_+\wedge_H X$.) $G$ acts via the left action on $G$.
	This is called the \term{induced $G$-action} on $G\times_H X$.
	\item The right adjoint is $F_H(G,X)$ (or $F_H(G_+, X)$ in the based case), the space of $H$-maps $G\to X$,
	with $G$-action $(gf)(g') = f(g'g)$. This is called the \term{coinduced $G$-action} on
	$F_H(G,X)$.\footnote{This actually \emph{is} a group action, since if $a,b,g\in G$, then $(a(b f))(g) =
	(bf)(ga) = f(gab) = (ab(f))(g)$.} \qedhere
\end{itemize}
\end{exm}

\begin{rem}
Here is a categorical perspective on ``change of group.'' Quite generally, a group homomorphism $G\morph^f H$
induces adjunctions
\[\dadjnctn[3em] {G\Top}{H\Top}{f_!}{f^*}{f_*}.\]
These are given by $f_!(X) \coloneqq H\times_G X$ and $f_*(X) \coloneqq F_G(H, X)$ for a $G$-space $X$, where $H$
is given the structure of a $G$-space by $f$. When $H=*$, an $H$-space is just a space, and $f_!(X) = X_G$ is the
space of orbits while $f_*(X) = X^G$ is the space of fixed points. Observe that similar statements hold for
categories of modules, given a ring homomorphism $R\morph^f S$.

In fact, these are both cases of very general abstract nonsense. Let $BG$ denote the category with one object $*$
with $\Hom(*,*)=G$; as we have said above, we can (naïvely) write $G\Top$ as the functor category $\Top^{BG}$. A group homomorphism $G\morph^f H$ induces a functor $BG\morph^F BH$ (it is not quite true that the two are equivalent---think about why this is). Now $f^*\colon H\Top\to G\Top$ is just restriction along $F$:
% 
\[\xymatrix{
	BG \ar[d]_-F \ar@{..>}[r]^-{f^*(Y)} & \Top\\
	BH \ar[ur]_-Y
}\]
% 
According to abstract nonsense, restriction along $F$ has left and right adjoints, called \term{left and right Kan
extension along $F$}, respectively:
\[
\vcenter{\xymatrix{
	BG \ar[d]_-F \ar[r]^-X \ar@{}[dr]|(0.33){\Downarrow\eta} & \Top\\
	BH \ar@{..>}[ur]_-{f_!(X) = \Lan_F X} &
}}
\qquad\qquad
\vcenter{\xymatrix{
	BG \ar[d]_-F \ar[r]^-X \ar@{}[dr]|(0.33){\Uparrow\epsilon} & \Top\\
	BH \ar@{..>}[ur]_-{f_*(X) = \Ran_F X} &
}}
\]
These diagrams do not commute, but there are natural transformations $X \twomorph^\eta f^*f_!(X)$ and $f^*f_*(X)
\twomorph^\epsilon X$. When $H$ is the trivial group, $BH$ is the trivial category, and it is known that left/right
Kan extensions of a functor $X$along a functor to the trivial category pick out the colimit/limit of $X$. That is,
still viewing a $G$-space $X$ as a functor $BG\to\Top$, we have $X_G = \mathop{\colim}\limits_{BG} X$ and $X^G =
\lim\limits_{BG} X$.

For an introduction to Kan extensions, we recommend \cite[Chapter 6]{RiehlCTC} (which is almost the same as \cite[Chapter 1]{RiehlCHT} but with some more amusing examples). Like much of category theory, this is ultimately all trivial, but it may be highly non-trivial to understand why it is trivial.
\end{rem}

\begin{exm}
Let $V$ be a finite-dimensional real representation of $G$, i.e.\ a real inner product space on which $G$ acts in a
way compatible with the inner product. (This is specified by a group homomorphism $G\to\O(V)$.) The one-point
compactification of $V$, denoted $S^V$, is a based $G$-space; the unit disc $D(V)$ and unit sphere $S(V)$ are
unbased spaces, but we have a quotient sequence
\[\xymatrix{
	S(V)_+\ar[r] & D(V)_+\ar[r] & S^V.
}\]
If $V = \R^n$ with the trivial $G$-action, $S^V$ is $S^n$ with the trivial $G$-action, so these generalize the
usual spheres; thus, these $S^V$ are called \term{representation spheres}.
\end{exm}
We will let $S^n$ denote $S^{\R^n}$, our preferred model for the $n$-sphere with trivial $G$-action.
\subsection*{Beginnings of homotopy theory.}
\begin{defn}
A \term{$G$-homotopy} is a map $h\colon X\times I\to Y$ in $G\Top$, where $G$ acts trivially on $I$. We generally
think of it, as usual, as interpolating between $h(\bl, 0)$ and $h(\bl, 1)$. This is the same data as a path in
$G\Map(X,Y)$. A \term{$G$-homotopy equivalence} between $X$ and $Y$ is a map $f\colon X\to Y$ such that there
exists a $g\colon Y\to X$ and $G$-homotopies $gf\sim \id_X$ and $fg\sim\id_Y$.
\end{defn}
The (well, a) natural question that might arise: what are $G$-weak equivalences and $G$-CW complexes? This closely
relates to obstruction theory: CW complexes are test objects.

To define $G$-CW complexes, we need cells. One choice is $G/H\times D^{n+1}$ and $G/H\times S^n$, where the actions
on $D^{n+1}$ and $S^n$ are trivial. This is a plausible choice (and in fact, will be the right choice), but it's
not clear why --- why not $G\times_H D(V)$ or $G\times_H S(V)$ for some $H$-representation $V$? Ultimately, this
comes from a (quite nontrivial) theorem that these can be triangulated in terms of the cells $G/H\times D^{n+1}$
and $G/H\times S^n$.\footnote{Illman's thesis~\cite{IllmanThesis} is a reference, albeit not the most accessible
one.} This is one of several triangulation results proven in the 1970s which are now assumed without comment, but
if you like this kind of math then it's a very interesting story.
\begin{defn}
A \term{$G$-CW complex} is a sequential colimit of spaces $X_n$, where $X_{n+1}$ is a pushout
\[\xymatrix{\bigvee G/H\times S^n\ar[r]\ar[d] & X_n\ar[d]\\
\bigvee G/H\times D^{n+1}\ar[r] & \pushout X_{n+1},
}\]
where $H$ varies over all closed subgroups of $G$.
\end{defn}
That is, it's formed by attaching cells just as usual, though now we have more cells.

This immediately tells you what the homotopy groups have to be: $[G/H\times S^n, X]$, which by an adjunction game
is isomorphic to $\pi_n(X^H)$. We let $\pi_n^H(X)\coloneqq \pi_n(X^H)$. Thus, we can define weak equivalences.
\begin{defn}
A map $f\colon X\to Y$ of $G$-spaces is a \term{weak equivalence} if for all subgroups $H\subset G$,
$f_*\colon\pi_n^H(X)\to\pi_n^H(Y)$ is an isomorphism.
\end{defn}
These homotopy groups have a more complicated algebraic structure: they're indexed by the lattice of subgroups of
$G$ and the integers. This is fine (you can do homological algebra), but some things get more complicated,
including asking what the analogue of connectedness is! (We'll broach this in \cref{equivariant_connected}.)

One quick question: do we need all subgroups $H$? What if we only want finite-index ones? The answer, in a very
precise sense, is that if you're willing to use fewer subgroups, you get fewer cells $G/H\times S^n$, and that's
fine, and you get a different kind of homotopy theory.

Finally, the Whitehead theorem (\cref{eqWhite}) is true for $G$-CW complexes. This follows for the same reason as
in May's course: it follows word-for-word after proving the equivariant HELP (homotopy extension lifting property)
lemma (\cref{HELP}) , which is true by the same argument as in the nonequivariant case.

We'll next talk about presheaves on the orbit category, leading to Bredon cohomology.
