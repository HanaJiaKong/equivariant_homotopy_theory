Today, we're going to finish the computation of the $\RO(C_2)$-graded cohomology of a point with coefficients in
the constant Mackey functor $\underline\Z$ and compute $H_{C_2}^{p,q}(*; A)$, where $A = A_{C_2}$ is the Burnside
Mackey functor.

First, though, let's recall some Mackey functors from \cref{Mackeyexm}. We'll write them out carefully so as to not
make a mistake. The constant Mackey functor $\underline\Z$ is
\[\xymatrix{
	\Z\ar@/^0.4cm/[d]^2\ar@(ul,ur)^\id\\
	\Z,\ar@/^0.4cm/[u]^\id
}\]
where the top entry corresponds to the orbit $C_2/e$, and the bottom to the orbit $C_2/C_2$. There's another Mackey
functor, denoted $\underline\Z\op$, given by the diagram
\[\xymatrix{
	\Z\ar@/^0.4cm/[d]^\id\ar@(ul,ur)^\id\\
	\Z.\ar@/^0.4cm/[u]^2
}\]
If $\pi\colon C_2\times X\to X$ is projection onto the second factor and $M$ is a Mackey functor, we get maps going
both ways:
\[\xymatrix{
	H_{C_2}^{p,q}(X;M)\ar@<0.1cm>[r]^-{\pi^*} & H_{C_2}^{p,q}(C_2\times X; M).\ar@<0.1cm>[l]^-{\pi_!}
}\]
Finally, it'll be good to remember a fact about the dimension axiom: the transfers in a Mackey functor $M$ coincide
with the transfer maps in cohomology induced by maps between orbits.

Anyways, we were computing $H_{C_2}^{p,q}(*)$, where $p$ is the \term{fixed dimension} and $q$ is the \term{sign
dimension}, i.e.\ indexed by the $C_2$-representation $V = \R^p\oplus\sigma^q$, where $\R$ is the trivial
representation and $\sigma$ is the sign representation. First, we computed that
\[H_{C_2}^{p,q}(C_2)\cong H^{p+q}(\pt) = \begin{cases}
	\Z, &p+q = 0\\
	0, &\text{otherwise.}
\end{cases}\]
If $q > 0$, then we computed
\[H^{-p.-q}_{C_2}(*) \cong \wH_{C_2}^{p,0}(S^{\sigma q})\cong\wH^p(\Sigma\RP^{q-1}),\]
because the Eilenberg-Steenrod axioms meant that for a free action, we can compute the nonequivariant cohomology of
the quotient.

To compute $H_{C_2}^{p,q}(*)$, we use duality:
\[H_{C_2}^{p,q}(*) \cong \wH^{p,q}_{C_2}(S^0)\cong \wH_{-p,-q}(S^0)\cong \wH_{-p,0}(S^{\sigma q}),\]
and that's where we left off. To continue, we'll use the cofiber sequence~\eqref{signed_cofiber}, as well as its
based version
\begin{equation}
\label{based_signed}
\xymatrix{
	S(\sigma q)_+\ar[r] & D(\sigma q)_+\ar[r] & S^{\sigma q},
}
\end{equation}
and it's worth noting that the middle space is equivalent to $S^0$, which is sometimes implicit in the literature.

If $k\ne 0,1$, apply $\wH_{k,0}$ to~\eqref{signed_cofiber} to conclude that
\[\wH_{k,0}(S^{q\sigma})\cong \wH_{k-1}(S(\sigma q))\]
and therefore that
\[\wH_{C_2}^{p,q}(*)\cong \wH_{-p-1}(\RP^{q-1}).\]
Recall that the cohomology of $\RP^m$ is
\[H^i(\RP^m) = \begin{cases}
	\Z, &i = 0\\
	\Z, &i = m\text{ and $m$ is odd}\\
	\Z/2, &0 < i < m\text{ and $i$ is odd}\\
	0, &\text{otherwise,}
\end{cases}\]
so we get a bunch of copies of $\Z$ and $\Z/2$.

That leaves $k = 0,1$, for which we must look at the long exact sequence
\[\xymatrix{
	0\ar[r] & \wH_{1,0}(S^{q\sigma})\ar[r] & \wH_{0,0}(S(q\sigma))\ar[r]^-\gamma &\Z\ar[r] &
	\wH_{0,0}(S^{q\sigma})\ar[r] & 0.
}\]
\begin{claim}
$\gamma$ is multiplication by $2$, and therefore $\wH_{1,0}(S^{q\sigma})\cong 0$ and
$\wH_{0,0}(S^{q\sigma})\cong\Z/2$.
\end{claim}
\begin{proof}
The idea is to identify the map $\gamma$ as the transfer map for the Mackey functor $\underline\Z$. Said another
way, we want to see $\gamma$ as coming from the map $H_{0,0}(C_2)\to H_{0,0}(\pt)$ associated to the map $C_2\to
*$. This follows from the dimension axiom and the fact that $S(\sigma)\cong C_2$.
\end{proof}
So now, how do we express this in terms of Mackey functors? This is only nontrivial in the $C_2$ orbit and when the
cohomology is nonzero (the zero Mackey functor is not hard to define). Thus, we care about the case $p+q = 0$.

Consider the long exact sequence
\[\xymatrix{
	H_{C_2}^{-(2n-1), 2n-1}(*) & H_{C_2}^{-2n,2n}(C_2)\ar[l]\\
	& H_{C_2}^{-2n,2n}(C_2)\ar[u]_\zeta & H_{C_2}^{-2n,2n-1}(*),\ar[l]
}\]
which comes from~\eqref{signed_cofiber} when $q = 1$, i.e.\ the cofiber sequence $C_2\to S^0\to S^\sigma$. The map
$\zeta$ is the transfer in the Mackey functor.
\begin{ex}
Show, using the double coset formula, that when $n > 0$, $\zeta$ is an isomorphism, and when $n < 0$, it's
multiplication by $2$.
\end{ex}
That is, when $n > 0$, it comes from the Mackey functor $\underline\Z$, and when $n < 0$, it comes from
$\underline\Z$.
\subsection*{The Burnside Mackey functor.}
Now we'll repeat this all in the slightly harder case of the Burnside Mackey functor $A_{C_2}$. The published
source is~\cite{Lew88}, building on unpublished work of Stong. The answer is somewhat crazy, and we'll have to
define a whole bunch of Mackey functors along the way.
\begin{beast}\hfill
\label{best_bestiary}
\begin{enumerate}
	\item In the case $G = C_2$,\footnote{For $C_p$, it's the same, but replace $x+2y$ with $x+py$.} the Burnside
	Mackey functor $A$ takes the form\footnote{\TODO: I want to work through this and understand it.}
	\[\xymatrix{
		\Z\oplus\Z\ar@/_0.4cm/[d]_{(x,y)\mapsto x+2y}\\
		\Z.\ar@(dl,dr)_\id\ar@/_0.4cm/[u]_{(0,1)}
	}\]
	\item For any abelian group $G$, there's a Mackey functor $\ang G$ for which all of the maps are $0$:
	\[\xymatrix{
		G\ar@/_0.4cm/[d]_0\\
		0.\ar@(dl,dr)_\id\ar@/_0.4cm/[u]_0
	}\]
	\item Given a $C_2$-Mackey functor, we can forget to a $\Z[C_2]$-module by evaluating at $C_2/e$. This has
	both left and right adjoints: if $M$ is a $\Z[C_2]$-module, the left adjoint, denoted $L(M)$, is\footnote{\TODO:
	are the maps wrong?}
	\[\xymatrix{
		M/C_2\ar@/_0.4cm/[d]_\tr\\
		M,\ar@(dl,dr)_\id\ar@/_0.4cm/[u]_\pi
	}\]
	where $\pi$ is projection. The right adjoint, denoted $R(M)$, is
	\[\xymatrix{
		M^{C_2}\ar@/_0.4cm/[d]\\
		M,\ar@(dl,dr)_\id\ar@/_0.4cm/[u]
	}\]
	where the map down is the inclusion of fixed points, and the map up is the trace: if $\gamma\in C_2$ is the
	nontrivial element, send $x\mapsto x+\gamma x$.
	\item Regarding $\Z$ as a trivial $\Z[C_2]$-module, we obtain $L\coloneqq L(\Z)$ and $R\coloneqq R(\Z)$, which
	are explicitly
	\[\gathxy{
		\Z\ar@/_0.4cm/[d]_{\cdot 2}\\
		\Z\ar@(dl,dr)_\id\ar@/_0.4cm/[u]_\id
	}\qquad\text{and}\qquad
	\gathxy{
		\Z\ar@/_0.4cm/[d]_\id\\
		\Z,\ar@(dl,dr)_\id\ar@/_0.4cm/[u]_{\cdot 2}
	}
	\]
	respectively.
	\item If $\Z\sigma$ denotes $\Z$ as a $\Z[C_2]$-module with the sign action, we obtain $L_-\coloneqq
	L(\Z\sigma)$ and $R_-\coloneqq R(\Z\sigma)$, which are explicitly
	\[\gathxy{
		\Z/2\ar@/_0.4cm/[d]_0\\
		\Z\ar@(dl,dr)_{-1}\ar@/_0.4cm/[u]_\pi
	}\qquad\text{and}\qquad
	\gathxy{
		0\ar@/_0.4cm/[d]_0\\
		\Z,\ar@(dl,dr)_{-1}\ar@/_0.4cm/[u]_0
	}
	\]
	respectively.
\end{enumerate}
\end{beast}
\begin{ex}
Show that $L(\bl)$ and $R(\bl)$ are really left and right adjoints to the forgetful map
$\cat{Mac}\to\Mod_{\Z[C_2]}$.
\end{ex}
It's possible to write down lots of Mackey functors, and it's entertaining to do so. But the key examples arise as
$\pi_0 X$, where $X$ is a $G$-spectrum. You might ask if all Mackey functors arise this way --- and for once we
know the answer: we constructed Eilenberg-Mac Lane spectra for every Mackey functor, so the answer is yes!

The next question: do functors on the category of Mackey functors arise as $\pi_0$ of functors on $\Spc^G$? You can
also ask lots of questions from commutative algebra, e.g.\ how to think of rings and modules in this context, and
the answers are generally harder than in the purely algebraic case.

Anyways, using the bestiary, we can state the answer. It's complicated, but the dimension axiom is complicated in
equivariant cohomology too.
\begin{thm}
\label{Aval}
\[H_{C_2}^{p,q}(*; A_{C_2}) = \begin{cases}
	A, &p = q = 0\\
	R, &p+q = 0,\ p < 0,\ p\text{\rm{} even}\\
	R_-, &p+q = 0,\ p \le 1,\ p\text{\rm{} odd}\\
	L, &p+q = 0,\ p > 0,\ p\text{\rm{} even}\\
	L_-, &p+q = 0,\ p < 1,\ p\text{\rm{} odd}\\
	\ang\Z, &p+q\ne 0, p = 0\\
	\ang{\Z/2} & p+q > 0,\ p < 0,\ p\text{\rm{} even}\\
	\ang{\Z/2} & p+q < 0,\ p > 1,\ p\text{\rm{} even}\\
	0, &\text{\rm otherwise.}
\end{cases}\]
\end{thm}
\begin{rem}
The description of $H_{C_2}^*(*;A_{C_2})$ in~\cite[Thm.~2.1]{Lew88} use a different grading, indexing by $(p+q,p)$.
\end{rem}
So the good news is, the answer is determined by the fixed and total dimensions. This also generalizes nicely to
$C_p$ when $p$ is odd. Though this looks kind of frightening, Lewis~\cite{Lew88} proved that if $X$ has
even-dimensional cells, then its cohomology is free over that of a point, which is good. Moreover, the
$\RO(C_2)$-grading makes these statements cleaner; things like this just aren't true for the $\Z$-graded theories.

The computation follows by analyzing the following cofiber sequence(s),\footnote{In the nonequivariant case, you
probably wouldn't dwell on this so explicitly, just like we mentioned Spanier-Whitehead duality for a point to make
it clearer what's going on.} and proceeds in a similar way as for $\underline\Z$, just with a harder
answer.\footnote{\TODO: some confusion as to which cofiber sequences work out correctly.}
\begin{gather}
%\xymatrix{
%	(C_2)_+\ar[r] & S(\eta)_+\ar[r] & \Sigma(C_2)_+
%}\\
% \xymatrix{
% 	S(\eta)_+\ar[r] & D(\eta)_+\ar[r] & S^\eta
% }\\
\label{sigmacofib}
\xymatrix{
	(C_2)_+\cong S(\sigma)_+\ar[r] & D(\sigma)_+\ar[r] & S^\sigma
}
\end{gather}
Here $\sigma$ is the sign representation, and we'd like $\eta$ to be the nontrivial irreducible complex
$C_2$-representation, regarded as a two-dimensional real representation.

At this point, things got confused, and we'll fix them Tuesday.
