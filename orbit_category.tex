%!TEX root = m392c_EHT_notes.tex

\begin{quote}\textit{
	``What's bad about this proof?''\\
	``It appeals to machinery we didn't develop in this class?''\\
	``No, that's perfectly fine.''
}\end{quote}
We'll start by reviewing the connection between the orbit category and $G$-sets.

Let $X$ be a finite $G$-set. Then, $X$ is the coproduct (disjoint union) of a bunch of orbits:
\[x \cong \coprod_i G/H_i.\]
The way you see this is that for any $x\in X$, its orbit is isomorphic to $G/G_x$. This is yet another
manifestation of the slogan that ``orbits are points.'' But it also implies that, rather than just presheaves on
$\sO_G$, one could work with certain presheaves on the category of finite $G$-sets, and this perspective will turn
out to be useful. By ``certain'' we mean a compatibility with orbits.

Last time, we talked about Elmendorf's theorem in the form of Theorem~\ref{revElmen}. It's also possible to state
it in a more general form.
\begin{thm}[Elmendorf]
\label{elmendorf}
The functor $G\Top\to\Fun(\sO_G\op, \Top)$ determined by $X\mapsto \paren{G/H\mapsto X^H}$ induces an equivalence
of $(\infty, 1)$-categories, where the weak equivalences on the left and right are specified by a family $\sF$.
\end{thm}
Without delving into $(\infty,1)$-categories, this means
\begin{itemize}
	\item the homotopy categories are equivalent, and
	\item homotopy limits and colimits behave identically.
\end{itemize}
In other words, from the perspective of abstract homotopy theory, these are the same.
\begin{defn}
By a \term{family} of subgroups $\sF$ of $G$, we mean a collection of subgroups of $G$ closed under conjugation and
taking subgroups.
\end{defn}
Examples include the set of all subgroups, the set of just the identity, and the set of finite subgroups. The
latter is useful for some $S^1$-equivariant spaces, where one tends to lose control of the $S^1$-fixed points, but
the finite subgroups behave better.
\begin{defn}
Let $\sF$ be a specified family of subgroups of $G$.
\begin{itemize}
	\item In $G\Top$, the weak equivalences specified by $\sF$ are the maps $f:X\to Y$ such that $f^H:X^H\to Y^H$
	is a weak equivalence for all $H\in\sF$.
	\item For $\Fun(\sO_G\op, \Top)$, a weak equivalence specified by $\sF$ is a pointwise weak equivalence at
	$G/H$ for all $H\in\sF$.
\end{itemize}
\end{defn}
We'll give two proofs of Theorem~\ref{elmendorf}. The first will be model-categorical.

Recall\footnote{If this is not review to you, then exercise: learn this material!} if $F: \fC\rightleftarrows \fD:
G$ is a Quillen adjunction, then the left and right derived functors $(\LD F, \RD G)$ is an adjunction on the
homotopy categories $(\operatorname{Ho}\fC, \operatorname{Ho}\fD)$. If $K$ denotes fibrant replacement in $\fD$ and
$Q$ denotes cofibrant replacement in $\fC$, then the derived functors are $\LD F = FQ$ and $\RD G =
GK$.\footnote{This does require cofibrant and fibrant replacement to be functorial, which is not true in every
model category, but will be true for pretty much everything we study.}
\begin{defn}
That $(F, G)$ is a \term{Quillen equivalence} means that for any cofibrant $X\in\fC$ and fibrant $Y\in\fD$, then
$FX\to Y$ is a weak equivalence iff its adjoint $X\to GY$ is.
\end{defn}
This is equivalent to asking that $(\LD F, \RD G)$ are equivalences of categories.

This is a kind of curious way to look at an equivalence of categories. One says that $G$ \term{creates the weak
equivalences} of $\fD$ if the weak equivalences in $\fD$ are the maps $f:A\to B$ such that $Gf$ is a weak
equivalence.
\begin{lem}
\label{forgetWElem}
If $G$ creates the weak equivalences of $\fD$ and for all cofibrant $X$ the unit map $X\to GFX$ is a weak
equivalence, then $(F,G)$ is a Quillen equivalence.
\end{lem}
This is a useful tool for extending model categories along free-forgetful adjunctions; for example, if you have a
model category and want to understand abelian group or ring objects in this category, often their weak equivalences
are detected by the forgetful functor.
\begin{proof}[Proof sketch of Theorem~\ref{elmendorf}]
We want to apply Lemma~\ref{forgetWElem} to the adjunction
\[\theta: \Fun(\sO_G\op, \Top)\rightleftarrows G\Top: \psi,\]
where $\theta:X\mapsto X(G/e)$ is evaluation at $G/e$ and $\psi:Y\mapsto\set{Y^H}$. The first condition, that
$\psi$ detects the weak equivalences, is straightforward, so we need to check that $X\mapsto\set{X(G/e)^H}$ is a
weak equivalence for all cofibrant $X$.

Cellular objects model the generating cofibrations, so cofibrant objects are retracts of cellular objects. Since
weak equivalences are preserved under retracts, then we can check on cellular objects. Here it's easier, since
$(\bl)^H$ commutes with the relevant colimits and is suitably cellular.
\end{proof}
The missing steps in this proofs can be filled in by explicitly identifying the cofibrant objects in
$\Fun(\sO_G\op, \Top)$. These are free diagrams on the orbit category; not hard to write down, but messy enough to
avoid on the chalkboard.
\begin{rem}
Elmendorf's original proof of his theorem was in the 1980s did not use model categories, even though Quillen had
already introduced them at the time. Until the mid-1990s (30 years after Quillen introduced them), many homotopy
theorists avoided them, thinking of them as formal gobbledygook. However, about the time EKMM introduced a
symmetric monoidal category of spectra, people began realizing they were unavoidable.
\end{rem}
You might not like the given proof of Elmendorf's theorem because it's extremely inexplicit: cofibrant replacement
is an infinite process, and many of the steps involved are quite abstract. The next proof will be more explicit,
building a (homotopical) right adjoint to $\psi$.

This proof will go through the \term{bar construction}, a categorical tool that's extremely useful. References for
it include May's ``Geometry of iterated loop spaces'' \cite{MayGILS}, Riehl's monograph~\cite{RiehlCHT}, and Vogt's ``Tensor
products of functors.''
\begin{proof}[Second proof of Theorem~\ref{elmendorf}]
Let $M:\sO_G\to\Top$ realize orbits as spaces: $G/H$ is sent to the topological space $G/H$, and an equivariant map
$f$ is forgotten to a continuous map $f$.

Given an $X\in\Fun(\sO_G\op, \Top)$, let
\[\Phi(X) \coloneqq \abs{B_\bullet(X, \sO_G, M)}\]
denote the geometric realization of the simplicial bar construction. Let's be a little more explicit about this.
$B_\bullet(X, \sO_G, M)$ is a simplicial space that sends
\[[n]\mapsto \coprod_{G/H_{n-1}\to\dotsb\to G/H_0} X(G/H_0)\times M(G/H_{n-1}).\]
As usual, the face maps are defined by composition, and the degeneracies by inserting the identity map. Since $G$
acts on $M(\bl)$ simplicially (i.e., in a way compatible with the face and degeneracy maps), then
$\abs{B_\bullet(X, \sO_G, M)}$ is a $G$-space (passing through the coend formula for the geometric realization).

If $H\subseteq G$, we want to understand $\Phi(X)^H$. Because the $G$-action passed through geometric realization,
\[\Phi(X)^H\cong \abs{B_\bullet(X, \sO_G, M^H)}\cong \abs{B_\bullet(X, \sO_G, \Map_{\sO_G}(G/H, \bl))}.\]
Let $X(G/H)$ denote the constant simplicial space $[n]\mapsto X(G/H)$. Then, by general theory of the bar
construction for any corepresented functor, there's a simplicial map
\begin{equation}
\label{extra_degeneracy}
B_\bullet(X, \sO_G, \Map_{\sO_G}(G/H, \bl))\longrightarrow X(G/H)
\end{equation}
defined by composing and applying $X$, and this is a simplicial homotopy equivalence (you can write down a
retraction).\footnote{This is called an \term{extra degeneracy argument} in the literature. There's an observation
probably due to John Moore which approximately says that if you have a simplicial object with an extra degeneracy
condition playing well with the preexisting ones, then it must be contractible; this argument is applied to the
fiber of~\eqref{extra_degeneracy}.} Thus, $\Phi(X)^H\cong X(G/H)$. In other words, $\Phi$ is a homotopy inverse,
since taking $H$-fixed points of $\Phi(X)$ gives back what you started with.
\end{proof}
$\Phi(X)$ is still an infinite-dimensional object, but it's much more explicit, and you can work with it.
\subsection*{Applications of this perspective.}
We'll be able to use Elmendorf's theorem to make some constructions that would be hard to imagine without the orbit
category.
\begin{defn}
Let $\sF$ be a family of subgroups of $G$. Then, the \term{classifying space} for $\sF$ is specified by the
universal property that if $Z$ has $\sF$-isotropy, then $[Z, E\sF]$ has a unique element. An explicit construction
is to let $\widetilde E\sF$ denote the presheaf on the orbit category where
\[\widetilde E\sF(G/H) \coloneqq
\begin{cases}
	*, &H\in\sF\\
	\varnothing, &H\not\in\sF,
\end{cases}\]
and let $E\sF\coloneqq \Phi(\widetilde E\sF)$.
\end{defn}
If you unwind the definition, this is the bar construction applied to $G$ in the category of $G$-spaces with weak
equivalences given by $\sF$, meaning it deserves to be called a classifying space.

Another useful notion is the $G$-connected components.
\begin{defn}
Let $X$ be a $G$-space and $x\in X^G$. Let $Y_x$ be the presheaf on the orbit category sending $H$ to the connected
component containing $x\in X^H$. Then, the \term{$G$-connected component} of $x$ is $\Phi(Y_x)$.
\end{defn}
The third useful application is defining Eilenberg-Mac Lane spaces. This will lead us to cohomology (and then to
Smith theory and other things). These will be constructed by working pointwise, then applying $\Phi$.
\begin{defn}
Let $G$ be a finite group, A \term{coefficient system} is a presheaf $X\in\Fun(\sO_G\op, \Ab)$.\footnote{For $G$ a
compact Lie group, the definition is almost the same, but we need to use $\Fun(h\sO_G\op, \Ab)$, taking presheaves
on the homotopy category. For finite groups these definitions coincide.}
\end{defn}
Elmendorf's theorem says that for any coefficient system, we have an Eilenberg-Mac Lane $G$-space. You could say
here that (Bredon) cohomology is completely determined: cohomology is the things represented by Eilenberg-Mac Lane
spaces. But it will be good to see it explicitly. Bredon cohomology is explicit, but there are serious drawbacks:
it has poor formal properties, and you need a lot of geometric insight to compute things. We'll later see that this
abelian category (meaning we can do homological algebra) is the wrong one; we'll later see this is a $\Z$-graded
cohomology theory (or rather graded on subgroups of $\Z$); this will be the wrong answer, especially if you want
Poincaré duality, and the right answer uses a grading by the representation ring. But we'll get there.
\begin{rem}
Another application of Elmendorf's theorem, which we will not discuss in detail (unless we get to the slice
filtration), is Postnikov towers. They're constructed in the same way, by either using the small object argument or
killing homotopy groups.
\end{rem}
Here are some examples of coefficient systems (which are often denoted with underlines).
\begin{comp}{exm}{enumerate}
	\item For a $G$-space $X$, the coefficient system $\underline\pi_n(X)$ ($n\ge 2$) sends $G/H\mapsto\set{\pi_nX^H}$. This is an example of
	a general formula: given a functor $\Top\to\Ab$, we can compose to obtain a map $\Fun(\sO_G\op,
	\Top)\to\Fun(\sO_G\op, \Ab)$.
	\item In the same way, $\underline H_n(X)$ sends $G/H\mapsto\set{H_n(X^H;\Z)}$.\qedhere
\end{comp}
We will now define Bredon cohomology, which is what you think it is.
\begin{defn}
Let $X$ be a $G$-CW complex and $X_n$ denote its $n$-skeleton. Let
\[\underline C_n(X) \coloneqq \underline H_n(X_n, X_{n-1};\Z),\]
i.e.\ this coefficient system sends $G/H\mapsto H_n((X^H)_n, (X^H)_{n-1};\Z)$. The connecting
homomorphism comes as usual from the triple $((X^H)_n, (X^H)_{n-1}, (X^H)_{n-2})$.\footnote{This requires knowing
how to obtain a CW structure on $X^H$ given a $G$-CW structure on $X$. If $G$ is finite, this is easy to see; for
general compact Lie groups, though, this requires a triangulation argument. One wants the resulting coefficient
system to be independent of the choice of triangulation, but as in the nonequivariant case, this is proven via an
axiomatic characterization of cohomology.}

The \term{Bredon cohomology} for the coefficient system $M$ is
\[H^n_G(X;M) \coloneqq H^n\paren{\Hom_{\Fun(\sO_G\op, \Ab)}(\underline C_\bullet(X), M)}.\]
\end{defn}
That is, it's the cohomology of the cochain complex of natural transformations from $\underline C_\bullet(X)$ to $M$.
\begin{exm}
\label{S2pi}
Give $S^2$ the $C_2$-action that rotates by $\pi$, and call it $Z$, and let $M$ denote the constant coefficient
system at $\Z$. First, we compute the cells:
\begin{itemize}
	\item $\underline C_2(Z)(C_2/e) = \Z\oplus\Z$ and $\underline C_2(Z)(C_2/C_2) = 0$.
	\item $\underline C_1(Z)(C_2/e) = \Z\oplus\Z$ and $\underline C_1(Z)(C_2/C_2) = 0$.
	\item $\underline C_0(Z)(C_2/e) = \Z\oplus\Z$ and $\underline C_0(Z)(C_2/C_2) = \Z\oplus\Z$.\qedhere
\end{itemize}
\end{exm}
\begin{ex}
Figure out the differentials in the above example.
\end{ex}
We'll continue to discuss Bredon cohomology next lecture, and introduce an axiomatic viewpoint. This is basic and
fundamental, but not too relevant to the rest of the class. It's also good to calculate; there are surprisingly few
examples out in the world.
