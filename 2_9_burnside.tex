%!TEX root = m392c_EHT_notes.tex
Today, we're going to talk about the Burnside category; there are at least five definitions of it, and we'll
provide one. We'll also talk about spectra; some things will be precise, and others will be made precise next time.

Recall that if $M$ is a $G$-manifold, we can equivariantly embed it in a finite-dimensional real $G$-representation
$V$, so that we obtain a sequence of maps $S^V\to T\!\nu\to T(\nu\oplus\tau)\cong S^V\wedge M_+ = \Sigma^V M_+$.
(Here $\nu$ is the normal bundle of $M$ and $\tau$ is the tangent bundle.) Thinking of this as a transfer $M\to *$,
we'd like to do something similar for $G/H$. Namely, if $K\subset H$ are subgroups of $G$, we'd like to define a
transfer map $G/K\to G/H$, a stable map (i.e.\ one that lives in $\colim\Map(\Sigma^V G/H, \Sigma^V G/K)$). This
will be a ``wrong-way'' map.

Applying $G\times_H\bl$ to $H/K\to *$, we get a map $G/K\to G/H$. The transfer for $G/K\to G/H$ is via the transfer
$H/K\to *$, induced up from a map $H/K\inj V$. Here, $V$ is a $G$-representation, and the embedding is
$H$-equivariant, where $H$ acts on $V$ by restriction. Thus, we obtain a map $S^V\to S^V\wedge H/K$.

We want to understand how induction works on $H$-spaces whose structure has been defined by restriction from
$G$-spaces.
\begin{ex}
Show that if $X$ is a $G$-space and $i_H^*X$ is the $H$-space defined by restriction, $G\wedge_H i_H^*X\cong
G/H_+\wedge X$.
\end{ex}
So our map $G\wedge_H S^V\to G\wedge_H(S^V\wedge H/K_+)$ is identified with a map $G/H_+\wedge S^V\to G/K_+\wedge
S^V$.

Consider the maps in $\Hom_{\sO_G}(G/H_1, G/H_2)$ that are determined by subconjugacy, i.e.\ $G/H_1\congto
G/g^{-1}H_1g\to G/H_2$ (we assume $g^{-1}H_1g\subset H_2$). There's a covariant functor from $\sO_G$ to the
(equivariant) $S$-category which sends a map $f$ to its representative in the colimit, and we also have a
contravariant functor: $G/H_1\to G/H_2$ takes the transfer map, then the inverse of the conjugation isomorphism.
\begin{defn}
The \term{Burnside category} $B_G$ is the full subcategory of the $S$-category spanned by the orbits $G/H$.
\end{defn}
It's a theorem that every map in $B_G$ is a composition of two maps in the images of the two functors $\sO_G\to\cat
S$ described above, which is pretty neat.

The Burnside category is enriched in $\Top$, because the $S$-category is.
\begin{defn}
The \term{algebraic Burnside category} is $\pi_0(B_G)$, which is enriched in $\Ab$. A \term{Mackey functor} is a
functor $\pi_0B_G\to\Ab$.
\end{defn}
Mackey functors will be our replacement for coefficient systems. They arise in other contexts, so you could care
about Mackey functors without caring about equivariant stable homotopy theory (well, you probably secretly do
anyways). For that reason, there are definitions of Mackey functors that are less homotopical.

We'd like to define the category of $G$-spectra to be ``spectral Mackey functors,'' i.e.\ functors from $B_G$ to
spectra. We need to develop tools for this first: we haven't developed a good model for spectra (in particular, a
point-set model, not a model of the stable homotopy category), and we want these functors to be enriched, which
will require some more work. We'll now begin to develop the tools to surmount these problems.
\subsection*{Recollections about transfers.}
One concrete example of a transfer map comes from a finite cover $\widetilde X\to X$: summing over the fibers
defines a transfer map $H^*(\widetilde X;\Q)\to H^*(X;\Q)$.

Transfers also arise in group cohomology $H^*(G;M)$, where $M$ is a $G$-module. This is computed by resolving $\Z$
as a $\Z[G]$-module by some resolution $P_\bullet$, then computing $H^*(\Hom_G(P_\bullet, M))$. Now suppose
$H\subset G$; there are two adjoints to the restriction functor $\Mod_G\to\Mod_H$.
\begin{itemize}
	\item The left adjoint sends an $H$-module $N\mapsto \Z[G]\otimes_{\Z[H]} N$. This is called the \term{induced
	$G$-module} $\Ind_H^G N$.
	\item The right adjoint sends $N\mapsto\Map_H(\Z[G], N)$. This is called the \term{coinduced $G$-module},
	written $\CoInd_H^GN$.
\end{itemize}
These definitions, and their names, should look familiar.

Now suppose $H\subset G$ is finite-index, e.g.\ when $G$ is a finite group. Then
\[\Ind_H^G N\cong\bigoplus_{g\in G/H} gN\qquad\text{and} \CoInd_H^G N\cong\prod_{g\in G/H} gN.\]
In an additive category, finite sums and finite products are isomorphic, so $\Ind_H^GN\cong\CoInd_H^G N$. When $G$
is finite, the analogous statement about {\color{red}TODO} certain sums and products uniquely characterizes the
equivariant stable category.\footnote{This is an instance of some life advice from Mike Hill: in order to avoid
being confused by constructions in equivariant stable homotopy theory, try computing the analogous construction in
group cohomology first. Group cohomology is very concrete, and so it's definitely worth thinking this through in
this case.}

So if $H\subset G$, we'd like to build a transfer map $H^*(H;M)\to H^*(G;M)$, where $M$ is a $G$-module made an
$H$-module by restricting. Induction is left adjoint to the restriction functor $i_H^*$, so the counit of the
adjunction is a map $\eta\colon\Ind_H^G i_H^*M\to M$. In particular, $H^*(H;M)\cong H^*(G;\Ind_H^G
M)\stackrel\eta\to H^*(G;M)$, and this is the desired transfer map.

As group cohomology $H^*(G;M)$ is the cohomology of $BG$ in the local system defined by $M$, you could ask whether
the transfer map is induced from a map $BG\to BH$ (in $\Top$). This is true, and the map is a
fibration.\footnote{If you don't restrict to finite subgroups, this is not quite true: if $H\subset G$ is a
subgroup of infinite index, the map in question only exists on the spectrum level.}
\subsection*{Diagram spectra.}
This section is motivated by~\cite{MMSS}, an excellent paper that constructs the point-set stable category using
diagram spectra. You should absolutely read this paper; it's a masterwork of exposition and making things look
simple and clear in retrospect.

The approach of diagram spectra is different from, but equivalent to, the approaches taken
in~\cite{LMS,AlaskaNotes}.

Our goal is to define a complete, cocomplete, symmetric monoidal category $\Spc$ such that
\begin{itemize}
	\item the $S$-category is a full subcategory of $\Spc$, and
	\item there is a symmetric monoidal functor $\Sigma^\infty\colon\Top\to\Spc$, which is left adjoint to a right
	adjoint $\Omega^\infty\colon\Spc\to\Top$.
\end{itemize}
There's a sense in which $\Spc$ is the smallest category satisfying these hypotheses, or that you get it by adding
limits and colimits to $\cat S$. In particular, we are constructing a stable analogue of the category of
topological spaces, \emph{not} its homotopy category.
\begin{rem}
Historically, Boardman constructed the stable homotopy category as a formal completion of the $S$-category. Then,
people tried to find ``point-set models,'' stable model categories whose homotopy categories are isomorphic to
Boardman's category. There are several options, but explicit proofs that their homotopy categories are equivalent to
Boardman's are rare in the literature.
\end{rem}
\begin{defn}
By a \term{diagram} $\fD$ we mean a small category, which we assume is enriched in $\Top$ and symmetric monoidal.
The category of \term{$\fD$-spaces} is the category $\Fun(\fD,\Top)$ of enriched functors.
\end{defn}
The category of $\fD$-spaces is symmetric monoidal under \term{Day convolution}. The idea is to build a symmetric
monoidal product via left Kan extension: if $F$ and $G$ are $\fD$-spaces, the functor $F\overwedge
G\colon(d_1,d_2)\mapsto F(d_1)\wedge G(d_2)$ is a $(\fD\times\fD)$-space. To produce a $\fD$-space from this, let
$\boxtimes\colon\fD\to\fD$ be the monoidal product on $\fD$, and consider the left Kan extension
\[\xymatrix{
	\fD\times\fD\ar[r]^{F\overwedge G}\ar[d]_\boxtimes & \Top\\
	\fD\ar@{-->}[ur]_{F\wedge G}
}\]
This is our symmetric monoidal product $F\wedge G$. More explicitly,
\begin{align*}
( F\wedge G)(z) &= \colim_{x\boxtimes y = z} F(x)\wedge G(y)\\
								&\coloneqq \int^{x, y\in \fD} F(x) \wedge G(y) \wedge \fD(x\boxtimes y, z).
\end{align*}
This looks like the usual convolution, and the analogy with harmonic analysis can be taken further, e.g.\ in an
unpublished paper of Isaksen-Behrens.

For any $d\in\fD$, there's an \term{evaluation} functor $\Ev_d\colon\Fun(\fD,\Top_*)\to\Top_*$ sending $X\mapsto
X(d)$. It's adjoint to $F_d\colon\Top_*\to\Fun(\fD,\Top_*)$ defined by
\[(F_dA)(e)\coloneqq \Map_\fD(d, e)\wedge A_+.\]
The unit for the symmetric monoidal structure on $\Fun(\fD,\Top)$ is $F_0S^0$.

Let $R$ be a \term{commutative monoid object} in $\Fun(\fD,\Top)$, which approximately means there are maps
$F_0S^0\to R$, $S^0\to R(0)$, and a unital, associative, commutative map $R(d)\wedge R(e)\to R(d\boxtimes e)$. For
example, the unital condition is that the composition
\[\xymatrix{
	R(d)\wedge S^0\ar[r] & R(d)\wedge R(0)\ar[r] & R(d\boxtimes 0)\cong R(d)
}\]
must be the identity.

In this case, we can define the category $\Mod_R$ of \term{$R$-modules} in $\Fun(\fD,\Top)$, those $\fD$-spaces $M$
with an action map $\mu\colon R\wedge M\to M$ (satisfying the usual conditions). This is also a symmetric monoidal
category (this requires $R$ to be commutative), defined in the same way as the tensor product of modules over a
ring: $M\wedge_R N$ is the coequalizer
\[\xymatrix{
	M\wedge R\wedge N\dblarrow[r] &M\wedge N\ar[r] &M\wedge_R N.
}\]
\begin{exm}[Prespectra]
\label{prespectra}
Let $\fD = \N$, with only the identity maps. This is symmetric monoidal under addition: $[m]\boxtimes [n]\coloneqq
[m+n]$. The assignment $S_\N\colon[n]\mapsto S^n$ is a monoid in $\N$-spaces, and the category of $S_\N$-modules is
classically called \term{prespectra}; the monoidal structure is the identification of $S^m\wedge S^n\cong S^{m+n}$.

\begin{warn}
$S_\N$ is \emph{not} a commutative monoid! $S^n\wedge S^m\not\cong S^m\wedge S^n$.
\end{warn}
This was the cause of thirty years of pain and suffering in the community --- they didn't know they were unhappy.
People knew what the smash product should be on the homotopy category, and wanted a point-set model that's
symmetric monoidal, unlike this example.
\end{exm}
Symmetric spectra are one answer, which we won't use in this class. If all of this had been stated in terms of the
Day convolution from the get-go, people probably would have figured out symmetric spectra as early as the 1960s,
but hindsight is always clearer, and here we are. Symmetric spectra were introduced in~\cite{HSS};
see~\cite{SchwedeSymmSpec} for a detailed introduction.
\begin{exm}[Orthogonal spectra]
\label{orthogonal_spectra}
Let $\sI$ denote the category whose objects are finite-dimensional real inner product spaces $V$, and whose
morphisms $\sI(V,W)$ are the linear isometric isomorphisms $V\to W$.

In this category, $V\oplus W$ and $W\oplus V$ aren't equal, but are isomorphic, and the isomorphism between them is
reflected in the flip between $S^n\wedge S^m$ and $S^m\wedge S^n$. In particular, the assignment $S_\sI\colon
V\mapsto S^V$ (the one-point compactification of $V$) is a \emph{commutative} monoid, so the category of
$S_\sI$-modules is a symmetric monoidal category, called the category of \term{orthogonal spectra}. This is the
model of the stable category that we will use.
\end{exm}
\begin{exm}[$\sW$-spaces]
\label{wspaces}
Let $\sW$ be the category of finite CW complexes (with either all maps or cellular maps; it doesn't really matter).
$\sW$-spaces are already like spectra, in a sense, in that they're modules over the identity functor
$i\colon\sW\inj\Top$. There's a map $\vp\colon A\to\Map(B, A\wedge B)$ sending $a\mapsto (b\mapsto a\wedge b)$, so
if $F$ is a
$\sW$-space, we have a sequence of maps
\[\xymatrix{
	A\ar[r]^-\vp & \Map(B, A\wedge B)\ar[r] &\Map(F(B), F(A\wedge B)).
}\]
Taking its adjoint defines a map
\[A\wedge F(B)\longrightarrow F(A\wedge B),\]
so $F$ is a module over $i$.
\end{exm}
The assignment $n\mapsto\R^n$ defines a functor $\N\to\sI$, and therefore a functor from prespectra to orthogonal
spectra; with the right model structures, this induces an equivalence of their homotopy category. Similarly, the
assignment $V\mapsto S^V$ defines a functor $\sI\to\sW$, hence a functor from orthogonal spectra to $\sW$-spaces,
and this also will induce a homotopy equivalence.
\begin{defn}
A prespectrum is an \term{$\Omega$-prespectrum} if for all $n$, $X_n\congto\Omega^mX_{m+n}$.
\end{defn}
\begin{defn}
If $X$ is a prespectrum and $q\in\Z$, the $q^{\text{th}}$ \term{homotopy group} of $X$ is
\[\pi_q(X)\coloneqq \colim_n \pi_{n+q}X(n).\]
A \term{$\pi_*$-isomorphism} of prespectra is a map that induces an isomorphism on all homotopy groups.
\end{defn}
Notice that negative homotopy groups exist, and may be nontrivial.
\begin{rem}
This is one approach to defining the stable category, and is not the only one. In~\cite{AdamsStableHomotopy} (which
is an excellent book), Adams uses a more naïve viewpoint of ``cells first, maps later'' which doesn't require such
abstraction, but it would be a huge mess to prove that his model is complete or cocomplete. The diagram spectra
approach rigidly separates point-set techniques (easy, but not as useful) from operations on the homotopy category
(more useful, but harder), and this separation is often useful. The $\infty$-categorical perspective mashes it all
together, which can be confusing, but is the only setting in which you can prove things such as the stable
category being initial.
\end{rem}
We'll reintroduce $G$-actions soon, and this is pretty slick using orthogonal spectra: we can replace $\sI$ with
the category of finite-dimensional $G$-representations with invariant inner products. Orthogonal spectra also have
a really nice homotopy theory relative to symmetric spectra (which have other advantages that don't apply as much
to us).
