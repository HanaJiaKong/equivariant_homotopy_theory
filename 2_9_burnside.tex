%!TEX root = m392c_EHT_notes.tex
\begin{quote}\textit{
	``The Burnside category is really nice, but the Sideburn category is a bit hairier.''
}\end{quote}
The Burnside category is a souped-up version of the orbit category that accounts for transfers. There are at least
five ways of defining it: we'll start with \cref{burn1} and later provide an alternative, \cref{burn2}.

Recall that if $M$ is a $G$-manifold, we can equivariantly embed it in a finite-dimensional real $G$-representation
$V$, so that we obtain a sequence of maps $S^V\to T\!\nu\to T(\nu\oplus\tau)\cong S^V\wedge M_+ = \Sigma^V M_+$.
(Here $\nu$ is the normal bundle of $M$ and $\tau$ is the tangent bundle.) Thinking of this as a transfer $M\to *$,
we'd like to do something similar for $G/H$. Namely, if $K\subset H$ are subgroups of $G$, we'd like to define a
transfer map $G/K\to G/H$, a stable map (i.e.\ one that lives in $\colim\Map(\Sigma^V G/H, \Sigma^V G/K)$). This
will be a ``wrong-way'' map.

Applying $G\times_H\bl$ to $H/K\to *$, we get a map $G/K\to G/H$. The transfer for $G/K\to G/H$ is via the transfer
$H/K\to *$, induced up from a map $H/K\inj V$. Here, $V$ is a $G$-representation, and the embedding is
$H$-equivariant, where $H$ acts on $V$ by restriction. Thus, we obtain a map $S^V\to S^V\wedge H/K$.

We want to understand how induction works on $H$-spaces whose structure has been defined by restriction from
$G$-spaces.
\begin{ex}
Show that if $X$ is a $G$-space and $i_H^*X$ is the $H$-space defined by restriction, $G\wedge_H i_H^*X\cong
G/H_+\wedge X$.
\end{ex}
So our map $G\wedge_H S^V\to G\wedge_H(S^V\wedge H/K_+)$ is identified with a map $G/H_+\wedge S^V\to G/K_+\wedge
S^V$.

Consider the maps in $\Hom_{\sO_G}(G/H_1, G/H_2)$ that are determined by subconjugacy, i.e.\ $G/H_1\congto
G/g^{-1}H_1g\to G/H_2$ (we assume $g^{-1}H_1g\subset H_2$). There's a covariant functor from $\sO_G$ to the
(equivariant) $S$-category which sends a map $f$ to its representative in the colimit, and we also have a
contravariant functor: $G/H_1\to G/H_2$ takes the transfer map, then the inverse of the conjugation isomorphism.
\begin{defn}
\label{burn1}
The \term{Burnside category} $B_G$ is the full subcategory of the $S$-category spanned by the orbits $G/H$.
\end{defn}
It's a theorem that every map in $B_G$ is a composition of two maps in the images of the two functors $\sO_G\to\cat
S$ described above, which is pretty neat.

The Burnside category is enriched in $\Top$, because the $S$-category is.
\begin{defn}
The \term{algebraic Burnside category} is $\pi_0(B_G)$, which is enriched in $\Ab$. A \term{Mackey functor} is a
functor $\pi_0B_G\to\Ab$.
\end{defn}
Mackey functors will be our replacement for coefficient systems. They arise in other contexts, so you could care
about Mackey functors without caring about equivariant stable homotopy theory (well, you probably secretly do
anyways). For that reason, there are definitions of Mackey functors that are less homotopical.

We'd like to define the category of $G$-spectra to be ``spectral Mackey functors\footnote{As devloped in the model categorical setting in \cite{bGjpM2013}, and in the $\infty$-categorical setting in \cite{cB2017}.},'' i.e.\ functors from $B_G$ to
spectra. We need to develop tools for this first: we haven't developed a good model for spectra (in particular, a
point-set model, not a model of the stable homotopy category), and we want these functors to be enriched, which
will require some more work. We'll now begin to develop the tools to surmount these problems.
\begin{rem}
Once we have a good category $\Spc^G$ of $G$-spectra, we can provide another, equivalent definition of the Burnside
category, as the full subcategory of $\Spc^G$ spanned by $\Sigma_+^\infty G/H$.
\end{rem}
\subsection*{Recollections about transfers.}
One concrete example of a transfer map comes from a finite cover $\widetilde X\to X$: summing over the fibers
defines a transfer map $H^*(\widetilde X;\Q)\to H^*(X;\Q)$.

Transfers also arise in group cohomology $H^*(G;M)$, where $M$ is a $G$-module. This is computed by resolving $\Z$
as a $\Z[G]$-module by some resolution $P_\bullet$, then computing $H^*(\Hom_G(P_\bullet, M))$. Now suppose
$H\subset G$; there are two adjoints to the restriction functor $\Mod_G\to\Mod_H$.
\begin{itemize}
	\item The left adjoint sends an $H$-module $N\mapsto \Z[G]\otimes_{\Z[H]} N$. This is called the \term{induced
	$G$-module} $\Ind_H^G N$.
	\item The right adjoint sends $N\mapsto\Map_H(\Z[G], N)$. This is called the \term{coinduced $G$-module},
	written $\CoInd_H^GN$.
\end{itemize}
These definitions, and their names, should look familiar.

Now suppose $H\subset G$ is finite-index, e.g.\ when $G$ is a finite group. Then
\[\Ind_H^G N\cong\bigoplus_{g\in G/H} gN\qquad\text{and} \CoInd_H^G N\cong\prod_{g\in G/H} gN.\]
In an additive category, finite sums and finite products are isomorphic, so $\Ind_H^GN\cong\CoInd_H^G N$. When $G$
is finite, the analogous statement about {\color{red}TODO} certain sums and products uniquely characterizes the
equivariant stable category.\footnote{This is an instance of some life advice from Mike Hill: in order to avoid
being confused by constructions in equivariant stable homotopy theory, try computing the analogous construction in
group cohomology first. Group cohomology is very concrete, and so it's definitely worth thinking this through in
this case.}

So if $H\subset G$, we'd like to build a transfer map $H^*(H;M)\to H^*(G;M)$, where $M$ is a $G$-module made an
$H$-module by restricting. Induction is left adjoint to the restriction functor $i_H^*$, so the counit of the
adjunction is a map $\eta\colon\Ind_H^G i_H^*M\to M$. In particular, $H^*(H;M)\cong H^*(G;\Ind_H^G
M)\stackrel\eta\to H^*(G;M)$, and this is the desired transfer map.

As group cohomology $H^*(G;M)$ is the cohomology of $BG$ in the local system defined by $M$, you could ask whether
the transfer map is induced from a map $BG\to BH$ (in $\Top$). This is true, and the map is a
fibration.\footnote{If you don't restrict to finite subgroups, this is not quite true: if $H\subset G$ is a
subgroup of infinite index, the map in question only exists on the spectrum level.}
