\begin{quote}\textit{
	``It seems that the people registered for the class and the people showing up for class are disjoint.''
}\end{quote}
Smith theory is an example application of Bredon cohomology. This theorem is very old, from the 1940s, so none of
the cohomology in the statement is equivariant.
\begin{thm}[Smith]
\label{smith}
Let $G$ be a finite $p$-group and $X$ be a finite $G$-CW complex such that (the underlying topological space of)
$X$ is an $\F_p$-cohomology sphere.\footnote{A space $X$ is an \term{$\F_p$-cohomology sphere} if there is an
isomorphism of graded abelian groups $H^*(X;\F_p)\cong H^*(S^n;\F_p)$ for some $n$.} Then, $X^G$ is either empty or
an $\F_p$-cohomology sphere of smaller dimension.
\end{thm}
There are sharper statements, but we can prove this one. It's the start of a long program to understand $H_*(X^G)$
using algebraic data calculated from $X$ and the action of $G$ on $X$. One useful tool in this is the \term{Borel
construction} (well, \emph a Borel construction) $EG\times_G X$ (where this is the usual balanced product, not a
pullback). This is a ``fattened up'' version of $X/G$.
\begin{defn}
The \term{Borel cohomology} of $X$ is $H_G^*(X) \coloneqq H^*(EG\times_G X)$.
\end{defn}
\begin{warn}
This notation is potentially confusing. Outside of the field of equivariant homotopy theory, ``equivariant
cohomology'' generally means Borel cohomology, not Bredon cohomology. In equivariant homotopy theory, $H_G^*$ can
be used to denote both. We will always make the coefficient system for Bredon cohomology explicit, so that
$H_G^*(X)$ unambiguously refers to Borel cohomology, as is standard in the literature.\footnote{The nLab uses a
different convention: Borel cohomology is $H_G^*(X;A)$, and Bredon cohomology is $\mathbf H^*(X_G)$.}
\end{warn}
The finiteness in \cref{smith} is key: Elmendorf's theorem lets us build a $C_p$-complex with non-equivariant
homotopy type $S^n$ and any set of fixed points. Thus, in the infinite-dimensional case, we should be looking at a
different thing than the fixed points, namely the homotopy fixed points.
\begin{defn}
The \term{homotopy fixed points} of a $G$-space $X$ is $X^{hG}\coloneqq \Map(EG, X)^G$.
\end{defn}
This can also be defined as a homotopy limit. This is ``easy'' to compute (relative to the rest of equivariant
homotopy theory, that is), as there are spectral sequences, nice models for $EG$, and other tools.

The terminal map $EG\to *$ induces a map $X^G\to X^{hG}$. The Sullivan conjecture is all about when this happens:
when you $p$-complete, things become really nice. (There will be a precise statement next lecture.)

Returning to \cref{smith}, we can use Bredon cohomology to give an easy, modern proof. The idea is to adroitly
choose coefficient systems such that we recover $H^n(X)$, $H^n(X^G)$, and $H^n((X/G)/G)$ from Bredon cohomology.
They'll fit into exact sequences, and using tools like Mayer-Vietoris, we'll get inequalities on the ranks of
groups. This will also use the fact that a short exact sequence of coefficient systems induces a long exact
sequence on $H_G^*$. The proof will be beautiful and short, unlike Smith's original proof!

\Cref{smith} is sufficiently classical that there are several different proofs. Ours illuminates Bredon cohomology
at the expense of obscuring the overarching goal of Smith theory. We'll try to be consistent with the notation
in~\cite{MaySmithTheory, AlaskaNotes}. We'll also discuss another proof by Dwyer and Wilkerson in ``Smith theory
revisited'' \cite{SmithRevisit}, a short, beautiful paper which is highly recommended. It uses the unstable
Steenrod algebra to prove \cref{smith} and more.
\begin{proof}[Proof of \cref{smith}]
First, we can quickly reduce to the case where $G = \Z/p$: if $H\subset G$ is a normal subgroup, then $X^G\cong
(X^H)^{G/H}$, so you can induct on the order of the group using the Sylow theorems. Thus, we will assume $G =
\Z/p$, whose orbit category is simple:
\[\xymatrix{
	\bullet\ar@(ul, ur)^g\ar[d]\\
	\bullet
}\]
There is a cofiber sequence
\[\xymatrix{
	X_+^G\ar[r] & X_+\ar[r] & X/X^G,
}\]
which is not particularly deep. We're going to construct three special coefficient systems $L$, $M$, and $N$ such
that
\begin{align*}
	H_G^*(X;L) &\cong \wH^*((X/X^G)/G; \F_p)\\
	H_G^*(X;M) &\cong H^*(X;\F_p)\\
	H_G^*(X;N) &\cong H^*(X^G;\F_p).
\end{align*}
These will fit into exact sequences which will imply the inequalities we wanted.\footnote{{\color{red}TODO}: $M$ is
$H^0$ applied to some spaces and map. What are they?}

How to we construct custom coefficient systems? Since these constructions commute with colimits, it suffices to
determine them via computation at $n = 0$ and $X = G/H$ for subgroups $H\subset G$, meaning just $e$ and $G$.

For $L$, we want to recover $\wH^0((X/X^G)/G;\F_p)$ for $X = G/e$ and $X = G/G$.
\begin{itemize}
	\item For $X = G/e$, we want $\wH^0((G/\varnothing)/G;\F_p)\cong \wH^0(G_+/G)\cong\F_p$.
	\item For $X = G/G$, we get $\wH^0((*/*)/G;\F_p) = 0$.
\end{itemize}
So we conclude $L(G/e) = \F_p$ and $L(G/G) = 0$.

For $M$, a similar calculation shows we need $H^0(G/e)\cong\F_p[G]$\footnote{Here, $\F_p[G]$ is the \term{group
ring}, the $\F_p$-algebra of functions $G\to\F_p$ with addition taken pointwise and multiplication determined by
requiring $\delta_g\cdot\delta_h = \delta_{gh}$, where $\delta_g$ is a delta function, equal to $1$ at $g$ and $0$
elsewhere.} and $H^0(G/G) = \F_p$, and for $N$, we need $N(G/e) = 0$ and $N(G/G)\cong\F_p$.
\begin{rem}
Almost everything in this proof generalizes; we will only need $X$ to be an $\F_p$-cohomology sphere in order to
know dimensions of a few things. But this technique of customized coefficient systems can be used elsewhere.
\end{rem}
Let $I$ denote the \term{augmentation ideal} of $\F_p[G]$, i.e. the kernel of the map $\F_p[G]\to\F_p$ sending all
$g\mapsto 1$. We will let $I^n$ refer to the coefficient system which assigns $I^n$ to $G/e$ and $0$ to $G/G$.

As coefficient systems, $M/I\cong\F_p$, and therefore there is a short exact sequence of coefficient systems
\[\shortexact{I}{M}{N\oplus L}.\]
This implies a long exact sequence in Bredon cohomology:
\[\xymatrix{
	\dotsb\ar[r] & H_G^q(X;I)\ar[r] & H_G^q(X;M)\ar[r] & H_G^q(X;N\oplus L)\ar[r] & H_G^{q+1}(X;I)\ar[r] & \dotsb
}\]
Exactness at $H^q(X;N\oplus L)$ implies
\[\rank H_G^q(X;N) + \rank H_G^q(X;L) \le \rank H_G^q(X;M) + \rank H_G^{q+1}(X;I).\]
That is,
\begin{equation}
\label{1st_smith_ineq}
\rank H^q(X^G;\F_p) + \rank\wH^q((X/X^G)/G;\F_p) \le \rank H^q(X;\F_p) + \rank H_G^{q+1}(X;I).
\end{equation}
We'll use this to strongly constrain $\rank H^q(X^G;\F_p)$, but first we need another inequality coming from
another exact sequence. Namely, the following sequence of coefficient systems is exact:
\[\shortexact{L}{M}{I\oplus N}.\]
This is because $I^p = 0$ and for $0\le n\le p-1$, $I^n/I^{n+1}\cong\F_p$. In particular, $I^{p-1}\cong\F_p\cong
L$, so we can think of $M/L$ as $M/I^{p-1}$. Now we play the same game: the induced long exact sequence is
\[\xymatrix{
	\dotsb\ar[r] & H_G^q(X;L)\ar[r] & H_G^q(X;M)\ar[r] & H_G^q(X;I\oplus N)\ar[r] & H_G^{q+1}(X;L)\ar[r] & \dotsb
}\]
which implies
\[\rank H_G^q(X;N) + \rank H_G^q(X;I) \le \rank H_G^q(X;M) + \rank H_G^{q+1}(X;L),\]
i.e.
\begin{equation}
\label{2nd_smith_ineq}
\rank H^q(X^G;\F_p) + \rank H_G^q(X;I) \le \rank H^q(X;\F_p) + \rank H^{q+1}((X/X^G)/G;\F_p).
\end{equation}
Let's use this to prove
\begin{equation}
\label{3rd_smith_ineq}
\rank \wH^q((X/X^G)/G;\F_p) + \sum_{i=q}^{q+r} \rank H^i(X^G;\F_p) \le \rank\wH^{q+r+1} + \sum_{i=q}^{q+r}\rank
H^i(X;\F_p).
\end{equation}
Let
\begin{align*}
	a_q &\coloneqq \rank H^q(X^G;\F_p)\\
	b_q &\coloneqq \rank H^q(X;\F_p)\\
	c_q &\coloneqq \rank H^q((X/X^G)/G;\F_p)\\
	d_q &\coloneqq \rank H_G^q(X;I).
\end{align*}
Then, \eqref{1st_smith_ineq} and~\eqref{2nd_smith_ineq} say
\[a_q + c_q\le b_q + d_{q+1}\qquad\text{and}\qquad a_q + d_q \le b_q + c_{q+1}.\]
Now, adding~\eqref{1st_smith_ineq} for $q$ even and~\eqref{2nd_smith_ineq} for $q$ odd
proves~\eqref{3rd_smith_ineq}.
% TODO: I did not follow this in class!

When $q = 0$ and $r$ is large, the finite-dimensionality of $X$ implies that
\begin{equation}
\label{interesting_bound}
\sum_i\rank H^i(X^G;\F_p) \le \sum_i \rank H^i(X;\F_p).
\end{equation}
This is already an interesting bound, especially relative to the amount of work we've put in.

Specializing to $X$ an $\F_p$-cohomology sphere,~\eqref{interesting_bound} means
\[\sum_i \rank H^i(X^G;\F_p) \le 2.\]
We want to show this sum isn't $1$ (so that we get the cohomology of a sphere) and that the top nonzero rank is at
most $n$. We will do this with another short exact sequence of coefficient systems:
\[\shortexact{I^{n+1}}{I^n}{L}.\]
From this, we get another long exact sequence. Applying the Euler characteristic, we obtain that
\begin{equation}
\label{euler_char_ineq}
\chi(X) = \chi(X^G) + p\widetilde\chi((X/X^G)/G).
\end{equation}
Here
\[\widetilde\chi(Y)\coloneqq \sum_i (-1)^i\rank \wH^i(Y)\]
is the \term{reduced Euler characteristic}.

Equation~\eqref{euler_char_ineq} already implies that $\chi(X) \equiv \chi(X^G)\bmod p$, so $\sum\rank
H^*(X^G;\F_p)\ne 1$ in our case.
\begin{comp}{ex}{enumerate}
	\item Think about choices of $q$ and $r$ that allow you to deduce $m\le n$, finishing the proof.
	\item Small changes need to be made to this argument when $p = 2$; what are they?\qedhere
\end{comp}
\end{proof}
This is an appealing proof: some fairly simple calculations and a dash of formal theory very effectively led to the
result. We'll give another proof with different advantages and disadvantages.

Smith theory naturally leads to questions about how to recover $H^*(X^G)$ algebraically from some equivariant
cohomology theory on $X$. For example, we could ask about Borel cohomology $H_G^*(X) \coloneqq H^*(EG\times_G X)$,
where $EG$ is a free $G$-space that's nonequivariantly contractible (which is simple to construct with the bar
construction or through Elmendorf's theorem).

\emph{In the following, all cohomology is understood to have coefficients in $\F_p$.} Recall that the Bredon
cohomology of $X$ is defined to be $H_G^*(X)\coloneqq H^*(EG\times_G X)$. For a subgroup $H\subset G$, let
$S_H\subset H^*(BG)$ be the multiplicative set generated by the classes in $H^2(BG)$ that are images of the
Bockstein homomorphism $H^1(BG)\to H^2(BG)$ of the elements that are nontrivial in $H^1(BH)$. This uses the fact
that Borel cohomology is an $H^*(BG)$-module: $H^*(EG\times_G *) \cong H^*(EG/G) = H^*(BG)$, and using the terminal
map $X\to *$ we get a map $H^*(BG)\to H^*(EG\times_G X)$.
\begin{thm}
\label{localization}
Let $G$ be a finite $p$-group and $H\subset G$ be a subgroup. Then, there is an isomorphism
$S_H^{-1}H_G^*(X)\isom S_H^{-1}H_G^*(X^H)$.
\end{thm}

There's a rich theory of unstable modules over the Steenrod algebra $\mathcal A_p$, which could fill a whole
semester. There's a functor $\Un$ which produces unstable $\mathcal A_p$-modules, in a sense by only keeping the
unstable part.
\begin{thm}[{Dwyer-Wilkerson~\cite{SmithRevisit}}]
\[H^*(X^G) \cong \F_p\otimes_{H^*(BG)} \Un(S_G^{-1}H^*(EG\times_G X)).\]
\end{thm}
The proof uses arguments that were hard to think of, but easy to follow.

We'll use these theorems to prove Smith's theorem using the Serre spectral sequence for
\[\xymatrix{
	X\ar[r] & EG\times_G X\ar[r] & BG.
}\]
This will be the nicest kind of spectral sequence argument: everything degenerates.

\Cref{localization} is an example of a general class of \term{localization theorems} in equivariant cohomology. In
these theorems, one considers the fiber sequence $X_+^G\to X_+\to X/X^G$, and wants to show that for some functor
$E$, $E(X_+^G)\cong E(X_+)$. This boils down to showing $E(X/X^G)$ vanishes, which will always follow from showing
that $E$ vanishes on $G$-spaces whose $G$-actions are free away from the basepoint. In general, this will reduce to
considering cells, so one considers $E(G/H_+\wedge \bigvee_\alpha S^{q_\alpha})$ for some wedge of spheres.

In our case, $E(G/H_+\wedge \bigvee_\alpha S^{q_\alpha})$ is
\[S_H^{-1} H_G^*\paren{G/H_+\wedge\bigvee_\alpha S^{q_\alpha}} = S_H^{-1} H^*\paren{EG\times_G
\paren{G/H_+\wedge\bigvee_\alpha S^{q_\alpha}}}\]
Now, the term $EG\times_G G/H\cong EG/H\cong BH$, so we have a piece that looks like $H^*(BH)$, which is how $BH$
inserts itself into the argument.
\begin{ex}
Finish the Serre spectral sequence proof of \cref{smith}. Hint: there's a simple geometric reason why the
spectral sequence collapses, which is what makes this whole thing go.
\end{ex}
Recall that the Borel cohomology $H^*(EG\times_G X)$ is an $H^*(BG)$-module, through the map $EG\times_G X\to
EG\times_G * = BG$. Thus, we should compute $H^*(BG)$ as a ring. In the case we care about, $G = (\Z/p)^n$; let's
start with $n = 1$.

There's a nice geometric model for $B\Z/p$ as an ``infinite-dimensional lens space:'' let $S^\infty$ denote the
unit sphere in an infinite-dimensional complex Hilbert space,\footnote{Alternatively, you could choose $S^\infty$
to be the colimit of $S^n$ for all $n$, through the inclusion $S^n\inj S^{n+1}$ at the equator. These two choices
are not homeomorphic, but produce homotopy equivalent models of $B\Z/p$.} and give it a $C_p$-action by $z\mapsto
e^{2\pi i/p}z$. Then, the infinite-dimensional lens space is $S^\infty/C_p$.

The quotient $S^\infty\surj S^\infty/C_p$ is a covering map, and $S^\infty$ is contractible. The proof goes through
some version of the Eilenberg swindle:
\begin{itemize}
	\item There is a homotopy from $\id_{S^\infty}$ to $s: (x_1,x_2,\dotsc)\mapsto (0, x_1, x_2, \dotsc)$. In fact,
	it's a straight-line homotopy.
	\item There is a straight-line homotopy from $s$ to $(0,x_1,\dotsc)\mapsto (1, 0, 0,\dotsc)$.
\end{itemize}
Thus $S^\infty/C_p$ is the quotient of a contractible space by a free $G$-action, so it deserves to be called
$B\Z/p$.

You can set up a cell structure on $B\Z/p$ as with finite-dimensional lens spaces, and therefore compute that
\[H^k(B\Z/p) \cong\begin{cases}
	\Z/p, &k\text{ even}\\
	0, &k\text{ odd.}
\end{cases}\]
Using the universal coefficient theorem, you can then deduce that
\[H^k(B\Z/p;\Z/p) \cong \Z/p\]
for all $k$. Now we want to deduce the ring structure.

Recall that if
\[\shortexact{M}{L}{N}{}\]
is a short exact sequence, it induces a long exact sequence in cohomology:
\[\xymatrix{
	\dotsb\ar[r] & H^k(\bl;M)\ar[r] &H^k(\bl;L)\ar[r] &H^k(\bl;N)\ar[r]^\beta &H^{k+1}(\bl;M)\ar[r] & \dotsb
}\]
Let's apply this to the short exact sequences
\[\xymatrix{
	0\ar[r] & \Z\ar[r]^{\cdot p}\ar[d]&\Z\ar[r]\ar[d] & \Z/p\ar[r]\ar[d] &0\\
	0\ar[r] & \Z/p\ar[r]^{\cdot p} &\Z/p^2\ar[r] & \Z/p\ar[r] &0.
}\]
This is a map of short exact sequences, inducing a map of their long exact sequences.
\begin{equation}
\gathxy{
	\dotsb\ar[r] & H^q(X;\Z)\ar[r]\ar[d] & H^q(X;\Z)\ar[r]\ar[d] & H^q(X;\Z/p)\ar[r]\ar[d] &
	H^{q+1}(X;\Z)\ar[r]\ar[d] &
	\dotsb\\
	\dotsb\ar[r] & H^q(X;\Z/p)\ar[r] & H^q(X;\Z/p^2)\ar[r] & H^q*(X;\Z/p)\ar[r]^\beta &H^{q+1}(X;\Z/p)\ar[r] &
	\dotsb
}
\end{equation}
The map $\beta:H^q(X;\Z/p)\to H^{q+1}(X;\Z/p)$ will be called the \term{Bockstein homomorphism}, and is a simple
example of a cohomology operation.

We now assume $p$ is odd.
\begin{lem}
If $n$ is odd, the Bockstein for $B\Z/p$ is an isomorphism; if $n$ is even, it's $0$.
\end{lem}
\begin{proof}
First, observe that the diagram
\[\xymatrix{
	H^n(B\Z/p;\Z)\ar[r]^f & H^n(B\Z/p;\Z/p)\ar[r]^g\ar[dr]_\beta & H^{n+1}(B\Z/p;\Z)\ar[d]^\pi\ar[r] &
	H^{n+1}(B\Z/p;\Z)\\
	&& H^{n+1}(B\Z/p;\Z/p)
}\]
commutes, and the top row is exact.
\begin{itemize}
	\item If $n$ is even, $f$ is an isomorphism, so $g = 0$, so $\beta = 0$.
	\item if $n$ is odd, $g$ and $\pi$ are surjections, so $\beta$ is a surjection between two $\F_p$-vector spaces
	of the same rank, so $\beta$ is an isomorphism.\qedhere
\end{itemize}
\end{proof}
Next, a nice way to compute the cup product. The map $B\Z/p\to\CP^\infty$ is cellular, and is a homeomorphism when
restricted to even-dimensional cells. As the cell structure determines the cup product structure, the cup products
on $H^{\text{even}}(B\Z/p;\Z/p)$ and $H^{\text{even}}(\CP^\infty;B\Z/p)\cong \Z/p[y_1]$ agree.\footnote{By
$\CP^\infty$, we mean the colimit of $\CP^n$ over all $n$, and similarly for $\RP^\infty$. However, there are
models for them based on the unit sphere in an infinite-dimensional Hilbert space modulo an $S^1$ (resp.\ $\Z/2$)
action. These are homotopic to, but not homeomorphic to, the colimit realizations of $\CP^\infty$ and $\RP^\infty$.
} On the odd-dimensional cells, the cup product is graded commutative rather than strictly commutative, so we get
an exterior algebra. Therefore we conclude that
\[H^*(B\Z/p;\Z/p)\cong \Lambda(x_1)\otimes \Z/p[y_1],\]
where $\abs{x_1} = 1$, $\abs{y_1} = 2$, and $\beta x_1 = y_1$.

By essentially the same argument, one can compute the cohomology ring for $B(\Z/p)^n$.
\begin{prop}
As rings, there is an isomorphism
\[H^*(B(\Z/p)^n; \Z/p) \cong \Lambda(x_1,\dotsc,x_n)\otimes\Z/p[y_1,\dotsc,y_n],\]
where $\beta x_i = y_i$.
\end{prop}
\begin{ex}
Figure out the slight changes needed for $p = 2$.
\end{ex}
Recall that the localization theorem asserted that if $G$ is a finite $p$-group, $S_H^{-1}H^*(EG\times_G
X)\cong S_H^{-1}H^*(EG\times_G X^G)$, where $S$ is the multiplicative system generated by images of the Bockstein
homomorphism in $H^*(BG;\Z/p)$. We'll prove this in the case when $G$ is abelian.

We mentioned above that it's possible to inductively reduce to considering $H^*(BH)\otimes H^*\paren{\bigvee
S^{q_i}}$. It's now clear\footnote{{\color{red}TODO}: I didn't follow this proof in class.} that something in $S$
restricts to $0$ in $H^*(BH)$, completing the proof (in the abelian case).

The localization fails terribly for infinite nonabelian compact Lie groups $G$. For example, for any topological
space $K$, there exists a $G$-CW complex $X$ such that $X$ is nonequivariantly contractible, $X$ is
finite-dimensional, and $X^G\simeq K$.
